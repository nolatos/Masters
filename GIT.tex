Consider the following example:
\begin{example}\label{naive-quotient}
	Let $k$ be an algebraically closed field and let $k^*$ act on $k^2$ by $\lambda \cdot (p,q):= (\lambda^{-1}p, \lambda q)$. The orbits consist of the axes without the origin, the origin and for every $t \in k^*$ the curve $xy = t$. If we consider the orbit space, it resembles $k$, indeed each nonzero $t$ will represent the curve $xy = t$; however where the origin should be, we find three orbits, one of which is closed and two of which has closure equal to the union of the three orbits. But if the orbit space of $k^2 = \A^2(k)$ was given the structure of a scheme (i.e. we have a morphism $\A^2 \rightarrow Y$ and the map of $k$-valued points is a bijection between orbits of $k^2$ and $Y(k)$), then it cannot be separated over $k$, and hence no variety (integral scheme, separated and of finite type over $k$) could be an orbit space (in fact we will show that there is no orbit space).
	
	But observe that if we consider the action restricted to $k^2\setminus\{(0,0)\}$, the orbit space is canonically identified with $k \setminus \{0\}$, which are the $k$-valued points of the variety $\Spec k[t, t^{-1}]$.
\end{example}
This raises the question, what is the \enquote{best} approximate quotient of an action and what sort of properties does it have? And can we always throw away \enquote{bad} orbits, like the restriction of $k^2$ to $k^2 \setminus \{(0,0)\}$, so that an orbit space does exist? Geometric invariant theory allows us to answer these questions under certain circumstances.

Being a vast and difficult subject, however, we do not have the time to develop the subject in detail. Our focus will be on the action of affine algebraic groups (to be defined) on quasi-projective varieties over $k$. For a full account, see \autocite{GIT}. Our exposition roughly follows the one found in \autocite{ModuliNotes}.
\section{Algebraic Groups and Actions}
\par{}We begin by formalising the notion of a \textit{group action}, since the notion of a group acting on a set is not sufficient for our purposes. Indeed, a morphism of schemes is not determined by where it sends its points (for example, there are two endomorphisms of the one-point scheme $\Spec k[\varepsilon]/\varepsilon^2$), and even if we define an action of some $G$ on some $X$ as a group homomorphism into $\Aut X$, it is possible for $G$ to have some scheme-theoretic structure (as in differential geometry where there is a concept of a Lie group), which will not be picked up if we treat $G$ as a collection of points. Hence we need the stronger notion of an \textit{algebraic group} acting on a scheme. We will begin with a very elementary approach to the subject, laying out all the formalisms, and doing calculations and examples extremely na\"ively and from scratch.
\begin{definition}\index{algebraic group}
	An \textit{algebraic group} over $k$, also known as a \textit{group scheme} over $k$ is a scheme $G$ over $k$ equipped with a $k$-point $e: \Spec k \rightarrow G$, known as the \textit{identity element} and morphisms $\mu: G \times G \rightarrow G$ and $\iota: G \rightarrow G$ known as \textit{multiplication} and \textit{inversion} respectively such that the following three diagrams commute:
	\begin{enumerate}
		\item (Asssociativity) 
		\begin{equation*}
			\begin{tikzcd}[row sep = huge]
				G \times G \times G \arrow[r, "\id \times \mu"] \arrow[d, "\mu \times \id"]& G \times G \arrow[d, "\mu"]\\
				G \times G \arrow[r, "\mu"] &G
			\end{tikzcd}
		\end{equation*}
		\item (Identity) 
		\begin{equation*}
			\begin{tikzcd}[row sep = huge]
				\Spec k \times G \arrow[r, "e \times \id"] \arrow[rd, "\cong"]& G \times G \arrow[d, "\mu"] &\arrow[l, "\id \times e"] G\times \Spec k \arrow[ld, "\cong"] \\
				&G
			\end{tikzcd}
		\end{equation*} 
		\item (Inverse) 
		\begin{equation*}
			\begin{tikzcd}[row sep = huge]
				G \arrow[r, "\iota\times \id"] \arrow[d] & G \times G \arrow[d, "\mu"]& \arrow[l, "\id \times \iota"] G\arrow[d] \\
				\Spec k \arrow[r, "e"] & G & \Spec k \arrow[l, "e"]
			\end{tikzcd}
		\end{equation*}
	\end{enumerate}
	An algebraic group $G$ is \textit{affine} if $G$ is an affine scheme. 
\end{definition}
\begin{remark}
	There are subtle variations in the definition of algebraic group and group scheme from one source to another (indeed, they are usually different!). For example, algebraic groups are sometimes required to be of finite type over $k$, the definition of a group scheme is usually a \enquote{group-valued functor represented by a scheme} (which we will make sense of very shortly), and sometimes algebraic groups are required to be varieties. However, since we will only ever deal with algebraic groups which are affine varieties, these subleties do not matter in this thesis.
\end{remark}
It is important to emphasise that while an algebraic group does have the structure of a group, in this thesis they will \textbf{not} be considered groups; instead they will be thought of as \textbf{group-valued functors}, since their functor of points factors through the category of groups. Indeed, for any scheme $S$, the set $\Hom(S, G)$ has a group structure, with group operation \[(f: S \rightarrow G) \cdot (g: S \rightarrow G) := (\mu\circ(f,g)\circ\Delta: S\rightarrow G)\] where $\Delta$ is the diagonal map. The identity morphism is the composition  $S \rightarrow \Spec k \xrightarrow{e} G$ where the first map is the unique morphism $S \rightarrow \Spec k$, and the inverse of $f: S \rightarrow G$ is simply $\iota \circ f$. In particular, using Yoneda's lemma, one may show that all of the statements deduced from the usual group axioms (uniqueness of identity and inverses, commpatibility of inversion with homomorphisms, etc.) are valid for algebraic groups.

%To see more concretely the connection with group theory, we may think of an algebraic group as a contravariant functor from $\Sch/k$ to $\Gps$; indeed for any scheme $S$, the set $\Hom(S, G)$ has a group structure, with group operation \[(f: S \rightarrow G) \cdot (g: S \rightarrow G) := (\mu\circ(f,g)\circ\Delta: S\rightarrow G)\] where $\Delta$ is the diagonal map. The identity morphism is the composition  $S \rightarrow \Spec k \xrightarrow{e} G$ where the first map is the unique morphism $S \rightarrow \Spec k$, and the inverse of $f: S \rightarrow G$ is simply $\iota \circ f$. In particular, $G(k)$ itself is a group, and when we think of $G$ as a group, we usually think of $G(k)$.

Additionally, $G$ has a scheme structure, so for example the group operations induce co-operations of $k$-algebras. In fact, if $G$ is affine, we may completely work with $k$-algebras, since there is an equivalence of categories. In this case, we will call $\Gamma(G, \O_G)$ the \textit{associated co-group}.
\begin{example}
	Let $G = \Spec k[t, t^{-1}]$, whose $k$-points are of course is canonically identified with $k^*$. Now $k^*$ obviously has a canonical multiplication which makes it a group; we will now extend this to an algebraic group structure on $G$. Note that since a morphism of varieties is detemined by its $k$-points (\autocite[II Proposition 2.6]{Hart}), this extension is unique, if it exists. Define the co-multiplication 
	\begin{align*}
		\mu^\sharp: k[t, t^{-1}] &\rightarrow k[t, t^{-1}]\otimes k[t, t^{-1}]  \\
		t &\mapsto t\otimes t
	\end{align*}
Using the identification $ k[t, t^{-1}]\otimes k[t, t^{-1}] \cong k[t_1^{\pm1}, t_2^{\pm1}]$, we may write this in the more familiar way as $t\mapsto t_1t_2$. Taking $\Spec$, we get a morphism $G \times G \rightarrow G$, and we observe the induced map of $k$-points is simply $(p,q) \mapsto pq$ as expected. Then the co-identity is the $k$-algebra map $k[t, t^{-1}] \rightarrow k$ given by $t \mapsto 1$, and co-inversion is the endomorphism $t \mapsto t^{-1}$. We check the group axioms by checking the co-axioms on the co-group: note that \[(\id \otimes \mu^\sharp)\circ \mu^\sharp(t) = t\otimes t \otimes t = (\mu^\sharp \otimes \id)\circ \mu^\sharp(t) \] and extending algebraically this proves associativity. To check identity, we observe  \[ (e^\sharp\otimes \id)\circ\mu^\sharp(t) = 1 \otimes t \cong t\cong t \otimes 1 =  (\id \otimes e^\sharp)\circ \mu^*(t)\] as desired, where, by abuse of notation, $\cong$ denotes the image of $1\otimes t$ under the isomorphisms $k \otimes k[t^{\pm1}] \cong k[t^{\pm1}]\cong k[t^{\pm1}] \otimes k$. Finally, we check inversion: \[(\iota^\sharp \otimes \id)\circ\mu^\sharp(t) = 1 = e^\sharp(t)\] and similarly in the other direction. Thus we have our expected group axioms.
\end{example}
Henceforth, we will denote $\Spec k[t, t^{-1}]$ by $\mathbb{G}_m$ (note that $m$ stands for multiplication, not for any specific number).

Before stating our next example, a few comments about notation are in order. We will use $\GL_n(R)$ to denote the (abstract) group of invertible $n$-by-$n$ matrices over $R$, and we will use $V$ to describe the vector space $k^n$. In particular, these are \textbf{not} schemes. We will use $\A^n$ to describe the variety/scheme $\Spec k[x_1,...,x_n]$, and we will use $\GL_n$ or $\GL_V$ to describe the variety/scheme, which is defined below. 
\begin{example}
	Observe $\GL_n(k)$ may be identified as the $k$-points of the affine scheme $\Spec k[x_{ij}, 1 \leq i,j \leq n, \det(x_{ij})^{-1}]$, which we will denote $\GL_n$. Co-multiplication is given by \[\mu^\sharp(x_{ij}) := \sum_{k = 1}^n x_{ik}\otimes x_{kj}\] or, once again, making the identification \[k[x_{ij}, 1 \leq i,j \leq n, \det(x_{ij})^{-1}] \otimes k[x_{ij}, 1 \leq i,j \leq n, \det(x_{ij})^{-1}] \cong k[x_{ij}, y_{ij}, \det(x_{ij})^{-1}, \det(y_{ij})^{-1}]\] this is just the familiar \[\mu^\sharp(x_{ij}) = \sum_{k = 1}^n x_{ik} y_{kj} \] Similarly, the co-identity is given by $e^\sharp(x_{ij}) := \delta_{ij}$. The co-inversion is a little difficult to write explicitly, but $\iota^\sharp(x_{ij})$ is the $i,j$-th entry of the $n$-by-$n$ matrix $(x_{k\ell})_{1 \leq k,\ell \leq n}$, which may be shown to be algebraic. Once again, we can check the group axioms for this by checking the co-group co-axioms:
	\begin{align*}
		(\mu^\sharp \otimes \id)(\mu^\sharp(x_{ij})) &= (\mu^\sharp \otimes \id)(\sum_k x_{ik} \otimes x_{kj}) \\
		&= \sum_k \sum_{\ell} x_{i\ell} \otimes x_{\ell k} \otimes x_{kj} \\
		&= \sum_k \sum_{\ell} x_{ik} \otimes x_{k \ell} \otimes x_{\ell j} \\
		&= (\id \otimes \mu^\sharp)(\sum_k x_{ik} \otimes x_{kj}) \\
		&= (\id \otimes \mu^\sharp)(\mu^\sharp(x_{ij}))
	\end{align*}
	proves associativity and \[(e^\sharp\otimes \id) (\mu^\sharp(x_{ij})) = (e^\sharp\otimes \id)(\sum_k x_{ik} \otimes x_{kj}) = \sum_k \delta_{ik}\otimes x_{kj} = \sum_i 1\otimes x_{ij} \cong x_{ij}\] and similarly proves identity. We will not prove inversion, but it follows from the properties of multiplying matrices. $\SL_n$ and $\PGL_n$ are defined similarly.
\end{example}
Observe that the group of invertible $n$-by-$n$ matrices is also identified with the group of $R$-valued points of $\GL_n$, hence both interpretations of the notation $\GL_n(R)$ agree.
\begin{example}
	Let $G$ be a finite group. We will endow $G$ with a natural algebraic group structure. Write $n:= |G|$. By Corollary \ref{cayleys-theorem}, we may embed $G$ into $S_n$, the symmetric group on $n$ letters, and $S_n$, in turn, embeds into $\GL_n(k)$ via permutation matrices. We therefore may interpret $G$ as a closed subscheme of $\GL_n$, and the group axioms inherit from the group axioms in $\GL_n$.
\end{example}
Next we will discuss algebraic group actions. 
\begin{definition} \index{algebraic group! algebraic actions}
	Let $G$ be an algebraic group and $X$ a scheme over $k$. An \textit{action} of $G$ on $X$ is a morphism $\sigma: G \times X \rightarrow X$ such that the following diagrams commute: 
	\begin{enumerate}
		\item (Associativity)
		\begin{equation*}
			\begin{tikzcd}[row sep = huge]
				G \times G \times X \arrow[r, "\id_G \times \sigma"]\arrow[d, "\mu \times \id_X"] & G \times X \arrow[d, "\sigma"]\\
				G \times X \arrow[r, "\sigma"] & X
			\end{tikzcd}
		\end{equation*}
		\item (Identity)
		\begin{equation*}
			\begin{tikzcd}[row sep = huge]
				\Spec k \times X \arrow[r, "e \times \id_X"]  \arrow[d, "\cong"]&G \times X \arrow[ld, "\sigma"]\\
				X
			\end{tikzcd}
		\end{equation*}
	\end{enumerate}
\end{definition}
\begin{example}
	The most straightforward example is $G$ acting on itself via $\mu$. Associativity and identity follow directly from the corresponding group axioms.
\end{example}
Let $G$ act on $X$. We observe a few things: passing to $k$-points, the group $G(k)$ acts (as an abstract group, in the usual sense) on $X(k)$. Any $k$-point $g: \Spec k \rightarrow G$ induces an automorphism of $X$ given by $\sigma(g\times \id)$, which we will denote $\sigma_g$ (on the level of $X(k)$, this is simply multiplication by $g$). Similarly, any $k$-point $p: \Spec k \rightarrow X$ induces a morphism $G \rightarrow X$. 

Now assuming $G$ and $X$ are both affine, equal to $\Spec R$ and $\Spec A$ respectively the action induces a co-action homomorphism of $k$-algebras $\sigma^\sharp A \rightarrow R \otimes A$. However, the group $G(k)$ also has an induced action (in the usual sense) on $A$ via automorphisms: indeed, an element $g$ of $G(k)$ is dual to a ring homomorphism $g^\sharp:R \rightarrow k$, and thus composing with the co-action homomorphism we have an automorphism of $A$: \[A \rightarrow R \otimes A\rightarrow k\otimes A \cong A. \] However, since the scheme-ring duality is contravariant, the composition works in the opposite direction. Hence the action is given by \[g\cdot f = (g^\sharp)^{-1}\circ\sigma^\sharp. \] It is easy to check that on $k$-points, the action is given by \[ g\cdot f(p) = f(g^{-1}\cdot p).\]

\begin{example} \label{naive-quotient-scheme}
	Consider the action described in Example \ref{naive-quotient}. We will extend this to an algebraic group action of $\bb{G}_m$ on $\A^2$. The associated coordinate rings are $k[t^{\pm1}]$ and $k[x,y]$. We now define the co-action $\sigma^\sharp: k[x,y]\rightarrow k[t^{\pm1}]\otimes k[x,y]$ by $x \mapsto t^{-1}\otimes x$ and $y \mapsto t\otimes y$; it is not hard to check the axioms and to show that this induces the action in the aforementioned example. Now we compute the action of $\bb{G}_m(k)$ on $k[x,y]$: let $\lambda\in k^* \cong \bb{G}_m(k)$ be given. This induces the map $k[t^{\pm1}] \rightarrow k$ defined by $t\mapsto \lambda$, hence we have \[ \sigma_{\lambda}^\sharp(x) = \lambda^{-1}\otimes x = 1 \otimes (\lambda^{-1}x) \cong \lambda^{-1} x \] and \[ \sigma_{\lambda}^\sharp(y)= \lambda \otimes y = 1 \otimes \lambda y \cong \lambda y \] which is easily seen to be a group action.
\end{example}
\begin{example}\label{gl-action}
	We consider the natural action of $\GL_n(k)$ (the group of invertible $n$-by-$n$ matrices) on $V = k^n$. We will lift this to an algebraic group action of $\GL_n$ on $\A^n$. The coordinate rings are $k[x_{ij}, \det(x_{ij})^{-1}]$ and $k[v_1,...,v_n]$. We define the co-action \[\sigma^\sharp: k[v_1,...,v_n] \rightarrow k[x_{ij}, \det(x_{ij})^{-1}]\otimes k[v_1,...,v_n] \] by \[\sigma^\sharp(v_i):= \sum_{j = 1}^n x_{ij} \otimes v_j\] and for a $k$-point $g = (g_{ij})\in \GL_n(k)$ (which is induced by $x_{ij}\mapsto g_{ij}$), we have the following automorphism $\sigma^\sharp_{g}$ of $k[v_1,...,v_n]$: \[\sigma^\sharp_g(v_i) = \sum_j g_{ij}\otimes v_j = \sum_j 1 \otimes g_{ij}v_j \cong \sum_j g_{ij}v_j  \] and in order for it to be a group action, we have to take the inverse of this map.
\end{example}

\begin{definition}
	Let $G$ and $H$ be algebraic groups. A \textit{homomorphism} of algebraic groups is a morphism of schemes $f:G \rightarrow H$ such that the following diagram commutes:
	\begin{equation*}
		\begin{tikzcd}[row sep = large]
			G \times G \arrow[r, "\mu_G"] \arrow[d, "f \times f"]&G \arrow[d, "f"] \\
			H \times H \arrow[r, "\mu_H"] & H
		\end{tikzcd}
	\end{equation*}
	A homomorphism $\rho: G \rightarrow \GL_n$ will be called a \textit{representation}. Composing this with the action on $\A^n$, we see that $\rho$ induces an action of $G$ on $\A^n$ and hence of $G(k)$ on $V = k^n$.
\end{definition}
\begin{example}
	Let $G$, $H$ be any affine algebraic groups. Then the \textit{trivial group homomorphism} $G \rightarrow H$ is the morphism of schemes 
	\begin{equation*}
		\begin{tikzcd}
			G \arrow[r] & \Spec k \arrow[r, "e_H"] & H.
		\end{tikzcd}
	\end{equation*}
	Clearly the square commutes.
\end{example}
\begin{example}
	Consider the representation $\rho: \bb{G}_m \rightarrow \GL_2$ induced by \[\rho^*(x_{ij}) = \delta_{ij}t^{2i-3} \] composing this with the natural action of $\GL_2$ on $\A^2$ described in Example \ref{gl-action} we obtain the action in Example \ref{naive-quotient-scheme}.
\end{example}

Note that the image of the induced morphism $k^* \rightarrow \GL_2(k)$ in the above representation is contained in the subgroup of diagonal matrices. This is no coincidence, as we will show below. This result is hugely important, and as we will soon see, representations of $\bb{G}_m$ play a huge role in our further discussions (for example, in our analysis of stability). For now, we end with the following theorem:
\begin{theorem}\label{torus-reductive}
	Let $\rho: \bb{G}_m \rightarrow \GL_n$ be a representation. Then there is a decomposition \[k^n =: V = \bigoplus_{i\in \Z} V_i \] where \[V_i:= \{v\in V \mid \lambda \cdot v = \lambda^i v \space, \forall \lambda \in k^*\} \]
\end{theorem}
\begin{proof}
	We follow the proof given in \autocite[p.17]{ModuliNotes}. Firstly, observe that $\rho$ induces a group homomorphism $\Hom(\Spec R, \bb{G}_m)\rightarrow \Hom(\Spec R, \GL_n)$ for every $k$-algebra $R$. In particular, letting $R = k[t^{\pm1}]$, we have a group homomorphism $\rho: \bb{G}_m(R)\rightarrow \GL_n(R)$, and hence $\rho(t)\in \GL_n(R)$. We interpret $\rho^* =\rho(t)$ as a linear map $\rho^*:V \rightarrow V \otimes k[t^{\pm1}]$. It can be shown (\autocite[22,23]{Waterhouse}) that the following diagrams commutes:
	\begin{equation}\label{comodule diagram}
		\begin{tikzcd}[row sep = huge]
			V \arrow[r, "\rho^*"] \arrow[d, "\rho^*"]&V \otimes k[t^{\pm1}] \arrow[d, "\id \times \mu^\sharp"]\\
			V \otimes k[t^{\pm1}] \arrow[r, "\rho^*\otimes \id"] & V \otimes k[t^{\pm1}]\otimes k[t^{\pm1}]
		\end{tikzcd}
	\end{equation}
	and
	\begin{equation}\label{comodule identity diagram}
		\begin{tikzcd}[row sep = huge]
			V \arrow[r, "\rho^*"] \arrow[d, "\cong"]& V \otimes k[t^{\pm1}] \arrow[ld, "\id \otimes e^\sharp"]\\
			V\otimes k
		\end{tikzcd}
	\end{equation}
	and it is clear that for any $\lambda\in \bb{G}_m(k) = k^*$ we have the following commutative diagram:
	\begin{equation*}
		\begin{tikzcd}[row sep = huge]
			V \arrow[r, "\rho^*"] \arrow[d, "\sigma_\lambda"] &V \otimes k[t^{\pm1}] \arrow[ld, "t\mapsto \lambda"] \\
			V
		\end{tikzcd}
	\end{equation*} 
	Now observe that there exist $f_i\in \End(V)$ for each $i\in \Z$ such that \[\rho^*(v) = \sum_{i\in \Z} f_i(v) \otimes t^i \] for every $v\in V$. We claim $f_i(v)\in V_i$. Indeed, by the commutativity of (\ref{comodule diagram}), we have \[\sum_{i\in \Z} f_i(v) \otimes t^i \otimes t^i = (\id \otimes \mu^\sharp)(\sum_{i\in \Z} f_i(v) \otimes t^i) = (\rho \times \id)(\sum_{i\in \Z} f_i(v) \otimes t^i) = \sum_{i\in \Z} \rho^*(f_i(v)) \otimes t^i\] and the claim follows since the $t^i$ are linearly independent. Next observe that  the commutativity of (\ref{comodule identity diagram}) implies \[v = \sum f_i(v) \] and hence we have the decomposition as desired. Finally, the observation that the $V_i$ intersect trivially pairwise implies the result.
\end{proof}
\section{Reductive Groups and the Affine GIT Quotient}
In this section, we will conduct a study on reductive groups and their action on affine schemes, and in the sequel we will be exclusively looking at the action of reductive algebraic groups. The reason for this is that the reductive hypothesis gives us many useful finiteness conditions. We will begin by studying the algebraic properties of the action of a reductive algebraic group on an affine scheme, and then translate these into geometric properties. First, the definition:
%However, we will first define what a quotient is:
%\begin{definition}
%	Let $G$ be a reductive algebraic group acting on a variety $X$. A \textit{quotient} of this action is a morphism $\varphi: X \rightarrow Y$ such that $\varphi \circ \sigma = \varphi \circ \pi_X$ as morphisms $G\times X \rightarrow Y$. A quotient is \textit{categorical} if it satisfies the following universal property: given any other quotient $\psi: X \rightarrow Z$, there is a unique morphism $f: Y \rightarrow Z$ such that $\psi = f\circ \varphi$.
%\end{definition}


\begin{definition}
	A group $G$ is \textit{(linearly) reductive} over $k$ if, for every representation $\rho: G \rightarrow \GL_n(k)$, we can decompose $V = k^n$ into irreducible subrepresentations (i.e. $G$-invariant subspaces). An algebraic group is reductive if its group of $k$-valued points is.
\end{definition} 
\begin{example}
	If $G$ is a finite group, then it is reductive if $\operatorname{char} k$ does not divide $|G|$, by Maschke's Theorem.
\end{example}
\begin{example}
	As we saw in Theorem \ref{torus-reductive}, $\bb{G}_m$ is reductive. Indeed, given a representation $\rho: \bb{G}_m \rightarrow \GL_V$, we have a weight space decomposition $V = \bigoplus V_i$ and any subspace of each $V_i$ is a subrepresentation.
\end{example}
\begin{example}
	In characteristic zero, it is well known that many \enquote{familiar} algebraic groups (such as $\GL_n, \SL_n, \PGL_n$) are reductive. 
\end{example}
Now let $G$ be a reductive group with a representation $G \rightarrow \GL_V(k)$, where $V$ finite-dimensional, and let $V^G\subseteq V$ be the subspace where $G$ acts trivially; it is clear this is a $G$-invariant subspace. Then by reductivity, we have a decomposition $V = V^G\oplus W$ for some subrepresentation $W$.
\begin{lemma}
	The $W$ above is unique.
\end{lemma}
\begin{proof}
	Suppose $V = V^G \oplus W = V^G \oplus W'$. By reductivity, we may decompose $W$ as $W = \bigoplus W_i$, where each $W_i$ is an irreducible subrepresentation. Let $w\in W_i$, so that $w = v_0 + w'$ for some $w'\in W'$. Since $w,w'\notin V^G$, it follows that there is some $g\in G$ which acts nontrivially on $w$; thus $g(w) - w \neq 0$. But $g(w) = v_0 + g(w')$ and hence \[g(w) - w = g(w') - w'\in W_i \cap W' \setminus \{0\}. \] Since $W_i$ is irreducible, it follows that $W_i \subseteq W'$. Since this holds for each $W_i$, it follows $W \subseteq W'$, and reversing their roles, we obtain $W' \subseteq W$ as well, as desired.
\end{proof}
\begin{definition}
	Let $G$ be an (abstract) group and let $\rho: G \rightarrow \GL_V(k)$ be a representation. The \textit{Reynolds operator} of $\rho$, denoted $E_\rho$, is defined to be the projection onto $V^G$. By the above lemma, this is well-defined. If $G$ is an algebraic group, and $\rho: G \rightarrow \GL_V$ is a representation, then the Reynolds operator of $\rho$ is the Reynolds operator of $\rho(k): G(k)\rightarrow \GL_V(k)$.
\end{definition}
We saw an example of a \enquote{Reynolds operator} in the proof of Theorem \ref{j-line-is-moduli}, where we projected onto our ring of invariant elements. This type of argument is very useful for showing something is invariant, since we need only exhibit it in the form $E_\rho(v)$, and so we would like to adapt this concept into infinite dimensions in certain circumstances; in particular the induced action on the coordinate ring of an affine scheme. To this end, let $G = \Spec R$ be an affine algebraic group acting on an affine scheme $X = \Spec A$ over $k$. Then as mentioned, there is an induced action of $G(k)$ as an abstract group on $A$ via $k$-algebra automorphisms, which must a-priori be $k$-linear. In particular, we have the following: 
\begin{lemma}
	Every element of $A$ is contained in a finite-dimensional $G(k)$-invariant subspace. In particular, we have \[A = \varinjlim V_r \] where the limit is taken across all $G(k)$-invariant subspaces of $A$.
\end{lemma}
\begin{proof}
	Following \autocite[p. 26]{GIT}, recall that the action induced by $g\in G(k)$ is given by composing the co-action homomorphism $\sigma^\sharp:A \rightarrow R \otimes A$ with the dual $k$-algebra homomorphism $g^\sharp\circ \iota^\sharp:R \rightarrow k$ induced by the inverse of $g$. Now fix some $f\in A$, let $R^*$ denote the vector space $\Hom(R, k)$ (which is essentially just $G(k)$), and let $V$ be the image of the map $\alpha:R^*\rightarrow A$ sending $u$ to $(u\otimes \id_A)(\sigma^\sharp(f))$. We claim $V$ satisfies our desired properties. To see this, observe firstly that by the identity axiom of group actions, the co-identity homomorphism $e^\sharp\in R^*$ satisfies $\alpha(e^\sharp) = f$, and so $f\in V$. To see that this is finite dimensional, simply observe that if $\sigma^\sharp(f) = \sum h_i\otimes f_i$ then $V\subseteq \Span \{f_i\}$. Finally, to see that $V$ is $G(k)$-invariant, observe that by co-associativity, we have \[\sigma^\sharp(\alpha(u)) = ( u\otimes \sigma^\sharp)(\sigma^\sharp(f)) = ((u\otimes \id_R\otimes \id_A)\circ (\mu^\sharp\otimes \id_A)\circ\sigma^\sharp)(f),  \] hence for any $u'\in R^*$, we have\[ u'\cdot \alpha(u) = (u\otimes u'\otimes \id_A)\circ(\mu^\sharp\otimes \id_A)(\sigma^\sharp(f)) = (((u\otimes u')\circ\mu^\sharp\otimes \id_A)\circ \sigma^\sharp)(f)  = \alpha((\mu^\sharp)^*(u\otimes u'))\] where $(\mu^\sharp)^*(u\otimes u')\in R^*$ is the map $h \mapsto( u\otimes u')(\mu(h))$, as desired.
\end{proof}
\begin{corollary}
	Every finite dimensional subspace is contained in a finite dimensional $G(k)$-invariant subspace.
\end{corollary}
\begin{proof}
	Apply the above argument to every element in a basis.
\end{proof}
\begin{definition}
	Let $G$ be an affine reductive algebraic group acting on an affine scheme $X = \Spec A$. The \textit{Reynolds operator} of this action is the $k$-linear map $E: A = \varinjlim V_r \rightarrow A$ induced by the Reynolds operator on each $V_r$. By the universal property of direct limits and the above lemma, it is well-defined. The \textit{ring of invariants}, denoted $A^G$ is the image of $E$.
\end{definition}
\begin{proposition}[Reynolds identity]
	Let $a\in A^G$ and $r\in A$. Then we have \[E(ar) = aE(r). \] In particular, the Reynolds operator is an $A^G$-module homomorphism.
\end{proposition}
\begin{proof}
	Firstly, let $V$ be an irreducible finite dimensional subrepresentation. Then $aV$ is also irreducible, and $v \mapsto av$ is a $k[G(k)]$-module homomorphism (where $k[G(k)]$ is the group algebra), and thus by Schur's lemma, is either trivial or an isomorphism. 
	
	Now let $V$ be a finite-dimensional subrepresentation of $A$ containing $a, r, E(ar),$ and $E(r)$, and let $V =  \bigoplus_{i = 0}^m V_i$ be an irreducible decomposition, with $V_0$ the subrepresentation of invariants. Then we may write $r = \sum r_i$, and hence \[E(ar) = E(a\sum r_i) = ar_0 + \sum_{i \neq 0} E(ar_i). \] Now for each $i\neq 0$, either $aV_i = 0$ or $aV_i \cong V_i$ in which case $v_i \mapsto av_i$ is an isomorphism. In the former case, $ar_i$ is already zero, in the latter case, there is some $g$ such that $g( r_i) - r_i\neq 0$ and hence \[g(ar_i ) - ar_i = ag(r_i) - ar_i = a(g(r_i)- r_i) \neq 0, \] whence $E(ar_i) = 0$, and hence \[E(ar) = ar_0 = aE(r)  \] as desired.
\end{proof}

\begin{corollary}\label{algebraic-properties-reductivity}
	Let $G$ be a reductive algebraic group acting on an affine scheme $X = \Spec A$.
	\begin{enumerate}
		\item If $B$ is an $A^G$-algebra, then there is an induced action on $B' :=A\otimes_{A^G}B$, and $B$ is the ring of invariants of $B'$, with the induced action.
%		\item If $S$ is a multiplicative subset of $A^G$, then the ring of invariants of $S^{-1}A$ is $S^{-1}A^G$.
		\item If $\{I_i\}$ is a collection of invariant ideals of $A$, then \[ (\sum I_i)\cap A^G = \sum (I_i \cap A^G). \]
		\item If $I$ is an invariant ideal of $A$, then the ring of invariants of $A/I$ is $A^G/(A^G \cap I)$.
	\end{enumerate}
\end{corollary}
\begin{proof}
	We follow \autocite[pp. 28-29]{GIT}. Firstly, note by the existence of the Reynolds operator $E$ on $A$ as an $A^G$-module homomorphism, we have the decomposition $A = A^G \oplus A'$ as $A^G$-modules, where $A' = \ker E$. Hence \[B' = (A^G \oplus A')\otimes_{A^G} B = B \oplus A'\otimes_{A^G} B,\] which means $B$ is a subring of $B'$ and so the statement makes sense. Now observe that $E$ induces a $B$-module homomorphism $E':B'\rightarrow B'$ sending $ \sum (a_i + r_i)\otimes b_i$, where $a_i\in A^G, r_i\in A'$, to $\sum a_ib_i$. The action of $G(k)$ on $B'$ is given by $g\cdot \sum a_i \otimes b_i= \sum g(a_i)\otimes b_i$.  It is not hard to check that $E'$ is the Reynolds operator of the induced action, and that $\im E' = B$. This proves (i).
	
	To prove (ii), firstly note that clearly $\sum (I_i \cap A^G) \subseteq  (\sum I_i)\cap A^G $. To prove the other inclusion, note that if $\sum f_i\in (\sum I_i)\cap A^G $, then \[\sum f_i = E(\sum f_i) = \sum E( f_i).\] It therefore suffices to prove that $E(I_i)\subseteq I_i$. To this end, let $f\in I_i$. Then $f$ is contained in some finite-dimensional subrepresentation $V$, and $V\cap I_i$ is finite-dimensional and $G(k)$-invariant. Thus by reductivity, we may decompose $V\cap I_i$ into subrepresentations $V\cap I_i = (V\cap I_i)^G \oplus W$, and hence \[E(f) = f - \pi_W(f)\in I_i \] as desired. 
	
	Finally, to prove (iii), observe that the action on $A/I$ is given by $g\cdot \bar{f} := \overline{(g\cdot f)}$, where $\bar{f} := f \mod I$. It is clear that the Reynolds operator on $A/I$ is given by \[\bar{E}(\sum_{i = 0}^n \bar{a_i}) := \overline{E(\sum_{i = 0}^n \bar{a_i})} = \bar{a_0}, \]  where $a_0\in A^G$, and from this the result is immediate.
\end{proof}


Henceforth, we will assume $\operatorname{char} k = 0$. Arguably the most important result about reductive groups acting on affine schemes is the following:
\begin{theorem}\label{ring-of-invariants-finitely-generated}
	Let $G$ be a reductive algebraic group acting on an affine scheme $X = \Spec A$ of finite type over $k$. Then $G(k)$ acts on $A$ via automorphisms. Then the ring of invariants $A^G$ is a finitely generated $k$-algebra. 
\end{theorem}
\begin{proof}
	Following \autocite[pp. 92-93]{Humphreys}, firstly, we claim $A^G$ is noetherian. To this end, let $I_1\subseteq I_2\subseteq ...$ be a chain of ideals in $A^G$. Then $I_1\otimes_{A^G} A\subseteq I_2\otimes_{A^G}\subseteq ...$ is an increasing chain of ideals in $A$, which terminates since $A$ is noetherian. Thus there is some $n_0\in \N$ such that $I_{i}\otimes_{A^G} A = I_{i+1}\otimes_{A^G} A$ for all $i \geq n_0$. But observe that for any ideal $I\subseteq A^G$, we have \[I = \sum_{f\in I} fA^G = \sum_{f\in I} (fA\cap A^G) = (\sum_{f\in I} fA)\cap A^G = (I\otimes_{A^G}A) \cap A^G \] and so \[I_i = (I_i\otimes_{A^G}A) \cap A^G = (I_{i+1}\otimes_{A^G}A) \cap A^G = I_{i+1}\] for all $i \geq n_0$ too, proving the claim.
	
	Now we first prove this result in the case $X = \A^n = \Spec k[x_1,...,x_n]$. Now since $G(k)$ acts via automorphisms, the action must preserve the graded pieces of the natural grading on $A =k[x_1,...,x_n]$, and thus $A^G$ will also be graded. It is clear that the irrelevant ideal $A^G_+ = A^G \cap A_+$ generates $A^G$, and by the above claim, $A^G_+$ is finitely generated, proving the result for $\A^n$.
	
	Finally, we prove the general case, following \autocite[p. 29]{GIT}. Let $f_1,...,f_n$ be generators of $A$ as a $k$-algebra, let $V$ be a finite-dimensional $G(k)$-invariant subspace containing them, and let $R:= \Sym(V)$ be the symmetric algebra on $V$, which is a polynomial ring. Then the action on $A$ induces an action on $R$ and there is a natural surjective equivariant map $\varphi: R \rightarrow A$. Clearly $\ker \varphi$ is invariant, and thus by Corollary \ref{algebraic-properties-reductivity} (iii), we have $A^G = R^G/(R^G \cap \ker \varphi)$. Since the action on $R$ preserves its degree 1 component, it follows that this is linear and hence algebraic (i.e. $G$ acts on $\Spec R$ as an algebraic group on a scheme), thus by the previous paragraph $R^
	G$ is finitely generated. The result then follows from the observation that if $\{v_1,...,v_m\}$ generate $R^G$, then $\{\varphi(v_1),...,\varphi(v_m) \} $ generate $A^G$.
\end{proof}
%We will prove this in several steps, first showing it is true for $X = \A^n$:
%\begin{lemma}
%	Theorem \ref{ring-of-invariants-finitely-generated} is true for $\A^n$.
%\end{lemma}
%\begin{proof}
%	We follow the proof in \autocite[pp. 92-93]{Humphreys}. Since $G(k)$ acts via automorphisms, it preserves the natural grading on $A = k[x_1,...,x_n]$.
%\end{proof}
We are now in a position to construct the quotient:

\begin{definition}\index{GIT quotient ! affine}
	Let $X$ be an affine variety, and $G$ a reductive algebraic group acting on $X$. The \textit{affine GIT quotient} is the map $\varphi^G:X\rightarrow \Spec A^G$ induced by the inclusion $A^G \rightarrow A$. We denote $\Spec A^G$ by $X\git G$.
\end{definition}
Observe that this map is $G$-invariant, in the sense that the following diagram commutes:
\begin{equation*}
	\begin{tikzcd}[row sep = huge]
			G \times X \arrow[r, "\sigma"]\arrow[d, "\pi_X"]& X\arrow[d, "\varphi^G"] \\
			X \arrow[r, "\varphi^G"]& X\git G
	\end{tikzcd}
\end{equation*}
Indeed, one can check this on the level of rings, and this follows since $A^G$ are exactly the elements $f\in A$ such that $\sigma^\sharp(f) = 1\otimes f$. In fact, the affine GIT quotient satisfies a stronger property, which we will formalise:
\begin{definition}
	Let $G$ be an algebraic group acting on a scheme $X$. A \textit{categorical quotient} of this action is a $G$-invariant morphism $\varphi:X \rightarrow Y$ such that if $\varphi':X \rightarrow Z$ is another $G$-invariant morphism then there exists a unique $f:Y \rightarrow Z$ such $\varphi' = f \circ \varphi$.
\end{definition}
As the reader may have guessed, the affine GIT quotient is a categorical quotient, which we will now prove in conjunction with other important properties:
\begin{theorem}\label{git-quotient-is-good-quotient}
	The affine GIT quotient satisfies the following:
	\begin{enumerate}
		\item The map $\varphi^G$ is surjective on $k$-points.
		\item For any open subset $U\subseteq X \git G$, the map $\O_{X \git G}(U)\rightarrow \O_X((\varphi^G)^{-1}(U))$ is an isomorphism onto $\O_X((\varphi^G)^{-1}(U))^G$.
		\item The image of every $G(k)$-invariant closed subset is closed.
		\item If $W_1$ and $W_2$ are disjoint, $G(k)$-invariant and closed, then there exists $f\in A^G$ such that for any $p_1\in W_1(k)$ and $p_2\in W_2(k)$, we have $f(p_1) = 0$ and $f(p_2) = 1$.
		\item $\varphi^G$ is affine.
		\item $\varphi^G$ is a categorical quotient.
	\end{enumerate}
\end{theorem}
\begin{proof}
	We follow \autocite[p. 31]{ModuliNotes} and \autocite[p. 28]{GIT}. Let $p\in (X\git G)(k)$ be a $k$-point with maximal ideal $\mf{m}$. We claim $\mf{m}\otimes{A^G}A$ is a proper ideal in $A$. Indeed if not, then $1 = \sum f_i\otimes a_i$ where $f_i\in \mf{m}, a_i\in A$, and so \[1 = E(1) = \sum E(f_i\otimes a_i) = \sum f_iE(a_i)\in \mf{m}, \] a clear contradiction. In particular, $\mf{m}\otimes{A^G}A$ is contained in a maximal ideal $\mf{m}'$ in $A$ associated to some $q\in X(k)$, and it thus follows that $\varphi^G(q) = p$ as desired. This proves (i).
	
	(ii) follows directly from (i) in Corollary \ref{algebraic-properties-reductivity}. 
	
	To prove (iii), let $W$ be an invariant closed subset corresponding to an invariant ideal $I \subseteq A$, and suppose for contradiction its image is not closed. We first claim that there must be a closed point in $\overline{\varphi^G(W)}\setminus \varphi^G(W)$. To this end, recall that by Chevalley's theorem, the image $\varphi^G(W)$ is contructible, and hence its complement in its closure $\overline{\varphi^G(W)}\setminus \varphi^G(W)$ is also constructible. Let $U \cap V$ be a nonempty locally open subset of $\overline{\varphi^G(W)}\setminus \varphi^G(W)$, where $U$ is open and $V$ is closed (both in $X$), and let $p\in U \cap V$. We claim that $U \cap V$ must contain some closed point of $\overline{\{p\}}$. Indeed, $V$ clearly does, and observe that if $U$ does not, then all the closed points of $\overline{\{p\}}$ are contained in the complement $X \setminus U$, which is closed. But this means the closure of these closed points is also contained in $X \setminus U$, and since $\Spec A^G$ is of finite type over $k$, these closed points are dense in $\overline{\{p\}}$ (\autocite[II Ex. 3.14]{Hart}), so in particular $p\notin U$, a clear contradiction. This proves the claim, and thus we may find some $q\in \overline{\varphi^G(W)}(k)\setminus \varphi^G(W)(k)$. Write $W' := (\varphi^G)^{-1}(q)$, and observe that $W'$ is closed, invariant and nonempty by (i). Denote its ideal by $I'\subseteq A$. By property (ii) of Corollary \ref{algebraic-properties-reductivity}, it follows that \[(I + I') \cap A^G = I \cap A^G + I' \cap A^G, \] and translated into geometric language, this says that \[\overline{\varphi^G(W\cap W')} = \overline{\varphi^G(W)}\cap \{q\} = \{q\}. \] In particular, this means that $W \cap W'\neq \emptyset$, contradicting the fact that $q$ is contained in the complement of $\varphi^G(W)$. This proves (iii).
	
	To prove (iv), let $I_1, I_2$ be the ideals corresponding to $W_1, W_2$ respectively. The assertion that they are disjoint is equivalent to the statement $I_1 + I_2 = A$, whence \[A^G =( A^G\cap I_1) + (A^G \cap I_2 ).\] Let $f\in A^G\cap I_1$ be such that $1-f\in A^G \cap I_2$. Then for any $p_1\in W_1(k), p_2\in W_2(k)$, we have $f(p_1) = 0$ and $1-f(p_2) = 0$, as desired.
	
	(v) follows directly from the definition.
	
	Finally, the proof of (vi) is essentially the second and third paragraphs of the proof of Theorem \ref{j-line-is-moduli}. Let $f:X \rightarrow Z$ be another $G$-invariant morphism, and let $\{U_i = \Spec A_i\}$ be an open affine cover of $Z$. Then as in the aforementioned proof, we can cover $X \git G$ with open subsets $\{V_i\}$ such that $(\varphi^G)^{-1}(V_i) = f^{-1}(U_i)$ and by  (ii) above, the restricted map $f|_{f^{-1}(U_i)}$ factors uniquely through $\varphi^G|_{(\varphi^G)^{-1}(V_i)}: (\varphi^G)^{-1}(V_i)\rightarrow V_i$, and one can check these glue into a global morphism. 
\end{proof}

Before we look at some examples, we will need to take a closer look at the action of $G(k)$ on $X(k)$. To this end, we first make the following definition:
\begin{definition}
	Let $G$ be a reductive group acting on $X$, and let $p\in X(k)$. The \textit{orbit} of $p$, denoted $G\cdot p$, is the set $\{g\cdot p \mid g\in G(k)\}\subseteq X(k)$. The \textit{stabiliser} of $p$, denoted $G_p$ is the fibred product $G \times_X \Spec k$, given by the following diagram:
	\begin{equation*}
		\begin{tikzcd}[row sep = huge]
			G \times_X \Spec k \arrow[d]\arrow[r] & \Spec k\arrow[d,"p"]  \\
			G \arrow[r, "\sigma(\id \times p)"]& X
		\end{tikzcd}
	\end{equation*}
\end{definition}

In fact, since $X$ is a variety, we can say more about $G\cdot p$: since $G\cdot p$ is just the set-theoretic image of the morphism $\sigma(-, p): G \rightarrow X$ in $X(k)$, and by Chevalley's theorem is a constructible subset of $X(k)$ (where $X(k)$ is identified as the closed points of $X$ and thus given the induced topology). So we can write \[G \cdot p = \bigcup_{i = 1}^n (U_i \cap V_i), \] where $U_i$ is open and $V_i$ is closed, and we may assume without loss of generality $V_i = \overline{U_i \cap V_i}$, the closure of $U_i \cap V_i$. Since closure commutes with unions, it follows \[\overline{G\cdot p} = \bigcup V_i. \] Now observe that $U =( \bigcup U_i) \cap G \cdot p$ is a dense open subset of $\overline{G \cdot p}$ (because $\overline{U}\cap \overline{G \cdot p}$ necessarily contains each $\overline{U_i \cap V_i} = V_i$ and $\bigcup V_i = \overline{G \cdot p}$ itself is closed), and since \[G\cdot p = \bigcup_{g\in G(k)} g\cdot U \] it follows that $G \cdot p$ is itself locally closed. Since $k$ is algebraically closed, this means we may identify $G\cdot p$ as the set of closed points of a closed subset of an open subscheme, and equipping it with the reduced closed subscheme structure of this open subscheme, we give $G\cdot p$ the natural structure of a scheme. By abuse of language, we will use the word \enquote{orbit} to denote both the set $G(k) \cdot p\subseteq X(k)$, and the scheme described above. It will either be clear from context, or unimportant which is meant.

Our next result is valid for any variety $X$, not just affine varieties.

\begin{proposition}
	Let $X$ be a variety and given any $k$-point $p\in X(k)$, the morphism $\sigma(-, p): G \rightarrow G \cdot p$ is flat. In particular, we have \[\dim G = \dim G_p + \dim G\cdot p. \] Moreover, $\sigma(-,p)$ is proper if and only if $G\cdot p$ is closed and $G_p$ is proper over $k$.
\end{proposition}
\begin{proof}
	\autocite[p. 19]{ModuliNotes} and \autocite[pp. 10-11]{GIT}
\end{proof}
With this in mind, we are now ready to describe the points of the action of $G$:
\begin{definition}
	A point $p\in X(k)$ is \textit{polystable} if $G\cdot p$ is closed. Furthermore $p$ is \textit{stable} if it is polystable, and $\dim G_p = 0$. An orbit is \textit{(poly)stable} if one (equivalently all) of its points is.
\end{definition}
With the terminology developed, we are ready to study our first example:
\begin{example}\label{naive-quotient-orbits}
	Recall Example \ref{naive-quotient}. As remarked, the orbits consist of the axes without the origin, the origin and for every $\lambda\in \bb{G}_m(k)$ the hyperbola $xy = \lambda$. Clearly the \enquote{typical} orbits are the hyperbolas, which are closed, and clearly the stabiliser for each such point is trivial, and hence they are stable. The origin itself is polystable, but not stable, and the axes are neither stable nor polystable. We note a few things: firstly the stable points form an open subset. Next, observe that closure of the union of the two non-closed orbits are $G$-invariant, and their intersection contains a polystable orbit (the origin). 
	
	Now it is easy to see that the ring of invariants is $k[x,y]^{\Gm} = k[xy]$, hence the GIT quotient is \[\A^2 = \Spec k[x,y] \rightarrow \Spec k[xy] = \A^1, \] where the map sends each stable orbit $\{xy = u\}$ to $u\in  k \cong\A^1(k)$ and the three orbits that are not stable, which intersect each other, to the origin. However, even though this is a categorical quotient, it is not an orbit space (since there are three orbits which are merged), but note that if we restrict to the open subset of stable points $(\A^2)^s :=\A^2\setminus\{xy=0\}$, the quotient is in fact an orbit space. Indeed, the restricted quotient is just \[(\A^2)^s = \Spec k[x,y,(xy)^{-1}] \rightarrow k[(xy)^{\pm 1}] = \A^1 \setminus \{0\} \] and the set-theoretic fibre of each $\lambda\in k^*$ is just the hyperbola $\{xy = \lambda\} $.
\end{example}

In fact, this example demonstrates very typical behaviour of the affine GIT quotient, which we will deduce as consequences of Theorem \ref{git-quotient-is-good-quotient}. Firstly observe that since the quotient $\varphi^G: X \rightarrow X\git G$ is, in fact, a quotient, orbits in $X(k)$ are contracted to points in $X\git G$, and in particular, property (iii) of Theorem \ref{git-quotient-is-good-quotient} implies that orbit closures are contracted to a point. By property (iv), the converse is also true: two points are mapped to the same point if and only if their orbit closures intersect. By the same property, it follows that the set-theoretic fibre (i.e. preimage) of every $p\in Y(k)$ contains a unique polystable orbit. In particular, every equivalence class of orbits (the relation being intersection of closure) contains a unique polystable orbit. In particular, this means that the set $Y(k)$ is in canonical 1-1 correspondence with the polystable orbits of $X$. 

The final observation in the above example (about restricting to the stable locus) is formalised thus:
%\begin{example}
%	Recall the action of $\bb{G}_m$ on $S =\Spec k[a,b,\Delta^{-1}]$, and let $(p,q)\in S(k)$ be a $k$-point. Then the orbit of $(p,q)$ is $\{(u^4p, u^6q)\mid u\in k^*\}$, and in particular is the set of $k$-points of the closed subscheme $\Spec k[a,b,\Delta^{-1}]/p^3y^2 = q^3x^3$. In particular, every orbit is closed and stable.
%\end{example}

%Let us investigate some consequences. Firstly, observe that any orbit closure is $G$-invariant; indeed let $G\cdot p$ be an orbit, and let $I$ denote the ideal of functions vanishing on this orbit. Now by definition, $I$ is the set of $f\in A$ such that $f\in \mathfrak{m}_{g\cdot p}$, or equivalently $g^{-1}\cdot f\in \mf{m}_p$ for all $g\in G(k)$, so in particular it is $G$-invariant, and hence $\overline{G\cdot p}$, which is the closed subset cut out by $I$, is $G$-invariant too. Since $\varphi^G$ is a quotient, property (iii) implies that orbit closures are contracted to a point. By property (iv), the converse is also true: two points are mapped to the same point if and only if their orbit closures intersect. By the same property, it follows that the set-theoretic fibre of every $p\in Y(k)$ contains a unique polystable orbit. In particular, every equivalence class of orbits (the relation being intersection of closure) contains a unique polystable orbit. In particular, this means that the set $Y(k)$ is in canonical 1-1 correspondence with the polystable orbits of $X$.  
\begin{definition}
	Let $G$ be a reductive algebraic group acting on a scheme $X$. A $G$-invariant morphism $X \rightarrow Y$ is a \textit{geometric quotient} if all the properties of Theorem \ref{git-quotient-is-good-quotient} are satisfied, and moreover $Y(k)$ is an orbit space of the $G(k)$-action on $X(k)$.
\end{definition}
%So in particular, the affine GIT quotient is a good quotient, and since it can also be shown that a good quotient is a categorical quotient (\autocite[Proposition 3.30]{ModuliNotes}), it follows that the GIT quotient is a categorical quotient. However, it is not always a geometric quotient:  
%\begin{example}
%	Let us consider Example \ref{naive-quotient} once more. We saw how the action is formalised as an algebraic action in Example \ref{naive-quotient-scheme}, and we studied its orbits in Example \ref{naive-quotient-orbits}. Now retaining the notation in Example \ref{naive-quotient-scheme}, observe that clearly the invariant ring is $A^G = k[xy]$, and hence the GIT quotient is \[\A^2 = \Spec k[x,y] \rightarrow \Spec k[xy] = \A^1, \] where the map sends each stable orbit $\{xy = \lambda\}$ to $\lambda$ and the three orbits that are not stable, which intersect each other, to the origin. Now it is not hard to check that this is a good quotient, however it is not geometric, because the set-theoretic fibre of the origin in $\A^1$ is the union of three orbits. However, if we restrict to the open subset of stable points $(\A^2)^s :=\A^2\setminus\{xy=0\}$, the quotient is in fact a geometric quotient. Indeed, the restricted quotient is just \[(\A^2)^s = \Spec k[x,y,(xy)^{-1}] \rightarrow k[(xy)^{\pm 1}] = \A^1 \setminus \{0\} \] and the set-theoretic fibre of each $\lambda\in k^*$ is just the hyperbola $\{xy = \lambda\} $.
%\end{example}
%In fact, this is typical:
\begin{theorem}
	Let $G$ be a reductive affine algebraic group acting on an affine variety $X$. Then the set of stable $k$-points are the $k$-points of a (possibly empty) open subscheme $X^s$, and the restricted map \[ X^s \rightarrow X^s \git G\] is a geometric quotient.
\end{theorem}
\begin{proof}
	This follows the proof in \autocite[p. 19, 32]{ModuliNotes}. Firstly, we claim that the set of points $p$ such that $\dim G_p >0$ is closed. Indeed, consider the following diagram:
	\begin{equation*}
		\begin{tikzcd}[row sep = huge]
			S = (G\times X)\times_{X\times X} X \arrow[r, "\varphi"] \arrow[d] & X\arrow[d,"\Delta"] \\
			G \times X \arrow[r, "(\sigma\times\pi_X)"] & X\times X
		\end{tikzcd}
	\end{equation*}
	where $\Delta: X \rightarrow X\times X$ is the diagonal map and $S$ is the fibred product of the diagram. Now observe that the $k$-points of $S$ are exactly the pairs $(g,p)$ such that $g\cdot p = p$. By Chevalley's semicontinuity theorem (\autocite[Th\'eor\`eme 13.1.3]{EGAIV}), the subset \[V:=\{(g,p)\in S(k) \mid \dim G_p = \dim S_{p} > 0 \} \] is closed. Now define $T$ to be the fibred product $S \times_{G\times X} X$ given by the following diagram:
	\begin{equation*}
		\begin{tikzcd}[row sep = huge]
			T \arrow[r] \arrow[d]&S \arrow[d]\\
			\Spec k \times X \cong X \arrow[r, "e \times \id"] &G \times X
		\end{tikzcd}
	\end{equation*}
	and observe that $V$ pulls back to the set $\{(e, p) \mid \dim G_p > 0\}$. Since $X$ is separated over $k$, the diagonal is a closed immersion, and in particular it is proper. It thus follows that $T \rightarrow X$ is a closed map, which proves the claim.
	
	Now let $p$ be a stable point. We will find an open neighbourhood of stable points containing $p$. Observe that $V$, which we saw was closed, and is clearly $G$-invariant, is disjoint from $p$, and hence by property (iv), $p$ and $V$ to disjoint sets in the GIT quotient. In particular, there is some invariant $f\in A^G$, such that $f(V) = 0$, and $f(p)\neq 0$ (for example, suppose take $1 - F$ where $F\in A^G$ vanishes exactly on $\varphi^G(p)$). We claim that the $k$-points of $X_f$ are stable. Indeed, since $X_f(k) \cap V = \emptyset$, it suffices to show that the orbits of $X_f$ are closed. So fix some $q\in X_f(k)$, suppose for contradiction its orbit is not closed. Since $G\cdot q$ is open and dense in $\overline{G\cdot q}$, its complement in $\overline{G\cdot q}$ must be of smaller dimension, and moreover contains a polystable orbit, say $G\cdot r$. It must therefore be that $\dim G_r > 0$ by, so $r\in V$, and hence $f(r) = 0$. But since $f$ is $G$-invariant and does not vanish on $q$, it does not vanish anywhere on $\overline{G\cdot q}$, which is a contradiction. It thus follows that the $k$-points of $X_f$ are all stable, and hence we conclude that $X^s$ is open. 
	
	Finally, we prove that this is a geometric quotient. To this end, we simply need to show that the set-theoretic fibre of each $k$-point in $ X^s \git G$ is a single stable orbit. But this is obvious; indeed for each $p\in X^s$, the set-theoretic fibre $(\varphi^G)^{-1}(\varphi^G(p))$ contains a unique polystable orbit, which must be $G\cdot p$. If it contains any other orbit, say $G\cdot q$, then $G\cdot p \cap \overline{G\cdot q}$ is nonempty, and since $\overline{G\cdot q}$ is $G$-invariant, it follows $G\cdot p$ is in the complement of $G\cdot q$ in $\overline{G\cdot q}$, and hence has dimension strictly less, which contradicts the fact that $p$ is stable.
\end{proof}
\begin{example}
%	As we previously saw, in the action of $\bb{G}_m$ on $\Spec k[a,b,\Delta^{-1}]$, every $k$-point is stable. Hence the affine GIT quotient is a geometric quotient, and as we calculated in chapter 1, the GIT quotient is $\Spec k[j]$.
	We are now ready to revisit the elliptic curves example from last chapter, and assimilate it into our new framework. In the previous chapter, we saw how $k^* = \bb{G}_m(k)$ acted on $S=\Spec k[a,b, \Delta^{-1}]$. This can be formalised as an algebraic action $\sigma:\bb{G}_m \times S \rightarrow S$ as dual to the homomorphism $\sigma^\sharp:k[a,b, \Delta^{-1}]\rightarrow  k[t^{\pm1}]\otimes k[a,b, \Delta^{-1}]$ given by $a\mapsto  t^{4}\otimes a$, $b\mapsto t^6\otimes b$. To check that this is a group action, observe that \[ ((\id_G^\sharp\otimes \sigma^\sharp)\circ\sigma^\sharp)(a) = t^4\otimes t^4 \otimes a = ((\mu^\sharp\otimes \id_S^\sharp)\circ\sigma^\sharp)(a) \] and similarly with $b$, and \[ ((e^\sharp \otimes \id_S^\sharp)\circ\sigma^\sharp)(a) = 1 \otimes a.\] Let $(p,q)\in S(k)$ be a $k$-point. Then the orbit of $(p,q)$ is $\{(u^4p, u^6q)\mid u\in k^*\}$, and in particular is the set of $k$-points of the closed subscheme $\Spec k[a,b,\Delta^{-1}]/p^3y^2 = q^3x^3$. In particular, every orbit is closed and stable. Now as we calculated in chapter 1, the GIT quotient is $\Spec k[j]$, and as we just saw, this is an orbit space since every $k$-point of $S$ is stable. In particular, this tells us the following: the fact that $S \rightarrow \Spec k[j]$ is a categorical quotient, combined with Lemma \ref{natural-transformation-elliptic-curve-is-k*-invariant} means that $\Spec k[j]$ satisfies property (ii) in the definition of a coarse moduli space, and the fact that this is an orbit space means property (i) is also satisfied. This demonstrates the usefulness of geometric quotients.
\end{example}
However, the affine GIT quotient is slightly oversimplified, since every $p\in X(k)$ has an image. In particular, no orbits are thrown away; simply merged. As we will see shortly, things are more complicated in the projective case.
\section{The Projective GIT Quotient}

Our next task is to extend the notion of a GIT quotient to a projective variety. There are, however, more issues, the first and perhaps most obvious is that a projective scheme does not have a canonical homogeneous coordinate ring; such a coordinate ring is induced by a projective embedding. Indeed, even in the case of $X = \P^1$, we may embed $X$ in $\P^2$ in the obvious way, with resulting coordinate ring $k[x_0, x_1]$, or via the 2-uple embedding, with resulting coordinate ring $k[y_0, y_1, y_2]/y_0y_2 - y_1^2$, but these rings are not isomorphic, since the former is a UFD but the latter is not. 

But even if we have an embedding $X \subseteq \P^n$, such that $X = \Proj S$, this is not enough, because the action does not lift canonically to $S$; indeed:
\begin{example}\label{linearisation-motivation}
	Let $X = \bb{P}^n$. Let $\bb{G}_m$ act on $X$ as follows: for any $\lambda\in \bb{G}_m(k), p = [p_0:...:p_n]\in X(k)$, we define \[\lambda\cdot p:= [\lambda^{-1}p_0:\lambda p_1:...:\lambda p_n] \] which extends uniquely to an algebraic action, since $k$ is algebraically closed. We embed $X$ in itself via the identity, with resulting coordinate ring $S = k[x_0,...,x_n]$. However, the action of $G$ may be lifted to one on $S$ in many ways; for example
	\begin{align*}
		\lambda\cdot x_0 := \lambda^{-1}x_0,\; \lambda \cdot x_i := \lambda x_i
	\end{align*}
	and 
	\begin{align*}
		\lambda \cdot x_0 := x_0, \; \lambda \cdot x_ := \lambda^2 x_i.
	\end{align*}
	just to name two.
\end{example}
We must therefore \textbf{choose} a lift of the action, which is accomplished as follows: recall that a projective embedding is equivalent to picking a very ample line bundle $\L$, and the resulting coordinate ring of this embedding is \[S = \bigoplus_{r \geq 0} H^0(X, \L^{\otimes r}). \] We will therefore choose a lift of our action on $X$ to one on $\L$, so that the action is linear in some sense. This is encapsulated in the following definition:
\begin{definition}\index{linearisation}
	Let $\sigma: G\times X \rightarrow X$ be an algebraic group action on a projective scheme $X$. A \textit{linearisation} of this action is a line bundle $\L$ and an isomorphism $\Phi:\sigma^*(\L)\cong \pi_X^*(\L)$, where $\pi_X: G \times X \rightarrow X$ is the projection onto the second factor, such that the following diagram commutes:
	\begin{equation}\label{linearisation cocycle condition}
		\begin{tikzcd}[column sep = huge, row sep = huge]
			(\sigma\circ(\id_G \times \sigma))^*\L  \arrow[r, "(\id_G \times \sigma)^*\Phi"] \arrow[d, "="]& (\pi_X \circ(\id_G \times \sigma))^*\L \arrow[r, "="]& (\sigma \circ \pi_{23})^*\L\arrow[d, "\pi_{23}^*\Phi"] \\
		(\sigma\circ(\mu\times \id_X))^*\L\arrow[r, "(\mu\times \id_X)^*\Phi"]&(\pi_X \circ(\mu\times \id_X))^*\L	\arrow[r, "="]& (\pi_X \circ \pi_{23})^*\L		
		\end{tikzcd}
	\end{equation}
	where $\pi_{23}: G \times G \times X\rightarrow G \times X$ is the projection onto the last two factors. A linearisation is \textit{very ample} if $\L$ is. By abuse of language, we will often refer to $\L$ itself as the linearisation.
\end{definition}
We unwrap this definition. Of course, for any $g\in G(k)$, we may pull back $\L$ along the isomorphism $\sigma_g: X \rightarrow X$. The linearisation $\Phi$ allows us to identify $\L$ before and after the pullback. More precisely: for any open subset $U\subseteq X$ where $\L$ is trivial, we identify $\L|_U \cong \O_{U}$. Now fix a $k$-point $g$ of $G$. Then $\sigma_g^*\L(U) \cong \Ox(gU)$ and $\pi_X^*\L(U) \cong \O_{X}(U)$. In particular, we have the following isomorphism: 
\begin{equation}\label{composition-clusterfuck}
	\begin{tikzcd}
		\sigma_g^*\L(U) \cong \Ox(gU) \arrow[r, "\Phi"] & \pi_X^*\L(U) \cong \Ox(U)
	\end{tikzcd}	
\end{equation} 
So in particular, $\Phi$ may be thought of as defining a way to \enquote{shift} $\L$ by $g$.

We now make sense of (\ref{linearisation cocycle condition}) a little. Firstly, observe that these are morphisms of sheaves on $G \times G \times X$, all of which are pullbacks of $\L$ by various maps. The equalities follow from the axioms, for example \(\sigma\circ (\id_G \times \sigma) = \sigma\circ(\mu\times \id_X)\) is just the associativity axiom of group actions. Without explicitly stating the equalities, the commutativity says
\begin{equation}\label{linearisation-cocycle-equation}
	(\mu\times \id_X)^*\Phi = \pi_{23}^*\Phi \circ (\id_G \times \sigma)^*\Phi 
\end{equation} 
which we make sense of as follows: Let $(g,h)$ be a $k$-point in $G\times G$. Then as in (\ref{composition-clusterfuck}), the map $(\mu\times \id_X)^*\Phi$ induces a map \[\sigma_{gh}^*\L(U) \cong \Ox(ghU) \rightarrow \pi_X^*\L(U)\cong \Ox(U). \] Now $(\sigma \circ \pi_{23})(g,h,-) = \sigma_h $, and so $\pi_{23}^*\Phi$ induces a map 
\begin{equation*}
	\begin{tikzcd}
		\sigma_{h}^*\L(U) \cong \Ox(hU) \arrow[r] &  \pi_X^*\L(U)\cong\Ox(U)
	\end{tikzcd}
\end{equation*}
and finally since $(\sigma\circ (\id_G \times \sigma))(g,h,-) = \sigma_g \circ \sigma_h$, the pullback $(\id_G \times \sigma)^*\Phi $ induces a map
 \begin{equation*}
 	\begin{tikzcd}
 		\sigma_g^*(\sigma_h^*\L)(U) \cong \Ox(ghU) \arrow[r]& \sigma_h^*\L(U) \cong \Ox(hU)
 	\end{tikzcd}
 \end{equation*}
Put together, this means the following diagram commutes:
\begin{equation*}
	\begin{tikzcd}
		\sigma_{gh}^*\L\arrow[d] \arrow[r] &\L\\
		\sigma_h^*\L \arrow[ru]
	\end{tikzcd}
\end{equation*}
so in particular $G(k)$ acts on $\L$ via automorphisms.

Of course, this means that $G(k)$ also acts on all tensor powers of $\L$, and in the case $\L$ is very ample, taking global sections shows that $G(k)$ acts on the graded homogeneous coordinate ring $S = \bigoplus_{r\geq 0}H^0(X, \L^{\otimes r})$, and moreover it is not hard to see that this action preserves the grading. Furthermore, there is a natural action induced on the dual bundle, which we may interpret as the affine cone of this embedding, and it is not hard to see that this action is linear.
\begin{example}\label{projective-git-example}
	There is a natural linearisation of the action described in Example \ref{linearisation-motivation} on $\Ox(1)$. To see this, we first note that both $\sigma^*(\Ox(1))$ and $\pi_X^*(\Ox(1))$ are abstractly isomorphic to $\O_{\bb{G}_m \times X}(1)$. We define the isomorphism $\sigma^*(\Ox(1))\rightarrow \pi_X^*(\Ox(1))$ to be $x_0 \mapsto t^{-1}x_0$ and $t_i \mapsto t x_i$ for $i\neq 0$. 
	
	Once again, we check that this is in fact a linearisation. Clearly it is an isomorphism, so it suffices to show that (\ref{linearisation-cocycle-equation}) holds. To this end,  we first observe that the various pullbacks of $\Ox(1)$ to $\bb{G}_m\times \bb{G}_m\times X$ are abstractly isomorphic to the $\O_{\bb{G}_m\times\bb{G}_m\times X}$-module $\O_{\bb{G}_m\times\bb{G}_m\times X}(1)$, and since \[f^*(\O_{\bb{G}_m \times X}(1)) =  \O_{\bb{G}_m\times\bb{G}_m\times X}\otimes_{f^{-1}\O_{\bb{G}_m \times X}}f^{-1}(\O_{\bb{G}_m \times X}(1))\] where $f: \bb{G}_m\times \bb{G}_m\times X \rightarrow \bb{G}_m\times X$ is any map, we may write its elements as sums of $f\otimes g \otimes hx_i $, where $f,g\in \O_{\bb{G}_m}$ and $h\in \O_{\bb{G}_m \times X}$ (the $X$ component of $\O_{\bb{G}_m\times \bb{G}_m\times X}$ is absorbed by $h$). With this in mind, we compute:\[(\mu\times \id_X)^*\Phi(1\otimes 1 \otimes x_0) =  1 \otimes 1 \otimes t^{-1}x_0 = t^{-1}\otimes t^{-1}\otimes x_0\] and similarly for any other $x_i$. We also have \[(\id_G\times \sigma)^*\Phi(1\otimes 1 \otimes x_0) = 1\otimes 1 \otimes t^{-1}x_0 = t^{-1}\otimes 1 \otimes x_0 \] and finally \[\pi^{*}_{23}\Phi(t^{-1} \otimes1 \otimes x_0)  = t^{-1}\otimes t^{-1}\otimes x_0 \] as desired.
		
	Now the homogeneous coordinate ring of this embedding is just \[S = \bigoplus_{r \geq 0} H^0(X, \Ox(1)^{\otimes r}) = k[x_0,...,x_n]\] as expected, and there is an induced action of $\bb{G}_m(k)$ on $S$ given by $\lambda\cdot x_0 = \lambda^{-1}x_0$ and $\lambda\cdot x_i = \lambda x_i$ for $i\neq 0$, and in particular observe that this action preserves the grading on $S$.
\end{example}
Now we may define our quotient. We fix the following data: let $X$ be a projective scheme, $G$ a reductive algebraic group, $G\times X \rightarrow X$ an action, $\L$ a very ample linearisation and $S = \bigoplus_{r \geq 0} H^0(X, \L^{\otimes r})$ the homogeneous coordinate ring. We denote $S^G$ the subring of invariant elements of $S$, and we write $S_+$ for the irrelevant ideal $\bigoplus_{r>0}S_{\deg r}$, and similarly write $S_+^G$ for the $S^G$-ideal $S_+\cap S^G$. 
\begin{definition}\index{stability}\index{GIT quotient ! projective}
	A $k$-point $p$ is \textit{semistable} (with respect to $\L$) if there is a homogeneous invariant $\sigma\in S^G$ of positive degree such that $\sigma(p) \neq 0$, or equivalently $p\in X_\sigma(k)$ where $X_\sigma = \Spec S[\sigma^{-1}]_{\deg 0}$. If $p$ is not semistable, then it is \textit{unstable}. The \textit{semistable locus}, denoted $X^{ss}$ is the open subscheme $X \setminus V(S_+^G)$, where $V(S_+^G)$ is the closed subset associated to the honogeneous ideal $\langle S_+^G \rangle$ in $S$. Note that the homogeneous elements of $S_+^G$ generate this ideal, so it is in fact homogenous. We say $p$ is \textit{polystable} if it is semistable, and its orbit is closed in the semistable locus. Furthermore, $p$ is \textit{stable} if it is polystable, and additionally its stabiliser has dimension zero. The \textit{projective GIT quotient} is the map \[ X^{ss} \rightarrow X \git_\L G:=\Proj S^G\] induced by the inclusion $S^G \subseteq S$.
\end{definition}
Let us compare the affine and projective GIT quotients. The main difference is that in the affine case, every $k$-point has an image in the quotient; in other words every point is \enquote{semistable}; this is obviously not so in the projective case. Their similarities are, however, much more abundant: if $\sigma\in S^G_+$ is homogeneous, it is not hard to check that $X_\sigma = \Spec S[\sigma^{-1}]_{\deg 0}$ is invariant, and that the restriction of the projective GIT quotient to $X_\sigma$ is just the affine GIT quotient $X_\sigma \rightarrow \Spec S[\sigma^{-1}]_{\deg 0}^G$, and since $X^{ss}$ is covered by these affine open subsets, it follows that the projective GIT quotient is just a collection of affine GIT quotients glued together. One can check that the statements of Theorem \ref{git-quotient-is-good-quotient} hold (indeed, statements (i) - (v) are local on the target, and (vi) can be checked using the same argument). By a similar argument to the affine case, it can also be shown that the stable locus is open and that the restriction to the stable locus is a geometric quotient.
\begin{example}
	Retain the notation and hypotheses in Example \ref{projective-git-example}. It can be shown (\autocite[p. 37]{GIT}) that the ring of invariants $S^G$ is just $k[x_0x_1,...,x_0x_n]$. It follows that $p = [p_0:...:p_n]$ is semistable if and only if $p_0$ is nonzero, and some other $p_i$ for $i > 0$ is nonzero. In particular, the semistable locus can be identified with $\A^n \setminus\{0\}$. Now on the semistable locus, the action is just multiplication by $\lambda^2$, so every point is polystable, the orbit just being the line passing through our point and the origin in $\A^n$, minus the origin itself. In fact, every point is stable, since the action is free. Of course, this makes sense because our projective GIT quotient is just \[\P^{n-1} = \Proj k[x_0x_1,...,x_0x_n]\] and this is a geometric quotient.
\end{example}
\begin{example}
	Of course, there is another linearisation on $\Ox(1)$ given by $x_0 \mapsto x_0$ and $x_i \mapsto t^2x_i$ for $i > 0$. Clearly $k[x_0]$ is the ring of invariants, so the projective GIT quotient with respect to this linearisation is simply $\Spec k$. Indeed, the semistable locus is the open set given by $x_0 \neq 0$, which is isomorphic to $\A^n$. With this interpretation, the action of $\bb{G}_m$ is just scaling, and the closure of every orbit contains the origin in $\A^n$ (or equivalent the point $[p_0:0:...:0]\in \P^n$), which is the unique polystable orbit of this linearisation. In particular, the stable locus is empty. This shows that projective GIT is heavily dependent on our choice of linearisation. However, we will often fix a single linearisation to work with, and the problem of choosing different linearisations will not be discussed in this thesis. 
\end{example}

\section{The Hilbert-Mumford Criterion}
As we have just seen, stability is very important. However, with our definition, it is rather difficult to calculate. In this section, we will develop the \textit{Hilbert-Mumford criterion}, which gives a numerical criterion for stability in terms of 1-parameter subgroups. We begin with a closer examination of our current definition for stability, which requires the following definition:
\begin{definition}\index{affine cone}
	Let $X$ be a projective scheme and let $\L$ be a very ample line bundle. Write $S$ for the homogeneous coordinate ring \[S:= \bigoplus_{r \geq 0} H^0(X, \L^{\otimes r}).\] We define the \textit{affine cone} of $X$ to be the affine scheme $\widetilde{X} := \Spec S$.
\end{definition}
To make sense of the affine cone, firstly recall that $\L$ embeds $X$ as a closed subscheme of $\P^n$, where $n = h^0(X, \L)-1 = \dim H^0(X, \L) - 1$. The $k$-points of $\P^n$ are just the 1-dimensional subspaces of $k^{n+1}$, and thus the $k$-points of $X$ may be interpreted as a collection of lines through the origin in $k^{n+1}$. The $k$-points of $\widetilde{X}$ may, in turn, be thought of as the union of these lines. For example, the affine cone of $\P^n$ is just $\A^{n+1}$.

There is a well-defined notion of an origin, which corresponds to the irrelevant ideal $S_+$, which is clearly maximal, and there is a natural map $\Spec S \setminus \{0\} \rightarrow \Proj S$, which we define as follows: let $f\in S_{\deg 1}$. Then there is an inclusion $S[f^{-1}]_{\deg 0}\subseteq  S[f^{-1}]$, which induces a morphism $\Spec S[f^{-1}]\rightarrow \Spec S[f^{-1}]_{\deg 0}$. Since $\L$ is very ample, it follows $S$ is generated by a finite set of these $f\in S_{\deg 1}$ as a $k$-algebra, and so $X_f = \Spec S[f^{-1}]_{\deg 0}$ cover $\Spec S \setminus\{0\}$. On the level of $k$-points, this is just $(p_0,...,p_n)\mapsto [p_0:...:p_n]$.

Now suppose $X$ is a projective variety, and let $G$ be a reductive affine algebraic group acting on $X$. Further, let $\L$ be a very ample linearisation, and let $S$ be the homogeneous coordinate ring. We claim the the linearisation naturally induces an action on the affine cone. Indeed, by the adjunction property of pullbacks and pushforwards, there is a natural map (the unit map of the adjunction) $\L \rightarrow \sigma_*\sigma^*(\L)$. Taking global sections, we have \[H^0(X, \L)\rightarrow H^0(G\times X, \sigma^*(\L))\cong H^0(G\times X, \pi_X^*(\L)) \cong H^0(G, \O_G)\otimes H^0(X, \L) \] where the final isomorphism comes from the K\"unneth formula (\autocite[Lemma 33.29.1]{stacks-project}). We can check that this induces a map $\tilde{\sigma}^*:S \rightarrow \O_G(G)\otimes S$ which satisfies the co-action axioms; and in particular this induces a group action $G \times \widetilde{X} \rightarrow \widetilde{X}$. Moreover, since the co-action homomorphism is linear on $H^0(X, \L)$, by linear algebra this means that $G$ acts linearly (i.e. via a representation $G \rightarrow \GL_{n+1}$) on $\widetilde{X}$.
\begin{example}
	Recall Example \ref{projective-git-example}. The induced action on $\widetilde{X} = \A^{n+1}$ is just \[\lambda \cdot (p_0,...,p_n) = (\lambda^{-1}p_0,\lambda p_1,...,\lambda p_n) \] on the level of $k$-points. More rigorously, the co-ordinate rings are $k[t^{\pm 1}]$ and $k[x_0,...,x_n]$, and co-action homomorphism is given by $x_0 \mapsto t^{-1}\otimes x_0$ and $x_i \mapsto t\otimes x_i$ for $i > 0$.
\end{example}
We now present our first criterion for stability:
\begin{theorem}[Topological criterion for stability] \index{stability ! topological criterion}
	Let $X$ be a projective variety, let $G$ be a reductive affine algebraic group with a linearisation on the very ample line bundle $\L$, and let $S$ denote the resulting coordinate ring. 
	\begin{enumerate}
		\item A $k$-point $p$ is semistable if and only if for any lift $\tilde{p}\in \widetilde{X}(k)$, the closure of the $\tilde{p}$ orbit in $\widetilde{X}$,  $\overline{G\cdot \tilde{p}}$, does not contain the origin.
		\item A $k$-point $p$ is polystable if and only if the orbit of any of its lifts is closed in $\widetilde{X}$.
		\item A $k$-point $p$ is stable if and only if for any lift $\tilde{p}$ its orbit is closed in $\widetilde{X}$ and $\dim G_{\tilde{p}} = 0$.
	\end{enumerate}
\end{theorem}
\begin{proof}
	Fix a $k$-point $p$ and a lift $\tilde{p}$. If $p$ is semistable, then there is some $r>0$ and $\sigma\in S_{\deg r}^G$ such that $\sigma(p) \neq 0$. Let $f = \sigma - \sigma(\tilde{p})$. Then $f$ is invariant, and hence constant on $G\cdot \tilde{p}$. Observe that $f(0) = \sigma(0) - \sigma(\tilde{p}) = - \sigma(\tilde{p})$ (since $\sigma$ is homogeneous of positive degree, it follows $\sigma(0) = 0$), which means that there is some function which vanishes on $G\cdot \tilde{p}$ but not 0, and hence $0$ is not in the orbit closure of $\tilde{p}$. Conversely, suppose $0$ is not in the orbit closure of $\tilde{p}$. Then $\overline{G\cdot \tilde{p}}$ and the origin are both $G$-invariant closed subsets of $\widetilde{X}$, and it can be shown (\autocite[Corollary 1.2]{GIT}) there is some invariant $f\in S$ such that $f(0)= 0$ for all $g\in G(k)$ but $f(g\cdot p)\neq 0$. Clearly then $f$ has no degree zero component. Now let $f = \sum_{i > 0} f_i$ be the homogeneous decomposition of $f$, with $f_i$ of degree $i$. In particular, some $f_r$ must not vanish on $\tilde{p}$, and since $G$ preserves each homogeneous component of $S$, it follows that $f_r$ is invariant, and hence $f_r(p)\neq 0$, so $p$ is semistable. This proves (i).
	
	To prove (ii), firstly suppose $p$ is semistable (if it is not, then it cannot be polystable, and the closure of its orbit contains the origin). Then $p\in X_\sigma(k)$ for some invariant homogeneous $\sigma$ of positive degree. Observe then that $X_\sigma$ is $G$-invariant, and so $G\cdot p\subseteq X_\sigma(k)$. Now pick a lift $\tilde{p}$ of $p$, and consider the closed subscheme $V = \Spec S/ \langle \sigma - \sigma(\tilde{p}) \rangle$ of $\widetilde{X}$, which clearly contains the orbit $G\cdot \tilde{p}$. Now the map $\widetilde{X} \setminus \{0\} \rightarrow X$ restricts to a map $\varphi: V\mapsto X_\sigma$, which is a morphism of affine schemes, induced by the canonical ring homomorphism \[S[\sigma^{-1}]_{\deg 0} \rightarrow S/\langle \sigma - \sigma(\tilde{p}) \rangle,  \] \[ \frac{f}{\sigma} \mapsto \frac{f}{\sigma(\tilde{p})}, \] and since the homomorphism is surjective, the morphism $\varphi$ is finite, and hence closed. In particular, if $G \cdot \tilde{p}$ is closed in $\widetilde{X}$, then it is closed in $V$, and hence $G\cdot p$ is closed in $X_\sigma$. Conversely, suppose $G \cdot p$ is closed in $X_\sigma$, and let $V$ be as above. We claim the preimage of $G \cdot p$ in $V$ is equal to $G \cdot \tilde{p}$. To this end, firstly observe that clearly $G \cdot \tilde{p}$ is contained in $\varphi^{-1}(G \cdot p) \cap V(k)$, so suppose $\tilde{q}\in \varphi^{-1}(G \cdot p) \cap V(k)$, and we may suppose without loss of generality $\varphi(\tilde{q}) = p$. Then $\tilde{q}$ and $\tilde{p}$ lie in the same fibre over $p$ and thus, considered as points in an ambient $\A^n(k) = k^n$, differ by a scalar multiple, say $\tilde{q} = u\tilde{p}, u\in \Gm(k)$, and so \[\sigma(\tilde{p}) =\sigma(\tilde{q}) = \sigma(u\tilde{p}) = u^{\deg \sigma}\sigma(\tilde{p}),\] whence $u = 1$, as claimed. In particular, this means $G \cdot \tilde{p}$ is closed in $V$, and hence in $X(k)$ as desired. This proves (ii).
	
	Finally, to prove (iii), it suffices to show that if $p$ is polystable then $\dim G_p = 0$ if and only if $\dim G_{\tilde{p}} = 0$. Clearly $\dim G_{\tilde{p}} \leq \dim G_p$, and so it suffices to show that if $\dim G_{\tilde{p}} = 0$ then $\dim G_p = 0$. To this end, observe that finite morphisms are stable under base change (\autocite[Proposiion 6.1.5]{EGAII}), and hence the leftmost down arrow below is also finite:
	\begin{equation*}
		\begin{tikzcd}
			G_{\tilde{p}} \arrow[d]\arrow[r] & G\times V \arrow[r]\arrow[d] & V \arrow[d, "\varphi"] \\
			G_{p}\arrow[r] & G \times X_{\sigma} \arrow[r] & X_{\sigma}.
		\end{tikzcd}
	\end{equation*}
	This completes the proof.
\end{proof}

The main issue with the above is that it is oftentimes very difficult to compute the closure of an orbit, and in fact it may even be unknown what the homogeneous coordinate ring of our linearisation is in the first place! And thus we will require a slightly different notion, which is more computation-friendly. This is the \textit{Hilbert-Mumford criterion}, which relates stability to 1-parameter subgroups, which we will now define:
\begin{definition}\index{1-parameter subgroup}
	Let $G$ be an algebraic group. A \textit{1-parameter subgroup}, or just \textit{1-PS}, is a group homomorphism $\lambda: \bb{G}_m \rightarrow G$. 
	
	Now let $X$ be a scheme, separated over $k$ and let $f: \bb{G}_m \rightarrow X$ be any morphism. Then by the valuative criterion for separation, $f$ has at most one extension to a morphism $f^\sharp: \A^1 \rightarrow X$. If this extension does exist, we define the \textit{limit} of $f$ at 0, denoted $\lim_{t\to 0} f(t)$, to be \[\lim_{t\to 0} f(t):= f^\sharp(0). \] If this extension does not exist, we say the \textit{limit does not exist}.
\end{definition}
%\begin{lemma}
%	Let $\lambda: \bb{G}_m \rightarrow G$ be a 1-PS. Then $\lim_{t \to 0} \lambda(t)$ exists if and only if $\lambda$ is trivial.
%\end{lemma}
%\begin{proof}
%	content...
%\end{proof}


The idea is now to reinterpret stability in terms of whether or not certain limits exist (which, in practice, usually amounts to checking if negative powers of $t$ turn up in an expression). However, since we will only ever take $G = \SL_m$ for some $m\in \N$ in practical applications in this thesis, and since this case is easier to prove, we will henceforth assume $G = \SL_m$. It is worth noting that this method works for $\GL_m$ as well.
\begin{theorem}\label{fundamental-theorem-git}
	Let $\SL_m$ act on a projective variety $X$ with a very ample linearisation $\L$ embedding $X$ in $\P^n$. Let $p\in X(k)$.
	\begin{enumerate}
		\item $p$ is stable if and only if for any lift $\tilde{p}$ and nontrivial 1-PS $\lambda: \Gm \rightarrow \SL_m$, the limit \[ \lim_{t \to 0} \lambda(t) \cdot \tilde{p} \] does not exist.
		\item $p$ is semistable if and only if for any $\tilde{p}$ and 1-PS $\lambda: \Gm \rightarrow \SL_m$, we have \[\lim_{t \to 0} \lambda(t) \cdot \tilde{p}\neq 0. \] 
	\end{enumerate}
\end{theorem}
Before we give the proof, we require the following result:
\begin{lemma}[Iwahori's Decomposition Theorem]
	Let $R$ be a DVR with valuation $v$, fraction field $K$ and uniformiser $\varpi$. Then given any $M\in \SL_m(K)$, there exists $A,B$ and $\Lambda = \diag(\varpi^{r_1},...,\varpi^{r_m})$ such that \[M = A\Lambda B. \]
\end{lemma}
\begin{proof}
	Following \autocite[p. 215]{Mukai}, we induct on $m$, with $m=1$ being trivial. Firstly, denote the entries of $M$ by $m_{ij}$, and observe that multiplying left and right by permutation matrices we may assume \[v(m_{1,1})\leq v(m_{ij})\] for all $i,j$, and  multiplying by diagonal elements of $\SL_m(R^*)$, where $R^*$ is the group of units in $R$, we may assume $m_{1,1} = \varpi^{v(m_{1,1})}$. Now considering only the top and left entries, we have \[\begin{pmatrix*}
		1 & 0 &...&0\\
		-\frac{m_{2,1}}{m_{1,1}} & 1 & ... & 0\\
		...&&...\\
		-\frac{m_{m,1}}{m_{1,1}} & 0 &...&1
	\end{pmatrix*}M \begin{pmatrix*}
	1 & -\frac{m_{1,2}}{m_{1,1}} &...&-\frac{m_{1,m}}{m_{1,1}}\\
	0 & 1 &...& 0\\
	...&&...\\
	0 & 0&... & 1
	\end{pmatrix*} = \begin{pmatrix*}
	m_{1,1} & 0 &...&0\\
	0\\
	...&& M^\flat\\
	0
\end{pmatrix*}, \] where $M^\flat$ is some $(m-1)\times(m-1)$ matrix, and the result follows from the inductiive hypothesis applied to $M^\flat$.
\end{proof}
\begin{proof}[Proof of Theorem \ref{fundamental-theorem-git}]
	Following \autocite[pp. 216-218]{Mukai}, we begin with some generalities. Since the action of $\SL_m$ on $\widetilde{X}\subseteq \A^{n+1}$ is linear, there is a representation $\rho:\SL_m \rightarrow \GL_{n+1}$. The diagonal subgroup $T\cong (\Gm)^m$ of $\SL_m$, induces a weight-space decomposition \[V = k^{n+1} = \bigoplus_{\gamma\in \Z^m} V_{\gamma}, \] where if $\gamma = (r_1,...,r_m)$, then \[ \diag(t_1,...,t_m)\cdot v = \prod_i t_i^{r_i}\cdot v\] for all $v\in V_\gamma$. To see this, observe we can apply the usual weight space decomposition of Theorem \ref{torus-reductive} to each $\Gm$ component of $T$ (which will look like $\diag(1,...,t_i,...,1)$), and take refinements as $i$ moves up, noting that if $\diag(1,...,t_i,...,1) \cdot v = t_i^{r_i}v$ and $\diag(1,...,t_j,...,1) \cdot v = t_j^{r_j} v$ then $\diag(1,...,t_i,...,t_j,...,1)\cdot v = t_i^{r_i}t_j^{r_j}v$. In particular, we have a basis $\{e_{\gamma,i}\}$ of $V$ where $e_{\gamma,i}\in V_{\gamma}$. 
	
	Now to prove (i), we will first show that if a 1-PS attains a limit then $p$ is not stable. To this end, suppose we have a 1-PS $\lambda: \Gm \rightarrow \SL_m$ and suppose $\lim_{t \to 0} \lambda(t) \cdot \tilde{p} = \tilde{q}$ for some $\tilde{q}\in X(k)$. If $\tilde{q}\notin (\SL_m) \cdot \tilde{p}$ we are done, otherwise we observe that the action of $\lambda$ fixes $\tilde{q}$, hence $\im \lambda \subseteq (\SL_m)_{\tilde{q}}$. But since $\lambda $ is nontrivial and connected, it follows that $(\SL_m)_{\tilde{q}}$, and hence $(\SL_m)_{\tilde{p}}$ cannot be finite. 
	
	Conversely, suppose that $p$ is not stable. Then either $\dim (\SL_m)_{\tilde{p}} > 0$ or $(\SL_m) \cdot \tilde{p}$ is not closed. In the first case, note that since $(\SL_m)_{\tilde{p}}$ is affine (it is the fibred product of two affine schemes over an affine scheme) of positive dimension, it is not proper over $\Spec k$, and similarly, if $(\SL_m)\cdot \tilde{p}$ is not closed, then the map $\SL_m \rightarrow \tilde{X}$ given by multiplication by $\tilde{p}$ is not closed. In either case, by the noetherian version of the valuative criterion for properness, there exists a DVR, say $R$ with residue field $k$, uniformiser $\varpi$, fraction field $K \supseteq k$ with the property $R = k \oplus \varpi R$ (in particular there is a subring $k[\varpi^{\pm 1}]\subseteq K$ isomorphic to $k[t^{\pm 1}]$), and an element $M\in \SL_m(K)\setminus \SL_m(R)$ such that $M\cdot \tilde{p}$ specialises to some $\tilde{q}$ (that is, the image of $M \cdot \tilde{p}$ in $V:=k^{n+1}$ is $\tilde{q}$), for which we will adopt the notation $M \cdot \tilde{p} \to 0$. In the former case, this $\tilde{q}$ is just $\tilde{p}$; in the latter it is some point in the boundary. Applying Iwahori's theorem, this means we can write $M = A\Lambda B$ where $\Lambda = \diag(\varpi^{r_1},...,\varpi^{r_m})$ and moreover $\Lambda$ is nontrivial, since $M$ is not contained in $\SL_m(R)$. In particular, we have \[(A\Lambda B)\cdot \tilde{p} \to \tilde{q}. \] We note that by definition $A$ specialises to a matrix of determinant 1, and since matrix multiplication commutes with ring homomorphisms (and group actions are associative), it follows $\Lambda B \cdot \tilde{p} \to \tilde{q}$. Now using the basis $\{e_{\gamma, i}\}$ constructed in the beginning of the proof, we may write 
	\begin{equation}\label{diagonalised-hilbert-mumford-proof}
		B \cdot \tilde{p} = \sum b_{\gamma, i} e_{\gamma, i}\in R^n,
	\end{equation}
	and the statement $\Lambda B \cdot \tilde{p} = \tilde{q}$ implies that for each $\gamma = (a_1,...,a_m)$, we have 
	\begin{equation}\label{valuation-nonnegative}
		v_R(\varpi^{\sum a_ir_i}b_{\gamma, i}) \geq 0, 
	\end{equation} 
	where $v_R$ is the valuation. In particular, this is saying that if $v_R(b_{\gamma, i}) = 0$ then $\sum a_ir_i \geq 0$. But that means that for any $\overline{b}_{\gamma, i}\neq 0$, where $\overline{b}_{\gamma, i}$ is the image of $b_{\gamma, i}$ in the residue field $k$, we have 
	\begin{equation}\label{second-valuation-nonnegative}
		v_R(\varpi^{\sum a_ir_i}\overline{b}_{\gamma, i})  \geq 0
	\end{equation} 
	and in particular, it follows that $\Lambda \overline{B} \cdot \tilde{p}$, where $\overline{B}$ is the reduction of $B$ mod $\varpi$, has a specialisation, or more precisely, it is of the form 
	\begin{equation}\label{Lambda-1-PS-has-specialisation}
		\Lambda \overline{B} \cdot \tilde{p}=u + \varpi v\in V\otimes R,
	\end{equation}
	where $u\in V$, $v\in V\otimes R$. Furthermore, we note that since $\Lambda\in \SL_m(k[\varpi^{\pm 1}])$, it follows that $\Lambda \overline{B}\cdot \tilde{p}$ is actually contained in $V\otimes k[\varpi]\cong V\otimes k[t]$. Now using the isomorphism $k[t^{\pm 1}]\cong k[\varpi^{\pm 1}]$, the image of $\Lambda\in \SL_m(k[\varpi^{\pm 1}])$ in $\SL_m(k[t^{\pm 1}])$ is the matrix \[\lambda =\diag(t^{r_1},...,t^{r_m})\in \SL_m(k[t^{\pm 1}]),\] and (\ref{Lambda-1-PS-has-specialisation}) may be reinterpreted as saying \[\lambda \overline{B} \cdot \tilde{p} = u + tv \in V \otimes k[t] \] for some $u\in V, v\in V \otimes k[t]$. But the morphism $\bb{G}_m = \Spec k[t^{\pm 1}]\rightarrow \SL_m$ induced by $\lambda$ is a 1-PS, and it thus follows that $\overline{B}^{-1}\lambda \overline{B}$ is a 1-PS with limit \[\lim_{t\to 0} \overline{B}^{-1}\lambda(t) \overline{B}\cdot \tilde{p} =\overline{B}^{-1}u  \] as desired.
	
	Finally, to prove (ii), observe that if $\lim_{t \to 0} \lambda(t) \cdot \tilde{p} = 0$ for some 1-PS $\lambda$ then 0 is in the closure of $(\SL_m) \cdot \tilde{p}$, whence $p$ is unstable. Conversely, if 0 is in the closure of $(\SL_m) \cdot \tilde{p}$, then similar to above, by the valuative criterion of properness there is some $M\in \SL_m(K) \setminus \SL_m(R)$ such that $M \cdot \tilde{p} \to 0$, and reasoning exactly as before, we deduce the analogue of (\ref{diagonalised-hilbert-mumford-proof}), but since $\Lambda B \cdot \tilde{p} \to 0$ this time, strict inequality holds in (\ref{valuation-nonnegative}), and hence (\ref{second-valuation-nonnegative}) too. It thus follows that the $\Lambda \overline{B} \cdot p\to 0$, and thus $\overline{B}^{-1}\lambda \overline{B} \cdot \tilde{p} \to 0$ too, as desired.
\end{proof}
%One can prove various analogues and converses to this result, so that stability can in general be encoded in whether the limits of 1-PS's exist. The idea is now to find a numerical criterion for determining whether the limit will exist; this is done using the following definition: %This is clear. What is not clear is that the converse is also true:
%\begin{theorem}\label{fundamental-theorem-git}
%	Let $G$ be a reductive affine algebraic group acting on $\A^n$, with the action induced by a representation $G \rightarrow \GL_n$. Then for any $p\in \A^n(k)$, a the origin is in the closure of $G\cdot p$ if and only if there exists a nontrivial 1-PS $\lambda: \bb{G}_m \rightarrow G$ such that \[\lim_{t\to 0} \lambda_p(t) = 0.\]
%\end{theorem}
%The {\color{red}proof relies on the Cartan-Iwahori decomposition theorem}, which is beyond the scope of this thesis, but it can be found in \autocite[p. 48]{ModuliNotes} or \autocite[p. 53]{GIT}. This result is highly analogous to the theorem in analysis, which states that a point is in the closure of a set in a metric space if and only if there is some sequence which converges to it. 
And in summary, we have shown that stability and semistability (although not polystability) are encoded in whether the induced limits of 1-PS's exist. Our final task is to find a numerical criterion that tells us whether these limits do indeed exists. To this end, we have the following definition:

\begin{definition}
	Let $G$ be a reductive affine algebraic group acting on $X \subseteq \P^n$, with a linearisation on $\O_{X}(1)$. Then we have a linear action of $G$ on $\widetilde{X}\subseteq \A^{n+1}$. Now given a 1-PS $\lambda: \bb{G}_m \rightarrow G$, we have a weight space decomposition \[ k^{n+1}=: V = \bigoplus_{r\in \Z} V_r. \] Choose a basis $\{e_i\}$ for $V$ such that $\lambda \cdot e_i = \lambda^{r_i} e_i$ for all $\lambda\in \bb{G}_m(k)$. Now let $p\in X(k)$, and let $\tilde{p}\in \widetilde{X}(k)$ be a lift. We may write $\tilde{p} = \sum p_i e_i$. The \textit{Hilbert-Mumford weight of $\lambda$ at $p$ with respect to $\Ox(1)$}, denoted $\mu^{\Ox(1)}(p, \lambda)$, is the integer \[\mu^{\Ox(1)}(p, \lambda) := \max\{-r_i \mid p_i \neq 0\}.\] Note that this does not depend on the choice of $\tilde{p}$.
\end{definition}
We prove a very useful property of the Hilbert-Mumford weight:
\begin{lemma}\label{hilbert-mumford-weight-conjugate}
	For any $g\in G(k)$, we have $\mu^{\Ox(1)}(p, \lambda) = \mu^{\Ox(1)}(gp, g\lambda g^{-1})$.
\end{lemma}
\begin{proof}
	Observe that if $v\in V_i$, then for any $u\in \Gm(k)$ we have \[g\lambda(u)g^{-1}(gv) = g\lambda(u)v = gu^{i}v = u^i gv.\] Hence writing $p = \sum p_ie_i$ and applying this to each $p_ie_i$ we deduce the result.
\end{proof}
Of course, it is a simple observation that if $\mu^{\Ox(1)}(p, \lambda) < 0$, then all the $r_i$ are positive, and we have $\lim_{t \to 0} \lambda(t) \cdot \tilde{p} = 0$. If $\mu^{\Ox(1)}(p, \lambda) = 0$, then all the $r_i$ are nonnegative, and hence the limit $\lim_{t \to 0} \lambda(t) \cdot \tilde{p}$ exists, but may not be zero. In summary, we have:
\begin{theorem}[The Hilbert-Mumford Criterion for $\SL_m$]
	Let $\SL_m$ act linearly on a projective variety $X$ and suppose we have a very ample linearisation on $\O_{X}(1)$. Let $p$ be a $k$-point. Then $p$ is semistable if and only if $\mu^{\Ox(1)}(p, \lambda)\geq 0$ for all nontrivial 1-PS $\lambda$, with stability holding if and only if the condition holds with strict inequality.
\end{theorem}
\begin{proof}
	This follows directly from Theorem \ref{fundamental-theorem-git} and the above observation.
\end{proof}
\begin{remark}
	The Hilbert-Mumford criterion is true for general reductive groups, not just for $\SL_m$. In fact, our argument could easily be adapted for $\GL_m$ too. However, the general statement requires a stronger version of Iwahori's decomposition theorem, such as \autocite[Theorem 1.1]{Cartan-Iwahori}, which is valid for reductive groups in general, but requires our DVR be complete, and replaces the diagonal $\Lambda$ with an element of $G(K)$ of the form $\Spec K \rightarrow  \Gm \rightarrow G$, where the second arrow is some 1-PS, and the first is dual to $t \mapsto \varpi$ as before. The argument is similar to our's, although differences include taking $R$ to be a complete DVR in the valuative criterion of properness (which is always possible, since we can just replace $R$ by its completion) in order to invoke the general Iwahori decomposition and replacing the matrix arguments with more abstract arguments (although they achieve the same effect). The proof can be found in \autocite[pp. 53-54]{GIT}.
\end{remark}
%Unfortunately the proof is far beyond the scope of the thesis, but let us break down the theorem a little. The \enquote{only if} is actually quite intuitive: let $\lambda$ be a 1-PS. If $\mu^{\Ox(1)}(p, \lambda) < 0$, this means that all the $r_i$ in the preceding definition are positive. In particular, \[\lim_{t\to 0} \lambda_p(t) = 0, \] and so $p$ is unstable. Conversely, if $\lim_{t\to 0} \lambda_p(t) = 0,$ then it must be the case that all the $r_i$ are positive. Hence $p$ is unstable for this induced $\bb{G}_m$-action if and only if $\mu^{\Ox(1)}(p, \lambda) < 0$. Now if $\mu^{\Ox(1)}(p, \lambda) = 0$, then all the $r_i$ are nonnegative, with at least one strictly zero, and so the limit does exist, and it easy to see \[\lim_{t\to 0} \lambda_p(t) = \sum_{r_i = 0}  p_i e_i. \] In particular, if $\lambda$ is not trivial, then either the limit lies outside the orbit, whence $p$ is not polystable, and thus not stable, or the 1-PS is contained in the stabiliser $G_p$, whence $p$ is still not stable. If all the weights are nonpositive, the same argument can be made for $\lambda^{-1}$ in place of $\lambda$, and so in order for $p$ to be stable, there must be a mixture of positive and negative weights. 
%Before we give the proof, let us give an intuitive explanation: let $\lambda$ be a 1-PS. If $\mu^{\Ox(1)}(p, \lambda) < 0$, this means that all the $r_i$ in the preceding definition are positive. In particular, \[\lim_{t\to 0} \lambda_p(t) = 0, \] and so $p$ is unstable. Conversely, if $\lim_{t\to 0} \lambda_p(t) = 0,$ then it must be the case that all the $r_i$ are positive. Hence $p$ is unstable for this induced $\bb{G}_m$-action if and only if $\mu^{\Ox(1)}(p, \lambda) < 0$. Now if $\mu^{\Ox(1)}(p, \lambda) = 0$, then all the $r_i$ are nonnegative, with at least one strictly zero, and so the limit does exist, and it easy to see \[\lim_{t\to 0} \lambda_p(t) = \sum_{r_i = 0}  p_i e_i. \] In particular, if $\lambda$ is not trivial, then either the limit lies outside the orbit, whence $p$ is not polystable, and thus not stable, or the 1-PS is contained in the stabiliser $G_p$, whence $p$ is still not stable. If all the weights are nonpositive, the same argument can be made for $\lambda^{-1}$ in place of $\lambda$, and so in order for $p$ to be stable, there must be a mixture of positive and negative weights. This is the \enquote{only if} direction. The significance of the theorem is that the \enquote{if} direction is also true; that is stability in general can be detected using 1-PS's. However, in this thesis, we will only use this in practice for $G = \SL_m$, and so we will only be proving it for this case. To begin, we extract the following result:
%\begin{theorem}[Iwahori's Decomposition Theorem]
%	Let $R$ be a complete DVR with fraction field $K$, uniformiser $\varpi$ and residue field $k$. Then for any $M\in \SL_m(K)$, there exists a diagonal matrix $\lambda =\diag(\varpi^{r_1},...,\varpi^{r_m})\in \SL_m(K)$ where $r_i\in \Z$ and $A,B\in \SL_m(R)$ such that \[A\lambda B = M. \]
%\end{theorem}
%\begin{proof}
%	\autocite[Theorem 1.1]{Cartan-Iwahori}
%\end{proof}
%Now it is easy to see from our above discussion that the Hilbert-Mumford Criterion follows from the following result:
%\begin{proposition}
%	Let $p\in \widetilde{X}(k)$ and let $\sigma_p: \SL_m \rightarrow \widetilde{X}\subseteq \A^{n+1}$ be the morphism determined by $p$. 
%	\begin{enumerate}
%		\item If $0\in \overline{\SL_m \cdot p}$ then there is a 1-PS $\lambda: \bb{G}_m\rightarrow \SL_m$ such that \[\lim_{t \to 0} \lambda(t) \cdot p = 0. \]
%		\item If $0\notin \overline{\SL_m \cdot p}$ but $\dim (\SL_m)_p \neq 0$, then there is a nontrivial 1-PS $\lambda: \Gm \rightarrow (\SL_m)_p$.
%	\end{enumerate}
%\end{proposition}
%\begin{proof}
%	Following \autocite[pp. 216-218]{Mukai}, we begin with some generalities. Observe that the action of $\SL_m$ on $\widetilde{X}\subseteq \A^{n+1}$ is linear, and hence there is a representation $\rho:\SL_m \rightarrow \GL_{n+1}$. The diagonal subgroup $T\cong (\Gm)^m$ of $\SL_m$, induces a weight-space decomposition \[V = k^{n+1} = \bigoplus_{\gamma\in \Z^m} V_{\gamma}, \] where if $\gamma = (r_1,...,r_m)$, then \[ \diag(t_1,...,t_m)\cdot v = \prod_i t_i^{r_i}\cdot v\] for all $v\in V_\gamma$. To see this, observe we can apply the usual weight space decomposition of Theorem \ref{torus-reductive} applied to each $\Gm$ component of $T$ (which will look like $\diag(1,...,t_i,...,1)$), and take refinements as $i$ moves up, noting that if $\diag(1,...,t_i,...,1) \cdot v = t_i^{r_i}v$ and $\diag(1,...,t_j,...,1) \cdot v = t_j^{r_j} v$ then $\diag(1,...,t_i,...,t_j,...,1)\cdot v = t_i^{r_i}t_j^{r_j}v$. In particular, we have a basis $\{e_{\gamma,i}\}$ of $V$ where $e_{\gamma,i}\in V_{\gamma}$. 
%	
%	Now note that by the valuative criterion for properness, there exists a DVR, say $R$ (which, taking its completion, we may assume to be complete) with residue field $k$ and fraction field $K \supseteq k$ (in particular, $R$ has the structure of a $k$-algebra), and an element $M\in \SL_m(K)\setminus \SL_m(R)$ such that $M\cdot p$ specialises to zero (that is, $M \cdot p \cong 0 \mod \varpi$, where $\varpi$ is the uniformiser). Applying Iwahori's theorem, this means we can write $M = A\Lambda B$ where $\Lambda = \diag(\varpi^{r_1},...,\varpi^{r_m})$ and moreover $\Lambda$ is nontrivial, since $M$ is not contained in $\SL_m(R)$. In particular, we have \[(A\Lambda B)\cdot p \to 0. \] We note that by definition $A$ specialises to a matrix of determinant 1, and since matrix multiplication commutes with ring homomorphisms, it follows $\Lambda B \cdot p \to 0$. Now using the basis $\{e_{\gamma, i}\}$ constructed in the previous paragraph, we may write 
%	\begin{equation}\label{diagonalised-hilbert-mumford-proof}
%		B \cdot p = \sum b_{\gamma, i} e_{\gamma, i}\in R^n,
%	\end{equation}
%	and the statement $\Lambda B \cdot p = 0$ is equivalent to saying that if $\gamma = (a_1,...,a_m)$, then we have \[v_R(\varpi^{\sum a_ir_i}b_{\gamma, i}) e_{\gamma, i} > 0, \] where $v_R$ is the valuation. In particular, this is saying that if $v_R(b_{\gamma, i}) = 0$ then $\sum a_ir_i > 0$. But that means that for any $\overline{b}_{\gamma, i}\neq 0$, where $\overline{b}_{\gamma, i}$ is the residue class of $b_{\gamma, i}$ mod $\varpi$, we have $v_R(\varpi^{\sum a_ir_i}\overline{b}_{\gamma, i})  > 0$, and in particular, we have \[\Lambda \overline{B} \cdot p \to 0, \] where $\overline{B}$ is the reduction of $B$ mod $\varpi$. Now observe that $\Lambda$ may be considered a composition \[\Spec K \rightarrow \Gm \xrightarrow{\lambda} \SL_m,\] where the first arrow is dual to $k[t^{\pm 1}]\rightarrow K$ sending $t$ to $\varpi$ and $\lambda$ is represented by the matrix $\diag(t^{r_1},...,t^{r_m})\in \SL_m(k[t^{\pm 1}])$. It thus follows that $\overline{B}^{-1}\lambda \overline{B}$ is our desired 1-PS. 
%\end{proof}

Now that the tools have been developed, we conclude this chapter with a revisit to a previous example, to see if our new tools can shed more light.
\section{Conics Revisited}
Recall that in the conics example in chapter 1, we remarked (Remark \ref{conics-jump-phenomenon}) it is crucial we are not defining conics up to projective transformations, since we get jump phenomena. Now that we have developed the techniques of GIT, we will try to fit this within our framework. We begin with a formal definition of our problem:
\begin{definition}
	Let $k$ be an algebraically closed field of characteristic zero. The \textit{moduli problem of conics in $\P^2$ up to projective transformations} is the functor \[ \M^\flat: \mathsf{FTSch}/k \rightarrow \Sets, \; S\mapsto \{\text{families over } S\}/\Aut_S(\P^2\times S). \]
\end{definition}
Note that $\M^\flat$ is essentially the same problem, the only difference being that the equivalence relation on our families have changed. In particular, Lemmas \ref{conics-locally-cut-out-single-polynomial} and \ref{flat-conics-generate-ring} still hold, and thus while the family $\mf{X}$ in the statement of Theorem \ref{conic-universal-family-theorem} is no longer universal, it is still a locally versal family. 

Now observe that $\SL_3$ acts on $\P^5$ (which parameterises $\mf{X}$) by acting inversely in the usual way on the variables $x,y,z$. More precisely, the usual action \textbf{composed with inversion} $\SL_3 \times \P^2 \rightarrow \P^2$ induces the following diagram:
\begin{equation*}
	\begin{tikzcd}[row sep = huge]
		(	\SL_3 \times \P^2 \times \P^5) \times_{\P^2 \times \P^5} \mf{X}\arrow[d]\arrow[r] & \mf{X}\arrow[d]\\
		\SL_3 \times \P^2 \times \P^5 \arrow[r]& \P^2 \times \P^5
	\end{tikzcd}
\end{equation*}
and it is clear that $(	\SL_3 \times \P^2 \times \P^5) \times_{\P^2 \times \P^5} \mf{X}$ is a family of conics over $\SL_3 \times \P^5$, and since $\P^5$ is a fine moduli space for the moduli problem of conics in $\P^2$, this is equivalent to a morphism $\SL_3 \times \P^5 \rightarrow \P^5$. One can check that this is a group action, and acts on $k$-points as described, and clearly two fibres in $\mf{X}$ are equivalent if and only if they lie over points in the same orbit. This action has an obvious linearisation on $\O_{\P^5}(1)$, and thus induces an action $\SL_3 \times \A^6\rightarrow \A^6$. So now we ask: what do stability semi-stability look like in this context? The first is very easy to answer: since $\dim \SL_3 = 8$ and $\dim \P^5 = 5$, for purely dimensional reasons no point is stable. We will now look at semistability.

To begin, observe that we may scale the $x,y,z$ uniformly (i.e. multiply them by the same scalar) as we please, hence we may scale any $\GL_3(k)$ operation on the $x,y,z$ so that it ends up in $\SL_3(k)$. For example, in order to interchange $x,y$, even though the usual permutation matrix \[ A = \begin{pmatrix*}
	0 & 1 & 0\\
	1 & 0 & 0 \\
	0 & 0 & 1
\end{pmatrix*}\] has determinant $-1$, we may multiply $A$ by $-1$ so that the resulting determiant is 1, and we end up with the desired operation. In particular, we may use any $\GL_3(k)$ operation, and we will do so without further comment.

Next observe that we may regard the equation $f =s_0x^2+...+s_5zx$ of any conic as a quadratic form $Q$ on $V = k^3$, and hence we have an associated symmetric bilinear form $\beta$ with the matrix \[J=	\begin{pmatrix}
	2s_0 & s_1 & s_5 \\
	s_1 & 2s_2 & s_3 \\
	s_5 & s_3 & 2s_4
\end{pmatrix}.\] Note that $J(x , y, z)^t$ is just the Jacobian of $f$. We can interpret $\beta$ as a map $\beta: V \rightarrow V^*$. Observe that $X = \Proj k[x,y,z]/f$ is nondegenerate if and only if $\beta$ is injective. 

%In fact, we have the following very useful result:
%\begin{proposition}
%	Any conic is equivalent to one defined by a polynomial the form $a_1 x^2+a_2y^2+a_3z^3$.
%\end{proposition}
%\begin{proof}
%	Let  be the equation of a conic. Then $f$ defines a quadratic form on $V =k^3$ and to this quadratic form we have an associated symmetric bilinear form, which we will can regard as a map $\beta: V \rightarrow V^*$. Now observe that $V$ splits (non-canonically) as $V = \ker \beta \oplus U$ for some $U\subseteq V$. Now to any basis $\{u_i\}$ of $U$ we can apply the Gram-Schmidt algorithm to get an orthogonal basis $\{v_i\}$, and together with a basis $\{w_i\}$ of $\ker \beta$ we get a basis of $V$. Now let $g\in \GL_3(k)$ be the element taking the standard basis $\{e_1,e_2, e_3\}$ to $\{v_i\}\cup \{w_i\}$. Then it is clear that $g\cdot f$ is diagonal.
%\end{proof}
%In fact, since we may change the $a_i$ as we wish, the only actual invariant is the number of the $a_i$ which are zero, and thus we have three cases, either zero, one or two of the $a_i$ are zero. In the first case, the conic is nondegenerate and nonsingular, in the latter two the conic is singular.
\begin{theorem}
	A conic $X$ over $k$ is semistable with respect to the linearisation of the $\SL_3$ action if and only if $X$ is nondegenerate.
\end{theorem}
\begin{proof}
%	Firstly, observe that by Theorem \ref{torus-reductive}, any 1-PS of $\SL_3$ is conjugate to a diagonal matrix. The observation that stability is only dependent on the orbit combined with Lemma \ref{hilbert-mumford-weight-conjugate} allows us to suppose without loss of generality that any 1-PS is diagonal. 
	
	Let $X$ be a conic defined by some $f\in k[x,y,z]_{\deg 2}$ and let $Q$ and $\beta: V \rightarrow V^*$ be as above. Suppose firstly $X$ is denerate, and let $v_0 \in \ker \beta\setminus\{0\}$. Let $v_1,v_2$ be orthogonal to each other (that is, $\beta(v_1, v_2) = 0$) so that $\{v_0, v_1, v_2\}$ is an orthogonal basis for $V$, and define a 1-PS as follows: let $g\in \SL_3(k)$ be (a scale of) the element sending $(v_0, v_1, v_2)$ to $(e_1, e_2, e_3)$, and define the 1-PS $\lambda: \Gm \rightarrow \SL_3$ represented as \[g^{-1}\diag(t^{2}, t^{-1}, t^{-1})g\in \SL_3(k[t^{\pm 1}]). \] Then one can check that $g\cdot f = ay^2+bz^2$, for some $a,b\in k$, and \[(g\lambda(u) g^{-1})\cdot (g\cdot X) = \diag(u^{2}, u^{-1}, u^{-1}) \cdot \Proj k[x,y,z]/ay^2+bz^2 = \Proj k[x,y,z]/u^2ay^2 + u^2bz^2 \] for any $u\in \Gm(k)$, and thus \[\mu(X, \lambda) = \lambda(g\cdot X, g\lambda g^{-1}) = -2 < 0 \] whence $X$ is unstable, as desired.
	
	Conversely, suppose $X = \Proj k[x,y,z]/f$ is unstable. Then there is some 1-PS $\lambda: \Gm \rightarrow \SL_3$ such that $\mu(X, \lambda) < 0$. The observation that stability is only dependent on the orbit combined with Lemma \ref{hilbert-mumford-weight-conjugate} allows us to suppose without loss of generality that $\lambda$ diagonal, and hence represented by some \[\diag(t^{r_1}, t^{r_2}, t^{r_3})\in \SL_3(k[t^{\pm 1}]), \] where $r_1+r_2+r_3 = 0$, and we may furthermore assume without loss of generality $r_1 \leq r_2\leq r_3$. Write $f =s_0x^2 +s_1 xy+s_2 y^2 + s_3 yz + s_4 z^2+s_5 zx$ and observe that \[ \lambda(u)\cdot f =s_0u^{-2r_1}x^2 +s_1u^{-r_1-r_2}xy+s_2u^{-2r_2} y^2 + s_3 u^{-r_2-r_3}yz + s_4 u^{-2r_3}z^2+s_5u^{-r_3-r_1} zx. \] In order for $\mu$ to be negative, it follows that all the powers of $u$ in the above expression with nonvanishing coefficient must be strictly positive. With this in mind, observe that since $r_1 < 0$ and $r_3 > 0$, it must be that $s_4 = 0$. Here we have a trichotomy about the sign of $r_2$: if $r_2 = 0$, then it follows $r_1 = -r_3$ and $s_3$ and $s_5$ are both 0. We then see \[\det J = 2s_2s_5^2 = 0. \] Now supposing that $r_2 > 0$, it once again follows $s_3 = 0$ and also $s_2 = 0$, and so $\det J = 0$. Finally, if $r_2 < 0$, then since $r_2 < r_3$, it follows $-r_2-r_3 < -2r_2 < 0$ and hence we must have $s_3 = 0$ one final time. Similarly, it follows $-r_3-r_1 = r_2 < 0$ and so $s_5 = 0$, whence $\det J = 0$ too. This completes the proof.
\end{proof}
Since any two nondegenerate conics are equivalent, it follows that $\P^5 \git_{\O_{\P^5}(1)} \SL_3 = \Spec k$. 
\begin{remark}
	We can make a little tweak to this situation as follows: firstly note that every quadratic form is diagonalisable (\autocite[IV, Theorem 1]{Serre}), and so we may assume every conic is of the form $\Proj k[x,y,z]/px^2+qy^2+rz^2$. Then analogously, the space of diagonal forms has a natural locally versal family parameterised by $\P^2 = \Proj k[a,b,c]$ with the family cut out by $ax^2+by^2+cz^2$. Then we need only consider the diagonal $\Gm^2\subseteq \SL_3$ (we only need two copies of $\Gm$ because the third diagonal entry is given by the reciprocal of the product of the first two, since we are working with $\SL_3$) acion on $\P^2$, with the obvious linearisation on $\O_{\P^2}(1)$, and with respect to this linearisation, doing the exact same calculation we find that nondegenerate conics are in fact \textbf{stable}, not just semistable (and in fact, stability and semistability coincide). Of course, the GIT quotient will be the same in both case.
\end{remark}
This innocuous calculation and result is an illustration of how geometric invariant theory is commonly used. Let $\M$ be a moduli problem, and suppose $X \rightarrow S$ is a locally versal family. Then if an algebraic group $G$ acts on $S$ parameterising equivalent families, restricting to the stable locus (possibly with respect to a very ample linearisation $\L$) can filter out \enquote{bad} points; this is particularly useful if $\M$ has a jump phenomenon, as we have just seen. Moreover, it is not uncommon for stability to coincide with a \enquote{natural} condition on the underlying na\"ive moduli problem of $\M$ (for example, nonsingularity in the above case). And furthermore, the GIT quotient $S^s \rightarrow S^s \git _{\L} G$ is a geometric quotient, so first and foremost it is a categorical quotient, and one can use the fact $X\rightarrow S$ is locally versal to build a natural transformation $\eta: \M \rightarrow \Hom(-, S^s\git_{\L} G)$ which will satisfy property (ii) in the definition of a coarse moduli space. Moreover, since it is an orbit space, property (i) will also be satisfied, and in particular, we can salvage a coarse moduli space of \enquote{stable} objects even if we have a jump phenomenon. In the next chapter, we will apply this idea to construct the moduli space of stable vector bundles.

%Observe that any $p_0x^2+...+p_5 zx$ can be written as $\sum p_{ij}x^{2-i-j}y^iz^j$, and so we will label the $k$-points on $\A^6$ as thus. Now by Theorem \ref{torus-reductive}, we know that any 1-PS $\lambda$ of $\SL_3$ is (up to change of basis) diagonal, and thus for any $u\in \bb{G}_m(k)$, we have  \[\lambda(u) = \diag(u^{r_1}, u^{r_2 }, u^{r_3}):=\begin{pmatrix}
%	u^{r_1} \\ & u^{r_2 }\\ && u^{r_3}
%\end{pmatrix}, \] and $r_1+r_2+r_3 = 1$, whence \[u\cdot \sum p_{ij}x^{2-i-j}y^iz_j = \sum p_{ij} u^{2r_1+i(r_2-r_1)+j(r_3-r_1)}x^{2-i-j}y^iz_j\] and \[\mu^{\O_{\P^5}(1)}(p, \lambda) = \max\{-(2r_1+i(r_2-r_1)+j(r_3-r_1)) \mid p_{ij}\neq 0\}. \] 