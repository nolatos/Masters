
% --------------------------------------------------------------------------------------
%                   LATEX TEMPLATE FOR DISSERTATION (HONS)
% --------------------------------------------------------------------------------------
\documentclass[11pt]{book}
\setlength{\parindent}{15pt}


\usepackage[hidelinks]{hyperref}
\usepackage{amsmath}
\usepackage{amsfonts,amsthm,amssymb,enumerate,mathrsfs}
\usepackage{times}
\usepackage{mathtools}
\usepackage[utf8]{inputenc}
\usepackage[english]{babel}
\usepackage{xfrac}
\usepackage{tikz-cd}
%\usepackage{tikz}
\usepackage{theoremref}
\usepackage[backend=biber,style=numeric,autocite=inline]{biblatex}
\addbibresource{bib.bib}
\usepackage{imakeidx}
\usepackage{enumitem}
\usepackage{bm}
\usepackage{xcolor}
\usepackage{csquotes}
\usepackage{CJKutf8}
\usepackage{MnSymbol}
\usepackage{CJKutf8}
%\usepackage{geometry}
%\usetikzlibrary{matrix,arrows}
%\usepackage[backref=page,pagebackref=true,linkcolor = blue,citecolor = red]{hyperref}
%\usepackage[backref=page]{backref}

\makeindex[columns = 3]

\theoremstyle{definition}
\newtheorem{definition}{Definition}[section]
\newtheorem{example}[definition]{Example}
\newtheorem{infdef}[definition]{Informal Definition}

\theoremstyle{plain}
\newtheorem{theorem}[definition]{Theorem}
\newtheorem{proposition}[definition]{Proposition}
\newtheorem{corollary}[definition]{Corollary}
\newtheorem{lemma}[definition]{Lemma}
\newtheorem{construction}[definition]{Construction}
\newtheorem{joke}[definition]{Joke}

\theoremstyle{remark}
\newtheorem{remark}[definition]{Remark}

\DeclareMathOperator{\Hom}{Hom}
\DeclareMathOperator{\im}{im}
\DeclareMathOperator{\coker}{coker}
\DeclareMathOperator{\Spec}{Spec}
\DeclareMathOperator{\Proj}{Proj}
\DeclareMathOperator{\Frac}{Frac}
\DeclareMathOperator{\Div}{Div}
\DeclareMathOperator{\nil}{nil}
\DeclareMathOperator{\ev}{ev}
\DeclareMathOperator{\height}{ht}
\DeclareMathOperator{\id}{id}
\DeclareMathOperator{\codim}{codim}
\DeclareMathOperator{\Cl}{Cl}
\DeclareMathOperator{\Pic}{Pic}
\DeclareMathOperator{\Ca}{Cart}
\DeclareMathOperator{\CaCl}{CaCl}
\DeclareMathOperator{\GL}{GL}
\DeclareMathOperator{\SL}{SL}
\DeclareMathOperator{\PGL}{PGL}
\DeclareMathOperator{\Ann}{Ann}
\DeclareMathOperator{\lcm}{lcm}
\DeclareMathOperator{\Lie}{Lie}
\DeclareMathOperator{\Ad}{Ad}
\DeclareMathOperator{\Hol}{Hol}
\DeclareMathOperator{\Fr}{Fr}
\DeclareMathOperator{\Deck}{Deck}
\DeclareMathOperator{\End}{End}
\DeclareMathOperator{\E}{\mathcal{E}}
\DeclareMathOperator{\Stab}{Stab}
\DeclareMathOperator{\diag}{diag}
\DeclareMathOperator{\Fun}{Fun}
\DeclareMathOperator{\Sym}{Sym}
\DeclareMathOperator{\spn}{span}
\DeclareMathOperator{\Gr}{Gr}
\DeclareMathOperator{\fancyExt}{\mathcal{E}xt}
\DeclareMathOperator{\Ext}{Ext}
\DeclareMathOperator{\fancyEnd}{\mathcal{E}nd}
\DeclareMathOperator{\fancyHom}{\mathcal{H}om}
\DeclareMathOperator{\Aut}{Aut}
\DeclareMathOperator{\yo}{よ}
\DeclareMathOperator{\tr}{tr}
\DeclareMathOperator{\vol}{vol}
\DeclareMathOperator{\Supp}{Supp}
\DeclareMathOperator{\rk}{rk}
\DeclareMathOperator{\Char}{char}
\DeclareMathOperator{\Quot}{Quot}
\DeclareMathOperator{\Hilb}{Hilb}
\DeclareMathOperator{\Tor}{Tor}


\renewcommand{\div}{\operatorname{div}}

\newcommand{\p}{\mathfrak{p}}
\newcommand{\R}{\mathbb{R}}
\newcommand{\Inn}{\text{Inn }}
\newcommand{\N}{\mathbb{N}}
\newcommand{\Q}{\mathbb{Q}}
\newcommand{\Z}{\mathbb{Z}}
\newcommand{\A}{\mathbb{A}}
\newcommand{\M}{\mathcal{M}}
\newcommand{\Sets}{\mathsf{Sets}}
\newcommand{\Ab}{\mathsf{Ab}}
\newcommand{\Top}{\mathsf{Open}}
\newcommand{\Sch}{\mathsf{Sch}}
\newcommand{\Gps}{\mathsf{Gps}}


\newcommand{\C}{\mathbb{C}}

\newcommand{\V}{\mathcal{V}}
\newcommand{\F}{\mathcal{F}}
\newcommand{\G}{\mathcal{G}}
\newcommand{\Hh}{\mathcal{H}}
\renewcommand{\L}{\mathcal{L}}
\renewcommand{\O}{\mathcal{O}}
\newcommand{\OxMod}{$\O_X$-{module}}
\newcommand{\Ox}{\mathcal{O}_X}
\newcommand{\Oy}{\mathcal{O}_Y}
\newcommand{\Om}{\mathcal{O}_M}
\newcommand{\K}{\mathcal{K}}
%\newcommand{\I}{\mathcal{I}}
\newcommand{\B}{\mathcal{B}}
\newcommand{\g}{\mathfrak{g}}
\newcommand{\delbar}{\bar{\partial}}
\renewcommand{\u}{\mathfrak{u}}
\newcommand{\git}{\mathbin{
		\mathchoice{/\mkern-4mu/}% \displaystyle
		{/\mkern-4mu/}% \textstyle
		{/\mkern-3mu/}% \scriptstyle
		{/\mkern-3mu/}}}% \scriptscriptstyle

\renewcommand{\P}{\mathbb{P}}
\newcommand{\scr}{\mathscr}
\newcommand{\cl}{\mathcal}
\newcommand{\mf}{\mathfrak}
\renewcommand{\sf}{\mathsf}
\newcommand{\bb}{\mathbb}
\renewcommand{\labelenumi}{\normalfont(\roman*)}

%\newcommand{\yo}{\text{\usefont{U}{min}{m}{n}\symbol{'210}}}
%\DeclareFontFamily{U}{min}{}
%\DeclareFontShape{U}{min}{m}{n}{<-> dmjhira}{}



\DeclareGraphicsExtensions{.pdf,.png,.jpg}

\usepackage{etoolbox}% http://ctan.org/pkg/etoolbox
\makeatletter
\patchcmd{\@makechapterhead}{\vspace*{50\p@}}{}{}{}% Removes space above \chapter head
\patchcmd{\@makeschapterhead}{\vspace*{50\p@}}{}{}{}% Removes space above \chapter* head
\makeatother

%\setlength{\oddsidemargin}{1.5cm}
%\setlength{\evensidemargin}{0cm}
%\setlength{\topmargin}{1mm}
%\setlength{\headheight}{1.00cm}
%\setlength{\headsep}{1.00cm}
%%\setlength{\textheight}{20.84cm}
%\setlength{\textheight}{20.5cm}
%\setlength{\textwidth}{15.5cm}
%\setlength{\marginparsep}{1mm}
%\setlength{\marginparwidth}{3cm}
%\setlength{\footskip}{2.36cm}

\setlength{\oddsidemargin}{1cm}
\setlength{\evensidemargin}{0cm}
\setlength{\topmargin}{.5mm}
\setlength{\headheight}{.5cm}
\setlength{\headsep}{1.00cm}
%\setlength{\textheight}{20.84cm}
\setlength{\textheight}{20.5cm}
\setlength{\textwidth}{15.5cm}
\setlength{\marginparsep}{1mm}
\setlength{\marginparwidth}{3cm}
\setlength{\footskip}{2.36cm}


\newcommand{\Mod}{\mathrm{mod}\ }
\newcommand{\lag}[2]{\mathchoice{\left(\frac{#1}{#2}\right)}{\bigl(\frac{#1}{#2}\bigr)}{??}{??}}
\renewcommand{\phi}{\varphi}
\renewcommand{\epsilon}{\varepsilon}
\renewcommand{\vec}[1]{\mathbf{#1}}
\newcommand*{\fullref}[1]{\hyperref[{#1}]{\nameref*{#1} (\autoref*{#1})}} % One single link





%\emph correction to bold
%\makeatletter
%\DeclareRobustCommand{\em}{%
%	\@nomath\em \if b\expandafter\@car\f@series\@nil
%	\normalfont \else \bfseries \fi}
%\makeatother







\begin{document}

\pagestyle{empty}

%: ----------------------------------------------------------------------
%:                  TITLE PAGE: name, degree,..
% ----------------------------------------------------------------------


\begin{center}

\vspace{1cm}

%%% Type the thesis title below%%%%%%%%%%%%%%%%
{\Huge  Moduli of Vector Bundles over an Algebraic Curve}

\vspace{35mm} 

\includegraphics[width=2cm]{logo}

 \vspace{45mm}

%%%%%Type Your Name Below%%%%%%%%%%%%
{\Large   Oliver Li    }

	\vspace{1ex}

Department of Mathematics

The University of Auckland

	\vspace{5ex}

 %%%%%Typing Your Supervisors Name Below%%%%%%%%%%%%
Supervisor:           Dr Pedram Hekmati

	\vspace{30mm}

A thesis  submitted in partial fulfillment of the requirements for the degree of MSc in Mathematics, The University of Auckland, 2021.

\end{center}


  \newpage



%: --------------------------------------------------------------
%:                  FRONT MATTER:  abstract,..
% --------------------------------------------------------------
\chapter*{Abstract}      

\setcounter{page}{1}
\pagestyle{headings}
 \pagenumbering{roman}

\addcontentsline{toc}{chapter}{Abstract}









%: --------------------------------------------------------------
%:                  END:  abstract
% --------------------------------------------------------------
\chapter*{Acknowledgements}       
\addcontentsline{toc}{chapter}{Acknowledgements}


%: ----------------------- contents ------------------------
\setcounter{secnumdepth}{3} % organisational level that receives a numbers
\setcounter{tocdepth}{2}    % print table of contents for level 3
{\small \tableofcontents }           % print the table of contents
% levels are: 0 - chapter, 1 - section, 2 - subsection, 3 - subsection



%: --------------------------------------------------------------
%:                  MAIN DOCUMENT SECTION
% --------------------------------------------------------------
	



\mainmatter

\chapter*{Introduction}
\addcontentsline{toc}{chapter}{Introduction}

\part{Algebraic Theory}

\chapter{Moduli Theory}
When studying a classification problem, a very natural thing to ask for is a geometric parameter space. Or in other words, we are asking for a geometric space whose points are in bijection with equivalence classes of the objects we are classifying. We begin with a very informal example, to illustrate the sort of thing we are looking for:
\begin{example}
	Consider the set of all circles in $\R^2$, up to equality. A circle is uniquely determined by its centre and its radius, and hence the set of circles is in natural 1-1 correspondence with the set $M=\R^2 \times \R_{>0}$ (we are discounting circles of radius zero), with the $\R^2$-component representing its centre and the $\R_{>0}$-component representing its radius. Note that points that are \enquote{close} in $M$ represent circles that are \enquote{similar} in $\R^2$. In particular, the geometry of $M$ reflects, in some vague sense, the structure of the set of circles.
\end{example}
Now that we have a rough idea of what we are looking for, we make our informal definition: \index{moduli problem ! na\"ive} \index{moduli space ! na\"ive}
\begin{infdef}
	A \textit{na\"ive moduli problem} is a pair $(\cal{M}, \sim)$ where $\cal{M}$ is a collection of objects and $\sim$ is an equivalence relation on $\cal{M}$. We will often just denote the problem by $\cal{M}$, and we often assume without loss of generality (for example, by replacing $\cal{M}$ by $\cal{M}/\sim$) that the equivalence relation is equality. A \textit{na\"ive moduli space} of $\cal{M}$ is a geometric space $M$ equipped with a bijection $\eta: (\mathcal{M}/\sim) \rightarrow M$.
\end{infdef}

Firstly, observe that \enquote{geometric space} is undefined in general, which is why this is an informal definition. But even if we insist that a geometric space is a (smooth/Riemannian/K\"ahler) manifold or a scheme or a stack, this is still not a very useful notion to work with, since it is literally just a question of cardinality; indeed, a moduli problem $\cal{M}$ with cardinality $2^{\aleph_0}$ always has a moduli space, and in fact \textbf{any} manifold or variety of positive dimension over $\C$ could be one such moduli space! Hence we have to insist that $\eta$ relates the geometry of $M$ and the \enquote{structure} of $\cal{M}$ in some way. We dedicate this rest of the chapter into formalising and studying this last condition in the context of algebraic geometry. 
%\begin{remark}
%	We will also, however, encounter na\"ive moduli spaces which do not fall under this formalism. In these cases, our construction will be very ad-hoc: we will take $\cal{M}$ as a set, and endow it with a \enquote{natural} (topological/smooth/Riemannian etc.) structure, and define our resulting na\"ive moduli space $M$ as \textbf{the} moduli space of $\cal{M}$. 
%\end{remark}


\section{The Functor of Points}

\index{functor of points}Let $X$ be a scheme over a base scheme $T$. We make the following definition:
\begin{definition}
	Let $S$ be another $T$-scheme. An \textit{$S$-valued point}, or simply \textit{$S$-point} of $X$ is a $T$-morphism $p:S \rightarrow X$. If $S$ is affine, equal to $\Spec R$ then an $S$-valued point will be called an \textit{$R$-valued point}. The set of $R$-valued points of $X$ will be denoted $X(R)$.
\end{definition}
This is perhaps a weird definition to make, since a geometric space is usually just a set of points endowed with some structure. To make some sense of it, we consider the following examples:
%fix a $k$-valued point, say $p$, and recall the following: the underlying topological space of $\Spec k$ consists of only one point (in the usual sense). The underlying continuous map of the morphism $p$ sends this point to a point (in the usual sense) of $X$. Call this image $P\in X$. The pushforward sheaf $p_*\O_{\Spec k}$ is just the skyscraper sheaf $k$ sitting over $P$, and for any open set $U \subseteq X$ containing $P$ we have the following homomorphism of $k$-algebras: 
%\begin{equation*}
%	\begin{tikzcd}
%		k \arrow[r]& \Ox(U) \arrow[r, "p^\sharp"] &k
%	\end{tikzcd}
%\end{equation*}
%where $p^\sharp$ is the induced map of sheaves of $p$. As $U$ varies over all the open sets containing $P$, we have the following:
%\begin{equation*}
%	\begin{tikzcd}
%		k \arrow[r]& \varinjlim_{U \ni P}\Ox(U) = \O_{X, P} \arrow[r, "p^\sharp"] &k
%	\end{tikzcd}
%\end{equation*}
%where $\O_{X,P}$ is the local ring of $P$, with maximal ideal, say $\mf{m}$ and residue field say $K$. Since the map $\O_{X,P} \rightarrow k$ is required to be a local homomorphism (that is $\mf{m} \mapsto 0$), this means we have a composition $k \rightarrow K \rightarrow k$ hence $K = k$. Conversely, given a (usual) point $P\in X$ with residue field $k$, one can easily show that there is a $k$-valued point sending $\Spec k$ to $P$. To wrap up, a $k$-valued point of $X$ is simply a usual point with residue field $k$.
\begin{example}
	Let $T = \Spec A$ for some noetherian ring $A$, let $R$ be an $A$-algebra, let $I = \langle f_1,...,f_r \rangle$ be an ideal of $A[x_1,...,x_n]$ (the noetherian hypothesis is just so that $I$ can be finitely generated), and let $X = \Spec A[x_1,...,x_n]/I$. Then \[X(R) = \{a = (a_1,...,a_n)\in R^n \mid f_i(a) = 0 \text{ for all }1\leq i \leq r\} \] (where, by abuse of notation, the equality above means \enquote{canonical identification}). Indeed, an element of $X(R)$ is just an $A$-algebra homomorphism $A[x_1,...,x_n]/I\rightarrow R$, and this is equivalent to giving a tuple $a = (a_1,...,a_n)$ such that $f_i(a) = 0$ for all $i$, with the homomorphism given by $x_i \mapsto a_i$. 
\end{example}
\begin{example}
	Retain the notation and hypothesis of the above example. In the special case $A = k$ for an algebraically closed field $k$ and $I$ is prime, the set $X(k)$ is just the maximal ideals of $k[x_1,...,x_n]/I$, by Hilbert's Nullstellensatz; so in other words $X(k)$ is the object that is classically known as an \enquote{affine variety}, as defined in \autocite[I, Section 1]{Hart}. In particular, given a variety $Y$, (that is, an integral, scheme, equipped with a separated morphism of finite type to $\Spec k$), the set $Y(k)$ is canonically identified with the closed points of $Y$ (as a locally ringed space). We will be making the use of this canonical identification without further comment.
\end{example}

The set $S$-valued points is (tautologically) the set $\Hom(S, X)$. As $S$ varies, we get a contravariant functor $\Hom(-, X)$, known as the \textit{functor of points} of $X$. We will study this functor, firstly showing that this determines $X$ up to isomorphism. We work in an arbitrary category $\sf{C}$, since our proofs are no more difficult, but the important case is $\mathsf{C} = \mathsf{Sch}/k$, the category of schemes over a ground field $k$.
\begin{proposition}\label{yoneda-up-to-iso}
	Let $X$ and $Y$ be objects in a category $\sf{C}$. Suppose $\alpha:\Hom(-,X) \rightarrow \Hom(-, Y)$ is a natural isomorphism (that is, a natural transformation with an inverse). Then $X \cong Y$ 
\end{proposition}
\begin{proof} 
	Consider the map $\alpha_X: \Hom(X,X) \rightarrow \Hom(X,Y)$. Then $\alpha_X(\id)$ is a morphism from $X$ to $Y$. Denote this morphism by $f$. Now applying $\alpha$ to the map $f: X \rightarrow Y$ we get the following commutative diagram:
	\begin{equation*}
		\begin{tikzcd}[row sep = huge]
			\Hom(Y,X) \arrow[r, "f^*"] \arrow[d, "\alpha_Y"]& \Hom(X,X) \arrow[d, "\alpha_X"]\\
			\Hom(Y,Y)\arrow[r, "f^*"] & \Hom(Y,X)
		\end{tikzcd}
	\end{equation*}
	where $f^*$ is the map $u \mapsto u \circ f$. Consider $g:= \alpha_Y^{-1}(\id)\in \Hom(Y,X)$. The commutativity of the diagram says that $g \circ f = \id$. Now reversing the roles of $X$ and $Y$ will show that $f \circ g = \id$ too, hence $f$ is an isomorphism.
\end{proof}

We now consider the functor category $\Fun(\sf{C}^\sf{opp}, \Sets)$, whose objects are contravariant functors $\sf{C} \rightarrow \Sets$ and whose morphisms are natural transformations (this is also known as the \textit{presheaf category} on $\sf{C}$; recall that a \textit{presheaf} is just a contravariant functor $\sf{C}\rightarrow \Sets$). There is a natural functor $\sf{C} \rightarrow \Fun(\sf{C}^\sf{opp}, \Sets)$ which sends $X$ to $\Hom(-, X)$ and $X \rightarrow Y$ to the obvious natural transformation $\Hom(-,X) \rightarrow \Hom(-,Y)$. This functor is known as the \textit{Yoneda embedding}. The name is justified by our first corollary to the following proposition: 
\begin{proposition}[Yoneda's Lemma]\index{Yoneda's Lemma}
	Let $\sf{C}$ be a category, let $F\in \Fun(\sf{C}^\sf{opp}, \Sets)$ be a contravariant functor from $\sf{C}$ into $\Sets$ and let $A$ be an object in $\sf{C}$. Then there is a canonical bijection between the set of natural transformations $\Hom(-, A)\rightarrow F$ and $F(A)$ given by $\alpha \mapsto \alpha_A(\id)$.
\end{proposition} 
\begin{proof}
	\autocite[p. 5]{ModuliNotes}.
\end{proof}
Before we state our corollaries, recall that a functor $F: \sf{C} \rightarrow \sf{C'}$ is \textit{fully faithful} if for every pair of objects $A,B$ in $\sf{C}$, the induced map $\Hom_{\sf{C}}(A,B) \rightarrow \Hom_{\sf{C'}}(A,B)$ is a bijection.
\begin{corollary}\label{fully-faithful-yoneda}
	The Yoneda embedding is fully faithful.
\end{corollary}
\begin{proof}
	For any given objects $A$ and $B$, take $F = \Hom(-, B)$ in the above proposition. Then there is a bijection between the set of natural transformations $\Hom(-, A) \rightarrow \Hom(-, B)$ and $\Hom(A,B)$, given by $\alpha \mapsto \alpha_A(\id)$. Fix some $\alpha$ and write $f:= \alpha_A(\id)$. Applying the Yoneda embedding to $f: A \rightarrow B$ gives the natural transformation $f^*: \Hom(-, A) \rightarrow \Hom(-,B)$, where for an object $X$ and $\varphi\in \Hom(X,A)$, we have $f^*_X(\varphi) = f \circ \varphi$. Taking $X = A$ and $\varphi = \id$, we observe $$f^*_A(\id) = f \circ \id = f = \alpha_A(\id)$$ By Yoneda's Lemma, this means $f^* = \alpha$. Conversely, given some $f: A \rightarrow B$, we see that $f^*_A(\id) = f$ hence the Yoneda embedding induces the bijection between $\Hom(A,B)$ and the set of natural transformations $\Hom(-, A) \rightarrow \Hom(-, B)$ described in Yoneda's Lemma. But the set of natural transformations $\Hom(-, A) \rightarrow \Hom(-, B)$ is exactly the set of morphisms from $\Hom(-, A)$ to $\Hom(-, B)$ in the functor category, and hence the embedding is fully faithful.
\end{proof}
\begin{remark}
	In fact, Proposition \ref{yoneda-up-to-iso} follows easily from the above corollary too.
\end{remark}
\begin{corollary}[Cayley's Theorem]\label{cayleys-theorem}
	Let $G$ be a group. Then $G$ is isomorphic to a subgroup of $\Sym(G)$.
\end{corollary}
\begin{proof}
	We can interpret $G$ as a groupoid $\sf{G}$ with one object, say $x$, and automorphism group equal to $G$. In other words, $\Hom_{\sf{G}}(x,x) = G$. Now $\Hom_{\sf{G}}(-,x)$ is the image of $x$ via the Yoneda embedding in the functor category $\Fun = \Fun(\sf{G}^\text{opp}, \Sets)$, and by the above corollary $\Aut_{\Fun}(\Hom_{\sf{G}}(-,x)) \cong G$. Now each such natural isomorphism induces a bijection of sets $\Hom_{\sf{G}}(x,x) \rightarrow \Hom_{\sf{G}}(x,x)$, in other words an element of $\Sym(G)$, and this is clearly a group homomorphism. It is also injective, since by Yoneda's lemma, a natural transformation $\Hom(-,x)\rightarrow \Hom(-,x)$ is completely determined by its value on $x$.
\end{proof}
\begin{definition}
	A functor is \textit{representable} if it is in the image of the Yoneda embedding. More precisely, a functor $F$ is representable if there exists an object $X$ such that $\Hom(-, X) \cong F$ as functors. If such an $X$ exists, we say that $X$ \textit{represents} $F$.
\end{definition}
Of course, a natural question to ask is whether every functor is representable. The answer is an emphatic \textbf{NO}, and as we will see in the next few sections, finding a (fine) moduli space is equivalent to finding a representative of a certain $\sf{Set}$-valued contravariant functor from $\Sch$. Such a representative is unique, by Proposition \ref{yoneda-up-to-iso}.

\section{Moduli Problems and Spaces}

We will begin with stating our definition:
\begin{definition}
	Let $\sf{Sch}/k$ be the category of schemes over a ground field $k$, and let $\sf{C}$ be a subcategory of $\sf{Sch}/k$ (in practice, this will often be limited to the category of (locally) noetherian schemes, the category of schemes of finite type over $k$, or the category of varieties over $k$). A \textit{moduli problem} is a contravariant functor $\mathcal{M}: \sf{C}\rightarrow \Sets$. An element of $\mathcal{M}(S)$ is known as an (equivalence class of) \textit{families over} $S$ and $\mathcal{M}(S)$ is the \textit{set of families} (up to equivalence) over $X$. For a morphism $f: T \rightarrow S$, the induced morphism $\mathcal{M}(f): \mathcal{M}(S) \rightarrow \mathcal{M}(T)$ is known as the \textit{pullback map}. If $Y$ is in the image of $\mathcal{M}(f)$, then we say $Y$ \textit{is obtained by pullback through} $f$. In the case $T = \Spec k$ and $f$ is a $k$-valued point which we will denote $p$, we will write $X_p\in \mathcal{M}(\Spec k)$ for $\mathcal{M}(p)(X)$, and we will call $X_p$ the \textit{fibre of $X$ over $p$}.
\end{definition}
This is obviously a very general definition, but in practice our moduli problems will have a certain \enquote{flavour} to them. This is probably best illustrated by an example:
\begin{example}\label{projective-space-problem}
	Consider the problem of classifying 1-dimensional quotient spaces of $V = k^n$. We now carefully define our moduli problem: Firstly, our na\"ive moduli problem is simply the set of surjective linear maps $k^{n+1}\rightarrow k$, where two such maps are equivalent if and only if they have the same kernel, or equivalently, $\varphi \sim \psi$ if and only if there is a fixed $\lambda \in k^*$ such that $\varphi(v) = \lambda \psi(v)$ for all $v\in k^n$.
	
	Now let $S$ be a scheme over $k$. We will define a family over $S$ to be a line bundle $\L$ equipped with a surjection $\O_S^{n+1} \rightarrow \L$. Two families are equivalent if and only if they have the same kernel. Now given a morphism $f:T \rightarrow S$ and a family $\L$ over $S$, we define the pullback to be $f^*(\L)$, and it is not difficult to check that this satisfies the required conditions. Thus we define \[\mathcal{M}(S):= \{\O_S^{n+1}\rightarrow \L\}/\sim \] and \[\mathcal{M}(f: T \rightarrow S) := (\L \mapsto f^*(\L)).\]
	Now observe that over $\Spec k$, the trivial rank $n$ vector bundle is exactly $k^n$, and hence the families over $\Spec k$ are exactly the 1-dimensional quotient spaces of $k^n$. 
\end{example}
This sets the stage for what most of our moduli problems look like. Firstly, we will state what our na\"ive moduli problem is. A family over $S$ will commonly be a morphism $X \rightarrow S$ (in this case, $\L$ is a line bundle), sometimes equipped with some extra structure, such as a coherent sheaf (in this case, a surjection $\O_S^{n+1}\rightarrow \L$), satisfying some conditions (usually including some sort of flatness), and the fibres $X_p$ in our definition are literally just the fibres of the morphism. Now over $\Spec k$ itself, the set of families will literally be our na\"ive moduli problem.
\begin{remark}
	This is not the usual definition for a moduli problem. A moduli problem is usually defined as the data of a family over $S$ for every $S$ and a way to pull them back, to which we associate the functor $$\mathcal{M}(S) := \{\text{families over }S\}/ \sim$$However, the author did not wish to define it this way, since using this definition it is not possible to answer the question \enquote{what is \textbf{not} a moduli problem?}. As a consequence though, we gain a \textbf{lot} of things we call \enquote{moduli problems}, which would otherwise not be the case. However, the author believes this is fine: it matters little if there are a lot of obsolete moduli problems, as we simply focus our attention on interesting moduli problems (in fact, the author believes this is also a theme in mathematics: there are a lot of uninteresting problems, but we focus on the interesting ones).
\end{remark}
Next, we define the notion of a coarse moduli space:
\begin{definition}\index{moduli space ! fine}
	Let $\mathcal{M}:\sf{C}\rightarrow \Sets$ be a moduli problem. A \textit{coarse moduli space} for $\cal{M}$ is a scheme $M$ in $\sf{C}$ equipped with a natural transformation $\eta: \mathcal{M} \rightarrow \Hom(-, M)$, known as a \textit{moduli transformation} such that the following conditions hold:
	\begin{enumerate}
		\item $\eta: \mathcal{M}(\Spec k) \rightarrow \Hom(\Spec k, M)$ is a bijection.
		\item If $N$ is another scheme and $\eta': \mathcal{M} \rightarrow \Hom(-,N)$ another natural transformation, there is a unique morphism $e: M \rightarrow N$ such that $$\eta'_S(X) = e \circ \eta_S(X)$$ for any scheme $S$ in $\sf{C}$ and family $X$ over $S$. Note that this means $M$ is unique.
	\end{enumerate}
\end{definition}
We unpack this definition a little. Firstly, the $k$-points of $M$ are in bijection with the na\"ive moduli problem, which is as required. The functorial condition is a little more interesting: let $S$ be a scheme in $\sf{C}$. For a $k$-point $p$, we have the following diagram:
\begin{equation*}
	\begin{tikzcd}[row sep = huge]
		\mathcal{M}(S) \arrow[r, "\eta"] \arrow[d, "\mathcal{M}(p)"] & \Hom(S, M) \arrow[d, "p^*"]\\
		\mathcal{M}(\Spec k) \arrow[r, "\eta"] & \Hom(\Spec k, M)
	\end{tikzcd}
\end{equation*}
Let $X$ be a family over $S$. Then we get a morphism $\eta_S(X)\in \Hom(S,M)$; call this $f$. The commutativity of the diagram tells us that the fibre $X_p$ over $p\in S(k)$ is equal to the object in $\mathcal{M}(\Spec k)$ corresponding to the point $f(p)\in M(k)$. The second condition then tells us the $M$ is initial with respect to this property.

Of course, a natural question to ask is whether or not either of the two conditions above are obsolete. The answer is no, and we will see why in various contexts.
\begin{example}\label{projective-space-solution}
	Let us now show that $\P^n$ is the coarse moduli space of Example \ref{projective-space-problem}. Let $S$ be a scheme, and let $\O_S^{n+1}\rightarrow \L$ be a family, so in other words the images of $e_i\in \Gamma(S,\O_S)$ generate $\L$. We define $\eta_S(\L)$ as follows: let $U = \Spec A$ be a sufficiently small open affine subset of $S$, specifically one such that $\L|_{U}\cong \O_S|_{U}$. Then the images of $e_i$ in $A$ via $\L$ generate $A$. Now let $U_i = \Spec A[e_i^{-1}]$ be the corresponding open affine subset, and we define a morphism $U_i \rightarrow \P^n$ by composing the map $U_i\rightarrow \Spec k[x_0/x_i,...,x_n/x_i]\cong \A^n$ associated to the ring homomorphism $x_j/x_i\mapsto e_j/e_i\in A$ with the inclusion $\A^n \rightarrow \P^n$. It is not hard to check that this glues, and since the $e_i$ generate $A$, it follows that the $U_i$ covers $U$. Hence we have a morphism $U \rightarrow \P^n$. Now covering $X$ with these open affines, it is clear that this glues on overlaps, and hence we have a morphism $\eta_S(\L):X \rightarrow \P^n$. The key property of $\eta_S(\L)$ is that this is the unique morphism $X \rightarrow \P^n$ such that $\eta_S(\L)^*(\O_{\P^n}^{n+1}\rightarrow \O_{\P^n}(1))$ is equal to our family; or more precisely the following diagram commutes: 
	\begin{equation*}
		\begin{tikzcd}[row sep = huge]
			\O_{S}^{n+1}\arrow[r, "="] \arrow[rd]& \eta_S(\L)^*(\O_{\P^n}^{n+1}) \arrow[d]\\
			& \eta_S(\L)^*(\O_{\P^n}(1))
		\end{tikzcd}
	\end{equation*}
	and the kernel of the diagonal morphism above is equal to $\ker(\O_S^{n+1}\rightarrow \L)$. This is shown in the proof of \autocite[II Theorem 7.1]{Hart}.
	
%	Then the images of $e_i\in H^0(S,\O_S^{n+1})$ for $0 \leq i \leq n$ give rise to $n+1$ global sections $s_0,...,s_n$ of $\L$ which generate it. By \autocite[II Theorem 7.1]{Hart}, the choice of $\L$ and these sections, corresponding to the $n+1$ coordinates on $\P^n$, uniquely define a morphism $\varphi:S \rightarrow \P^n$ such that $\L \cong \varphi^*(\O_{\P^n}(1))$, and $s_i \mapsto x_i$ under this isomorphism. It is not hard to check that this does not depend on the equivalence class of the family. We define the moduli transformation $\eta$ to be $(\O_S^{n+1}\rightarrow \L) \mapsto \varphi$. Now given a morphism $f: T \rightarrow S$, we have to check that $\eta$ respects $f$. To see this, observe that, under $f$, the sections $s_i$ pull back to global sections $t_i$ on $T$. Now $(f\circ \varphi)^*(\O_{\P^n}(1)) = f^*(\L)$, and observe $x_i \mapsto s_i \mapsto t_i$, and since $\eta(f^*(\L))$ is the unique morphism $T \rightarrow \P^n$ such that $(\eta(f^*(\L)))^*(\O_{\P^n}(1)) = f^*(\L)$ and $x_i \mapsto t_i$, it follows $\eta(f^*(\L)) = f\circ \varphi$ as desired. 
	
	We now have to check that $\eta_{\Spec k}$ is bijective. Firstly, we show that it is surjective; so let $p = [p_0:...:p_n]$ be a $k$-point, and assume without loss of generality $p_0 \neq 0$. Then $p$ factors through the open affine subset $\Spec k[x_1/x_0,..., x_n/x_0]$ as the dual of the $k$-algebra homomorphism $k[x_1/x_0,..., x_n/x_0]\rightarrow k$ sending $x_i/x_0\mapsto p_i/p_0.$ Then it is clear from the definition of $\eta$ in the above paragraph that the family $L: e_i \mapsto p_i/p_0$ satisfies $\eta(L) = p$. This proves that $\eta_{\Spec k}$ is surjective. To prove that it is injective, suppose two families $L: e_i \mapsto p_i$ and $L': e_i \mapsto p_i'$ satisfy \[ [p_0:...:p_n] = \eta_{\Spec k}(L) =\eta_{\Spec k}(L') = [p_0':...:p_n'].\] Then $p_i p_j' = p_i' p_j$ for all $i,j$, hence $\ker L = \ker L'$, so in particular $L$ and $L'$ are equivalent, as desired. 
	
%	We claim the family $L:e_i \mapsto p_i$ over $\Spec k$ satisfies $\eta(L) = p$. Indeed, denoting the local ring of $\O_{\P^n}(1)$ at $p$ by $\O_{\P^n, p}(1)$, we obtain \[p^*(\O_{\P^n}(1)) = \O_{\P^n, p}(1)\otimes_{\O_{\P^n, p}} k = x_0k[\frac{x_1}{x_0},...,\frac{x_n}{x_0}]_{\sum\langle \frac{x_i}{x_0} - \frac{p_i}{p_0} \rangle} \otimes_{k[\frac{x_n}{x_0},...,\frac{x_1}{x_0}]_{\sum\langle \frac{x_i}{x_0} - \frac{p_i}{p_0} \rangle}} k \] but the $\O_{\P^n, p}$-module structure of $k$ is given by $x_i/x_0 \mapsto p_i/p_0$, and thus asserting $x_0 \mapsto p_0$, we fix a unique isomorphism such that $x_i \mapsto p_i$, and thus the family $\eta(L) = p$ as desired. Also note that by the calculation, this inverse of $p$ is unique, and hence $\eta_{\Spec k}$ is unique as desired.

	Finally, we show the universal property, starting with existence. Let $N$ be another scheme over $k$ and $\eta':\mathcal{M}\rightarrow \Hom(-,N)$ another natural transformation. Applying to $\P^n$, we obtain a map $\eta': \mathcal{M}(\P^n)\rightarrow \Hom(\P^n, N)$, and the natural family  $\O_{\P^n}^{n+1} \rightarrow \O_{\P^n}(1)$ induces a map $\eta'(\O_{\P^n}^{n+1} \rightarrow \O_{\P^n}(1)): \P^n \rightarrow N$. We claim $e := \eta'(\O_{\P^n}^{n+1} \rightarrow \O_{\P^n}(1))$ satisfies the desired property. Indeed, let $S$ be any scheme, and chasing $\O_S^{n+1}\rightarrow\L$ a family. Then $f = \eta(\O_S^{n+1}\rightarrow\L):S \rightarrow \P^n$ satisfies $f^*(\O_{\P^n}^{n+1} \rightarrow \O_{\P^n}(1)) =\O_S^{n+1}\rightarrow\L $. The following diagram commutes: 
	\begin{equation*}
		\begin{tikzcd}[row sep = huge]
			\mathcal{M}(\P^n) \arrow[r, "f^*"] \arrow[d, "\eta'"]&\mathcal{M}(S) \arrow[d, "\eta'"]\\
			\Hom(\P^n,N) \arrow[r, "f^*"] & \Hom(S,N)
		\end{tikzcd}
	\end{equation*}
	and $\O_{\P^n}^{n+1} \rightarrow \O_{\P^n}(1)$ in the above diagram, we find \[ \eta'(\O_S^{n+1}\rightarrow\L ) = e\circ f = e \circ \eta(\O_S^{n+1}\rightarrow\L), \] as desired. Uniqueness is immediate, since any such $e$ satisfies $e\circ \id = \eta'(\O_{\P^n}^{n+1} \rightarrow \O_{\P^n}(1))$. 
	
	This is now a good time to show that the second condition in the definition of a coarse moduli space is not obsolete. To this end, we take inspiration from \autocite[p. 4]{HarrisMorrison} and consider the moduli problem of one-dimensional subspaces of $k^2$, (which, if we consider the quotient of such a one-dimensional subspace, is just the moduli problem above for $n = 1$), and suppose $k$ is algebraically closed. As we saw above, the moduli space for this problem is $\P^1$. However, observe that there is a natural map from $\P^1 = \Proj k[x_0, x_1]$ to the cuspidal cubic $Y = \Proj k[x,y,z]/y^2z = x^3$, dual to the ring homomorphism $x\mapsto x_0^2x_1, y\mapsto x_0^3, z\mapsto x_1^3$ (on the level of $k$-points this is just $[p:q]\mapsto [p^2q: p^3:q^3]$). In particular, this map is bijective on $k$-points (in fact, a homeomorphism), and so composing the $\eta$ above with $\P^1\rightarrow Y$, we have a natural transformation $\mathcal{M}\rightarrow \Hom(-, Y)$ such that the map $\mathcal{M}(\Spec k)\rightarrow Y(k)$ is bijective. However, the cuspidal cubic is not the coarse moduli space, because $\P^1$ is, and coarse moduli spaces are unique up to isomorphism.
\end{example}

In fact, $\P^n$ satisfies a stronger condition above, in that every family is a unique pullback of $\O_{\P^n}^{n+1} \rightarrow \O_{\P^n}(1)$. This is formalised as follows:
\begin{definition}\index{moduli space ! fine}
	Let $\cal{M}$ be a moduli problem. A scheme $M$ is a \textit{fine moduli space} for $\cal{M}$ if $M$ represents $\cal{M}$. 
\end{definition}
Note that such an $M$ is unique, by Proposition \ref{yoneda-up-to-iso}. Also, as expected, a fine moduli space is also a coarse one:
\begin{proposition}
	If $M$ is a fine moduli space for $\cal{M}$, then for any representation $\eta: \cal{M} \rightarrow \Hom(-,M)$, the pair $(M, \eta)$ is a coarse moduli space.
\end{proposition}
\begin{proof}
	The first condition in the definition is satisfied automatically. Now let $N$ be another scheme and $\eta': \mathcal{M} \rightarrow \Hom(-,N)$ another natural transformation. Composing with $\eta^{-1}$, we get a natural transformation $\Hom(-,M) \rightarrow \Hom(-,N)$. Since the Yoneda embedding is fully faithful (Corollary \ref{fully-faithful-yoneda}), this is induced by a unique $e: M \rightarrow N$.
\end{proof}
\begin{definition}
	Let $\cal{M}$ be a moduli problem with coarse moduli space $M$ and moduli transformation $\eta$. A \textit{tautological family} is a family $X\in \mathcal{M}(M)$ such that for every $p\in M(k)$, we have $X_p = \eta_{\Spec k}^{-1}(p)$. If $M$ is a fine moduli space, the family $\eta_M^{-1}(\id)$ is known as the \textit{universal family}.
\end{definition}
\begin{proposition}
	Let $\mf{X}$ denote the universal family. Then it satisfies the following universal property: if $Y$ is a family over $S$ then there exists a unique morphism $f:S \rightarrow M$ such that $Y = \mathcal{M}(f)(\mf{X})$. In particular, taking $S = \Spec k$, we see that $\mf{X}$ is tautological.
\end{proposition}
\begin{proof}
	Since the moduli transformation $\eta$ is an isomorphism and hence $\eta_S$ is bijective, we see that $Y$ corresponds uniquely to a morphism $f: S \rightarrow M$. The following diagram commutes:
	\begin{equation*}
		\begin{tikzcd}[row sep = huge]
			\Hom(M,M) \arrow[r, "f^*"] \arrow[d, "\eta"] & \Hom(S,M) \arrow[d, "\eta"]\\
			\mathcal{M}(M) \arrow[r] & \mathcal{M}(S)
		\end{tikzcd}
	\end{equation*}
	The result then follows by chasing $\id\in \Hom(M,M)$ in the diagram.
\end{proof}
In fact, it is not hard to show the converse is true: if $\mf{X}$ is a family over a coarse moduli space $M$ such that every family is pulled back from $\mf{X}$ in a unique way, then $M$ is fine and $\mf{X}$ is universal.
\begin{example}
	The family $\O_{\P^n}^{n+1}\rightarrow \O_{\P^n}(1)$ is universal for Example \ref{projective-space-solution}, as every family is obtained uniquely by pullback from this family. In particular, $\P^n$ is fine.
\end{example}
\begin{example}\label{rational-curves-problem}
	Consider the moduli problem of classifying curves isomorphic to $\P^1$, up to isomorphism. Of course, there is only one, namely $\P^1$ itself. We define a family of genus 0 curves over $S$ to be a flat proper morphism $X \rightarrow S$ such that for any $p\in S(k)$ we have \[X_p := X\times_{S} \Spec k \cong \P^1 \] and if $f: T \rightarrow S$ is a morphism and $X\rightarrow S$ is a family, we define $f^*(X) = X\times_S T$. This defines our moduli problem. By \autocite[Proposition 25.1]{Deform} the coarse moduli space for this is just $\Spec k$, and it is easy to see that $\P^1 \rightarrow \Spec k$ is the tautological family. However, there is no universal family, since there exist nontrivial ruled surfaces (\autocite[V, Section 2]{Hart}).
\end{example} 
\begin{remark}
	Note that proving $\Spec k$ is the moduli space in the above example is actually nontrivial. Indeed, it is obvious that the moduli space, if it exists, is a one-point scheme and is thus necessarily equal to the spectrum of a local Artinian ring. However, in order to prove it is reduced, one must make sure that every family over a local Artinian ring is trivial; this follows from \autocite[Lemma 25.2]{Deform}.
\end{remark}
\section{Examples}
We will now study in detail some examples of moduli problems and spaces. Each example will illustrate interesting phenomena, and discuss concepts which will be used as motivation later on in the thesis.
\subsection{Conics in $\P^2$}
Our first example is the problem of conics in $\P^2 = \Proj k[x,y,z]$. In particular, we are not considering not just the conic, but the embedding in $\P^2$ as well (some texts will refer to a moduli space parameterising objects equipped with an embedding a \textit{parameter space}, but we will not make that distinction here). We will only consider schemes of finite type over $k$, and further we make the assumption that $k$ is algebraically closed. To formalise:
\begin{definition}
	A \textit{conic} is a closed subscheme of $\P^2$, cut out by a homogeneous ideal of degree 2 (in particular, we are allowing degenerate conics). Two conics are equivalent if and only if they are equal as subschemes of $\P^2$. Now let $S$ be a scheme of finite type over $k$. We define a \textit{family of conics} over $S$ to be a closed subscheme $X \subseteq \P^2 \times S$, flat over $S$ via the projection, whose scheme-theoretic fibres are conics in $\P^2$. Two families are \textit{equivalent} if they are equal as subschemes of $\P^2 \times S$. Now let $f:T \rightarrow S$ be a morphism of finite type, suppose $S$ and $T$ are of finite type over $k$ and let $X \rightarrow S$ be a family over $S$. We define the \textit{pullback} of $X$ along $f$ is the fibred product of the following diagram:
	\begin{equation*}
		\begin{tikzcd}[row sep = huge]
			f^*(X):= X \times_S T \arrow[r] \arrow[d]& X \arrow[d]\\
			T \arrow[r, "f"] & S
		\end{tikzcd}
	\end{equation*}
	The \textit{moduli problem of conics in $\P^2$} is the functor $\cal{M}$ defined by \[\cal{M}(S) = \{\text{families over }S\} \] and $\cal{M}$ maps a morphism $f:T \rightarrow S$ to $f^*: \mathcal{M}(S) \rightarrow \mathcal{M}(T)$.
\end{definition}

The key theorem of this section is:
\begin{theorem}\label{conic-universal-family-theorem}
	The scheme $\P^5 = \Proj[a_0,...,a_5]$ is a fine moduli space for the above moduli problem, and the family $\mf{X}\subseteq \P^2 \times \P^5$ cut out by the polynomial $a_0x^2 +a_1 xy+...+a_5 zx$ is the universal family.
\end{theorem}

Following the approach outlined in \autocite[Ex. 1.1]{Deform}, we will prove this after some lemmas. This approach works for general degree $d$ curves and $\P^{\binom{d+2}{2}-1}$ in place of $\P^5$, but for concreteness we will work with $d = 2$. To begin, we have the following:
\begin{lemma}
	Let $S$ be a scheme of finite type over $k$, and let $X\subseteq \P^2\times S$ be a family over $S$. Then $S$ can be covered by open affines $\{U=\Spec A\}$ such that the restricted family $X|_U\subseteq \P^2_A := \Proj A[x,y,z]$ is cut out by a single homogeneous polynomial (that is, the homogeneous ideal corresponding to $X|_U$ as a closed subscheme of $\P^2_A$ is principal), necessarily of degree 2.
\end{lemma}
\begin{proof}
	Write $\cal{I}$ for the sheaf of ideals of $X$, which is coherent since $S$ is of finite type over $k$. We have the following short exact sequence of sheaves on $\P^2\times S$:
	\begin{equation}\label{ses-conics-family}
		0 \rightarrow \mathcal{I} \rightarrow \O_{\P^2\times S} \rightarrow \Ox \rightarrow 0,
	\end{equation}
	where $\Ox$ is considered an $\O_{\P^2\times S}$-module. Let $p$ be a $k$-point of $S$ and let $\O_{S,p}$ be the local ring of $p$, with maximal ideal $\mf{m}_p$. Pulling back (\ref{ses-conics-family}) along the morphism $\Spec \O_{S,p}\rightarrow S$ and taking the associated graded objects, we have the following sequence of graded $\O_{S,p}[x,y,z]$-modules 
	\begin{equation}\label{ses-conics-local-ring}
		0 \rightarrow I_p \rightarrow \O_{S,p}[x,y,z]\rightarrow \Gamma_*(\O_{X})\otimes \O_{S,p}\rightarrow 0,
	\end{equation}
	 which is exact because $\Spec \O_{S,p}\rightarrow S$ is just localisation on the level of rings. Now since $X$ is flat over $S$, it follows $\Gamma_*(\O_{X})\otimes \O_{S,p}$ is flat over $\O_{S,p}$. In particular, we have  \[\Tor_1^{\O_{S,p}}(\Gamma_*(\O_{X})\otimes \O_{S,p},\O_{S,p}/\mf{m}_p) = 0,\] and so annihilating $\mf{m}_p$, the following sequence of graded $\O_{S,p}/\mf{m}_p[x,y,z] \cong k[x,y,z] $-modules is also exact: \[ 0 \rightarrow I_p \otimes k \rightarrow k[x,y,z] \rightarrow \Gamma_*(\O_{X_p})\rightarrow 0, \] where $X_p$ is the fibre over $p$. Since $I_p\otimes k$ is generated by its degree 2 component, $I_p$ must be too, and since $S$ is of finite type over $k$, it follows $I_p$ is finitely generated as an $\O_{S,p}[x,y,z]$-module, and hence $I_{p, \deg 2}$ is finitely generated as a $\O_{S,p}$-module. Finally, since the $\O_{S,p}/\mf{m}_p$-module $I_{p, \deg 2} \otimes k = I_{p, \deg 2} / \mathfrak{m}_p I_{p,\deg 2}$ is generated by a single element, it follows by Nakayama's lemma that $I_{p, \deg 2}$, as an $\O_{S,p}$-module is also generated by a single polynomial of degree 2.
	
	Now let $U' = \Spec A'\subseteq S$ be an open affine subset containing $p$, so that $\mf{m}_p$ may be considered a maximal ideal of $A$ and $\O_{S,p} = A'_{\mf{m}_p}$. Then, analogous to (\ref{ses-conics-local-ring}), we have the following short exact sequence of graded $A'[x,y,z]$-modules: 
	\begin{equation}\label{ses-conics-local-family}
		0 \rightarrow I_{A'} \rightarrow A'[x,y,z]\rightarrow \Gamma_*(\O_{X|_{U'}})\rightarrow 0,
	\end{equation}
	Now as established, $I_p$ is generated by a polynomial of the form \[f=\frac{s_0}{t_0}x^2 + \frac{s_1}{t_1}xy +...+ \frac{s_5}{t_5} zx, \] where $s_i, t_i\in A'$ and $\prod t_i \notin \mf{m}_p$, and hence by the universal property of localisation, the pullback of (\ref{ses-conics-local-ring}) from (\ref{ses-conics-local-family}) along $A' \rightarrow \O_{S,p}$ factors uniquely through $A' \rightarrow A = A'[\prod t_i^{-1}]$ and it is clear that $I_A$ is generated by $f$, as desired.
\end{proof}
Now that we know that any family is locally cut out by a single polynomial, the plan of attack is clear: we map the coordinates of $\P^5$ to the coefficients of our polynomial. In order for this to be possible, we present the next result:
\begin{lemma}
	Let $A$ be a finitely generated $k$-algebra, and suppose $X\subseteq \P^2_A$ is a flat family over $A$ cut out by $f = s_0x^2 +...+s_5 zx\in A[x,y,z]_{\deg 2}$. Then $s_0,...,s_5$ generate $A$.
\end{lemma}
\begin{proof}
	Observe that for every $d \geq 0$ the $A$-module $(A[x,y,z]/f)_{\deg d}$ must be flat. In particular, the map \[I \otimes (A[x,y,z]/f)_{\deg 2} \rightarrow (A[x,y,z]/f)_{\deg 2} \]is injective, where $I = \langle s_0,...,s_n \rangle$. This means $s_0\otimes x^2+...+s_5\otimes zx = 0$ in $I\otimes(A[x,y,z]/f)_{\deg 2}$, or in other words there is some $\lambda\in I$ such that $(1-\lambda)f = 0$ in $A[x,y,z]$. Now if $I \neq A$, then it is contained in some maximal ideal, say $\mf{m}$, and localising $A$ at $\mf{m}$, we deduce $(1-\lambda)f = 0$ in $A_\mf{m}$. But $1-\lambda$ is a unit in $A_\mf{m}$ and hence $f = 0$ in $A_\mf{m}$, which is absurd. Hence $I = A$.
\end{proof}

In particular, the $s_i\in A$ may be considered global sections of $\Spec A$, which generate the structure sheaf, and hence by Example \ref{projective-space-solution}, this corresponds uniquely to a morphism $\Spec A \rightarrow \P^5$ such that $\O_{\P^5}(1)$ pulls back to $\O_{\Spec A}$, and $a_i\in H^0(\P^5, \O_{\P^5}(1))$ pull back to $s_i$. It is then clear that the family $\mf{X}$ pulls back to the family $\Proj A[x,y,z]/f$. 
\begin{proof}[Proof of Theorem \ref{conic-universal-family-theorem}]
	Let $S$ be a scheme, and $X \subseteq \P^2\times S$ a family. Cover $X$ with open affine subsets $U_i = \Spec A_i$ such that $X|_{U_i} := X \times_{S} U_i\subseteq \P^2_{A_i}$ is cut out by a single polynomial. Then by our previous discussions, we have a collection $\varphi_i: U_i \rightarrow \P^n$ such that $\varphi_i^*(\mf{X}) = X|_{U_i}$, and hence it suffices to show that these glue. So we fix $U_i, U_j$. By \autocite[Proposition 5.3.1]{Vakil}, we can cover $U_i \cap U_j$ with affine open subsets $\{U_{ijk}\}$, which are distinguished open affine of both $U_i$ and $U_j$. Now fix some $U_{ijk} = \Spec A_{ijk}$, so that $A_i[g_1^{-1}] = A_{ijk} = A_j[g_2^{-1}]$. Now since the family $X|_{U_{ijk}}$ is also defined by a single polynomial (indeed, the same polynomial that defines $X|_{U_i}$ and $X|_{U_i}$), and the maps $\varphi_i:U_i \rightarrow \P^n$ and $\varphi_j:U_j \rightarrow \P^n$ both factor through $U_{ijk}\rightarrow \P^n$.  In particular, this means that the morphisms glue, as desired.
\end{proof}

A few comments on this problem. Note that we are crucially not defining conics up to abstract isomorphism; indeed consider the following family over $\A^1$: \[X = \Proj k[t,x,y,z]/ \langle tyz - x^2 \rangle \rightarrow \A^1 = \Spec k[t],\] where $k[t,x,y,z]$ is graded in $x,y,z$ (in other words, $t$ is degree 0). Flatness, which is equivalent to torsion-freeness, is obvious. For every nonzero $\lambda\in \A^1(k)$, the fibre $X_\lambda$ is a nondegenerate parabola defined by $\lambda yz = x^2$, and in particular is isomorphic to $\P^1$ via the 2-uple embedding followed by scaling. However, the fibre $X_0$ is the degenerate conic defined by $x^2 = 0$, which is clearly not isomorphic to the $\P^1$ (indeed, the former is not reduced but the latter is). This is an example of a \textit{jump phenomenon}, which is an obstruction to the existence of a moduli space: if a coarse moduli space $M$ exists, there would be a morphism $\A^1\rightarrow M$ which maps each nonzero $k$-point of $\A^1$ to some $s\in M(k)$, but maps $0$ to some $s' \neq s$. This is clearly not possible, and hence no coarse moduli space exists.

This is now a good time to show that the first condition in the definition of a coarse moduli space is also not obsolete, which we also take from from \autocite[p. 4]{HarrisMorrison}. Consider the moduli problem of reduced conics in $\P^2$, \textbf{up to isomorphism}, and make the further assumption $\Char k \neq 2$. As we know, the family $\Proj k[t,x,y,z]/xy = tz^2$ over $\Spec k[t]$ exhibits a jump phenomenon, and so there is no coarse moduli space. However, we claim that $M = \Spec k$, with the natural transformation $\eta$ sending a $k$-scheme $S$ to the morphism $S \rightarrow \Spec k$ is a natural transformation which satisfies property (ii) of the definition of a coarse moduli space. However, $\eta_{\Spec k}$ is not injective, since the nondegenerate conic and the union of two lines are both reduced conics (in fact, the only two), but $\Hom(\Spec k, \Spec k) = \{\id\}$. Since any scheme with property (ii) is unique, this gives another proof that this moduli problem has no coarse moduli space. 

So, let $N$ be a scheme equipped with a natural transformation $\eta'$ of our moduli problem into $\Hom(-, N)$. We want to show that there exists a unique $e: \Spec k \rightarrow N$ such that $\eta'(S) = e \circ \eta(S)$ for any relevant scheme $S$. Uniqueness is obvious, indeed, any such $e$ must also satisfy this property for families of nondegenerate conics, and since a family of nondegenerate conics is a family of nonsingular complete rational curves, uniqueness is guaranteed by Example \ref{rational-curves-problem}. So we now prove existence, that is, this $e$ above does satisfy the required property. Let $S$ be a scheme and let $X \subseteq \P^2\times S$ be a family. As above, let $U = \Spec A$ be a sufficiently small affine open subset of $S$, so that $X|_U$ is cut out by a single polynomial $f = s_0x^2 +...+s_5 zx\in A[x,y,z]_{\deg 2}$ (the difference is here we only care about $X$ up to isomorphism). Firstly, observe that the degenerate locus of $U$ (that is, the locus where very fibre is degenerate) is closed; indeed it is defined by the vanishing of the determinant of the following matrix:
\begin{equation*}
	Q = \begin{pmatrix}
		s_0 & \frac{s_1}{2} & \frac{s_5}{2} \\
		\frac{s_1}{2} & s_2 & \frac{s_3}{2} \\
		\frac{s_5}{2} & \frac{s_3}{2} & s_4
	\end{pmatrix}.
\end{equation*}
In particular, the nondegenerate locus of $U$, call it $U^{\flat}$ is open and {\color{red} (assuming without loss of generality $U$ is irreducible,)} dense, and taking the union across all such $U$, it follows that the nondegenerate locus of $S$, say $S^{\flat}$ is dense. Now by Example \ref{rational-curves-problem}, we have $\eta'(X|_{S^{\flat}}) = e \circ \eta(X|_{S^{\flat}})$, and since $S^{\flat}$ is dense and the image of $S^{\flat}$ in $N$ is a closed point, and moreover $\eta'(X|_{S^\flat}) = \eta'(X)|_{S^\flat}$ it follows that $\eta'(X) = e \circ \eta(X)$ too, as desired.
\\\\
\par{}We conclude with a remark. Observe that a family of conics (in the original problem of up to embedding) may alternatively thought of as a family of quotients (or equivalently subsheaves) of $\O_{\P^2}$. Indeed, a conic may be identified with its coherent sheaf of ideals $\cal{I}\subseteq \O_{\P^2}$. What distinguishes conics (or indeed degree $d$ curves for any $d > 0$), is their \textit{Hilbert polynomial}, a concept which will be discussed in Chapter 3. In general, given any projective variety $X$, a coherent sheaf $\F$ on $X$ and a numerical polynomial $P\in \Q[z]$ (that is, $P(n)\in \Z$ for all $n\in \Z$), there exists a fine moduli space, known as the \textit{Quot scheme} of $\F$, often denoted $\Quot_X^P(\F)$, parameterising quotients of $\F$ with Hilbert polynomial $P$. If $\F = \Ox$, then the Quot scheme is called a \textit{Hilbert scheme}; and in particular we have proven that the Hilbert scheme of $\P^2$ with Hilbert polynomial $P = 2z+1\in \Q[z]$ is $\P^5$.
\subsection{The Grassmannian}
We saw a brief glimpse of the Grassmannian already; indeed $\P^n$ is a special case of it. We will now study the subject in greater generality. Fix $0 < m \leq n$. Our na\"ive moduli problem is the set of surjections $k^n \rightarrow k^m$, up to equality of kernels. A family over $S$ is simply a vector bundle surjection $\O_S^{n}\rightarrow \F$ where $\F$ is of rank $n$.
\subsection{Elliptic Curves}

\indent Recall our general method in the previous example: we found a candidate space and a candidate universal family $\mathfrak{X} \rightarrow M$, showed that for a general $X \rightarrow S$, there is locally a unique morphism for an open subscheme $U \subseteq S$ such that $X\times_S U$ is pulled back from $\mathfrak{X}$, and finally we showed that these glue. This approach is illustrative of a typical approach for constructing moduli spaces, with some simplifications of course. The first and most obvious is that a moduli space need not be fine, and thus finding a candidate universal family is not always possible. What often instead happens is that we look for a \textbf{locally} universal family, that is, an overparameterised (i.e. there are repeated elements) family $\mathfrak{X} \rightarrow T$ with the property that for a general $X \rightarrow S$, there is an open cover $\{U_i\}$ of $S$ and for each $U_i$ a (not necessarily unique) morphism $\varphi_i: U_i \rightarrow T$ such that $X\times_S U_i$ is the pullback of $\mf{X}$ via $\varphi_i$. One then needs to find a way to contract the isomorphic fibres of $T$, which then defines a coarse moduli space. We will illustrate this technique now, in the context of elliptic curves. Throughout this section, we fix an algebraically closed ground field $k$ of characteristic neither $2$ nor 3. 

\begin{definition}
	An \textit{elliptic curve} is a complete nonsingular curve $X$ over $k$ of genus 1, equipped with a distinguished point $p_0\in X(k)$. A \textit{family of elliptic curves} over a scheme $S$ of finite type over $k$ is a scheme $X$ equipped with a flat morphism $X \rightarrow S$ and a section $s: S \rightarrow X$ such that for any $p\in S(k)$, the fibre $X_p$ is genus 1 curve, which is an elliptic curve with distinguished point $p^*s: \Spec k \rightarrow X_p$. Two families over $S$ are equivalent if they are isomorphic as $S$-schemes. It is clear how families pull back, and so we have the moduli problem of elliptic curves, which we will denote $\cal{M}_{1,1}$. 
\end{definition}
A detailed study of elliptic curves will take us too far afield, so we will focus solely on the study of their moduli space, and for that all we need to know is that for any family $(X \rightarrow S, s)$, there exists an open affine cover $\{U_i = \Spec A_i\}$ such that $X|_{U_i}$ can be embedded inside $\P^2_{A_i}$ with an equation of the form $y^2z = x^3+axz^2+bz^3$ for $a,b \in A_i$, and $4a^3+27b^2\neq 0$, and $s$ is the constant section $[0:1:0]\in X|_{U_i}\subseteq \P^2_{A_i}$ (see \autocite[p. 47]{Script}). In particular, the family over $k[a,b, (4a^3+27b^2)^{-1}]$ defined by the above equation, call it $X_0$, is a locally universal family.

The next question to ask is which fibres of ${X_0}$ are isomorphic as elliptic curves. It turns out that after an elementary (but tedious) calculation, two curves $ \Proj k[x,y,z]/y^2z = x^3+p_1xz^2+q_1z^3$ and $ \Proj k[x,y,z]/ y^2z = x^3+p_2xz^2+q_2z^3$ are isomorphic if and only if there is some $u\in k^*$ such that $p_1 = u^4 p_2$ and $q_1 = u^6 q_2$ (\autocite[III, Table 1.2]{Silverman}). This brings us to the following definition:
\begin{definition}
	Let $X =  \Proj k[x,y,z]/y^2z = x^3+pxz^2+qz^3$ be an elliptic curve. The \textit{j-invariant} of $X$ is the quantity \[j = 1728\frac{4p^3}{4p^3+27q^2}. \] Note that this only depends on the isomorphism class of $X$.
\end{definition}
\begin{proposition}
	Two elliptic curves are isomorphic if and only if they have the same $j$-invariant.
\end{proposition}
\begin{proof}
	We follow the proof in \autocite[pp. 51-52]{Silverman}. The \enquote{only if} follows from the above discussion. Conversely, let $X$ and $Y$ be elliptic curves given by the respective equations $y^2z = x^3+p_1xz^2+q_1z^3$ and $y^2z = x^3+p_2xz^2+q_2z^3$, let $S$ and $T$ be their respective homogeneous coordinate rings, and suppose they have the same $j$-invariant, that is, 
	\begin{equation*}
		4p_1^3(4p_2^3+27q_2^2) = 4p_2^3(4p_1^3+27q_1^2),
	\end{equation*}
	and rearranging we find \[p_1^3q_2^2 = p_2^3 q_1^2. \] Here we have a dichotomy: if $p_1= 0$, then $j = 0$, $q_1 \neq 0$, $p_2 = 0$ and $q_2 \neq 0$. One then finds that the graded ring isomomorphism $T \rightarrow S$ given by 
	\begin{equation}\label{same-j-1}
		x\mapsto (\frac{q_2}{q_1})^{1/3} x, \; y\mapsto (\frac{q_2}{q_1})^{1/2} y
	\end{equation}
	induces an isomorphism $X \rightarrow Y$. On the other hand, if $p_1 \neq 0$, then $j \neq 0$ and $p_2 \neq 0$ too. One then finds that the isomorphism $T \rightarrow S$ given by
	\begin{equation}\label{same-j-2}
		x\mapsto (\frac{p_2}{p_1})^{\frac{1}{2}}x, \; y\mapsto (\frac{p_2}{p_1})^{\frac{3}{4}}y
	\end{equation}
	induces an isomorphism $X \rightarrow Y$, as desired.
\end{proof}
In particular, there is a $k^*$ action on $\Spec k[a,b, (4a^3+27b^2)^{-1}]$ via automorphisms, such that the the orbit of a $k$-point $p$ are exactly the points $q$ whose fibre is isomorphic to the fibre at $p$. which dually induces an action on the coordinate ring $k[a,b, (4a^3+27b^2)^{-1}]$ also via automorphisms, such that $u\in k^*$ sends $a$ to $u^3 a$ and $b$ to $u^4b$. 

To kill any false hope that may be brewing, we have the following result:
\begin{proposition}
	The moduli problem of elliptic curves does not have a fine moduli space.
\end{proposition}
\begin{proof}
	Consider the family $y^2z = x(x-z)(x-\lambda z)$ over $\Spec k[\lambda, \lambda^{-1}, (\lambda-1)^{-1}]$ (often called the \textit{$\lambda$-line}), and consider the automorphism of $k[\lambda, \lambda^{-1}, (\lambda-1)^{-1}]$ given by $\lambda \mapsto \lambda^{-1}$, whence $(\lambda-1)^{-1}\mapsto \lambda(1-\lambda)^{-1}$. 
\end{proof}


%The theory of elliptic curve endows us two moduli spaces; the moduli space of elliptic curves themselves, and the so-called \textit{Jacobian variety}, which can be defined for any complete nonsingular curve. However, we will show that on an elliptic curve, the Jacobian is simply the curve itself. So for this section, we fix a complete nonsingular curve $X$ over $k$.
%
%We recall some definitions:
%\begin{definition}
%	The \textit{(geometric) genus} of $X$, denoted $g$, is defined to be $h^0(\omega_X):=\dim_k H^0(X, \omega_X)$, where $\omega_X$ is the canonical bundle (or in this case just the cotangent bundle).
%\end{definition}
%To make sense of it, observe that the cotangent bundle is just the bundle of algebraic 1-forms.



\chapter{Geometric Invariant Theory}
 Geometric invariant theory, or GIT developed by Mumford provides us a way of taking \enquote{nice} quotients of group actions in algebraic geometry. To see why this is may be desired, consider the following example:
\begin{example}\label{naive-quotient}
	Let $k$ be an algebraically closed field and let $k^*$, the multiplicative group of $k$ act on $k^2$ by $\lambda \cdot (p,q):= (\lambda^{-1}p, \lambda q)$. The orbits consist of the axes without the origin, the origin and for every $t \in k^*$ the curve $xy = t$. If we consider the orbit space, it resembles $k$, indeed each nonzero $t$ will represent the curve $xy = t$; however where the origin should be, we find three orbits, one of which is closed and two of which has closure equal to the union of the three orbits. 
\end{example}
Intuition thus tells us that the quotient space \enquote{should} be $k$ itself, but clearly something has gone wrong near the origin. GIT allows us to formalise this, and as we will see below, equipped with the formalism the GIT quotient is indeed $k$.

Being a vast and difficult subject, however, we do not have the time to develop the subject in detail. Our focus will be on linear actions (to be defined) on quasi-projective schemes of finite type over $k$. For a full account, see \autocite{GIT}. Our exposition follows the one found in \autocite{ModuliNotes} very closely.
\section{Algebraic Groups and Actions}
In this section, we will introduce the basics of algebraic groups. This essentially amounts to writing elementary group theory in the language of algebraic geometry!
\begin{definition}\index{algebraic group}
	An \textit{algebraic group} over $k$, also known as a \textit{group scheme} over $k$ is a scheme $G$ over $k$ equipped with a $k$-point $e: \Spec k \rightarrow G$, known as the \textit{identity element} and morphisms $\mu: G \times_k G \rightarrow G$ and $\iota: G \rightarrow G$ known as \textit{multiplication} and \textit{inversion} respectively such that the following three diagrams commute:
	\begin{enumerate}
		\item (Asssociativity) 
		\begin{equation*}
			\begin{tikzcd}[row sep = huge]
				G \times G \times G \arrow[r, "\id \times \mu"] \arrow[d, "\mu \times \id"]& G \times G \arrow[d, "\mu"]\\
				G \times G \arrow[r, "\mu"] &G
			\end{tikzcd}
		\end{equation*}
		\item (Identity) 
		\begin{equation*}
			\begin{tikzcd}[row sep = huge]
				\Spec k \times G \arrow[r, "e \times \id"] \arrow[rd, "\cong"]& G \times G \arrow[d, "\mu"] &\arrow[l, "\id \times e"] G\times \Spec k \arrow[ld, "\cong"] \\
				&G
			\end{tikzcd}
		\end{equation*} 
		\item (Inverse) 
		\begin{equation*}
			\begin{tikzcd}[row sep = huge]
				G \arrow[r, "\iota\times \id"] \arrow[d] & G \times G \arrow[d, "\mu"]& \arrow[l, "\id \times \iota"] G\arrow[d] \\
				\Spec k \arrow[r, "e"] & G & \Spec k \arrow[l, "e"]
			\end{tikzcd}
		\end{equation*}
	\end{enumerate}
	An algebraic group $G$ is \textit{affine} if $G$ is an affine scheme. 
\end{definition}
\begin{remark}
	There are subtle variations in the definition of algebraic group and group scheme from one source to another (indeed, they are usually different!). For example, algebraic groups are sometimes required to be of finite type over $k$, the definition of a group scheme may not require a base scheme, and sometimes algebraic groups are required to be varieties. However, since we will only ever deal with algebraic groups which are affine varieties, these subleties do not matter in this thesis; in fact the definition is given only for completeness!
\end{remark}
Of course, these resemble the usual group axioms written using commutative diagrams. To see more concretely the connection with group theory, we may think of an algebraic group as a functor from $\Sch/k$ to $\Gps$; indeed for any scheme $S$, the set $\Hom(S, G)$ has a group structure, with group operation \[(f: S \rightarrow G) \cdot (g: S \rightarrow G) := (\mu\circ(f,g)\circ\Delta: S\rightarrow G)\] where $\Delta$ is the diagonal map. The identity morphism is the composition  $S \rightarrow \Spec k \xrightarrow{e} G$ where the first map is the unique morphism $S \rightarrow \Spec k$, and the inverse of $f: S \rightarrow G$ is simply $\iota \circ f$. In particular, $G(k)$ itself is a group, and when we think of $G$ as a group, we usually think of $G(k)$.

Being a scheme, $G$ has more structure, for example the group operations induce co-operations of $k$-algebras. In fact, if $G$ is affine, we may completely work with $k$-algebras, since there is an equivalence of categories. In this case, we will call $H^0(G, \O_G)$ the \textit{associated co-group}.
\begin{example}
	Let $G = \Spec k[t, t^{-1}]$, whose $k$-points are of course is canonically identified with $k^*$. Now $k^*$ obviously has a canonical multiplication which makes it a group; we will now extend this to an algebraic group structure on $G$. Note that since a morphism of varieties is detemined by its $k$-points (\autocite[II Proposition 2.6]{Hart}), this extension is unique, if it exists. Define the co-multiplication 
	\begin{align*}
		\mu^\sharp: k[t, t^{-1}] &\rightarrow k[t, t^{-1}]\otimes k[t, t^{-1}]  \\
		t &\mapsto t\otimes t
	\end{align*}
Using the identification $ k[t, t^{-1}]\otimes k[t, t^{-1}] \cong k[t_1^{\pm1}, t_2^{\pm1}]$, we may write this in the more familiar way as $t\mapsto t_1t_2$. Taking $\Spec$, we get a morphism $G \times G \rightarrow G$, and we observe the induced map of $k$-points is simply $(p,q) \mapsto pq$ as expected. Then the co-identity is the $k$-algebra map $k[t, t^{-1}] \rightarrow k$ given by $t \mapsto 1$, and co-inversion is the endomorphism $t \mapsto t^{-1}$. We check the group axioms by checking the co-axioms on the co-group: note that \[(\id \otimes \mu^\sharp)\circ \mu^\sharp(t) = t\otimes t \otimes t = (\mu^\sharp \otimes \id)\circ \mu^\sharp(t) \] and extending algebraically this proves associativity. To check identity, we observe  \[ (e^\sharp\otimes \id)\circ\mu^\sharp(t) = 1 \otimes t \cong t\cong t \otimes 1 =  (\id \otimes e^\sharp)\circ \mu^*(t)\] as desired, where, by abuse of notation, $\cong$ denotes the image of $1\otimes t$ under the isomorphisms $k \otimes k[t^{\pm1}] \cong k[t^{\pm1}]\cong k[t^{\pm1}] \otimes k$. Finally, we check inversion: \[(\iota^\sharp \otimes \id)\circ\mu^\sharp(t) = 1 = e^\sharp(t)\] and similarly in the other direction. Thus we have our expected group axioms.
\end{example}
Henceforth, we will denote $\Spec k[t, t^{-1}]$ by $\mathbb{G}_m$ (note that $m$ stands for multiplication, not for any specific number).

Before stating our next example, a few comments about notation are in order. We will use $\GL_n(R)$ to denote the (abstract) group of $n$-by-$n$ matrices over $R$, and we will use $V$ to describe the vector space $k^n$. In particular, these are \textbf{not} schemes. We will use $\A^n$ to describe the variety/scheme $\Spec k[x_1,...,x_n]$, and we will use $\GL_n$ or $\GL_V$ to describe the variety/scheme, which is defined below.
\begin{example}
	We endow $\GL_n(k)$ with an algebraic group structure. Now $\GL_n(k)$ may be identified as the $k$-points of the affine scheme $\Spec k[x_{ij}, 1 \leq i,j \leq n, \det(x_{ij})^{-1}]$, which we will denote $\GL_n$. Co-multiplication is given by \[\mu^\sharp(x_{ij}) := \sum_{k = 1}^n x_{ik}\otimes x_{kj}\] or, once again, making the identification \[k[x_{ij}, 1 \leq i,j \leq n, \det(x_{ij})^{-1}] \otimes k[x_{ij}, 1 \leq i,j \leq n, \det(x_{ij})^{-1}] \cong k[x_{ij}, y_{ij}, \det(x_{ij})^{-1}, \det(y_{ij})^{-1}]\] this is just the familiar \[\mu^\sharp(x_{ij}) = \sum_{k = 1}^n x_{ik} y_{kj} \] Similarly, the co-identity is given by $e^\sharp(x_{ij}) := \delta_{ij}$. The co-inversion is a little difficult to write explicitly, but $\iota^\sharp(x_{ij})$ is the $i,j$-th entry of the $n$-by-$n$ matrix $(x_{k\ell})_{1 \leq k,\ell \leq n}$, which may be shown to be algebraic. Once again, we can check the group axioms for this by checking the co-group co-axioms:
	\begin{align*}
		(\mu^\sharp \otimes \id)(\mu^\sharp(x_{ij})) &= (\mu^\sharp \otimes \id)(\sum_k x_{ik} \otimes x_{kj}) \\
		&= \sum_k \sum_{\ell} x_{i\ell} \otimes x_{\ell k} \otimes x_{kj} \\
		&= \sum_k \sum_{\ell} x_{ik} \otimes x_{k \ell} \otimes x_{\ell j} \\
		&= (\id \otimes \mu^\sharp)(\sum_k x_{ik} \otimes x_{kj}) \\
		&= (\id \otimes \mu^\sharp)(\mu^\sharp(x_{ij}))
	\end{align*}
	proves associativity and \[(e^\sharp\otimes \id) (\mu^\sharp(x_{ij})) = (e^\sharp\otimes \id)(\sum_k x_{ik} \otimes x_{kj}) = \sum_k \delta_{ik}\otimes x_{kj} = \sum_i 1\otimes x_{ij} \cong x_{ij}\] and similarly proves identity. We will not prove inversion, but it follows from the properties of multiplying matrices. 
\end{example}
\begin{example}
	Let $G$ be a finite group. We will endow $G$ with a natural algebraic group structure. Write $n:= |G|$. By Corollary \ref{cayleys-theorem}, we may embed $G$ into $S_n$, the symmetric group on $n$ letters, and $S_n$, in turn, embeds into $\GL_n(k)$ via permutation matrices. We therefore may interpret $G$ as a closed subscheme of $\GL_n$, and the group axioms inherit from the group axioms in $\GL_n$.
\end{example}
Next we will discuss algebraic group actions. 
\begin{definition} \index{algebraic group! algebraic actions}
	Let $G$ be an algebraic group and $X$ a scheme over $k$. An \textit{action} of $G$ on $X$ is a morphism $\sigma: G \times X \rightarrow X$ such that the following diagrams commute: 
	\begin{enumerate}
		\item (Associativity)
		\begin{equation*}
			\begin{tikzcd}[row sep = huge]
				G \times G \times X \arrow[r, "\id_G \times \sigma"]\arrow[d, "\mu \times \id_X"] & G \times X \arrow[d, "\sigma"]\\
				G \times X \arrow[r, "\sigma"] & X
			\end{tikzcd}
		\end{equation*}
		\item (Identity)
		\begin{equation*}
			\begin{tikzcd}[row sep = huge]
				\Spec k \times X \arrow[r, "e \times \id_X"]  \arrow[d, "\cong"]&G \times X \arrow[ld, "\sigma"]\\
				X
			\end{tikzcd}
		\end{equation*}
	\end{enumerate}
\end{definition}
\begin{example}
	The most straightforward example is $G$ acting on itself via $\mu$. Associativity and identity follow directly from the corresponding group axioms.
\end{example}
Let $G$ act on $X$. We observe a few things: passing to $k$-points, the group $G(k)$ acts (as an abstract group, in the usual sense) on $X(k)$. Any $k$-point $g: \Spec k \rightarrow G$ induces an automorphism of $X$ given by $\sigma(g\times \id)$, which we will denote $\sigma_g$ (on the level of $X(k)$, this is simply multiplication by $g$). Similarly, any $k$-point $p: \Spec k \rightarrow X$ induces a morphism $G \rightarrow X$. 

Now assuming $G$ and $X$ are both affine, equal to $\Spec R$ and $\Spec A$ respectively the action induces a co-action homomorphism of $k$-algebras $A \rightarrow R \otimes A$. However, the group $G(k)$ also has an induced action (in the usual sense) on $A$: as previously mentioned, every $g\in G(k)$ induces an automorphism of $X$. However, by the equivalence of categories, this also induces a $k$-linear automorphism of $A$. We define it so that for any $g\in G(k)$ and $p\in X(k)$ we have $(g\cdot f)(p) = f(g^{-1}\cdot p)$, so that we end up with a group action.

\begin{example} \label{naive-quotient-scheme}
	Consider the action described in Example \ref{naive-quotient}. We will extend this to an algebraic group action of $\bb{G}_m$ on $\A^2$. The associated coordinate rings are $k[t^{\pm1}]$ and $k[x,y]$. We now define the co-action $\sigma^\sharp: k[x,y]\rightarrow k[t^{\pm1}]\otimes k[x,y]$ by $x \mapsto t^{-1}\otimes x$ and $y \mapsto t\otimes y$; it is not hard to check the axioms and to show that this induces the action in the aforementioned example. Now we compute the action of $\bb{G}_m(k)$ on $k[x,y]$: let $\lambda\in k^* \cong \bb{G}_m(k)$ be given. This induces the map $k[t^{\pm1}] \rightarrow k$ defined by $t\mapsto \lambda$, hence we have \[ \sigma_{\lambda}^\sharp(x) = \lambda^{-1}\otimes x = 1 \otimes (\lambda^{-1}x) \cong \lambda^{-1} x \] and \[ \sigma_{\lambda}^\sharp(y)= \lambda \otimes y = 1 \otimes \lambda y \cong \lambda y \] which is easily seen to be a group action.
\end{example}
\begin{example}\label{gl-action}
	We consider the natural action of $\GL_n(k)$ (the group of $n$-by-$n$ matrices) on $V = k^n$. We will lift this to an algebraic group action of $\GL_n$ on $\A^n$. The coordinate rings are $k[x_{ij}, \det(x_{ij})^{-1}]$ and $k[v_1,...,v_n]$. We define the co-action \[\sigma^\sharp: k[v_1,...,v_n] \rightarrow k[x_{ij}, \det(x_{ij})^{-1}]\otimes k[v_1,...,v_n] \] by \[\sigma^\sharp(v_i):= \sum_{j = 1}^n x_{ij} \otimes v_j\] and for a $k$-point $g = (g_{ij})\in \GL_n(k)$ (which is induced by $x_{ij}\mapsto g_{ij}$), we have the following automorphism $\sigma^\sharp_{g}$ of $k[v_1,...,v_n]$: \[\sigma^\sharp_g(v_i) = \sum_j g_{ij}\otimes v_j = \sum_j 1 \otimes g_{ij}v_j \cong \sum_j g_{ij}v_j  \] and in order for it to be a group action, we have to take the inverse of this map.
\end{example}
\begin{definition}
	Let $G$ and $H$ be algebraic groups. A \textit{homomorphism} of algebraic groups is a morphism of schemes $f:G \rightarrow H$ such that the following diagram commutes:
	\begin{equation*}
		\begin{tikzcd}[row sep = large]
			G \times G \arrow[r, "\mu_G"] \arrow[d, "f \times f"]&G \arrow[d, "f"] \\
			H \times H \arrow[r, "\mu_H"] & H
		\end{tikzcd}
	\end{equation*}
	A homomorphism $\rho: G \rightarrow \GL_n$ will be called a \textit{representation}. Composing this with the action on $\A^n$, we see that $\rho$ induces an action of $G$ on $\A^n$ and hence of $G(k)$ on $V = k^n$.
\end{definition}
\begin{example}
	Consider the representation $\rho: \bb{G}_m \rightarrow \GL_2$ induced by \[\rho^*(x_{ij}) = \delta_{ij}t^{2i-3} \] composing this with the natural action of $\GL_2$ on $\A^2$ described in Example \ref{gl-action} we obtain the action in Example \ref{naive-quotient-scheme}.
\end{example}
Note that the image of the induced morphism $k^* \rightarrow \GL_2(k)$ in the above representation is contained in the subgroup of diagonal matrices. This is no coincidence, as we will show below. This result is hugely important, and as we will soon see, representations of $\bb{G}_m$ play a huge role in our further discussions (for example, in our analysis of stability). For now, we end with the following theorem:
\begin{theorem}\label{torus-reductive}
	Let $\rho: \bb{G}_m \rightarrow \GL_n$ be a representation. Then there is a decomposition \[k^n =: V = \bigoplus_{i\in \Z} V_i \] where \[V_i:= \{v\in V \mid \lambda \cdot v = \lambda^i v \space, \forall \lambda \in k^*\} \]
\end{theorem}
\begin{proof}
	We follow the proof given in \autocite[p.17]{ModuliNotes}. Firstly, observe that $\rho$ induces a group homomorphism $\Hom(\Spec R, \bb{G}_m)\rightarrow \Hom(\Spec R, \GL_n)$ for every $k$-algebra $R$. In particular, letting $R = k[t^{\pm1}]$, we have a group homomorphism $\rho: \bb{G}_m(R)\rightarrow \GL_n(R)$, and hence $\rho(t)\in \GL_n(R)$. We interpret $\rho^* =\rho(t)$ as a linear map $\rho^*:V \rightarrow V \otimes k[t^{\pm1}]$. It can be shown (\autocite[22,23]{Waterhouse}) that the following diagrams commutes:
	\begin{equation}\label{comodule diagram}
		\begin{tikzcd}[row sep = huge]
			V \arrow[r, "\rho^*"] \arrow[d, "\rho^*"]&V \otimes k[t^{\pm1}] \arrow[d, "\id \times \mu^\sharp"]\\
			V \otimes k[t^{\pm1}] \arrow[r, "\rho^*\otimes \id"] & V \otimes k[t^{\pm1}]\otimes k[t^{\pm1}]
		\end{tikzcd}
	\end{equation}
	and
	\begin{equation}\label{comodule identity diagram}
		\begin{tikzcd}[row sep = huge]
			V \arrow[r, "\rho^*"] \arrow[d, "\cong"]& V \otimes k[t^{\pm1}] \arrow[ld, "\id \otimes e^\sharp"]\\
			V\otimes k
		\end{tikzcd}
	\end{equation}
	and it is clear that for any $\lambda\in \bb{G}_m(k) = k^*$ we have the following commutative diagram:
	\begin{equation*}
		\begin{tikzcd}[row sep = huge]
			V \arrow[r, "\rho^*"] \arrow[d, "\sigma_\lambda"] &V \otimes k[t^{\pm1}] \arrow[ld, "t\mapsto \lambda"] \\
			V
		\end{tikzcd}
	\end{equation*} 
	Now observe that there exist $f_i\in \End(V)$ for each $i\in \Z$ such that \[\rho^*(v) = \sum_{i\in \Z} f_i(v) \otimes t^i \] for every $v\in V$. We claim $f_i(v)\in V_i$. Indeed, by the commutativity of (\ref{comodule diagram}), we have \[\sum_{i\in \Z} f_i(v) \otimes t^i \otimes t^i = (\id \otimes \mu^\sharp)(\sum_{i\in \Z} f_i(v) \otimes t^i) = (\rho \times \id)(\sum_{i\in \Z} f_i(v) \otimes t^i) = \sum_{i\in \Z} \rho^*(f_i(v)) \otimes t^i\] and the claim follows since the $t^i$ are linearly independent. Next observe that  the commutativity of (\ref{comodule identity diagram}) implies \[v = \sum f_i(v) \] and hence we have the decomposition as desired. Finally, the observation that the $V_i$ intersect trivially pairwise implies the result.
\end{proof}
\section{Reductive Groups}
In this section, we will conduct a brief study into the theory of reductive groups, and in the sequel we will be exclusively looking at the action of reductive algebraic groups. The reason for this is that the reductive hypothesis provides us with a finitely-generated ring of invariants, and this will allow us to define the GIT quotient. So we begin with a definition:
%However, we will first define what a quotient is:
%\begin{definition}
%	Let $G$ be a reductive algebraic group acting on a variety $X$. A \textit{quotient} of this action is a morphism $\varphi: X \rightarrow Y$ such that $\varphi \circ \sigma = \varphi \circ \pi_X$ as morphisms $G\times X \rightarrow Y$. A quotient is \textit{categorical} if it satisfies the following universal property: given any other quotient $\psi: X \rightarrow Z$, there is a unique morphism $f: Y \rightarrow Z$ such that $\psi = f\circ \varphi$.
%\end{definition}

\begin{definition}
	Let $G$ be a group acting on a ring $A$ by ring automorphisms. The \textit{ring of invariants} of this action, denoted $A^G$ is the subring of elements fixed by $G$.
\end{definition}
\begin{definition}
	An algebraic group $G$ is \textit{reductive} over $k$ if, for every representation $\rho: G \rightarrow \GL_n$, we can decompose $V = k^n$ into irreducible $G(k)$-invariant subspaces.
\end{definition} 
\begin{example}
	If $G$ is a finite group, then it is reductive if $\operatorname{char} k$ does not divide $|G|$, by Maschke's Theorem.
\end{example}
\begin{example}
	As we saw in Theorem \ref{torus-reductive}, $\bb{G}_m$ is reductive.
\end{example}
\begin{example}
	In characteristic zero, many \enquote{familiar} groups (such as $\GL_n, \SL_n, \PGL_n$) {\color{red}are reductive}. 
\end{example}
\begin{remark}
	Our definition of reductive is more commonly known as \textit{linearly reductive}, the current definition of reductive is that radical of $G$ is a torus (isomorphic to $(\bb{G}_m)^r)$. However, {\color{red} in characteristic zero, they are equivalent}.
\end{remark}
Henceforth, we will assume $\operatorname{char} k = 0$. The key that makes this work is the following theorem:
\begin{theorem}\index{ring of invariants}
	Let $G$ be a reductive algebraic group acting on an affine scheme $X = \Spec A$ of finite type over $k$. Then the ring of invariants $A^G$ is a finitely generated $k$-algebra. 
\end{theorem}
\begin{proof}
	{\color{red}TODO }
\end{proof}
We are now in a position to construct the quotient. 
\section{The GIT Quotient}

In this section, we will construct the GIT quotient for the action of a reductive group $G$ on an affine or projective variety $X$. We begin with a general study of quotients, so we make the following definition:
\begin{definition}
	Let $G$ be a reductive group acting on $X$, and let $p\in X(k)$. The \textit{orbit} of $p$, denoted $G\cdot p$, is the set $\{g\cdot p \mid g\in G(k)\}\subseteq X(k)$. The \textit{stabiliser} of $p$, denoted $G_p$ is the fibred product $G \times_X \Spec k$, given by the following diagram:
	\begin{equation*}
		\begin{tikzcd}[row sep = huge]
			G \times_X \Spec k \arrow[d]\arrow[r] & \Spec k\arrow[d,"p"]  \\
			G \arrow[r, "\sigma(\id \times p)"]& X
		\end{tikzcd}
	\end{equation*}
	A \textit{quotient} of this action is a morphism $\varphi: X \rightarrow Y$ such that $\varphi \circ \sigma = \varphi \circ \pi_X$ as morphisms $G\times X \rightarrow Y$. A quotient is \textit{categorical} if it satisfies the following universal property: given any other quotient $\psi: X \rightarrow Z$, there is a unique morphism $f: Y \rightarrow Z$ such that $\psi = f\circ \varphi$.
\end{definition}

In fact, since $X$ is a variety, we can say more about $G\cdot p$. Firstly, observe that $X(k)$ has a natural topology. Now $G\cdot p$ is just the set-theoretic image of the morphism $\sigma(-, p): G \rightarrow X$ in $X(k)$, and by Chevalley's theorem is a constructible subset (i.e. a finite disjoint union of locally closed subsets) of $X(k)$. So we can write \[G \cdot p = \bigcup_{i = 1}^n (U_i \cap V_i), \] where $U_i$ is open and $V_i$ is closed, and we may assume without loss of generality $V_i = \overline{U_i \cap V_i}$, the closure of $U_i \cap V_i$. Since closure commutes with unions, it follows \[\overline{G\cdot p} = \bigcup V_i. \] Now observe that $U =( \bigcup U_i) \cap G \cdot p$ is a dense open subset of $\overline{G \cdot p}$ (because $\overline{U}\cap \overline{G \cdot p}$ necessarily contains each $\overline{U_i \cap V_i} = V_i$ and $\bigcup V_i = \overline{G \cdot p}$ itself is closed), and since \[G\cdot p = \bigcup_{g\in G(k)} g\cdot U \] it follows that $G \cdot p$ is open in its closure; that is it is itself locally closed. Since $k$ is algebraically closed, this means we may identify $G\cdot p$ as the set of $k$-points of a closed subset of an open subscheme, and equipping it with the reduced closed subscheme structure, we give $G\cdot p$ the natural structure of a scheme. By abuse of language, we will use the word \enquote{orbit} to denote both the set $G(k) \cdot p\subseteq X(k)$, and the scheme described above. It will either be clear from context, or unimportant which is meant.

\begin{proposition}
	For any $k$-point $p\in X(k)$, the morphism $\sigma(-, p): G \rightarrow G \cdot p$ is flat. In particular, we have \[\dim G = \dim G_p + \dim G\cdot p. \]
\end{proposition}
\begin{proof}
	\autocite[p. 19]{ModuliNotes}
\end{proof}
Now we are equipped to construct and study the affine GIT quotient. 
\subsection{Affine GIT Quotients}
Let $X$ be an affine variety and $G$ a reductive algebraic group acting on $X$. We make the following definition:
\begin{definition}
	A point $p\in X(k)$ is \textit{polystable} if $G\cdot p$ is closed. Furthermore $p$ is \textit{stable} if it is polystable, and $\dim G_p = 0$. An orbit is \textit{(poly)stable} if one (equivalently all) of its points is.
\end{definition}
As the name suggests, the stable points are the ones that are \enquote{nicest}, and we will shortly why that is the case. However, let us first look at this example:
\begin{example}\label{naive-quotient-orbits}
	Recall Example \ref{naive-quotient}. As remarked, the orbits consist of the axes without the origin, the origin and for every $\lambda\in \bb{G}_m(k)$ the hyperbola $xy = \lambda$. Clearly the \enquote{typical} orbits are the hyperbolas, which are closed, and clearly the stabiliser for each such point is trivial, and hence they are stable. The origin itself is polystable, but not stable, and the axes are neither stable nor polystable. We note a few things: firstly the stable points form an open subset. Next, observe that closure of the union of the two non-closed orbits are $G$-invariant, and their intersection contains a polystable orbit (the origin). 
\end{example}
In fact, the the situation described above is typical, and we will see how these properties of orbits correspond to points in the GIT quotient, which we will now define:
\begin{definition}\index{GIT quotient ! affine}
	Let $X$ be an affine variety, and $G$ a reductive algebraic group acting on $X$. The \textit{affine GIT quotient} is the map $\varphi^G:X\rightarrow \Spec A^G$ induced by the inclusion $A^G \rightarrow A$. We denote $\Spec A^G$ by $X\git G$.
\end{definition}
We will state without proof the key properties of the affine GIT quotient.
\begin{theorem}
	The affine GIT quotient is a quotient, and satisfies the following properties:
	\begin{enumerate}
		\item $\varphi^G$ is surjective.
		\item For any open subset $U\subseteq X \git G$, the map $\O_{X \git G}(U)\rightarrow \O_X((\varphi^G)^{-1}(U))$ is an isomorphism onto $\O_X((\varphi^G)^{-1}(U))^G$.
		\item The image of every $G$-invariant closed subset is closed.
		\item If $W_1$ and $W_2$ are disjoint, $G$-invariant and closed, then their images are disjoint. 
		\item $\varphi^G$ is affine.
	\end{enumerate}
\end{theorem}
\begin{proof}
	\autocite[p. 31]{ModuliNotes}
\end{proof}
Let us investigate some consequences. Firstly, observe that any orbit closure is $G$-invariant; indeed let $G\cdot p$ be an orbit, and let $I$ denote the ideal of functions vanishing on this orbit. Now by definition, $I$ is the set of $f\in A$ such that $f\in \mathfrak{m}_{g\cdot p}$, or equivalently $g^{-1}\cdot f\in \mf{m}_p$ for all $g\in G(k)$, so in particular it is $G$-invariant, and hence $\overline{G\cdot p}$, which is the closed subset cut out by $I$, is $G$-invariant too. Since $\varphi^G$ is a quotient, property (iii) implies that orbit closures are contracted to a point. By property (iv), the converse is also true: two points are mapped to the same point if and only if their orbit closures intersect. By the same property, it follows that the set-theoretic fibre of every $p\in Y(k)$ contains a unique polystable orbit. In particular, every equivalence class of orbits (the relation being intersection of closure) contains a unique polystable orbit. In particular, this means that the set $Y(k)$ is in canonical 1-1 correspondence with the polystable orbits of $X$.  
\begin{definition}
	Let $G$ be a reductive algebraic group acting on a scheme $X$. A quotient $\varphi^G: X \rightarrow Y$ is a \textit{good quotient} if the five conditions in the above theorem hold. The quotient $\varphi^G$ is said to be a \textit{geometric quotient} if it is a good quotient and additionally the mapping $G \cdot p \rightarrow \varphi^G(p)$ is a bijection between the orbit space and $Y(k)$.
\end{definition}
So in particular, the affine GIT quotient is a good quotient, and since it can also be shown that a good quotient is a categorical quotient (\autocite[Proposition 3.30]{ModuliNotes}), it follows that the GIT quotient is a categorical quotient. However, it is not always a geometric quotient:  
\begin{example}
	Let us consider Example \ref{naive-quotient} once more. We saw how the action is formalised as an algebraic action in Example \ref{naive-quotient-scheme}, and we studied its orbits in Example \ref{naive-quotient-orbits}. Now retaining the notation in Example \ref{naive-quotient-scheme}, observe that clearly the invariant ring is $A^G = k[xy]$, and hence the GIT quotient is \[\A^2 = \Spec k[x,y] \rightarrow \Spec k[xy] = \A^1, \] where the map sends each stable orbit $\{xy = \lambda\}$ to $\lambda$ and the three orbits that are not stable, which intersect each other, to the origin. Now it is not hard to check that this is a good quotient, however it is not geometric, because the set-theoretic fibre of the origin in $\A^1$ is the union of three orbits. However, if we restrict to the open subset of stable points $(\A^2)^s :=\A^2\setminus\{xy=0\}$, the quotient is in fact a geometric quotient. Indeed, the restricted quotient is just \[(\A^2)^s = \Spec k[x,y,(xy)^{-1}] \rightarrow k[(xy)^{\pm 1}] = \A^1 \setminus \{0\} \] and the set-theoretic fibre of each $\lambda\in k^*$ is just the hyperbola $\{xy = \lambda\} $.
\end{example}
In fact, this is typical:
\begin{theorem}
	Let $G$ be a reductive algebraic group acting on an affine variety $X$. Then the set of stable $k$-points are the $k$-points of a (possibly empty) open subscheme $X^s$, and the restricted map \[ X^s \rightarrow X^s \git G\] is a geometric quotient.
\end{theorem}
\begin{proof}
	This follows the proof in \autocite[p. 19, 32]{ModuliNotes}. Firstly, we claim that the set of points $p$ such that $\dim G_p >0$ is closed. Indeed, consider the following diagram:
	\begin{equation*}
		\begin{tikzcd}[row sep = huge]
			S = (G\times X)\times_{X\times X} X \arrow[r, "\varphi"] \arrow[d] & X\arrow[d,"\Delta"] \\
			G \times X \arrow[r, "(\sigma\times\pi_X)"] & X\times X
		\end{tikzcd}
	\end{equation*}
	where $\Delta: X \rightarrow X\times X$ is the diagonal map and $S$ is the fibred product of the diagram. Now observe that the $k$-points of $S$ are exactly the pairs $(g,p)$ such that $g\cdot p = p$. By Chevalley's semicontinuity theorem (\autocite[Th\'eor\`eme 13.1.3]{EGAIV}), the subset \[V:=\{(g,p)\in S(k) \mid \dim G_p = \dim S_{p} > 0 \} \] is closed. Now define $T$ to be the fibred product $S \times_{G\times X} X$ given by the following diagram:
	\begin{equation*}
		\begin{tikzcd}[row sep = huge]
			T \arrow[r] \arrow[d]&S \arrow[d]\\
			\Spec k \times X \cong X \arrow[r, "e \times \id"] &G \times X
		\end{tikzcd}
	\end{equation*}
	and observe that $V$ pulls back to the set $\{(e, p) \mid \dim G_p > 0\}$. Since $X$ is separated over $k$, the diagonal is a closed immersion, and in particular it is proper. It thus follows that $T \rightarrow X$ is a closed map, which proves the claim.
	
	Now let $p$ be a stable point. We will find an open neighbourhood of stable points containing $p$. Observe that $V$, which we saw was closed, and is clearly $G$-invariant, is disjoint from $p$, and hence by property (iv), $p$ and $V$ to disjoint sets in the GIT quotient. In particular, there is some invariant $f\in A^G$, such that $f(V) = 0$, and $f(p)\neq 0$ (for example, suppose take $1 - F$ where $F\in A^G$ vanishes exactly on $\varphi^G(p)$). We claim that the $k$-points of $X_f$ are stable. Indeed, since $X_f(k) \cap V = \emptyset$, it suffices to show that the orbits of $X_f$ are closed. So fix some $q\in X_f(k)$, suppose for contradiction its orbit is not closed. Since $G\cdot q$ is open and dense in $\overline{G\cdot q}$, its boundary must be of smaller dimension, and moreover contains a polystable orbit, say $G\cdot r$. It must therefore be that $\dim G_r > 0$ by, so $r\in V$, and hence $f(r) = 0$. But since $f$ is $G$-invariant and does not vanish on $q$, it does not vanish anywhere on $\overline{G\cdot q}$, which is a contradiction. It thus follows that the $k$-points of $X_f$ are all stable, and hence we conclude that $X^s$ is open. 
	
	Finally, we prove that this is a geometric quotient. To this end, we simply need to show that the set-theoretic fibre of each $k$-point in $ X^s \git G$ is a single stable orbit. But this is obvious; indeed for each $p\in X^s$, the set-theoretic fibre $(\varphi^G)^{-1}(\varphi^G(p))$ contains a unique polystable orbit, which must be $G\cdot p$. If it contains any other orbit, say $G\cdot q$, then $G\cdot p \cap \overline{G\cdot q}$ is nonempty, and since $\overline{G\cdot q}$ is $G$-invariant, it follows $G\cdot p$ is in the boundary of $G\cdot q$, and hence has dimension strictly less, which contradicts the fact that $p$ is stable.
\end{proof}
However, the affine GIT quotient is slightly oversimplified, since every $p\in X(k)$ has an image. In particular, no orbits are thrown away; simply merged. As we will see shortly, things are more complicated in the projective case.
\subsection{Projective GIT Quotients}

Our next task is to extend the notion of a GIT quotient to a projective variety. There are, however, more issues, the first and perhaps most obvious is that a projective scheme does not have a canonical homogeneous coordinate ring; such a coordinate ring is induced by a projective embedding. Indeed, even in the case of $X = \P^1$, we may embed $X$ in $\P^2$ in the obvious way, with resulting coordinate ring $k[x_0, x_1]$, or via the 2-uple embedding, with resulting coordinate ring $k[y_0, y_1, y_2]/y_0y_2 - y_1^2$, but these rings are not isomorphic, since the former is a UFD but the latter is not. 

But even if we we have an embedding $X \subseteq \P^n$, such that $X = \Proj S$, this is not enough, because the action does not lift canonically to $S$; indeed:
\begin{example}\label{linearisation-motivation}
	Let $X = \bb{P}^n$. Let $\bb{G}_m$ act on $X$ as follows: for any $\lambda\in \bb{G}_m(k), p = [p_0:...:p_n]\in X(k)$, we define \[\lambda\cdot p:= [\lambda^{-1}p_0:\lambda p_1:...:\lambda p_n] \] which extends uniquely to an algebraic action, since $k$ is algebraically closed. We embed $X$ in itself via the identity, with resulting coordinate ring $S = k[x_0,...,x_n]$. However, the action of $G$ may be lifted to one on $S$ in many ways; for example
	\begin{align*}
		\lambda\cdot x_0 &:= \lambda^{-1}x_0,\\ \lambda \cdot x_i &:= \lambda x_i
	\end{align*}
	and 
	\begin{align*}
		\lambda \cdot x_0 &:= x_0, \\\lambda \cdot x_i& := \lambda^2 x_i.
	\end{align*}
	just to name two.
\end{example}
We must therefore \textbf{choose} a lift of the action, which is accomplished as follows: recall that a projective embedding is equivalent to picking a very ample line bundle $\L$, and the resulting coordinate ring is \[S = \bigoplus_{r \geq 0} H^0(X, \L^{\otimes r}). \] We will therefore choose a lift of our action on $X$ to one on $\L$, so that the action is linear in some sense. This is encapsulated in the following definition:
\begin{definition}\index{linearisation}
	Let $\sigma: G\times X \rightarrow X$ be an algebraic group action on a projective scheme $X$. A \textit{linearisation} of this action is a line bundle $\L$ and an isomorphism $\Phi:\sigma^*(\L)\cong \pi_X^*(\L)$, where $\pi_X: G \times X \rightarrow X$ is the projection onto the second factor, such that the following diagram commutes:
	\begin{equation}\label{linearisation cocycle condition}
		\begin{tikzcd}[column sep = huge, row sep = huge]
			(\sigma\circ(\id_G \times \sigma))^*\L  \arrow[r, "(\id_G \times \sigma)^*\Phi"] \arrow[d, "="]& (\pi_X \circ(\id_G \times \sigma))^*\L \arrow[r, "="]& (\sigma \circ \pi_{23})^*\L\arrow[d, "\pi_{23}^*\Phi"] \\
		(\sigma\circ(\mu\times \id_X))^*\L\arrow[r, "(\mu\times \id_X)^*\Phi"]&(\pi_X \circ(\mu\times \id_X))^*\L	\arrow[r, "="]& (\pi_X \circ \pi_{23})^*\L		
		\end{tikzcd}
	\end{equation}
	where $\pi_{23}: G \times G \times X\rightarrow G \times X$ is the projection onto the last two factors. A linearisation is \textit{very ample} if $\L$ is. By abuse of language, we will often refer to $\L$ itself as the linearisation.
\end{definition}
We unwrap this definition. Of course, for any $g\in G(k)$, we may pull back $\L$ along the isomorphism $\sigma_g: X \rightarrow X$. The linearisation $\Phi$ allows us to identify $\L$ before and after the pullback. More precisely: for any open subset $U\subseteq X$ where $\L$ is trivial, we identify $\L|_U \cong \O_{U}$. Now fix a $k$-point $g$ of $G$. Then $\sigma_g^*\L(U) \cong \Ox(gU)$ and $\pi_X^*\L(U) \cong \O_{X}(U)$. In particular, we have the following isomorphism: 
\begin{equation}\label{composition-clusterfuck}
	\begin{tikzcd}
		\sigma_g^*\L(U) \cong \Ox(gU) \arrow[r, "\Phi"] & \pi_X^*\L(U) \cong \Ox(U)
	\end{tikzcd}	
\end{equation} 
So in particular, $\Phi$ may be thought of as defining a way to \enquote{shift} $\L$ by $g$.

We now make sense of (\ref{linearisation cocycle condition}) a little. Firstly, observe that these are morphisms of sheaves on $G \times G \times X$, all of which are pullbacks of $\L$ by various maps. The equalities follow from the axioms, for example \(\sigma\circ (\id_G \times \sigma) = \sigma\circ(\mu\times \id_X)\) is just the associativity axiom of group actions. Without explicitly stating the equalities, the commutativity says
\begin{equation}\label{linearisation-cocycle-equation}
	(\mu\times \id_X)^*\Phi = \pi_{23}^*\Phi \circ (\id_G \times \sigma)^*\Phi 
\end{equation} 
which we make sense of as follows: Let $(g,h)$ be a $k$-point in $G\times G$. Then as in (\ref{composition-clusterfuck}), the map $(\mu\times \id_X)^*\Phi$ induces a map \[\sigma_{gh}^*\L(U) \cong \Ox(ghU) \rightarrow \pi_X^*\L(U)\cong \Ox(U). \] Now $(\sigma \circ \pi_{23})(g,h,-) = \sigma_h $, and so $\pi_{23}^*\Phi$ induces a map 
\begin{equation*}
	\begin{tikzcd}
		\sigma_{h}^*\L(U) \cong \Ox(hU) \arrow[r] &  \pi_X^*\L(U)\cong\Ox(U)
	\end{tikzcd}
\end{equation*}
and finally since $(\sigma\circ (\id_G \times \sigma))(g,h,-) = \sigma_g \circ \sigma_h$, the pullback $(\id_G \times \sigma)^*\Phi $ induces a map
 \begin{equation*}
 	\begin{tikzcd}
 		\sigma_g^*(\sigma_h^*\L)(U) \cong \Ox(ghU) \arrow[r]& \sigma_h^*\L(U) \cong \Ox(hU)
 	\end{tikzcd}
 \end{equation*}
Put together, this means the following diagram commutes:
\begin{equation*}
	\begin{tikzcd}
		\sigma_{gh}^*\L\arrow[d] \arrow[r] &\L\\
		\sigma_h^*\L \arrow[ru]
	\end{tikzcd}
\end{equation*}
so in particular $G(k)$ acts on $\L$ via automorphisms.

Of course, this means that $G(k)$ also acts on all tensor powers of $\L$, and in the case $\L$ is very ample, taking global sections shows that $G(k)$ acts on the graded homogeneous coordinate ring $S = \bigoplus_{r\geq 0}H^0(X, \L^{\otimes r})$, and moreover it is not hard to see that this action preserves the grading. Furthermore, there is a natural action induced on the dual bundle, which we may interpret as the affine cone of this embedding, and it is not hard to see that this action is linear.
\begin{example}\label{projective-git-example}
	There is a natural linearisation of the action described in Example \ref{linearisation-motivation} on $\Ox(1)$. To see this, we first note that both $\sigma^*(\Ox(1))$ and $\pi_X^*(\Ox(1))$ are abstractly isomorphic to $\O_{\bb{G}_m \times X}(1)$. We define the isomorphism $\sigma^*(\Ox(1))\rightarrow \pi_X^*(\Ox(1))$ to be $x_0 \mapsto t^{-1}x_0$ and $t_i \mapsto t x_i$ for $i\neq 0$. 
	
	Once again, we check that this is in fact a linearisation. Clearly it is an isomorphism, so it suffices to show that (\ref{linearisation-cocycle-equation}) holds. To this end,  we first observe that the various pullbacks of $\Ox(1)$ to $\bb{G}_m\times \bb{G}_m\times X$ are abstractly isomorphic to the $\O_{\bb{G}_m\times\bb{G}_m\times X}$-module $\O_{\bb{G}_m\times\bb{G}_m\times X}(1)$, and since \[f^*(\O_{\bb{G}_m \times X}(1)) =  \O_{\bb{G}_m\times\bb{G}_m\times X}\otimes_{f^{-1}\O_{\bb{G}_m \times X}}f^{-1}(\O_{\bb{G}_m \times X}(1))\] where $f: \bb{G}_m\times \bb{G}_m\times X \rightarrow \bb{G}_m\times X$ is any map, we may write its elements as sums of $f\otimes g \otimes hx_i $, where $f,g\in \O_{\bb{G}_m}$ and $h\in \O_{\bb{G}_m \times X}$ (the $X$ component of $\O_{\bb{G}_m\times \bb{G}_m\times X}$ is absorbed by $h$). With this in mind, we compute:\[(\mu\times \id_X)^*\Phi(1\otimes 1 \otimes x_0) =  1 \otimes 1 \otimes t^{-1}x_0 = t^{-1}\otimes t^{-1}\otimes x_0\] and similarly for any other $x_i$. We also have \[(\id_G\times \sigma)^*\Phi(1\otimes 1 \otimes x_0) = 1\otimes 1 \otimes t^{-1}x_0 = t^{-1}\otimes 1 \otimes x_0 \] and finally \[\pi^{*}_{23}\Phi(t^{-1} \otimes1 \otimes x_0)  = t^{-1}\otimes t^{-1}\otimes x_0 \] as desired.
		
	Now the homogeneous coordinate ring of this embedding is just \[S = \bigoplus_{r \geq 0} H^0(X, \Ox(1)^{\otimes r}) = k[x_0,...,x_n]\] as expected, and there is an induced action of $\bb{G}_m(k)$ on $S$ given by $\lambda\cdot x_0 = \lambda^{-1}x_0$ and $\lambda\cdot x_i = \lambda x_i$ for $i\neq 0$, and in particular observe that this action preserves the grading on $S$.
\end{example}
Now we may define our quotient. We fix the following data: let $X$ be a projective scheme, $G$ a reductive algebraic group, $G\times X \rightarrow X$ an action, $\L$ a very ample linearisation and $S = \bigoplus_{r \geq 0} H^0(X, \L^{\otimes r})$ the homogeneous coordinate ring. We denote $S^G$ the subring of invariant elements of $S$, and we write $S_+$ for the irrelevant ideal $\bigoplus_{r>0}S_{\deg r}$, and similarly write $S_+^G$ for the $S^G$-ideal $S_+\cap S^G$. 
\begin{definition}\index{stability}\index{GIT quotient ! projective}
	A $k$-point $p$ is \textit{semistable} (with respect to $\L$) if there is a homogeneous invariant $\sigma\in S^G$ of positive degree such that $\sigma(p) \neq 0$, or equivalently $p\in X_\sigma(k)$ where $X_\sigma = \Spec S[\sigma^{-1}]_{\deg 0}$. If $p$ is not semistable, then it is \textit{unstable}. The \textit{semistable locus}, denoted $X^{ss}$ is the open subscheme $X \setminus V(S_+^G)$, where $V(S_+^G)$ is the closed subset associated to the honogeneous ideal $\langle S_+^G \rangle$ in $S$. Note that the homogeneous elements of $S_+^G$ generate this ideal, so it is in fact homogenous. We say $p$ is \textit{polystable} if it is semistable, and its orbit is closed in the semistable locus. Furthermore, $p$ is \textit{stable} if it is polystable, and additionally its stabiliser has dimension zero. The \textit{projective GIT quotient} is the map \[ X^{ss} \rightarrow X \git_\L G:=\Proj S^G\] induced by the inclusion $S^G \subseteq S$.
\end{definition}
Let us compare the affine and projective GIT quotients. The main difference is that in the affine case, every $k$-point has an image in the quotient; in other words every point is \enquote{semistable}; this is obviously not so in the projective case. Their similarities are, however, much more abundant: if $\sigma\in S^G_+$ is homogeneous, it is not hard to check that $X_\sigma = \Spec S[\sigma^{-1}]_{\deg 0}$ is invariant, and that the restriction of the projective GIT quotient to $X_\sigma$ is just the affine GIT quotient $X_\sigma \rightarrow \Spec S[\sigma^{-1}]_{\deg 0}^G$, and since $X^{ss}$ is covered by these affine open subsets, it follows that the projective GIT quotient is just a collection of affine GIT quotients glued together. Now observe that being a good quotient is local,and hence it follows that the projective GIT quotient is also a good quotient. By a similar argument to the affine case, it can also be shown that the stable locus is open and that the restriction is a geometric quotient.
\begin{example}
	Retain the notation and hypotheses in Example \ref{projective-git-example}. It can be shown (\autocite[p. 37]{GIT}) that the ring of invariants $S^G$ is just $k[x_0x_1,...,x_0x_n]$. It follows that $p = [p_0:...:p_n]$ is semistable if and only if $p_0$ is nonzero, and some other $p_i$ for $i > 0$ is nonzero. In particular, the semistable locus can be identified with $\A^n \setminus\{0\}$. Now on the semistable locus, the action is just multiplication by $\lambda^2$, so every point is polystable, the orbit just being the line passing through our point and the origin in $\A^n$, minus the origin itself. In fact, every point is stable, since the action is free. Of course, this makes sense because our projective GIT quotient is just \[\P^{n-1} = \Proj k[x_0x_1,...,x_0x_n]\] and this is a geometric quotient.
\end{example}
\begin{example}
	Of course, there is another linearisation on $\Ox(1)$ given by $x_0 \mapsto x_0$ and $x_i \mapsto t^2x_i$ for $i > 0$. Clearly $k[x_0]$ is the ring of invariants, so the projective GIT quotient with respect to this linearisation is simply $\Spec k$. Indeed, the semistable locus is the open set given by $x_0 \neq 0$, which is isomorphic to $\A^n$. With this interpretation, the action of $\bb{G}_m$ is just scaling, and the closure of every orbit contains the origin in $\A^n$ (or equivalent the point $[p_0:0:...:0]\in \P^n$), which is the unique polystable orbit of this linearisation. In particular, the stable locus is empty. This shows that projective GIT is heavily dependent on our choice of linearisation. However, we will often fix a single linearisation to work with, and the problem of choosing different linearisations will not be discussed in this thesis. 
\end{example}

\section{The Hilbert-Mumford Criterion}
As we have just seen, stability is very important. However, with our definition, it is rather difficult to calculate. In this section, we will develop the \textit{Hilbert-Mumford criterion}, which gives a numerical criterion for stability in terms of 1-parameter subgroups. We begin with a closer examination of our current definition for stability, which requires the following definition:
\begin{definition}\index{affine cone}
	Let $X$ be a projective scheme and let $\L$ be a very ample line bundle. Write $S$ for the homogeneous coordinate ring \[S:= \bigoplus_{r \geq 0} H^0(X, \L^{\otimes r}).\] We define the \textit{affine cone} of $X$ to be the affine scheme $\widetilde{X} := \Spec S$.
\end{definition}
To make sense of the affine cone, firstly recall that $\L$ embeds $X$ as a closed subscheme of $\P^n$, where $n = h^0(X, \L)-1 = \dim H^0(X, \L) - 1$. The $k$-points of $\P^n$ are just the 1-dimensional subspaces of $k^{n+1}$, and thus the $k$-points of $X$ may be interpreted as a collection of lines through the origin in $k^{n+1}$. The $k$-points of $\widetilde{X}$ may, in turn, be thought of as the union of these lines. For example, the affine cone of $\P^n$ is just $\A^{n+1}$.

There is a well-defined notion of an origin, which corresponds to the irrelevant ideal $S_+$, which is clearly maximal, and there is a natural map $\Spec S \setminus \{0\} \rightarrow \Proj S$, which we define as follows: let $f\in S_{\deg 1}$. Then there is an inclusion $S[f^{-1}]_{\deg 0}\subseteq  S[f^{-1}]$, which induces a morphism $\Spec S[f^{-1}]\rightarrow \Spec S[f^{-1}]_{\deg 0}$. Since $\L$ is very ample, it follows $S$ is generated by a finite set of these $f\in S_{\deg 1}$ as a $k$-algebra, and so $X_f = \Spec S[f^{-1}]_{\deg 0}$ cover $\Spec S \setminus\{0\}$. On the level of $k$-points, this is just $(p_0,...,p_n)\mapsto [p_0:...:p_n]$.

Now suppose $X$ is a projective variety, and let $G$ be a reductive affine algebraic group acting on $X$. Further, let $\L$ be a very ample linearisation, and let $S$ be the homogeneous coordinate ring. We claim the the linearisation naturally induces an action on the affine cone. Indeed, by the adjunction property of pullbacks and pushforwards, there is a natural map (the unit map of the adjunction) $\L \rightarrow \sigma_*\sigma^*(\L)$. Taking global sections, we have \[H^0(X, \L)\rightarrow H^0(G\times X, \sigma^*(\L))\cong H^0(G\times X, \pi_X^*(\L)) \cong H^0(G, \O_G)\otimes H^0(X, \L) \] where the final isomorphism comes from the K\"unneth formula (\autocite[Lemma 33.29.1]{stacks-project}). We can check that this induces a map $\tilde{\sigma}^*:S \rightarrow \O_G(G)\otimes S$ {\color{red} which satisfies the co-action axioms; and in particular this induces a group action $G \times \widetilde{X} \rightarrow \widetilde{X}$. Moreover, since the co-action homomorphism is linear on $H^0(X, \L)$, this means that $G$ acts linearly (i.e. via a representation $G \rightarrow \GL_{n+1}$) on $\widetilde{X}$.}
\begin{example}
	Recall Example \ref{projective-git-example}. The induced action on $\widetilde{X} = \A^{n+1}$ is just \[\lambda \cdot (p_0,...,p_n) = (\lambda^{-1}p_0,\lambda p_1,...,\lambda p_n) \] on the level of $k$-points. More rigorously, the co-ordinate rings are $k[t^{\pm 1}]$ and $k[x_0,...,x_n]$, and co-action homomorphism is given by $x_0 \mapsto t^{-1}\otimes x_0$ and $x_i \mapsto t\otimes x_i$ for $i > 0$.
\end{example}
We now present our first criterion for stability:
\begin{theorem}[Topological criterion for stability] \index{stability ! topological criterion}
	Let $X$ be a projective variety, let $G$ be a reductive affine algebraic group with a linearisation on the very ample line bundle $\L$, and let $S$ denote the resulting coordinate ring. 
	\begin{enumerate}
		\item A $k$-point $p$ is semistable if and only if for any lift $\tilde{p}\in \widetilde{X}(k)$, the closure of the $\tilde{p}$ orbit in $\widetilde{X}$,  $\overline{G\cdot \tilde{p}}$, does not contain the origin.
		\item A $k$-point $p$ is polystable if and only if the orbit of any of its lifts is closed in $\widetilde{X}$.
		\item A $k$-point $p$ is stable if and only if for any lift $\tilde{p}$, the map $\sigma(-, p): G \rightarrow X$ is proper.  %its orbit is closed in $\widetilde{X}$ and $\dim G_{\tilde{p}} = 0$.
	\end{enumerate}
\end{theorem}
\begin{proof}
	Fix a $k$-point $p$ and a lift $\tilde{p}$. If $p$ is semistable, then there is some $r>0$ and $\sigma\in S_{\deg r}^G$ such that $\sigma(p) \neq 0$. Let $f = \sigma - \sigma(\tilde{p})$. Then $f$ is invariant, and hence constant on $G\cdot \tilde{p}$. Observe that $f(0) = \sigma(0) - \sigma(\tilde{p}) = - \sigma(\tilde{p})$ (since $\sigma$ is homogeneous of positive degree, it follows $\sigma(0) = 0$), which means that there is some function which vanishes on $G\cdot \tilde{p}$ but not 0, and hence $0$ is not in the orbit closure of $\tilde{p}$. Conversely, suppose $0$ is not in the orbit closure of $\tilde{p}$. Then $\overline{G\cdot \tilde{p}}$ and the origin are both $G$-invariant closed subsets of $\widetilde{X}$, and it can be shown (\autocite[Corollary 1.2]{GIT}) there is some invariant $f\in S$ such that $f(0)= 0$ for all $g\in G(k)$ but $f(g\cdot p)\neq 0$. Clearly then $f$ has no degree zero component. Now let $f = \sum_{i > 0} f_i$ be the homogeneous decomposition of $f$, with $f_i$ of degree $i$. In particular, some $f_r$ must not vanish on $\tilde{p}$, and since $G$ preserves each homogeneous component of $S$, it follows that $f_r$ is invariant, and hence $f_r(p)\neq 0$, so $p$ is semistable. This proves (i).
	
	To prove (ii), firstly suppose $p$ is semistable (if it is not, then it cannot be polystable, and the boundary of its orbit contains the origin). Then $p\in X_\sigma(k)$ for some invariant homogeneous $\sigma$ of positive degree. Observe then that $X_\sigma$ is $G$-invariant, and so $G\cdot p\subseteq X_\sigma(k)$. Now pick a lift $\tilde{p}$ of $p$, and consider the closed subscheme $V = \Spec S/ \langle \sigma - \sigma(\tilde{p}) \rangle$ of $\widetilde{X}$, which clearly contains the orbit $G\cdot \tilde{p}$. Now the map $\widetilde{X} \setminus \{0\} \rightarrow X$ restricts to a map $\varphi: V\mapsto X_\sigma$, which is a morphism of affine schemes, induced by the canonical ring homomorphism \[S[\sigma^{-1}]_{\deg 0} \rightarrow S/\langle \sigma - \sigma(\tilde{p}) \rangle,  \] \[ \frac{f}{\sigma} \mapsto \frac{f}{\sigma(\tilde{p})}, \] and since the homomorphism is surjective, the morphism $\varphi$ is finite, and hence closed. In particular, if $G \cdot \tilde{p}$ is closed, then it is closed in $V$, and $G \cdot X$ is closed in $X_\sigma$. {\color{red} TO FINISH}.
\end{proof}

The main issue with the above is that it is oftentimes very difficult to compute the closure of an orbit, and in fact it may even be unknown what the homogeneous coordinate ring of our linearisation is in the first place! And thus we will require a slightly different notion, which is more computation-friendly. This is the \textit{Hilbert-Mumford criterion}, which relates stability to 1-parameter subgroups, which we will now define:
\begin{definition}\index{1-parameter subgroup}
	Let $G$ be an algebraic group acting on a scheme $X$, separated over $k$. A \textit{1-parameter subgroup}, or just \textit{1-PS}, is a group homomorphism $\lambda: \bb{G}_m \rightarrow G$. 
	
	Now let $f: \bb{G}_m \rightarrow X$ be any morphism. Then by the valuative criterion for separation, $f$ has at most one extension to a morphism $f^\sharp: \A^1 \rightarrow X$. If this extension does exist, we define the \textit{limit} of $f$ at 0, denoted $\lim_{t\to 0} f(t)$, to be \[\lim_{t\to 0} f(t):= f^\sharp(0). \] If this extension does not exist, we say the \textit{limit does not exist}.
\end{definition}
%\begin{lemma}
%	Let $\lambda: \bb{G}_m \rightarrow G$ be a 1-PS. Then $\lim_{t \to 0} \lambda(t)$ exists if and only if $\lambda$ is trivial.
%\end{lemma}
%\begin{proof}
%	content...
%\end{proof}

Now suppose $X$ is projective. Then given an embedding $X\subseteq \P^n$ and $p\in X(k)$, the morphism $\lambda_p$ may be considerd a morphism into $\P^n$, and if we define the lift $\lambda_p^\sharp$ as a morphism into $\P^n$, it is clear that the limit will be contained in the closed subscheme $X$, and thus we reduce to the case $X = \P^n$. 

Now suppose we have a linearisation $\L$, and hence a linear action of $G$ on the affine cone $\widetilde{\P^n} = \A^{n+1}$. Then given a $p\in \A^n(k)$ and a 1-PS $\lambda: \bb{G}_m \rightarrow G$, we have an induced morphism $\lambda_p: \bb{G}_m \rightarrow \A^{n+1}$, and if the limit does exist, then the limit is contained in the closure of $G\cdot p$. This is clear. What is not clear is that the converse is also true:
\begin{theorem}\label{fundamental-theorem-git}
	Let $G$ be a reductive affine algebraic group acting on $\A^n$, with the action induced by a representation $G \rightarrow \GL_n$. Then for any $p\in \A^n(k)$, a $k$-point $q$ is in the closure of $G\cdot p$ if and only if there exists a nontrivial 1-PS $\lambda: \bb{G}_m \rightarrow G$ such that \[\lim_{t\to 0} \lambda_p(t) = q.\]
\end{theorem}
The {\color{red}proof relies on the Cartan-Iwahori decomposition theorem}, which is beyond the scope of this thesis, but it can be found in \autocite[p. 48]{ModuliNotes} or \autocite[p. 53]{GIT}. This result is highly analogous to the theorem in analysis, which states that a point is in the closure of a set in a metric space if and only if there is some sequence which converges to it. 

The significance of this theorem is that we can study the closure of an orbit by studying linear actions of $\bb{G}_m$, which we know has a weight space decomposition, by Theorem \ref{torus-reductive}. More precisely, we have the following definition:
\begin{definition}
	Let $G$ be a reductive affine algebraic group acting on $\P^n$, with a linearisation on $\O_{\P^n}(1)$. Then we have a linear action of $G$ on $\A^{n+1}$. Now given a 1-PS $\lambda: \bb{G}_m \rightarrow G$, we have a weight space decomposition \[ k^{n+1}=: V = \bigoplus_{r\in \Z} V_r. \] Choose a basis $\{e_i\}$ for $V$ such that $\lambda \cdot e_i = \lambda^{r_i} e_i$ for all $\lambda\in \bb{G}_m(k)$. Now let $p\in \P^n(k)$, and let $\tilde{p}\in \A^{n+1}(k)$ be a lift. We may write $\tilde{p} = \sum p_i e_i$. The \textit{Hilbert-Mumford weight of $\lambda$ at $p$}, denoted $\mu(p, \lambda)$, is the integer \[\mu(p, \lambda) := \max\{-r_i \mid p_i \neq 0\}.\] Note that this does not depend on the choice of $\tilde{p}$.
\end{definition}

We unwrap this definition a little. If $\mu(p, \lambda) < 0$, this means that all the $r_i$ are postive. In particular, \[\lim_{t\to 0} \lambda_p(t) = 0, \] and so $p$ is unstable. If $\mu(p, \lambda) = 0$, then all the $r_i$ are nonnegative, with at least one strictly zero, and so the limit does exist, and it easy to see \[\lim_{t\to 0} \lambda_p(t) = \sum_{r_i = 0}  p_i e_i. \] In particular, if $\lambda$ is not trivial, then $p$ is {\color{red} not polystable}.





\chapter{Construction of the Moduli Space}
In this chapter, we will put together everything we have learnt so far to construct the moduli space of vector bundles. Let $X$ be a fixed nonsingular curve of genus $g$ over an algebraically closed field $k$ of characteristic zero (that is, an integral scheme of dimension 1 whose local rings at $k$-points are regular, quasi-projective over $k$). 


\section{Divisors and Line Bundles}
In our study of vector bundles, the natural place to start is with line bundles. In this section, we will review the relation between divisors and line bundles, which will ultimately help us define the degree of a vector bundle. Our exposition closely follows the one found in \autocite[II, Section 6]{Hart}
\begin{definition}\index{divisor ! Weil divisor}
	We define the \textit{group of Weil divisors} on $X$, denoted $\Div(X)$ to be the free abelian group on the $k$-points of $X$. The elements of $\Div X$ are known as \textit{Weil divisors}. The \textit{support} of a Weil divisor $D = \sum n_i p_i$, denoted $\Supp D$, is the set $\{p_i\mid n_i \neq 0\}$, and the \textit{degree} of $D$ is defined to be $\deg D:=\sum n_i$. We say $D$ is \textit{effective} if $n_p > 0$ for all $p\in \Supp D$. A \textit{prime divisor} is a Weil divisor of degree 1.
\end{definition}
Since $X$ is nonsingular, for any $k$-point $p$, the local ring $\O_{X, p}$ is regular of dimension 1, and in particular it is a DVR, with fraction field $K(X)$, the function field of $X$. Denote the valuation $v_p$. For any $f\in K(X)^*$, we define the \textit{divisor of $f$}, denoted $\div(f)$ to be \[\div f := \sum_{p\in X(k)} v_p(f) p \]we can show (\autocite[II Lemma 6.1]{Hart}) that all but finitely many of the $v_p(f)$ vanish, hence we get a Weil divisor. A divisor of the form $\div f$ is known as a \textit{principal divisor}; clearly the principal divisors form a subgroup. Two Weil divisors are \textit{linearly equivalent} if their difference is a principal divisor. The group of Weil divisors modulo linear equivalence is the \textit{divisor class group}, denoted $\Cl X$. A \textit{divisor class}, is an element of $\Cl X$. By a \textit{divisor}, we will abuse language and refer to either a Weil divisor or its divisor class; it will either be clear from context or unimportant which is meant. \index{divisor! divisor class}
\begin{example}
	If $X$ is affine, equal to $\Spec A$ where $A$ is necessarily a Dedekind domain (\autocite[I, Proposition 11.5]{Neuk}), then $\Cl X = 0$ if and only if $A$ is a PID. Indeed, if $A$ is a PID, then if $D = \sum n_p p$ is a divisor, then for each $p\in \Supp D$, the maximal ideal $\mf{m}_p$ of $p$ is principal, generated by, say $\varpi_p$, which must be a uniformiser of $\O_{X,p}$. Now let \[f = \prod_{p\in \Supp D} \varpi^{n_p}\in \Frac A = K(X) \] then clearly $D = \div f$. 
	
	Conversely, if $A$ is not a PID, then there is some maximal ideal $\mf{m}_p$ corresponding to some $p\in X(k)$ generated by two elements (\autocite[I, 3, Ex 6.]{Neuk}), say $\mf{m}_p = \langle f, g \rangle$. Now localising at $p$, it is clear that either $f$ or $g$ must be the uniformiser of $\O_{X,p}$; suppose it is $f$ without loss of generality. Now if $p$ was principal, then clearly $p$ must be the divisor of some associate of $f$ in $\O_{X,p}$, say $F$. But $F$ is not prime, and by the unique ideal factorisation property of Dedekind domains (\autocite[I, Theroem 3.3]{Neuk}), we may write \[\langle F \rangle = \mf{m}_p \prod \mf{m}_{p_i}^{n_i}\] where at least one $n_i$ is nonzero, but all but finitely many of them are. But that means \[\div F = p + \sum n_i p_i \] which is a contradiction.
	
	More concretely, if $X = \A^1 = \Spec k[x]$, then any $D = \sum n_i a_i$ can be written $D = \div \prod (x - a_i)$, but if $X = \Spec k[x,y]/ \langle y^2 = x^3-x\rangle$ and $\operatorname{char} k \neq 2$, then we claim $p = (0,0)$ is not principal. Indeed, $\mf{m}_p = \langle x, y \rangle$ and clearly $y^2 = x(x-1)(x+1)$, so $v_p(x) = 2$, so $y$ is a uniformiser of $\O_{X,p}$. But \[\div y = (0,0) + (1,0) + (-1,0) \neq p,\] and similarly for any other uniformiser.
\end{example}
We now state a key result, which will allow us to define the degree of a line bundle:
\begin{proposition}\label{degree-independent-divisor-representative}
	The degree map $\deg: \Div X \rightarrow \Z$ descends to a map $\Cl \rightarrow \Z$.
\end{proposition}
\begin{proof}
	\autocite[p. 138]{Hart}.
\end{proof}
\begin{example}
	Let $X = \P^1 = \Proj k[x_0,x_1]$. We claim that the degree map is an isomorphism; in particular its kernel is trivial. So suppose $D = \sum n_p p$ is a degree zero divisor. Now each $p$ can be written $[p_0: p_1]$, corresponding to $\langle x_0p_1 - x_1p_0 \rangle$. Write \[f = \prod_{p\in \Supp D} (x_0p_1-x_1p_0)^{n_p}\in k(x_0, x_1) \] Now observe that \[K(X) = k(\frac{x_0}{x_1}) \] in particular, $K(X)$ is the subfield of $k(x_0, x_1)$ consisting of degree 0 elements. Since $\sum n_p = 0$, it follows $\deg f = 0$ too, hence $D = \div f$ as desired.
\end{example}

Next we describe the relation between divisors and line bundles: Let $[D]$ be a divisor class. We define the \textit{line bundle associated to} $[D]$, denoted $\L(D)$ as follows: let $D = \sum n_p p$ be a divisor in the class of $[D]$. For any open set $U$, we define \[\L(D)(U):= \{f\in K(X) \mid v_p(f) + n_p \geq 0 \text{ for all }p\in U(k)\}\] Firstly, we need to check that this is indeed a line bundle. To see this, observe the following: if $U\cap \Supp D = \emptyset$ then $\L(D)(U) = \Ox(U)$. Now for any $p\in \Supp D$, we choose some uniformiser $\varpi_p$ of $\O_{X, p}$, and we may assume $\varpi_p\in \Ox(U_p)$ for some open set $U_p$ containing $p$. We may pick $U_p$ sufficiently small that $(\div \varpi_p^{n_p})|_{U} = n_p p$ (in other words, $U_p \cap \Supp (\div \varpi_p) = \{p\}$). Now observe $\L(D)(U_p) = \varpi_p^{-1} \Ox(U)$, and in particular they are isomorphic. Now covering $X$ with the open sets $U_p$ and open sets which do not intersect $\Supp D$, we deduce that $\L(D)$ is a line bundle. 

Finally, we need to check that $\L(D)$ does not depend on our choice of $D$. So let $D' = \sum n_p' p$ be a divisor linearly equivalent to $D$, so that $D - D' = \div F$. Observe that then $v_p(F) = n_p - n_p'$. Now we define the isomorphism $\varphi: \L(D)\rightarrow \L(D')$ to be $f\mapsto  Ff$. To see that this is an isomorphism, observe that given an open set $U$, we have \[ \L(D)(U)= \{f\in K(X) \mid v_p(f) + n_p \geq 0 \text{ for all }p\in U(k)\} \] and \[\L(D')(U):= \{f\in  K(X) \mid v_p(f) + n_p' \geq 0 \text{ for all }p\in  U(k)\} \] now if $f\in \L(D)(U)$, then \[v_p(fF) + n_p' = v_p(f) + v_p(F) + n_p' \geq -n_p + v_p(F) + n_p' = 0 \] hence we have a well-defined morphism of sheaves, and moreover, clearly division by $F$ is its inverse. 
\par Before we state our next result, we recall that the \textit{Picard group}, denoted $\Pic X$ is the group of line bundles on $X$, with the group operation given by tensor product, and inversion given by dualising. \index{Picard group}%Using \v{C}ech cohomology, we can show $\Pic X \cong H^1(\Ox^*)$, where $\Ox^*$ is the sheaf of nowhere-vanishing functions on $X$. Indeed, the transition functions of a line bundle $\L$ on a sufficiently fine cover define a \v{C}ech 1-cocycle of $\Ox^*$, and is not hard to show two line bundles are isomorphic if and only if their transition functions are cohomologous, and that the $\v{C}ech$ cohomology group remains constant under refeinement. 
\begin{theorem}
	The map $D \mapsto \L(D)$ is an isomorphism $\Cl X \rightarrow \Pic X$.
\end{theorem}
\begin{proof}
	This follows the proof given in \autocite[pp. 144, 145]{Hart}. First of all, let $\K$ denote the constant sheaf $K(X)$, which is an $\Ox$-module. We claim $\L \otimes \K\cong \K$. To see this, we consider the base of the topology $\{U_\alpha = \Spec A_\alpha\}$ on $X$ consisting of the affine open sets where $\L$ is trivial. Then $K(X) = \Frac A_\alpha$ for each $\alpha$, so locally we have the natural isomorphism $A_\alpha \otimes K(X) \cong K(X)$, given by $a\otimes b \mapsto b$, and it is very easy to see that these agree on overlaps, and since the $U_\alpha$ form a base, by \autocite[Proposition I-12]{EisHar}, this extends to an isomorphisms of sheaves. It thus follows that $\L$ can be embedded inside $\K$. 
	
	First we prove surjectivity. By our discussion above, we may consider $\L$ as a subsheaf of $\K$; in particular, every local section is an element of $K(X)$. Since $\L$ is invertible, it follows that $\L(U)$ is a free $\Ox(U)$-submodule of $K(X) = \Frac \Ox(U)$, for sufficiently small $U$, and in particular it is generated by some $\varpi_U^{-1}\in K(X)$. Now we define the divisor $D$ locally as $\div \varpi_U$, and let $U$ vary across a cover. To see this is well-defined, note that if $\varpi_U^{-1}$ generates $\L(U)$ and $\varpi_V^{-1}$ generates $\L(V)$, then both $\varpi_U^{-1}$ and $\varpi_V^{-1}$ generate $\L(U \cap V)$, and in particular they are associates in $\Ox(U\cap V)$, hence they generate the same Weil divisor on $U \cap V$. It is then clear that $\L(D) = \L$.
	
	Next we prove the homomorphism property. But this is obvious since if $\L(D_1)$ and $\L(D_2)$ are generated locally by $\{\varpi_1^{-1}\}$ and $\{\varpi_2^{-1}\}$ respectively, then $\L(D_1 + D_2)$ are generated locally by $\{\varpi_1^{-1}\varpi_2^{-1}\}$. But $\L(D_1)\otimes \L(D_2)$ is also generated locally by $\{\varpi_1^{-1}\varpi_2^{-1}\}$, and hence they are isomorphic.
	
	Finally, we prove injectivity. Since we have shown this is a group homomorphism, it suffices to show that the kernel is trivial. So suppose $\L(D) \cong \Ox$, and fix an isomorphism $\varphi: \Ox \rightarrow \L(D)$. We claim $D = \div \varphi(1)^{-1}$. Indeed, choosing a sufficiently fine open affine cover $\{U_\alpha\}$ such that $\#(U_\alpha \cap \Supp D) \leq 1$, we suppose $D|_{U_\alpha}$ is the divisor of $\varpi_\alpha$. Then $\varpi_\alpha^{-1}$ generates $\L(D)(U_\alpha)$. But $\varphi(1)$ also generates $\L(D)(U_\alpha)$, hence $\varphi(1)$ and $\varpi_\alpha^{-1}$, and hence both generate the same Weil divisor. In particular, $\varphi(1)^{-1}$ generates $D$, as desired.
\end{proof}

Now let $\L$ be a line bundle. We will take a look at $H^0(X, \L)$, the space of global sections. For each nonzero $s\in H^0(X, \L)$, we define the \textit{divisor of zeroes} of $s$, denoted $\div s$ as follows: on any open subset $U$ on which $\L$ is trivial, we let $\Phi_U: \L|_U \rightarrow \O_U$ be an isomorphism, and define \[\div s|_U:= \div \Phi_U(s). \] Of course, this is well-defined: let $p\in U(k)$, and let $\Phi_V: \L|_V \rightarrow \O_V$ be another trivialisation, with $p\in V(k)$. Then at $p$, $\Phi_V(s)$ and $\Phi_U(s)$ differ by an invertible element of $\O_{X, p}$, and thus they have the same valuation, and since $p$ is arbitrary, this means we get a well-defined Weil divisor. Furthermore, observe that $\div s$ is effective, since it is locally the divisor of some section of $\Ox$, which must have nonnegative valuation at any $k$-point. We are now in a position to state our result:
\begin{proposition}\label{divisor-of-zeroes-effective}
	Let $\L = \L(D)$ be the line bundle associated to a divisor $D$. Then:
	\begin{enumerate}
		\item The divisor of zeroes of any nonzero $s\in H^0(X, \L)$ is linearly equivalent to $D$.
		\item Any effective divisor $D_0$ linearly equivalent to $D$ is the divisor of zeroes of some nonzero $s\in H^0(X, \L)$.
	\end{enumerate}
\end{proposition}
\begin{proof}
	\autocite[p. 157]{Hart}.
\end{proof}
We now come to the definition of the degree:
\begin{definition}\index{vector bundle ! degree}
	Let $\E$ be a line bundle. We define the \textit{degree} of $\E$, denoted $\deg \E$ as follows: if $\E$ is a line bundle, we define $\deg \E$ to be the degree of the divisor corresponding to $\E$ via the isomorphism $\Pic X \cong \Cl X$. This is well-defined, by Proposition \ref{degree-independent-divisor-representative}. In general, we define the \textit{determinant line bundle} of a rank $n$ vector bundle $\E$ to be the line bundle \[\det \E := \wedge ^n \E\] and we define $\deg \E := \deg(\det \E)$. Note that $\det \L = \L$ if $\L$ is a line bundle, hence our definition is consistent. We define the \textit{signature} of $\E$ to be the pair $(\rk \E, \deg \E)$.
\end{definition}
\begin{example}
	Let $X = \P^1 = \Proj k[x_0, x_1]$. We will show $\deg \Ox(n) = n$. Let $D =n [0:1]$. We compute $\L(D)$ as follows: on $U_1 = \{x_1\neq 0\} = \Spec k[x_0/x_1]$, we have \[D|_{U_1} = \div (\frac{x_0}{x_1})^n \] and on $U_0 = \{x_0 \neq 0\} = \Spec k[x_1/x_0]$, we have $D|_{U_0} = 0$, hence \[\L(D)(U_0) = \Ox(U_0) = k[x_1/x_0] \] and \[\L(D)(U_1) = (\frac{x_1}{x_0})^nk[x_0/x_1] \] Meanwhile, \[\Ox(n)(U_0) = x^n_0 k[x_1/x_0]\] and \[\Ox(n)(U_1) = x_1^n k[x_0/x_1] \] hence we define the isomorphism $\L(D) \rightarrow \Ox(n)$ by $f\mapsto x_0^n f$. It is easy to check that this is well-defined and agrees on overlaps (which is just localising), hence we get an isomorphism of line bundles.
	
	We also observe that $D$ is the divisor of zeroes of $x_0^n\in H^0(X, \Ox(n))$, as expected.
\end{example} 
\begin{example}
	Let $X$ be the elliptic curve $X=\Proj k[x,y,z]/ \langle y^2z = x^3 - xz^2 \rangle$, and let $\Ox(1)$ be the pullback of $\O_{\P^2}(1)$ induced by the embedding (or equivalently the twisting sheaf of Serre). We will show $\deg \Ox(1) = 3$ and is isomorphic to $\L = \L(3[0:1:0])$. Note that $x\neq 0$ implies $z\neq 0$, hence we can cover $X$ by two open sets $U_z = \{z\neq 0\}$ and $U_y = \{y \neq 0\}$. Thus we have \[\Ox(1)(U_z) = zk[y/x, z/x]/ (y^2z/x^3 = 1 - z^2/x^2) = z\Ox(U_z) \] and \[\Ox(1)(U_y) = yk[x/y, z/y]/ (z/y = (x/y)^3 - xz^2/y^3) = y\Ox(U_y).\] Now observe that \[\L(U_z) = \Ox(U_z)\] but \[\L(U_y) = (y/z)\Ox(U_y) \] since $D = 3[0:1:0]$ is the divisor of $z/y$ on $U_y$. It then follows we have a global isomorphism $\L \rightarrow \Ox(1)$ defined by $f \mapsto zf$.
\end{example}

We conclude this section with some technical results about the degree:


\begin{proposition}
	Let \[0 \rightarrow \E \rightarrow \F \rightarrow \G \rightarrow 0 \] be a short exact sequence of vector bundles. Then \[\deg \F = \deg \E + \deg \G\]
\end{proposition}
\begin{proof}
	If $\{g_{\alpha\beta}\in \GL_n(\Ox(U_\alpha\cap U_\beta)) \}$ is the set of transition morphisms on a sufficiently fine cover representing $\F$, then it is not difficult to check $\{\det g_{\alpha\beta}: U_\alpha\cap U_\beta \rightarrow \bb{G}_m \}$ is the set of transition functions of $\det \F$. Now it can be shown (\autocite[p. 68]{GriHa}) that each $g_{\alpha\beta}$ has the shape \[\begin{pmatrix*}
		h_{\alpha\beta} & k_{\alpha\beta} \\
		0 & j_{\alpha\beta}
	\end{pmatrix*} \] where $\{h_{\alpha\beta} \}$ and $\{j_{\alpha\beta}\}$ are the transition morphisms representing $\E$ and $\F$ respectively. Hence\[ \det g_{\alpha\beta} = \det h_{\alpha\beta} \det j_{\alpha\beta} \] But $\{\det h_{\alpha\beta} \det j_{\alpha\beta}\}$ is the set of transition functions for $\det \E \otimes \det \G$, which means $\det \F = \det \E \otimes \det \G$, and thus the result follows.
\end{proof}
\begin{corollary}
	Let $\E$ and $\F$ be vector bundles. then $\deg (\E\oplus \F) = \deg \E + \deg \F$.
\end{corollary}
\begin{proof}
	Apply the above proposition to \[0 \rightarrow \E \rightarrow \E \oplus \F \rightarrow \F \rightarrow 0 \] and the result follows immediately.
\end{proof}
\begin{corollary}\label{twisting-by-line-bundle-degree}
	Let $\E$ be a vector bundle and $\L$ a line bundle. Then \[\deg(\E \otimes \L) = \deg \E + \deg \L \rk \E \] 
\end{corollary}
\begin{proof}
	As in the proof of the proposition, we look at the transition functions, and the result immediately follows.
\end{proof}


\begin{lemma}
	Let $\E$ and $\F$ be vector bundles of rank $n$, and suppose there is a homomorphism $ \E \rightarrow \F$ with nonzero determinant (i.e. the determinant is not identically zero). Then $\deg \E \leq \deg \F$.
\end{lemma}
\begin{proof}
	Taking the determinant, we get a nonzero homomorphism $\det \E \rightarrow \det \F$, hence we may assume without loss of generality $n = 1$. Now a nonzero homomorphism $\E \rightarrow \F$ is nothing more than a global section of $\fancyHom(\E, \F) = \E ^\vee \otimes \F$, and since we are assuming $\E$ and $\F$ are line bundles, it follows $\deg(\E ^\vee \otimes \F) = \deg \F - \deg \E$; and thus we reduce to proving that if $\L$ is a line bundle with a nonzero global section, then $\deg \L \geq 0$. 
	
	So let $\L$ be a line bundle, associated to a divisor $D$ and suppose $s\in H^0(X, \L)$ is a nonzero global section. Then by Proposition \ref{divisor-of-zeroes-effective}, the divisor of zeroes of $s$, say $D_0$, is effective, and linearly equivalent to $D$. But since $D_0$ is effective, it has nonnegative degree, and thus $\L = \L(D) = \L(D_0)$ has nonnegative degree, as desired.
\end{proof}
\begin{proposition}\label{canonical-extension-proposition}
	Let $\E \rightarrow \F$ be a nonzero homomorphism of vector bundles. Then there is a factorisation:
	\begin{equation}
		\begin{tikzcd}
			0 \arrow[r] & \cal{E'}\arrow[r] & \E \arrow[r] \arrow[d, "\varphi"]& \E'' \arrow[d]\arrow[r]& 0\\
			0  & \F''\arrow[l]& \F \arrow[l] & \F'\arrow[l] &\arrow[l] 0
		\end{tikzcd}
	\end{equation}
	where each sheaf above is locally free, the rows are exact, and $\E' \cong \ker \varphi$, $\E''\cong \im \varphi$ and $\rk \E'' = \rk \F'$, $\deg \E'' \leq \deg \F'$.
\end{proposition}
\begin{proof}
	Define $\E'$ and $\E''$ as in the theorem statement, so that the top row is exact. Since $\F$ is locally free and since $X$ is covered by the spectra of Dedekind domains (which are hereditary), it follows that $\E'' = \im \varphi\subseteq \F$ is locally free. Now the sheaf $\coker \varphi$ is coherent, and hence if $U = \Spec A$ is an affine open subset of $X$, where $A$ is a Dedekind domain, then $\coker \varphi$ is isomorphic to $\widetilde{M}$ for some finitely-generated $A$-module $M$. Let $M'$ be the torsion submodule of $M$ and we define $\F''$ to locally be the sheaf $(M/M')^\sim$. It is not hard to see that this is well-defined. Observe that since $A$ is a Dedekind domain and $M/M'$ is a torsion-free module, it is also a projective module, and hence $\F''$ is locally free, by the Serre-Swan Theorem. We then define $\F'$ to be the kernel of $\F \rightarrow \F''$. Now since the map $\E'' \rightarrow \F''$ defined by composing the obvious maps is zero, by the universal property of kernels there is unique homomorphism $\E'' \rightarrow \F'$ making everything commute. This unique homomorphism has nonzero determinant, because $\varphi$ is nonzero, and $\E'' \rightarrow \F$ is the inclusion of the image. Finally, in light of the above lemma, it suffices to show that $\rk \E'' = \rk \F'$. But this follows directly from the observation that the local ring at any $k$-point is a DVR, and thus a PID, and so any finitely generated module splits into its torsion and free parts.
\end{proof}


\section{Stable and Semistable Bundles}
We now have the tools we need to define stability. To see why we require this definition in the first place though, we consider the moduli problem of vector bundles over $X$. To formalise:
\begin{definition}
	The \textit{moduli problem of vector bundles of signature $(n,d)$ on $X$} is the functor $\mathcal{V}_{n,d}: \Sch/k \rightarrow \Sets$ defined as follows: for any scheme $S/k$, we define a family of vector bundles over $S$ to be a coherent sheaf $\E$ on $X\times S$, flat over $S$ such that for any $s\in S(k)$, the fibre $\E_s $ defined to be the pullback of $\E$ along the map $s\times \id: \Spec k \times X \rightarrow S \times X$ is a locally free sheaf of signature $(n,d)$ on $X$. Two families $\E, \F$ are \textit{equivalent}, written $\E \sim \F$, if there is a line bundle $\L$ on $S$ such that \[\E \cong \F \otimes \pi_S^*\L. \] Note that dualising commutes with pullbacks, so this is an equivalence relation. We may also pull back families in the obvious way, and clearly equivalent families are pulled back to equivalent families. We then define $\mathcal{V}_{n,d}$ to be \[\mathcal{V}_{n,d}(S):= \{\text{families over }S\}/ \sim. \]
\end{definition}
However, even in the simple case of signature (2,0)-bundles over $\P^1 = \Proj k[x,y]$, there is a jump phenomenon:
\begin{example}
	We consider the following family $\E_t$ over the affine line $\A^1 = \Spec k[t]$: let $U_0$ (resp. $U_1$) be the open subset where $x$ (resp. $y$) does not vanish. Then we glue $\O_{U_0\times \A^1}^2$ and $\O_{U_1\times \A^1}^2$ on the overlap via the transition function \[g_{0,1}:= \begin{pmatrix*}
		\frac{x}{y} & t \\
		0 & \frac{y}{x}
	\end{pmatrix*} \in \GL_2(k[t, \frac{y}{x}, \frac{x}{y}])= \GL_2(\Gamma(U_0 \cap U_1, \O_{\A^1\times \P^1})). \] More precisely, on $U_0$ we have the standard basis $r_1, r_2\in \Gamma(U_0, \O_{U_0\times \A^1}^2)$, and on $U_1$ the standard basis $s_1, s_2\in \Gamma(U_0, \O_{U_1\times \A^1}^2)$. We then define isomorphism given by $s_1 \mapsto ( x/y)r_1$ and $s_2 \mapsto rs_1 + (y/x) s_2$, and we define $\E_t$ to be the gluing of the two sheaves on the overlap. Since $\E_t$ is locally free, it is flat over $\A^1 \times \P^1$ and by the transitivity of flatness is also flat over $\A^1$. 
	
	Now we claim that the fibre $\E_0$ is isomorphic to $\O_{\P^1}(-1) \oplus \O_{\P^1}(1)$, but every other fibre is isomorphic to $\O_{\P^1} \oplus \O_{\P^1}$ (and of course, these are not isomorphic, the latter has a nowhere-vanishing global section, but the former does not). To see this, first observe that clearly when $t = 0$, the transition map is just $\diag(y/x, x/y)$, which is the transition map of $\O_{\P^1}(-1) \oplus \O_{\P^1}(1)$. However, for a nonzero $\lambda$, we will define an isomorphism $\varphi: \E_\lambda \rightarrow \O_{\P^1} \oplus \O_{\P^1}$ as follows: let $e_1, e_2$ be the standard basis of $\O_{\P^1}\oplus \O_{\P^1}$. We define $\varphi$ to be the map 
	
	\begin{align*}
		r_1 &\mapsto \frac{y}{x}e_1 - e_2, \\ r_2 &\mapsto \lambda e_1
	\end{align*} 
	on $U_0$ and
	\begin{align*}
		s_1& \mapsto e_1 - \frac{x}{y}e_2,\\   s_2 &\mapsto \lambda e_2
	\end{align*}
	on $U_1$. To see that these glue, observe that $r_1 = (y/x)s_1$ and $r_2 = \lambda s_1 + (x/y)s_2$, and so on the overlap we have \[\varphi(\frac{y}{x}s_1) = \frac{y}{x}(e_1- \frac{x}{y}e_2) =\frac{y}{x}e_1 - e_2  \] \[\varphi(\lambda s_1 + \frac{x}{y} s_2) = \lambda e_1 - \frac{\lambda x}{y} e_2 + \frac{x}{y} \lambda e_2 = \lambda e_1 \] as desired. Finally, observe that $\varphi$ maps local free generators to local free generators, and thus this is an isomorphism.
\end{example}
\begin{remark}
	Of course, the fibres of the family $\E_t$ given above are canonically identified with the elements of the group \[\Ext^1(\O_{\P^1}(1), \O_{\P^1}(-1)) \cong \Ext^1(\O_{\P^1},\O_{\P^1}(-2) ) \cong H^1(\P^1, \O_{\P^1}(-2)) \cong k.\] {\color{red} We may generalise this idea as follows}: let $\E, \F$ be two vector bundles over $X$, so that $\Ext^1(\F, \E) \cong H^1(X, \fancyHom(\F, \E))$. Let $\{U_\alpha\}$ be a sufficiently fine open cover of $X$, let $e_{\alpha\beta}$ (resp. $f_{\alpha\beta}$) be a set of transition functions representing $\E$ (resp. $\F$), and let $\{g_{\alpha\beta}\}$ be a \v{C}ech 1-cocycle representing some element of $H^1(X, \fancyHom(\G, \F))$. 
\end{remark}
\begin{definition}\index{vector bundle ! slope}
	Let $\E$ be a vector bundle. The \textit{slope} of $\E$, denoted $\mu(\E)$, is defined to be \[\mu(\E):= \frac{\deg \E}{\rk \E} \]
\end{definition}
\begin{lemma}\label{subbundle-quotient-slope}
	Let \[0 \rightarrow \E \rightarrow \F \rightarrow \G \rightarrow 0 \] be a short exact sequence of vector bundles. Then $\mu(\E) \leq \mu(\F)$ if and only if $\mu(\G) \geq \mu(\F)$, with equality holding in one if and only if in the other.
\end{lemma}
\begin{proof}
	Trivial.
\end{proof}
And finally, the definition:
\begin{definition}\index{vector bundle ! stability}
	Let $\E$ be a vector bundle. Then $\E$ is \textit{stable} (resp. \textit{semistable}) if for every proper subbundle $\F$, we have 
	\begin{align*}
		\mu(\F) &\,< \mu(\E)\\
		&(\leq)
	\end{align*}
	$\E$ is \textit{polystable} if it is a direct sum of stable bundles of the same slope.
\end{definition}
We observe a few things. Firstly, note that this definition resembles, in some way, the Hilbert Mumford criterion; and indeed {\color{red} we will show later that this is not a coincidence}. Next observe that we can alternatively define stability in terms of quotient bundles: by Lemma \ref{subbundle-quotient-slope}, $\E$ is (semi) stable if and only if for every quotient bundle $\G$ we have:
\begin{align*}
	\mu(\E) &\,< \mu(\G)\\
	&(\leq)
\end{align*}
\par Let us mention some more basic properties of stability:
\begin{lemma}\label{stability-properties}
	Let $\E$ be a vector bundle. 
	\begin{enumerate}
		\item $\E$ is (semi) stable if and only if $\E \otimes \L$ is (semi) stable for every line bundle $\L$.
		\item If $\rk \E = 1$ (i.e. $\E$ is a line bundle) then $\E$ is always stable.
		\item If $\E'$ is another vector bundle then $\E \oplus \E'$ is not stable. It is semistable only if $\mu(\E) = \mu(\E')$ and both are semistable. 
		\item If $\varphi: \E\rightarrow \F$ is a nonzero morphism of vector bundles, and both bundles are semistable, then $\mu(\E)\leq \mu(\F)$.
		\item If $\E$ is stable, then it is simple (that is, $\End(\E) = k$).
	\end{enumerate}
\end{lemma}
\begin{proof}
	Note that every subbundle of $\E\otimes \L$ has the form $\F \otimes \L$ (indeed, if $\G$ is any subbundle of $\E \otimes \L$ we take $\F = \G \otimes \L^\vee$). Now observe \[\mu(\F \otimes \L) = \frac{\deg \F + \rk \F \deg \L}{\rk \F} = \mu(\F) + \deg \L\] and similarly for $\mu(\E \otimes \L)$ and hence \[\mu(\E \otimes \L) - \mu(\F \otimes \L) = \mu(\E) - \mu(\F) \] which proves (i). (ii) is trivial. To prove (iii), suppose without loss of generality $\mu(\E) \leq \mu(\E')$. Now \[\mu(\E \oplus \E') = \frac{\deg(\E) + \deg(\E')}{\rk(\E) + \rk(\E')} \leq \mu(\E') \] hence $\E \oplus \E'$ is not stable. If it is semistable, then equality must hold above, and clearly both must both be semistable, which proves (iii).
	
	To prove (iv), observe that by Proposition \ref{canonical-extension-proposition}, the map $\varphi$ factors as follows:
	\begin{equation}\label{proper-factorisation-equation}
		\begin{tikzcd}
			0 \arrow[r] & \cal{E'}\arrow[r] & \E \arrow[r] \arrow[d, "\varphi"]& \E'' \arrow[d]\arrow[r]& 0\\
			0  & \F''\arrow[l]& \F \arrow[l] & \F'\arrow[l] &\arrow[l] 0
		\end{tikzcd}
	\end{equation}
	where $\E' \cong \ker \varphi$, $\E''\cong \im \varphi$ and $\rk \E'' = \rk \F'$, $\deg \E'' \leq \deg \F'$. Since $\E$ and $\F$ are both stable, it follows 
	\begin{equation}\label{chain-of-slop-inequalities-equation}
		\mu(\E) \leq \mu(\E'') \leq \mu(\F') \leq \mu(\F)
	\end{equation}
	as desired. 
	
	Finally, to prove (v), suppose $\varphi: \E \rightarrow \E$ is an endomorphism. If $\varphi$ is zero, we are done. If not, we must have equality in (\ref{chain-of-slop-inequalities-equation}), and by the stability of $\E$ we have $\E = \E'' = \im \varphi$, so in particular $\varphi$ is an isomorphism. In particular, this implies that any nonzero endomorphism of $\E$ is an isomorphism. Now fix some $p\in X(k)$; we have an induced map of fibres $\E_p/\mf{m}_p\E_p\rightarrow \E_p/\mf{m}_p\E_p$, and since $k$ is algebraically closed, this map has an eigenvector, say $\lambda$. But $\varphi - \lambda: \E \rightarrow \E$ is no longer an isomorphism, hence must be zero, and thus $\varphi = \lambda$ as desired. 
\end{proof}

\section{The Construction}
Our first job is to construct a bounded family of semistable bundles; that is, a family that parameterises every semistable bundle of given signature, so that we may take a GIT quotient of the parameter space. To this end, we will first require the Riemann-Roch theorem:
\begin{theorem}[Riemann-Roch for Vector Bundles]\index{Riemann-Roch Theorem}
	Let $\F$ be a vector bundle on $X$ of signature $(n,d)$. Then \[\chi(\F) := h^0(\F) - h^1(\F) = d + n(1-g). \]
\end{theorem}
\begin{proof}
	We induct on $n$. For $n=1$, the result is classical, and is proven in, for example, pages 295-296 of \autocite{Hart}. Now supposing true up to some $n-1$, suppose $\F$ has rank $n$. Let $\L$ be a line subbundle of maximal degree. Then we claim $\F/\L$ must be locally free. Indeed, we may apply Proposition \ref{canonical-extension-proposition} taking $\varphi$ in the proposition statement to be the inclusion $\L\subseteq \F$, so that there is a nonzero map $\L \rightarrow \F'$ (where $\F'$ is as in the proposition statement). But both these bundles are line bundles, and hence are stable, and so $\deg \L \leq \deg \F'$, and since $\L$ is of maximal degree, equality must hold, and hence $\F/\L = \F''$ is locally free as claimed. Now applying the inductive hypothesis we have \[\chi(\F) = \chi(\L) + \chi(\F/\L) = \deg \L + 1 - g + (d - \deg \L) +(n-1)(1-g) = d + n(1-g) \] as desired.
\end{proof}
\begin{corollary}[Classical Riemann-Roch Theorem]
	For any line bundle $\L$ of degree $d$, we have \[h^0(\L) + h^0(\L^\vee\otimes \omega_X) = d + 1-g, \]  where $\omega_X$ is the canonical bundle (which is just the cotangent bundle in this case). In particular, the degree of the canonical bundle is $2g-2$.
\end{corollary}
\begin{proof}
	Combine the above theorem and the Serre duality theorem (\autocite[III, Corollary 7.7]{Hart}). 
\end{proof}
The importance of the Riemann-Roch theorem is that it allows us to relate the quantities we are interested in. More precisely, by the Serre vanishing theorem, for any coherent sheaf $\F$, and any ample line bundle $\L$, we have $H^i(X, \F\otimes \L^{\otimes m}) = 0$ for any sufficiently large $m$ and any $ i > 0$. Moreover, by the definition of ampleness, we know that $\F\otimes \L^{\otimes m}$ is generated by global sections, for any sufficiently large $m$. In particular, if $\F$ is locally free then the Riemann-Roch theorem tells us exactly how many global sections generate $\F\otimes \L^{\otimes m}$. The main issue is that the \enquote{sufficiently large} criterion depends on $\F$ and $\L$. This is remedied by the following result:
\begin{lemma}\label{bounding-family}
	Let $\E$ be a semistable vector bundle of signature $(n,d)$.
	\begin{enumerate}
		\item If $d >n(2g-2)$ then $H^1(X, \E) = 0$.
		\item If $d > n(2g-1)$ then $\E$ is generated by global sections.
	\end{enumerate} 
\end{lemma}
\begin{proof}
	This follows the proof given in \autocite[p. 68]{ModuliNotes}. Suppose for contradiction $H^1(X, \E)\neq 0$. By the Serre duality theorem, we have \[H^1(X, \E)\cong H^0(X, \E^\vee \otimes \omega_X) = \Hom(\E, \omega_X),\] where $\omega_X$ is the canonical bundle (which is just the cotangent bundle here). This means that there is a nonzero homomorphism $\varphi: \E \rightarrow \omega_X$. Now by Lemma \ref{stability-properties} (iv), we have \[2g-2 = \frac{n(2g-2)}{n} <\frac{d}{n} = \mu(\E) \leq \mu(\omega_X) = 2g-2 \] which is a contradiction. This proves (i). 
	
	To prove (ii), we note that since $k$ is algebraically closed, we need only check the stalk is generated by global sections at $k$-points. So let $p$ be a $k$-point, with local ring $\O_{X,p}$ and maximal ideal $\mf{m}_p$. The composition $\Ox(U) \rightarrow \O_{X,p} \rightarrow \O_{X,p}/\mf{m}_p$ induces the following short exact sequence of sheaves:
	\[ 0 \rightarrow \mathcal{I}_p \rightarrow \Ox \rightarrow k_p \rightarrow 0, \] where $k_p$ is the skyscraper sheaf $k$ sitting over $p$, and $\mathcal{I}_p$ is the kernel. Since $\E$ is locally free, tensoring is exact, and moreover since it is of rank $n$, and $k_p$ is a skyscraper sheaf, it follows $k_p\otimes \E\cong k_p \otimes \Ox^n \cong k_p^n$. Thus tensoring the above with $\E$ we have 
	\begin{equation}\label{skyscraper-ses}
		0 \rightarrow \mathcal{I}_p\otimes \E \rightarrow \E \rightarrow k_p^n \rightarrow 0.
	\end{equation}  
	Now we claim that $\mathcal{I}_p\cong \L(-p)$. To see this, let $U = \Spec A$ be an open affine subset. If $p\notin U$, then $\mathcal{I}_p|_U = \Ox|_U$. Otherwise, $p$ is cut out by some $f\in A$, and hence $\mathcal{I}_p|_U = f\Ox|_U$. But this is exactly the definition of $\L(-p)$, as claimed. Now observe that $\mathcal{I}_p\otimes \E = \E \otimes \L(-p)$ is semistable and has degree $n(2g-2)$ and thus by part (i) we have $H^1(X, \E\otimes \L(-p)) = 0$. Now taking cohomology of (\ref{skyscraper-ses}), it follows we have a surjection \[H^0(X, \E ) \rightarrow H^0(X, k_p^n) = k^n. \] Finally, we apply Nakayama's lemma (\autocite[Corollary 4.8]{EisAlg}) on the local ring $\O_{X,p}$ to deduce that the map $H^0(X, \E)\rightarrow \E_p$ is surjective too.
\end{proof}
The consequence of the above lemma combined with the Riemann-Roch theorem is that every semistable vector bundle of signature $(n,d)$ for $d$ sufficiently large is a quotient of $\Ox^{d+n(1-g)}$. Now it just so happens that there is a scheme that parameterises all quotients of a given coherent sheaf, with certain constraints. To define it however, we require the following result:
\begin{proposition}
	Let $Y$ be a projective variety, let $\F$ be a coherent sheaf on $Y$ and let $\O(1)$ be a very ample line bundle. For any $m\in \Z$, write $\F(m):=\F\otimes \O(m):= \F \otimes \O(1)^{\otimes m}$. Then the map \[P:m\mapsto \chi(\F(m)) \] coincides with a polynomial with rational coefficients.
\end{proposition}
\begin{proof}
	Embed $Y$ into $\P^n = \Proj k[x_0,...,x_n]$ via $\O(1)$. We induct on $r=\dim \Supp \F$. If $r = 0$, then $\F$ is supported on a discrete subset, and hence tensoring with $\O(1)$ does nothing, and so $P$ is constant, equal to $\chi(\F)$. Now supposing true for some $r > 0$, we suppose $\dim \Supp \F = r+1$, and write $M$ for the graded module $M:= \bigoplus_{m\in \Z} H^0(X, \F(m))$, so that $\widetilde{M} \cong \F$ (\autocite[II. Proposition 5.15]{Hart}). We note that the map $M(-1)\rightarrow M$ given by multiplication by $x_i$ induces the following short exact sequence of sheaves \[ 0 \rightarrow \E \rightarrow \F(-1)\rightarrow \F \rightarrow \G \rightarrow 0 \] where $\E$ and $\G$ are the kernel and cokernel respectively. Now if $\dim \Supp \E = r+1$, then {\color{red} TO FINISH }.
\end{proof}
\begin{definition}\index{Hilbert polynomial}
	The polynomial $P$ above is known as the \textit{Hilbert polynomial} of $\F$ with respect to $\O(1)$. 
\end{definition}
\begin{example}
	Take $Y = \P^r$, take $\F = \Oy$ and take $\O(1)$ to be the usual twisting sheaf of Serre. Observe that if $m \geq 0$ then $h^0(\O(m)) = \binom{m+r}{r},$ and {\color{red}$h^i(\O(m)) = 0$} for all $i > 0$, and hence\[ P = \binom{z+r}{r}:= \frac{1}{r!}\prod_{i = 1}^{r} (z+i)\in \Q[z].\]
\end{example}
\begin{example}
	Let $\Ox(1)$ be a very ample line bundle on $X$ of degree $m$, and let $\E$ be a of signature $(n,d)$. By the Riemann-Roch theorem we know \[\chi(\E(m)) = d+nm + n(1-g), \] and so $\E$ has Hilbert polynomial $P = d + nz + n(1-g)\in \Q[z]$ with respect to $\Ox(1)$. Conversely, if $\F$ is a vector bundle with Hilbert polynomial $P = d + nz + n(1-g)\in \Q[z]$, then $\chi(\F) = d+n(1-g)$ and $\chi(\F(1)) = d+nm+n(1-g)$, and so we know $\rk(\F) = n$ by Corollary \ref{twisting-by-line-bundle-degree} and $\deg(\F) = d$ by Riemann-Roch. In particular, the data of the Hilbert polynomial on $\F$ (with respect to $\Ox(1)$) is equivalent to the data of the signature of $\F$.
\end{example}
The last example illustrates that the degree of $\E$ is encoded by the Hilbert polynomial. We now have what we need to define a bounded family. First, we define the following moduli problem:
\begin{definition}
	Let $\F$ be a coherent sheaf on a projective scheme $Y$ with an embedding $Y\subseteq \P^n$, and let $P\in \Q[z]$ be a numerical polynomial. For any scheme $S$ of finite type over $k$, a \textit{family of quotients of $\F$ with Hilbert polynomial $P$ parameterised by $S$} is a coherent sheaf $\G$ on $S \times Y$, flat and with proper support over $S$, equipped with a surjective map $\F_S \rightarrow \G$, where $\F_S$ is the pullback of $\F$ to $S\times X$ via the projection, such that all closed fibres $\G_p$ are coherent quotient sheaves of $\F$ with Hilbert polynomial $P$. Two families over $S$ are \textit{equivalent} if they have the same kernel. It is clear how families pull back, and so we have a functor \[\mathcal{Q}uot^{P}_Y(\F): \sf{C} \rightarrow \Sets, \] where $\sf{C}$ is the category of schemes of finite type over $k$, sending a scheme $S$ to the equivalence classes of families over $S$.
\end{definition}
\begin{theorem}[Grothendieck]
	The $\mathcal{Q}uot^{P}_Y(\F)$ functor is representable. 
\end{theorem}
\begin{definition}
	Let $P\in \Q[z]$ be a numerical polynomial. The \textit{quot scheme of $\F$ with respect to $P$}, denoted $\Quot_Y^P(\F)$, is the fine moduli space of $\mathcal{Q}uot^{P}_Y(\F)$. A \textit{Hilbert scheme} is a Quot scheme of the form $\Quot_Y^P(\Oy)$, which we will simply denote $\Hilb_Y^P$.
\end{definition}
The idea of the construction, which can be found in \autocite{HilbQuot}, is to define an injective natural transformation from $\mathcal{Q}uot^{P}_Y(\F)$ into a certain Grassmannian functor, and show that this defines a scheme structure. However, this construction is far beyond the scope of this thesis, so we will be content with our examples from Chapter 1, where we showed that $\Quot_{\Spec k}^1(k^n) = \P^n$ and $\Hilb_{\P^2}^{2z+1} = \P^5$.

In particular, from Lemma \ref{bounding-family}, it follows that all semistable vector bundles of signature $n,d$ with $d > n(2g-1) =: N$ are found in the universal family over the scheme $Q = \Quot_X^{d+nz +n(1-g)}(\Ox^N).$ However, $Q$ also parameterises other sheaves, which we would like to ignore. The following result allows us to do so:
\begin{proposition}
	Let $S$ be a scheme and $\F$ a family of quotients of $\Ox^N$ with Hilbert polynomial $d+nz +n(1-g)$ over $S$. Then the locus $s$ of locally free semistable quotients $s: \Ox^N \rightarrow \F_s$ such that $H^0(s)$ is an isomorphism $k$-vector spaces is open.
\end{proposition}
\begin{proof}
	\autocite[p. 35 and p. 45]{HuyLehn}.
\end{proof}
In particular, taking $S = Q$ and $\F$ as the universal family, say $\mf{E}$ over $Q$, we have an open subscheme $Q^{ss}$, and a restricted family $\mf{E}^{ss}$ whose fibres over $k$-points are all semistable vector bundles of signature $(n,d)$, and conversely, every such vector bundle is contained in this family, by Lemma \ref{bounding-family}. Such a family is known as a \textit{bounded family}, and our plan of attack from here is clear: show that this family is locally universal, find a group action parameterising isomorphic fibres, linearise this action and take a GIT quotient. We will sketch these steps below:



\part{Analytic Theory}

\chapter{Hodge Theory}
\par We have constructed the spaces $V^s_{n,d}$ of stable vector bundles on a curve. In the second part, we turn our attention to the case $k = \C$ (in particular, our vector bundles are \textbf{complex} vector bundles). This choice is necessary, as it gives us access to methods from analysis and differential geometry. In this setting, it turns out $V^s_{n,d}$ has a nice characterisation, known as the \textit{Narasimhan-Seshadri Correspondence}, or the \textit{Narasimhan-Seshadri Theorem}, which will be stated in the next chapter. In this chapter, we will be concerned with the theory of complex manifolds, and in particular various topics such as Hodge theory and Sobolev theory will be discussed. 
\section{Hodge Theory}

The basic idea of Hodge theory is to find certain repre

\chapter{The Narasimhan Seshadri Theorem}	
%We have constructed the spaces $V^s_{n,d}$ of stable vector bundles on a curve. The next question to ask is \enquote{what does this space look like}? It turns out in the case $k = \C$ (This choice is necessary, as it gives us access to methods from analysis and differential geometry), there is a correspondence known as the \textit{Narasimhan-Seshadri Correspondence}, or the \textit{Narasimhan-Seshadri Theorem}, which asserts a bijection between stable holomorphic vector bundles (which is equivalent to algebraic vector bundles, by GAGA) of degree zero and irreducible unitary representations of the fundamental group, which turns out to be an homeomorphism of moduli spaces, {\color{red}although we do not show this. }So let us fix once and for all a compact Riemann surface $X$ of genus $g$ (which may be identified as the $\C$-valued points of a nonsingular projective curve over $\C$, given the complex topology). 
In Part I, we have contructed the moduli spaces $V^s_{n,d}$ as a projective GIT quotient. In Part II, we will give another na\"ive moduli space construction for the underlying na\"ive problem of $\mathcal{V}^s_{n,0}$ in the special case $k = \C$. Specifically, if $X$ is a compact Riemann surface of genus $g$ (which may be identified with the $\C$-points of a nonsingular projective curve over $\C$) which we will fix in this chapter, one can identify stable vector bundles of degree zero on $X$ with two other spaces, the basic result being the following:
\begin{theorem}[Narasimhan-Seshadri, 1965]
	Let $X$ be a compact Riemann surface of genus $g$ and suppose $g\geq 2$. Then there is a bijection between $V^s_{n,0}$ with irreducible representations $\pi_1(X) \rightarrow U(n)$ up to conjugation.
\end{theorem}
This is first given as Corollary 1 of \autocite{NS}. Since then, Donaldson gave a different proof in \autocite{Donaldson} by studying unitary connections, and using a correspondence theorem known as the \textit{Riemann-Hilbert Correspondence}. His result is stated as follows:
\begin{theorem}[Donaldson, 1983]
	An indecomposable holomorphic bundle $\E$ over $X$ with a hermitian metric $h$ is stable if and only if there is a unitary connection on $\E$ with curvature equal to a constant multiple of the volume form. Such a connection is unique up to isomorphism.
\end{theorem}
In this chapter, we will use these bijections to give the space of stable bundles another geometric structure, specifically a manifold structure, inherited from the character space $\Hom(\pi_1(X), U(n))$. It turns out that this second structure is homeomorphic to $V^s_{n,0}(\C)$, the latter given the usual complex topology, but very unfortunately we will not be proving this.
\\\\
\indent Of course, since we are moving into analytic territory, some comments are in order. \textit{Open} and \textit{closed} will always mean with respect to the usual complex topology. We will make use of the correpondence between vector bundles and locally free sheaves (of an appropriate structure sheaf) without comment, and we will also use without comment the correspondence between holomorphic vector bundles on $X$ and algebraic vector bundles on $X$. 

\section{Holomorphic Structures on a Smooth Bundle}
\indent The goal of this section is to study the space of holomorphic bundles that restrict to a given smooth bundle. Indeed, as we will see, the degree of a holomorphic bundle is actually a smooth invariant, and in fact, along with the rank, completely classifies $E$! Thus $V_{n,d}(\C)$, the isomorphism classes of all holomorphic bundles with signature $(n,d)$, is equal to the set of holomorphic structures on $E$, up to isomorphism. It turns out that this space is, in turn, canonically identified with the space of unitary connections (to be defined) on $E$, and this correspondence, known as the \textit{Chern Correspondence}, gives us a tool to turn the study of holomorphic bundles into the study of connections. 
\subsection{The Chern Correspondence}
Recall some notation. Let $\Omega^{p,q}_X$ denote the sheaf of smooth $(p,q)$-forms on $X$, and let \[\Omega^{p,q}_E:= \Omega^{p,q}_X \otimes E.\] Note that $\Omega^0_E \cong E$.
\begin{definition}\index{Dolbeault operator}
	A \textit{Dolbeault operator} on $E$ is a homomorphism of abelian sheaves (in particular, \textbf{NOT} as $\Ox$-modules) \[\delbar_E: \Omega^0_E \rightarrow \Omega^{0,1}_E\] such that for any smooth $f\in C^\infty(U)$ and local section $s\in \Omega^0_U$, we have \[\delbar_E(fs) =  \delbar(f)\otimes s + f\delbar_E(s).\]
\end{definition}
\begin{example}
	Let $E$ be the trivial bundle $E = X \times \C^n$. Then the usual $\delbar$ operator is a Dolbeault oeprator
\end{example}
\begin{example}
	Let $\nabla$ be any connection on $E$. Composing $\nabla$ with the projection to its $(0,1)$-component we obtain a Dolbeault operator. This is often known as the \textit{$(0,1)$-component of} $\nabla$.
\end{example}
\begin{example}
	Let $\E$ be a holomorphic bundle whose underlying smooth bundle is $E$. We may think of $\E$ as $E$ equipped with a collection of distinguished local frames, which we deem to be holomorphic. Then there is a natural Dolbeault operator, known as the \textit{canonical Dolbeault operator}, characterised by $\delbar_E(s) = 0$ for any homomorphic section $s$. To see that this is well-defined, let $u: U \rightarrow E$ be a smooth section. Covering $U$ with sufficiently small open subsets $\{U_\alpha\}$ we may assume there are local holomorphic frames $\{(s_i)_\alpha\}$. Now we can write $u_\alpha:=u|_{U_\alpha} = \sum a_i s_i$ where $s_\alpha = (s_i)_\alpha$ is a holomorphic frame on $U_\alpha$ and $a = (a_i)$ is smooth. Then we see \[\delbar_E(u_{\alpha}) = \delbar_E(\sum a_i s_i) = \sum \delbar(a_i)\otimes s_i \] Repeating on $U_\beta$ with local holomorphic frame $t_\beta = (t_i)_\beta$ such that $u_\beta = \sum b_i t_i$ we have $\delbar_E(u_\beta) = \sum \delbar(b_i)\otimes t_i$. Now if $g = g_{\alpha\beta}$ is the transition map, letting $g_{ij}$ denote the $i,j$-th entry of $g$, we observe \[\sum_j \delbar(b_j) \otimes t_j = \sum_j\sum_i \delbar(b_j)\otimes (g_{ij}s_i)=  \sum_i \sum_j\delbar(g_{ij}b_j)\otimes s_i = \sum_i \delbar(a_i)\otimes s_i \] as desired (note that since $g$ is holomorphic we have $\delbar(g) = 0$) and hence by the sheaf axioms this defines $\delbar_E(u)$ uniquely. 
\end{example} 
\begin{caution}
	However, it is important to note that this Dolbeault operator on $E$ is in fact dependent on the \textbf{choice} of frames, and not just the isomorphism class of $\E$. Indeed, we will see very soon that there are different Dolbeault operators which can give rise to isomorphic holomorphic bundles, and in fact we will describe exactly when two distinct Dolbeault operators give rise to the same holomorphic bundle.
\end{caution}
In fact, the converse of the above example is true for Riemann surfaces: if $\delbar_E$ is a Dolbeault operator and there exists an open cover $U_\alpha$ with frames $s_\alpha$ such that $\delbar_E(s_\alpha) = 0$, then it is not hard to show that the transition maps are holomorphic and hence the $\{s_\alpha\}$ define a holomorphic structure on $E$. Such a Dolbeault operator is said to be \textit{integrable}. It can be shown (\autocite[p.555]{AtiBot}) that every Dolbeault operator on a Riemann surface is integrable. Hence Dolbeault operators parameterise holomorphic structures on $E$ (however, we will soon see that they actually over-parameterise holomorphic structures).
\\\\
Next we recall the following definition:
\begin{definition}
	A \textit{Hermitian metric} on $E$, denoted $h$ is the assignment of a complex inner product $h_p(\cdot, \cdot): E_p^2 \rightarrow E_p$ to every $p\in X$ such that for any smooth sections $s,t$ we have that $p \mapsto h_p(s,t)$ is smooth. A \textit{hermitian bundle} is a vector bundle $E$ equipped with a hermitian metric. A \textit{morphism} of hermitian bundles is a smooth morphism of bundles $\varphi: E \rightarrow F$ such that $h_F(\varphi(s),\varphi(t)) = h_E(s,t)$ for any sections $s,t$ if $E$. A hermitian automorphism is known as a \textit{unitary gauge transformation}, and the group of all such automorphisms is known as the \textit{unitary gauge group}. 
	
	A frame $(s_i)_\alpha$ is \textit{unitary} if $h(s_i, s_j) = \delta_{ij}$. A \textit{unitary connection} is a connection $\nabla$ such that \[dh(s,t) = h(\nabla s, t )+ h(s, \nabla t)\] for any smooth sections $s,t$. Given a local frame $(s_i)_\alpha$ on $U_\alpha$, we define the \textit{local matrix of} $h$, denoted $h_\alpha$ such that $(h_\alpha)_{ij}:= h(s_i, s_j)$. 
\end{definition}
\begin{lemma}\label{unitary-iff-skew-hermitian}
	Let $\nabla$ be a connection. Then $\nabla$ is a unitary connection if and only if for any unitary frame $(s_i)_\alpha$, the local 1-form $\omega_\alpha$ is skew-Hermitian.
\end{lemma}
\begin{proof}
	Suppose $\nabla$ is unitary. Note that for any $i,j$, we have $$0 = dh(s_i, s_j) = h(\sum_k(\omega_\alpha)_{ki} s_k, s_j) + h(s_i, \sum_k\omega_\alpha)_{kj} s_k) = (\omega_\alpha)_{ij} + (\overline{\omega}_\alpha)_{ji}$$ and thus $(\omega_\alpha)_{ij} =- (\overline{\omega}_\alpha)_{ji}$ as required.
	
	Conversely, suppose $\omega_\alpha$ is skew-Hermitian and let $s = \sum a_i s_i$ and $t = \sum b_i s_i$ be local sections. Then we can check
	\begin{align*}
		h(\nabla(s), t) + h(s, \nabla(t)) &= h(\nabla(\sum a_i s_i), \sum b_i s_i) + h(\sum a_i s_i, \nabla(\sum b_i s_i))\\
		&=h(\sum_i da_i s_i + a_i\sum_j (\omega_\alpha)_{ij} s_j, \sum_i b_i s_i) \\
		&+ h(\sum_i a_i s_i, \sum_i db_i + b_i \sum_j((\omega_\alpha)_{ij}, s_j))\\
		&= (\sum_i b_ida_i + a_idb_i) + (\sum_j\sum_i a_i (\omega_\alpha)_{ij} b_j) + (\sum_j\sum_i a_j (\overline{\omega}_\alpha)_{ij}b_i)\\
		&= (\sum_i b_ida_i + a_idb_i) + (\sum_j\sum_i a_i (\omega_\alpha)_{ij} b_j) - (\sum_j\sum_i a_j (\omega_\alpha)_{ji}b_i)\\
		&= \sum_i b_ida_i + a_idb_i\\
		&= dh(s,t)
	\end{align*}
	as desired.
\end{proof}
It can be shown by a partition of unity argument (\autocite[III Theorem 1.2]{Wells}) that hermitian metrics exist on any bundle. Similarly, a standard Gram-Schmidt argument will show that smooth unitary frames always exist locally, and finally, we will show that unitary connections always exist:
\begin{theorem}[Chern Correspondence]\index{Chern correspondence}
	Let $\E$ be a holomorphic vector bundle, let $\delbar_E$ be a Dolbeault operator on $E$ giving rise to $\E$, and $h$ a hermitian metric. Then there is a unique unitary connection $\nabla$ on $\E$ with $(0,1)$ component $\delbar_E$. Moreover, if $(s_i)_\alpha$ is a local holomorphic frame, then this connection is described by $\omega_\alpha =(\partial h_\alpha) h_\alpha^{-1}$
\end{theorem}
\begin{proof}
	We first prove uniqueness. Suppose $\nabla$ is such a connection, and let $(s_i)_\alpha$ be a holomorphic frame defined on $U_\alpha$. Then the corresponding matrix of 1-forms $\omega_\alpha$ satisfies \[\nabla(s_i) = \sum_j (\omega_\alpha)_{ij}s_j.\] Observe that since all the $s_i$ are holomorphic, all the entries of $\omega_\alpha$ must be of type $(1,0)$. Now we compute: 
	\begin{align*}
		dh(s_i, s_j) &= h(\nabla s_i, s_j) + h(s_i, \nabla s_j) \\
		&= h(\sum_k (\omega_\alpha)_{ik}s_k, s_j) + h(s_i, \sum_k(\omega_\alpha)_{jk}s_k) \\
		&= \sum_k(\omega_\alpha)_{ik} h(s_k, s_j) + \overline{(\omega_\alpha)_{jk}} h(s_i,s_k)
	\end{align*}
	But $dh(s_i, s_j)  = \partial (h_\alpha)_{ij} + \delbar (h_\alpha)_{ij}$, and thus comparing types we must have $(\partial h_\alpha)_{ij} = \sum_k(\omega_\alpha)_{ik} (h_{\alpha})_{kj}$ and letting $i,j$ vary, we observe \[ \omega_\alpha h_\alpha = \partial h_\alpha \] as required. This proves uniqueness.
	
	To prove existence, we define the connection to be $\omega_\alpha:=  (\partial h_\alpha)h_\alpha^{-1}$ for any holomorphic frame $(s_i)_\alpha$, and extend by the Leibniz rule. By the proof of uniqueness, this satisfies the properties in the theorem, thus we just need to check that this is well-defined. So let $(t_i)_\beta$ be another holomorphic frame, and suppose $g$ satisfies $t_i = \sum g_{ij}s_j$. Then it is not hard to see \[h_\beta = gh_\alpha g^*\] where $g^*$ is the conjugate transpose of $g$. By Proposition \ref{connection-transformatino-rule}, it suffices to show that $\omega_\beta = (dg)g^{-1}+ g\omega_\alpha g^{-1}$. We compute:
	\[\partial h_\beta = \partial (gh_\alpha g^*) = (\partial g) h_\alpha g^* + g(\partial h_\alpha)g^* + gh_\alpha (\partial g^*) = (\partial g) h_\alpha g^* + g(\partial h_\alpha)g^* \]  since $g$ is holomorphic. Thus \[\omega_\beta = (\partial h_\beta )h_\beta^{-1} = ((\partial g) h_\alpha g^* + g(\partial h_\alpha)g^* )(g^*)^{-1}h_\alpha g^{-1} = (\partial g)g^{-1}+ g(\partial h_\alpha)h_{\alpha} ^{-1}g^{-1} = (dg)g^{-1}+ g\omega_\alpha g^{-1} \] as desired. 
\end{proof}
\begin{definition}\index{Chern connection}
	The unitary connection in the above theorem is the \textit{Chern connection}.
\end{definition}
\begin{remark}
	Again, it is important to note that different unitary connections could give rise to the same holomorphic bundle. This is why we will often use the phrase \enquote{a Chern connection}.
\end{remark}
We sum up the above results as follows: as established, there is a 1-1 correspondence between holomorphic structures and Dolbeault operators. Now the natural follow-up question to that is given a Dolbeault operator $\delbar_E$, is there a canonical connection we can put on $E$ with (0,1)-component $\delbar_E$? The Chern correspondence answers this in the affirmative, with the choice of a hermitian metric placed on $E$. Thus, in order to study holomorphic structures, we can study unitary connections instead. 

Finally, we describe the gauge group action on the space of connections. To begin, we consider the gauge group action on the space of Dolbeault operators. So let $u$ be a gauge transformation, and $\delbar_E$ a Dolbeault operator. We define \[(u \cdot \delbar_E)(s):= u\delbar_E(u^{-1}(s))\] it is easy to check that this is a group action that results in a Dolbeault operator. 
\begin{proposition}
	Let $\delbar_1, \delbar_2$ be two Dolbeault operators on $E$, and $\E_1, \E_2$ their associated holomorphic bundles. Then $\E_1 \cong \E_2$ if and only if there is a gauge transformation $u$ such that $\delbar_2 = u \cdot \delbar_1$. Moreover, $u: \E_1 \rightarrow \E_2$ is one such isomorphism.
\end{proposition}
\begin{proof}
	Suppose firstly that $\delbar_2 = u\cdot\delbar_1 = u\delbar_1u^{-1}$. Now let $s$ be a holomorphic section of $\E_1$. Observe \[0 = \delbar_1(s) = \delbar_1(u^{-1} u(s)) = u\delbar_1(u^{-1}u(s)) = \delbar_2(u(s))\] and hence $u(s)$ is a holomorphic section of $\E_2$. By the same argument, if $t$ is a holomorphic section of $\E_2$, then $u^{-1}(t)$ is holomorphic in $\E_1$ as desired.
	
	Conversely, suppose $\E_1 \cong \E_2$ and let $u: \E_1 \rightarrow \E_2$ be an isomorphism. Now let $s_\alpha = (s_i)_\alpha$ be a holomorphic frame of $\E_1$; whence by the above calculation we have $u\delbar_1(u^{-1}u(s_i)) = 0$. But also $\delbar_2(u(s_i))= 0$ (since $u(s_i))$ is holomorphic). Since $u\delbar_1u^{-1}$ and $\delbar_2$ agree on a collection of frames on an open cover of $X$, they must be equal.
\end{proof}
We now want to extend this to the space of all unitary connections, so let $\nabla$ be a unitary connection with Dolbeault operator $\delbar_E$. By the Chern correspondence, the Dolbeault operator $u\cdot \delbar_E$ corresponds to a unique unitary connection; thus we simply need to find a unitary connection with $(0,1)$-component $u\cdot \delbar_E$. To this end, observe that as in the space of all connections, the space of unitary connections is an affine space. The underlying vector space is the subspace of $H^0(X, \Omega^1_{\fancyEnd(E)})$ consisting of 1-forms with values in a skew-hermitian endomorphism; that is, an endomorphism $F$ such that \[h(F(s), t) + h(s, F(t)) = 0\] Observe then, that \[-(\delbar_{\fancyEnd E}u)u^{-1} +  ((\delbar_{\fancyEnd E}u)u^{-1} )^*\] is clearly skew-hermitian, (where $(\delbar_{\fancyEnd E}u)(s):= \delbar_E(u(s))- u\delbar_E(s)$ is the \textit{induced Dolbeault operator}) and $u^*$ satisfies $h(us, t) = h(s, u^*t)$) and has $(0,1)$-component equal to \[(-(\delbar_{\fancyEnd E}u)u^{-1} +  (\delbar_{\fancyEnd E}u)u^{-1} )^*)^{0,1} = -(\delbar_{\fancyEnd E}u)u^{-1} \] since $((\delbar_{\fancyEnd E}u)u^{-1} )^*$ is of type (1,0). Hence $u\cdot \nabla$ defined by
\begin{equation}\label{gauge-group-unitary-connection}
	(u\cdot \nabla)(s) := \nabla(s) -(\delbar_{\fancyEnd E}u)u^{-1}(s) +  ((\delbar_{\fancyEnd E}u)u^{-1} )^*(s)
\end{equation}
is a unitary connection, and its associated Dolbeault operator is equal to  \[\delbar_E(s) -(\delbar_{\fancyEnd E}u)u^{-1}(s) = \delbar_E(s) - \delbar_E(uu^{-1}(s)) + u\delbar_E(u^{-1}(s)) = u\delbar_E(u^{-1}(s)) \] as desired. Hence defining the action of the gauge group by the formula in (\ref{gauge-group-unitary-connection}) (and it is easy to check that this is indeed a group action) extends the gauge group action on Dolbeault operators, and in particular two unitary connections induce isomorphic holomorphic structures if and only if they lie in the same gauge orbit. In summary, we have the following bijection:\[\{\text{Holomorphic structures on }E\}/\text{isomorphism} \leftrightarrow \{\text{unitary connections on }E \}/ \text{gauge equivalence}. \]
{\color{red} We conclude this section with a discussion about the Sobolev completions of various spaces we have been working with, and how the above bijection extends. To begin, we first recall that since $X$ is a Riemann surface, it is K\"ahler with its Fubini-Study form. This fixes a volume form $\vol$ and hence a Hodge star product $\star$. The \textit{Hodge inner product} is defined on $\Omega^p_X$ as \[\langle\alpha, \beta\rangle := \int_X \alpha \wedge \star \beta. \] Completing this with }
\subsection{Degree of a Smooth Bundle}
In this section, we will be giving a classification of all smooth vector bundles on $X$. In particular, we will show that the degree of a bundle is actually a \textbf{smooth} invariant, and in fact, along with the rank, the only smooth invariant there is! Thus for each signature $(n,d)\in \N \times \Z$, there is a unique smooth bundle with that signature. The vehicle for showing this is the invariant known as the \textit{first Chern class}. To define it, we fix a smooth bundle $E$. We begin with a result:
\begin{lemma}
	Let $\nabla_1, \nabla_2$ be connections on $E$ with curvature forms $\Theta_1, \Theta_2$ respectively. Then $\tr \Theta_1$ and $\tr \Theta_2$ are cohomologous.
\end{lemma}
\begin{proof}
	Since $X$ is a Riemann surface, clearly $\tr \Theta_i$, being a two-form, is closed, thus the statement makes sense. Now $\nabla_1-\nabla_2$ is a global $\fancyEnd(E)$-valued 1-form; call it $A$. Let $\omega_1, \omega_2$ be respective local 1-forms of $\nabla_1$ and $\nabla_2$ on a local frame. Then locally, we have \[\Theta_2 = d\omega_2 + \omega_2\wedge \omega_2 = d(\omega_1+A) + (\omega_1 + A)\wedge (\omega_1 + A) = \Theta_1 + dA + A\wedge A + \omega_1 \wedge A + A \wedge \omega_1 \] hence \[\tr(\Theta_2) = \tr(\Theta_1) + \tr(dA) + \tr(A\wedge A) + \tr(\omega_1 \wedge A + A \wedge \omega_1). \] It is not hard to show that $\tr(A\wedge A)$ and $\tr(\omega_1 \wedge A + A \wedge \omega_1)$ are both zero, from the antisymmetry of the wedge product. Since $A$ is a global form, the $dA$ glue to a global exact 2-form and hence the result follows.
\end{proof}
\begin{definition}\index{Chern class! first}
	The \textit{first Chern class} of $E$ is defined to be the cohomology class \[ c_1(E):=\left[\tr(\frac{i}{2\pi} \Theta)\right]\in H^2_{\text{DR}}(X) \] where $\Theta$ is the curvature form for any connection. By the above lemma, this does not depend on the connection.
\end{definition}
The key theorem we will be proving in this section is the following:
\begin{theorem}\label{chern-weil-degree}
	For any holomorphic bundle $\E$ with underlying smooth bundle $E$, we have \[\int_X c_1(E) = \deg(\E),\] where $X$ is given the standard orientation $idz \wedge d\bar{z}$ for any holomorphic coordinate $z$.
\end{theorem}
Let us take a moment to appreciate this result. We defined the degree of a line bundle $\L$ on a curve $X$ over an algebraically closed field $k$ to be the degree of its corresponding divisor class, and we defined the degree of a vector bundle $\E$ to be the degree of $\det(\E)$. Note that this is purely algebraic. The theorem states that in the case $k = \C$, where we have access to analytic tools, when we equip $\E$ with a connection (any connection, in fact), take its curvature, take the trace of the curvature, multiply by $i/2\pi$ and integrate it, we get, not only an integer, but the same integer representing the degree of the divisor associated to its determinant bundle!

The way to prove this is to reduce to the case of line bundles, and to do this, we present the following well-known result:
\begin{theorem}[Structure Theorem of Smooth Vector Bundles]
	There is a diffeomorphism \[E \cong \det E \oplus \Ox^{\rk E - 1}. \]
\end{theorem}
Before we give the proof, we first recall it can be shown (\autocite[II, Theorem 15.3]{Bredon}) that every section $s$ has a section $s'$ which intersects $s$ transversally (i.e. if $s$ and $s'$ intersect at $P\in X$, then $\im s_{*,P}\oplus \im s'_{*,P} = T_{s(P)}E$; in other words the images of the differential of $s, s'$ at $P$ generate the tangent space of $s(P)$ in $E$). 
\begin{proof}
	We proceed by induction on the rank, with the rank 1 case being trivial. Now supposing $rk E > 1$, we observe that by the above there is a section $s$ which intersects the zero section transversally. However, the real dimension of $E$ is $\dim_\R E = 2+ 2\rk E > 4$ and thus $s$ and the zero section cannot intersect at all (otherwise the space spanned by the images of their differential is at most 4), or in other words $s$ is nonvanishing. In particular, the bundle spanned by $s$, which is a trivial line bundle, is a subbundle of $E$, and thus we may write $E = \Ox \oplus E'$ for some complement $E'$ (for example, placing a hermitian metric $h$ on $E$ and taking the orthogonal complement of $s$ with respect to $h$), and by the inductive hypothesis the result follows.
\end{proof}
\par Before we proceed, we review the Snake Lemma; in particular how the \enquote{snake} is constructed. We recall the statement:
\begin{proposition}[Snake Lemma]\index{Snake Lemma}
	Suppose we have the following diagram of $A$-modules with exact rows:
	\begin{equation*}
		\begin{tikzcd}
			0 \arrow[r] &A^0  \arrow[r]\arrow[d, "d^A"] &B^0\arrow[d, "d^B"] \arrow[r]& C^0 \arrow[d, "d^C"]\arrow[r] & 0 \\
			0 \arrow[r] &A^1  \arrow[r] &B^1 \arrow[r, "f"]& C^1 \arrow[r] & 0
		\end{tikzcd}
	\end{equation*}
	Then there is a map $\delta: \ker d^C \rightarrow \coker d^A$ such that the following sequence is exact:
	\begin{equation*}
		\begin{tikzcd}[column sep = small]
,`				0 \arrow[r] & \ker d^A \arrow[r] & \ker d^B  \arrow[r] & \ker d^C \arrow[dll, 
			rounded corners,
			to path={ -- ([xshift=2ex]\tikztostart.east)
				|- (Z) [near end]\tikztonodes
				-| ([xshift=-2ex]\tikztotarget.west)
				-- (\tikztotarget)}] \\ 
			& \coker d^A \arrow[r] & \coker d^B \arrow[r] & \coker d^C \arrow[r] & 0
		\end{tikzcd}
	\end{equation*}
\end{proposition}
The $\delta$ above is constructed as follows: Take $c\in \ker d^C$. Since the top row is exact, there exists some $b\in B^0$ that maps to $c$. Now observe that since $d^C(c)
= 0$, it must follow that the image of $d^B(b)\in \ker f$, by the commutativity of the diagram. Since the bottom row is exact, this pulls back to some unique $a\in A^1$, and moreover it can be shown that the image of $a$ in $\coker d^A$ does not depend on our choice of $b$, and in particular it is well-defined. Hence we define $\delta(c):= b$, and one can show that the resulting sequence is exact.
\\\\
\par Now we begin our investigation. We clearly have the short exact sequence of abelian groups:
\begin{equation*}
	\begin{tikzcd}
		0 \arrow[r] & \Z \arrow[r] & \C \arrow[r, "\exp"] & \C^* \arrow[r] &0
	\end{tikzcd}
\end{equation*}
where $\exp$ above is the map $x \mapsto \exp(2\pi i x)$. Now let $\scr{P}$ denote the property of smoothness or holomorphicity (in particular, statements made about $\scr{P}$ will be valid in both the smooth and holomorphic settings). Taking the sheaf of $\scr{P}$-functions with values in the above groups, we have the following short exact sequence of sheaves, known as the \textit{exponential sheaf sequence}: 
\begin{equation}\label{exponential-sequence}
	\begin{tikzcd}
		0 \arrow[r] & \underline{\Z} \arrow[r] & \Ox \arrow[r, "\exp"] & \Ox^* \arrow[r] &0
	\end{tikzcd}
\end{equation}
where $\Ox$ is the sheaf of $\scr{P}$-functions on $X$, and $\underline{\Z}$ is the constant sheaf $\Z$. Recall that $H^i(X, \underline{\Z}) \cong H^i_{\text{sing}}(X, \Z)$ (\autocite[pp. 42-43]{GriHa}). Now taking cohomology of (\ref{exponential-sequence}), we have:
\begin{equation}\label{exponential-sequence-cohomology}
	\begin{tikzcd}[column sep = small]
		0 \arrow[r] & H^{0}(X, \Z) \arrow[r] & H^{0}\left(X, \Ox\right)  \arrow[r] & H^{0}\left(X, \Ox^*\right) \arrow[dll, 
		rounded corners,
		to path={ -- ([xshift=2ex]\tikztostart.east)
			|- (Z) [near end]\tikztonodes
			-| ([xshift=-2ex]\tikztotarget.west)
			-- (\tikztotarget)}] \\ 
		& H^{1}\left(X, \Z\right) \arrow[r] & H^{1}\left(X, \Ox\right) \arrow[r] & H^{1}\left(X, \Ox^*\right) ...
	\end{tikzcd}
\end{equation}
and in particular, we have a map $\delta: H^1(X, \Ox^*)\rightarrow H^2(X, \Z)$. Now recall that $H^1(X, \Ox^*)$ is the Picard group of $X$; in particular it parameterises the isomorphism classes of $\scr{P}$-line bundles on $X$. 
\begin{proposition}\label{exp-sequence-chern-class}
	Under the inclusion $H^2(X, \Z) \subseteq H^2(X, \R) \cong H^2_{\operatorname{DR}}(X)$, we have \[\delta(L) = -c_1(L) \] for any $\scr{P}$-line bundle $L$
\end{proposition}
\begin{proof}
	This follows the proof given in \autocite[pp. 141-142]{GriHa}. Let $\{U_\alpha\}$ be a sufficiently fine open cover (in particular, one where we can take local logarithms on overlaps) and let $\{g_{\alpha\beta}\}$ be a \v{C}ech cocycle $L$. We define local inverses \[h_{\alpha\beta}:= \frac{1}{2\pi i}\log g_{\alpha\beta} \] and by the construction of the snake map of the Snake Lemma, it follows that $\{h_{\alpha\beta} - h_{\alpha\gamma} + h_{\beta\gamma}\}$ is a 2-cocycle of $\underline{\Z}$ representing $\delta(L)$. 
	
	Next we look at $c_1(L)$. Fix a connection and let $\{\omega_\alpha\}$ be the associated 1-forms. By Proposition \ref{connection-transformatino-rule}, we have \[\omega_\beta = (dg_{\alpha\beta})  g_{\alpha\beta}^{-1} + g_{\alpha\beta}\omega_\alpha g_{\alpha\beta}^{-1} = g_{\alpha\beta} dg_{\alpha\beta} + \omega_\alpha \] hence \[\omega_\beta - \omega_\alpha = g_{\alpha\beta}^{-1}dg_{\alpha\beta} = d \log g_{\alpha\beta} \] and in particular $d\omega_\alpha = d\omega_\beta$. Now we compute the curvature. Since we are working with a line bundle, $\omega$ is just a usual 1-form; it follows $\omega \wedge \omega = 0$, hence \[\Theta = d\omega_\alpha = d \omega_\beta. \] Finally, we reconcile de Rham and \v{C}ech cohomology. Note that by the Poincare Lemma we have the following short exact sequences of sheaves (recall exactness of sheaves is measured locally):
	\begin{equation}\label{poincare-zero-forms}
		0 \rightarrow \underline{\R} \rightarrow \Omega^0_X \rightarrow \mathcal{Z}^1 \rightarrow 0
	\end{equation}
	and
	\begin{equation}\label{poincare-1-forms}
		0 \rightarrow \mathcal{Z}^1 \rightarrow \Omega^1_X \rightarrow \mathcal{Z}^2 \rightarrow 0
	\end{equation}
	where $\mathcal{Z}^p$ denotes the sheaf of closed $p$-forms on $X$. Now it can be shown \autocite[p. 42]{GriHa} that $\Omega^p_X$ is acyclic, hence we have 
	\begin{equation}\label{0-cohomology-2-forms}
		H^0(X, \mathcal{Z}^2)/ H^0(X, \Omega^1_X) \cong H^1(X, \mathcal{Z}^1)
	\end{equation}
	by the long exact sequence of (\ref{poincare-1-forms}) and 
	\begin{equation}\label{2-cohomology-1-forms}
		H^1(X, \mathcal{Z}^1) \cong H^2(X, \R) 
	\end{equation}
	by the long exact sequence of (\ref{poincare-zero-forms})
	Now we compute. Using (\ref{0-cohomology-2-forms}), and the construction of the snake map, the image of $\Theta$ in $H^1(X, \cal{Z}^1)$ is represented by the \v{C}ech 1-cocycle $\{i(\omega_\beta - \omega_\alpha)/ 2\pi\}$, and using (\ref{2-cohomology-1-forms}), the image of $\Theta$ in $H^2(X, \R)$ is \[\{\frac{i}{2\pi} (\log g_{\alpha\beta} - \log g_{\alpha\gamma} + \log g_{\beta\gamma})\}= -\{h_{\alpha\beta} - h_{\alpha\gamma} + h_{\beta\gamma}\} = - \delta(L) \] as desired.
\end{proof}
\begin{corollary}
	In the smooth category, or in the holomorphic category with $g = 0$, every line bundle is uniquely determined by its first Chern class.
\end{corollary}
\begin{proof}
	In both cases $\Ox$ is acyclic.
\end{proof}

Next we prove Theorem \ref{chern-weil-degree} for line bundles:
\begin{theorem}\label{chern-weil-degree-line}
	If $\L$ is a holomorphic line bundle with underlying smooth bundle $L$, then we have:
	\begin{equation}\label{chern-degree-equation-line}
		\int_X c_1(L) = \deg \L.
	\end{equation}
\end{theorem}
\begin{proof}
	We first prove the case where $\L$ is the line bundle of a prime divisor. Let $D$ be a prime divisor, supported on $P\in X$, suppose $\L = \L(D)$. Since $D$ is effective, by Proposition \ref{divisor-of-zeroes-effective}, it is the divisor of zeroes of some global section $s\in H^0(X, \L)$ that vanishes exactly once, at $P$. In particular, $s$ is a frame on $X \setminus \{P\}$. Now fix a hermitian metric $h$. By the Chern correspondence, a Chern connection of $(\E, h)$ is given locally by $h^{-1}_\alpha\partial h_\alpha$ with respect to a holomorphic frame, and the curvature by $dh^{-1}_\alpha \partial h_\alpha= d\partial \log h_\alpha$ (here we may take logarithms because $h_\alpha > 0$). Now for an $\varepsilon> 0$, write $U_\varepsilon$ for the open set $\{p\in X\mid h(s(p), s(p)) > \varepsilon \}$ and $\overline{U}_\varepsilon$ for its closure, define $s_\varepsilon$ to be the frame $s|_{U_\varepsilon}$ and finally write $h_\varepsilon:= h(s_\varepsilon, s_\varepsilon) > 0$. We compute: \[\int_X c_1(L) = \lim_{\varepsilon \to 0} \int_{\overline{U}_\varepsilon}c_1(L) = \lim_{\varepsilon \to 0} \int_{\overline{U}_\varepsilon}\frac{i}{2\pi} d\partial \log h_\varepsilon = \lim_{\varepsilon \to 0}\int_{\partial U_\varepsilon} \frac{i}{2\pi}\partial \log h_\varepsilon \] where the final equality follows from Stoke's theorem. Now since $s|_{X \setminus \overline{U}_\varepsilon}$ vanishes exactly at $P$, we may pick a holomorphic coordinate chart centred at $P$ such that $s = z$. Hence 
	\begin{equation}\label{orientation-chern-class}
		\int_{\partial U_\varepsilon} \partial \log h_\varepsilon= - \int_{|z| = \varepsilon} \partial (\log z + \log \bar{z} + \log h_z(1,1))
	\end{equation} integrating anticlockwise as usual. The negative sign is there due to our choice of orientation. Now observe that since $\log h_z(1,1)$ is smooth, it follows $\partial \log (h_z(1,1))/\partial z$ is continuous, and since $\{|z| \leq \delta \}$ is compact, for some sufficiently small $\delta$, it follows $\partial \log (h_z(1,1))/\partial z$ is bounded on $\{|z| \leq \delta \}$. Thus \[\lim_{\varepsilon \to 0}\left| \int_{|z| = \varepsilon} \partial \log h(1,1)\right| = \lim_{\varepsilon \to 0} \left|\int_{|z| = \varepsilon} \frac{\partial \log h_z(1,1)}{\partial z} dz\right| \leq \lim_{\varepsilon\to 0} 2\pi \varepsilon \sup\{ \left|\frac{\partial \log h_z(1,1)}{\partial z}\right| \mid |z|\leq \delta \} = 0. \] Finally, we have, by Cauchy's integration formula, \[\int_{|z| = \varepsilon} \partial (\log z + \log \bar{z}) =  \int_{|z| = \varepsilon} \frac{dz}{z} = 2\pi i  \] as desired. 
	
	Finally, we observe that $H^2(X, \Z)\cong \Z$, and by Proposition \ref{exp-sequence-chern-class}, it follows \[c_1(L_1 \otimes L_2) = c_1(L_1)+ c_2(L_2)\] Hence if $D = \sum n_i P_i$ is any divisor and $\L = \L(D)$, it follows $\L = \otimes \L(P_i)^{\otimes n_i}$ hence \[\int_X c_1(L) = \int_X c_1(\otimes \L(P_i)^{\otimes n_i}) = \int_X \sum n_i c_1(\L(P_i)) = \sum n_i  = \deg \L \] as desired.
\end{proof}

\begin{example}\label{chern-class-line-bundle-on-p1}
	Suppose $X = \P^1$. We will compute the Chern class of $\Ox(n)$. Suppose the homogeneous coordinates on $X$ are $[x_0:x_1]$, and let $U_0 = \{x_0\neq 0\}$ and similarly with $U_1$. We define $z_0:= x_1/x_0$ to be the affine coordinate on $U_0 \cong \A^1$ and similarly for $z_1$. Now by definition of $\Ox(n)$, we may find frames $s_0, s_1$ on the corresponding affine patches such that \[g_{0,1}:= \frac{s_1}{s_0} = z_0^n = z^{-n}_1\] We define the hermitian metric on $\Ox(n)$ to be \[h_0 := h(s_0, s_0) = (1+|z_0|^2)^{-n} \]  on $U_0$, and \[h_1:= h(s_1, s_1) = (1+|z_1|^2)^{-n}. \] Of course, we need to check this is well-defined, that is \[h_0 = h(s_0, s_0) = h(z_1^n s_1, z_1^ns_1) = |z_1|^{2n}h_1. \] To this end observe \[h_0 = (1+|z_0|^2)^{-n} = (1+\frac{1}{|z_1|^2})^{-n} = (\frac{|z_1|^2+1}{|z_1|^2})^{-n} = |z_1|^{2n}(|z_1|^2+1)^{-n} = |z_1|^{2n}h_1\] as claimed. Hence for brevity, we will simply write \[h = (1+|z|^2)^{-n}\] understanding that this works for any affine patch, and finally observe that both affine patches have complement measure zero, and thus we may ultimately just work on one affine patch. 
	
	We continue. By the Chern correspondence, the unitary connection $\omega$ is given by \[\omega = h^{-1}\partial h = -\frac{n\bar{z}}{1+ |z|^2}dz \] and now computing the curvature $\Theta$: \[\Theta = d\omega = \frac{-n}{(1+|z|^2)^2} d\bar{z} \wedge dz =\frac{n}{(1+|z|^2)^2} dz\wedge d\bar{z} \]	and hence \[c_1(\Ox(n)) = \frac{ni}{2\pi(1+|z|^2)^2} dz\wedge d\bar{z} \] Finally, we integrate. Write $z = x+iy$ and we interpret the real coordinates $x,y$ as the standard coordinates of $\R^2$. Then $idz\wedge d\bar{z} = 2dx\wedge dy$, and letting $r,\theta$ denote the polar coordinates, we deduce \[\int_X c_1(\Ox(n)) = \int_U \frac{n}{\pi(1+|z|^2)^2}dx\wedge dy = \int_0^ \infty \int_0^{2\pi} \frac{nr}{\pi(1+r^2)^2}d\theta dr =n \] as desired.
\end{example}

\begin{proof}[Proof of Theorem \ref{chern-weil-degree}]
	By the structure theorem we know $E = \det E \oplus \Ox^{\rk E -1}$. Then given connections on $\det E$ and $\Ox^{rk E - 1}$, we can build a connection on $E$ with a diagonal matrix of associated 1-forms. More precisely, if $\nabla$ is a connection on $\det E$ with local 1-forms $\{\omega_\alpha\}$, then $\{\diag(\omega_\alpha,0,...,0)\}$ is the matrix of 1-forms for a connection on $\det E \oplus \Ox^{\rk E - 1}= E$. Thus \[\int_X c_1(E) = \int_X c_1(\det E) = \deg(E) \] as desired.
\end{proof}

\begin{corollary}
	For each $(n,d)\in \N \times \Z\cong \N \times H^2(X, \Z)$, there exists a unique smooth bundle $E$ over $X$ with signature $(n,d)$.
\end{corollary}

To summarise, we have given an interpretation of $\mathcal{V}_{n,d}$, the set of isomorphism classes of holomorphic vector bundles of signature $(n,d)$ as the set of holomorphic structures on the unique smooth bundle $E$ of signature $(n,d)$. Our earlier work on the Chern correspondence in turn describes this set as equal to the affine space of unitary connections on $E$ modulo gauge equivalence. Now the next question is, where does stability fit in all of this? 
\section{The Riemann-Hilbert Correspondence}
In this section, we will be constructing and demonstrating the equivalence between the moduli space of flat unitary connections and the so-called \textit{character variety} (which, despite the name, is not actually a variety by any definition) of $U(n)$, the space of representations of $\pi_1(X)$ modulo conjugacy. Since some of the more enlightening examples will not be on a Riemann surface, we will instead work with an arbitrary smooth manifold $Y$. To begin, we give the following informal example:
\begin{example}\label{circle-riemann-hilbert}
	Being the free group on one element, there is a canonical isomorphism $\Hom_{\Gps}(\Z, G )\cong G$ for any group $G$. In the case $G = \R^*$, there is a perverse way to see this: firstly, let $U_1 = S^1 \setminus \{1\}$ and $U_2 = S^1\setminus\{-1\}$. Let $u\in \R^*$, and let $E\rightarrow S^1$ be the (real) line bundle defined by gluing $U_1\times \R$ and $U_2 \times \R$ together with locally constant transition function \[g_{1,2}(z):=\begin{cases}
		1&\text{ if } \Im(z) > 0\\
		u&\text{ if } \Im (z) < 0
	\end{cases} \] on the intersection of $U_1$ and $U_2$ (which is just the intersection of $S^1$ and the union of the two half-planes). Observe that if $u < 0$ we get a M\"obius strip, but if $u > 0$ we get a cylinder. Now for any $x\in \R$, define the section $s_x$ to be the section $s_x(z) := (z, x)\in U_1 \times \R$, and define the path $\gamma: [0,1)\rightarrow E$ as \[\gamma(t) := \begin{cases}
	(1, x) \in U_2 \times \R&\text{ if } t = 0 \\
	s_x(\exp(2i\pi t))&\text{ otherwise}
	\end{cases}. \] Observe that the section is smooth, and \[\lim_{t\to 1} \gamma(t) = (1, ux)\in U_2\times \R; \] and in particular by \enquote{transporting} the vector $x$ along the loop, we have caused it to increase by a factor of $u$. In particular, this may be thought of as a map $\pi_1(S^1) \rightarrow \R^*$, taking a loop and seeing how much it deforms a vector transported along the loop. In this particular case, $\pi_1(S^1) =\Z$, and the map $\pi_1(S^1) \cong \Z\rightarrow \R^*$ is given by $1\mapsto u$. 
\end{example}
And indeed, this is a very simple example of the Riemann-Hilbert correspondence, which relates flat connections to representations of the fundamental group. The key concept here is the idea of \textit{parallel transport}, which we will now study, beginning with the following definition:
\begin{definition}
	Let $\nabla$ be a connection on a smooth vector bundle $E$, let $\gamma:[0,1]\rightarrow U$ be a piecewise smooth path and suppose $\gamma^\sharp: [0,1] \rightarrow E$ is a lift of $\gamma$ (that is, $\pi\circ\gamma^\sharp = \gamma$). We say $\gamma^\sharp$ is \textit{parallel} if \[\nabla_{\gamma'(t)}(\gamma^\sharp(t)) = 0 \] for every $t\in [0,1]$ such that $\gamma(t)$ is smooth.  
\end{definition}
\begin{proposition}
	Let $\gamma$ be a piecewise smooth path in $Y$ and let $(p,v)\in E$. Then there exists a unique parallel lift $\gamma^\sharp$ such that $\gamma^\sharp(0) = (p,v)$.
\end{proposition}
\begin{proof}
	Since we may break up our path into finitely many pieces, we may suppose without generality $\gamma$ is contained in some open subset $U_\alpha$ on which $E$ is trivial. Let $(s_1,...,s_n)_\alpha$ be a frame for $E$ on $U_\alpha$ and suppose $(\omega_\alpha)_{ij}$ is the associated matrix of 1-forms for $\nabla$. We are solving the first-order linear ODE \[\sum_j da_j(\gamma(t))(\gamma'(t))\otimes s_j(\gamma(t)) + \sum_i a_i(\gamma(t)) (\omega_\alpha)_{ij}(\gamma'(t))s_j \] in $t$ with an initial condition, and hence there exists a unique solution.
\end{proof}
\begin{example}
	We will construct a connection $\nabla$ such that that $s_x\circ \gamma$ in Example \ref{circle-riemann-hilbert} is a parallel section with respect to the loop $\gamma(t) = \exp(2\pi i t)$. To this end, let $s,t$ be the sections $s(z): = (z,x)\in U_1 \times \R$ and $t(z):=(z,x)\in U_2\times\R$. One then defines both 1-forms of $\nabla$ to be zero, it is clear they satisfy the transformation rule and that $s_x\circ \gamma$ is parallel. 
\end{example}
And now we can define parallel transport:
\begin{definition}
	Let $\gamma$ be a piecewise smooth path in $Y$, let $P\in E$ and let $\gamma^\sharp$ be the unique parallel lift along $\gamma$ beginning at $P$. The \textit{parallel transport} of $P$ is the element $\gamma^\sharp(1)$. 
\end{definition}
Now observe that since the ODE for parallel transport is linear, this means that parallel transport itself is linear. In fact, we have the following:
\begin{proposition}
	Let $E$ be a bundle over $X$, and let $\nabla$ be a unitary connection. Then for any $P\in X$ and any loop beginning and ending at $P$, parallel transport $T: E_P \rightarrow E_P$ is unitary. 
\end{proposition}

\section{An Overview of Donaldson's Proof}
\par Finally, we will conclude the thesis with an exposition of Donaldson's paper \autocite{Donaldson}. This builds on earlier work by Atiyah and Bott in \autocite{AtiBot}, and provides {\color{red} a short proof of the theorem of Narasimhan and Seshadri}. We fix the following data: let $\E$ be an indecomposable holomorphic vector bundle of signature $(n,d)$ with underlying smooth bundle $E$ and fix a hermitian metric $h$. Furthermore, since $X$ is a compact Riemann surface, it is K\"ahler, and we make the further assumption that the volume of $X$ is 1 (that is, we fix a volume form such that $\int_X \vol = 1$). The result is the following:
\begin{theorem}[Donaldson-Narasimhan-Seshadri]
	The bundle $\E$ is stable if and only if there is some unitary connection $\nabla$ on $E$ giving rise to $\E$ with curvature $\Theta\in H^0(\Omega^2_X)\otimes \End E$ satisfying 
	\begin{equation}\label{NS-equality}
		\Theta = -2\pi i \mu \vol \otimes \id_E 
	\end{equation}
	Moreover, $\nabla$ is unique up to the action of the unitary gauge group.
\end{theorem}
\begin{example}
	Of course, over $\P^1$ the only stable bundles are line bundles. So let $\Ox(n)$ be a line bundle. In Example \ref{chern-class-line-bundle-on-p1}, we defined a hermitian metric and computed the Chern class, Chern connection and curvature. Now we will need to compute the volume form. Of course, $X$ is obviously K\"ahler with its Fubini-Study metric (which can be realised as the metric of $\mathcal{T}_X = \Ox(2)$ or its dual $\Ox(-2) = \Omega_X^1$ described in Example \ref{chern-class-line-bundle-on-p1}), and so by taking the real part of this complex inner product, we have a natural Riemannian structure. Locally, if we pick an affine patch with holomorphic coordinate $z = x + iy$, and frame $(\partial x, \partial y)$ of $T_X$ (the real smooth tangent space) the metric is given by \[g = \begin{pmatrix*}
		\frac{1}{\sqrt{\pi}(1 + x^2+y^2)^2} & 0 \\
		0 & \frac{1}{\sqrt{\pi}(1 + x^2+y^2)^2} 
	\end{pmatrix*} \] The $\sqrt{\pi}$ is there so that the resulting volume is 1. An orthonormal frame is given by $(\sqrt{\pi}(1 + x^2+y^2)\partial x , \sqrt{\pi}(1 + x^2+y^2)\partial y)$, and hence the volume form is \[\vol = \frac{dx \wedge dy}{\pi(1 + x^2 + y^2)^2} = \frac{idz \wedge d\bar{z}}{2\pi(1 + |z|^2)^2}. \] Now we computed the curvature of the Chern connection on $\Ox(n)$ to be \[ \Theta =\frac{n}{(1+|z|^2)^2} dz \wedge d\bar{z} = -2\pi i \deg(\Ox(n)) \vol \] as expected. Hence the theorem is true for $\P^1$.
\end{example}
In fact, we will first prove the theorem for line bundles in general:
\begin{theorem}
	The Donaldson-Narasimhan-Seshadri theorem is true for line bundles.
\end{theorem}
\begin{proof}
	Of course, if $\L$ is a line bundle with hermitian metric $h$, underlying smooth bundle $L$ and Chern connection $\nabla$, then it is already stable. Thus we reduce to showing that a connection $\nabla'$ in the orbit of $\nabla$ with curvature in the form (\ref{NS-equality}) exists. 
	
	To this end, we observe that the curvature $\Theta$ of $\nabla$ is just an imaginary global (1,1)-form (since $\fancyEnd \L$ is trivial; the identity endomorphism is a global frame), so $i\Theta$ differs from its harmonic representative $i\Theta'$ by a real exact  1-form, say $i\Theta - d\eta= i\Theta'$. Now observe that since $\Theta'$ is harmonic, $d\star \Theta' = 0$, and so $\star\Theta'$ is a constant; necessarily equal to $-2\pi i\mu$. So we reduce once again to showing that there is a gauge transformation $g$ such that $\Theta'$ is the curvature of $g \cdot \nabla$. 
	
	Observe that $d\eta$ is real and closed and therefore the following Poisson equation has a solution (\autocite[Theorem 4.7]{Aubin}): \[2\delbar\partial f = \Delta f = id\eta. \] Now write $g:= \exp f$, let $\nabla' = g\cdot \nabla$, and write $\Theta'$ for the curvature of $\nabla'$. Firstly, observe that since $\fancyEnd \L$ is trivial, the operators induced by the connection, $\partial_{\fancyEnd \L}$ and $\delbar_{\fancyEnd \L}$, are just the usual $\partial$ and $\delbar$ operators. Hence \[\Theta' = \theta - d(\delbar g)g^{-1} + d\overline{(\delbar g)g^{-1}} = \Theta - \partial \delbar f +  \]
\end{proof}

Observe that the condition (\ref{NS-equality}) is a little awkward to work with, so we introduce the \textit{Donaldson $J$-functional} on the {\color{red} Sobolev }space of unitary connections, which satisfies the property that $J(\nabla) = 0$ if and only if $\nabla$ satisfies (\ref{NS-equality}). It is defined as follows: Firstly recall that the \textit{trace norm} (which despite its name, is not a norm in general) of a square matrix $M\in \C^{r\times r}$ is defined to be \[\nu(M) := \tr((MM^*)^{\frac{1}{2}}), \] where $(MM^*)^{\frac{1}{2}}$ is the unique positive semidefinite matrix $B$ such that $B^2 = MM^*$, which exists since $MM^*$ is hermitian (and hence diagonalisable) and positive semidefinite. In fact, if $M$ is diagonalisable, it is easy to see that \[\nu(M) = \sum |\lambda_i|, \] where the sum is taken across all eigenvalues of $M$, counting multiplicity. The key property is the following:
\begin{lemma}
	For any hermitian matrix $M$, we have \[\nu(M) = \sup_{\{s_i\}}\sum_{i = 1}^n |\langle Ms_i, s_i \rangle|, \] where the supremum is taken across all unitary bases $\{s_i\}$ of $\C^n$.
\end{lemma}
\begin{proof}
	We first observe that $M$ has a unitary basis of eigenvectors, say $\{v_i\}$ and letting $\{s_i\} = \{v_i\}$ we deduce \[\nu(M) = \sum|\lambda_i| = \sum_{i = 1}^n |\langle Mv_i, v_i \rangle| \leq  \sup_{\{s_i\}}\sum_{i = 1}^n |\langle Ms_i, s_i \rangle|. \] For the reverse inequality, let $\{s_i\}$ be a unitary basis, and let $(g_{ij})\in U(n)$ denote the matrix taking $\{v_i\}$ to $\{s_i\}$; that is, $s_i = \sum g_{ij}v_j.$ We compute:
	\begin{align*}
		\sum_{i = 1}^n |\langle Ms_i, s_i\rangle | &= \sum_{i = 1}^n|\langle M\sum_{j = 1}^n g_{ij} v_j,\sum_{k = 1}^n g_{ik}v_k \rangle | \\
		&= \sum _{i = 1}^n |\sum_{j = 1}^ng_{ij}\langle Mv_j, \sum_{k = 1}^n g_{ik}v_k \rangle| \\
		&= \sum _{i = 1}^n |\sum_{j = 1}^ng_{ij}\langle Mv_j, g_{ij}v_j \rangle|\\
		&= \sum _{i = 1}^n |\sum_{j = 1}^n\lambda_j\langle g_{ij}v_j, g_{ij}v_j \rangle|\\
		&\leq \sum_{i = 1}^n\sum_{j = 1}^n |\lambda_j||\langle g_{ij}v_j, g_{ij}v_j \rangle|\\
		&= \sum_{i = 1}^n \sum_{j = 1}^n |\lambda_j||g_{ij}|^2 = \sum_{j = 1}^n |\lambda_j|
	\end{align*} 
	as desired.
\end{proof}
Of course, this in itself is not particularly interesting or useful, but it does give us two very important corollaries:
\begin{corollary}\label{trace-norm-properties}
	Let $H(n)$ denote the vector space of hermitian $n$-by-$n$ matrices.
	\begin{enumerate}
		\item $\nu$ is a norm on $H(n)$.
		\item If $M\in H(n)$ is can be written in the form \[M = \begin{pmatrix*}
			A & B\\
			B^* & C
		\end{pmatrix*}, \] then $\nu(M) \geq |\tr A| + |\tr C|$.
	\end{enumerate}
\end{corollary}
\begin{proof}
	To prove (i), we need only check the triangle inequality. So suppose $M,N\in H(n)$ are given. Then \[\nu(M +N) = \sup_{\{e_i\}} \sum |\langle (M+N)e_i, e_i \rangle| = \leq \sup_{\{e_i\}} \sum |\langle Me_i, e_i \rangle|+ |\langle Ne_i, e_i \rangle|\leq \nu(M) + \nu(N) \] as desired. To prove (ii), let $\{e_i\}$ denote the standard basis of $\C^n$. Then \[\nu(M) \geq \sum_{i = 1}^n |\langle M e_i, e_i \rangle | \geq |\sum_{i = 1}^{\rk A} \langle M e_i, e_i \rangle | + |\sum_{i = \rk A + 1}^n \langle M e_i, e_i \rangle | = |\tr A | + |\tr C| \] as desired.
\end{proof}

With this in mind, we define the \textit{$N$-norm} on the space of self-adjoint smooth endomorphisms of $E$, as \[N(s):= \left(\int_X \nu^2(s) \vol\right)^{\frac{1}{2}}. \] By the above corollary, this is a norm.

Now let $\nabla$ be a unitary $W^{1,2}$-connection with curvature $\Theta\in H^0(\Omega_{\fancyEnd E}^{2})$. Since the matrix of a unitary connection with respect to a unitary frame is skew-hermitian, its curvature $\Theta$ is also skew-hermitian, and since the volume form is real, it follows that $\star\Theta$ is also skew-hermitian. In particular, it follows that $\frac{\star \Theta}{2\pi i}$ is actually \textbf{hermitian}. Thus we define the \textit{Donaldson $J$-functional} as \[J(\nabla):= N(\frac{\star \Theta}{2\pi i} + \diag(\mu))= \left(\int_X \nu^2\left(\frac{\star \Theta}{2\pi i} + \diag(\mu)\right)\vol\right)^{\frac{1}{2}}, \] where $\nu^2(s) := (\nu(s))^2$. Observe that $J=0$ if and only if $\nabla$ is a unitary connection of the type we want (known as, \textit{projectively flat}, or \textit{Yang-Mills connections}). Thus we have turned our problem into one of finding zeroes of $J$.

The rough idea of the proof is as follows: we fix a reference Chern connection $\nabla_0$ of $\E$, and use gauge transformations to find our desired $\nabla$. Denote the $W^{2,2}$-gauge orbit of $\nabla_0$ by $O_{\nabla_0}$. We will show that if $\E$ is stable, then the infimum of $J(O_{\nabla_0})$ is attained; that is there is some $\nabla\in O_{\nabla_0}$ such that $J(\nabla) = \inf J(O_{\nabla_0})$. One then deduces that the infimum must be zero, by looking at near $\nabla$. In order to deduce that the infimum is attained, we take a minimising sequence (that is, a sequence $\nabla_i$ such that $J(\nabla_i)\to \inf J(O_\nabla)$) in $O_{\nabla_0}$ and extract, using Uhlenbeck's weak compactness theorem (to be stated), a weakly convergent subsequence that converges to $\nabla$. Now $\nabla$ defines a holomorphic bundle, say $\F$, and the key property is that $\Hom(\E, \F)\neq 0$. So we take a nonzero $\varphi: \E \rightarrow \F$, and apply Proposition \ref{canonical-extension-proposition} to get a factorisation of $\varphi$ through two exact rows, and apply estimates to these rows to deduce that $\E$ is stable if and only if $\E \cong\F$. The converse (that if there is some connection annihilating $J$ then $\E$ is stable) also follows from these estimates.
\\\\
\par Now we begin with a {\color{red} statement of Uhlenbeck's compactness theorem}:
\begin{theorem}[Uhlenbeck's weak compactness]
	Let $(\nabla_i)$ be a sequence of $W^{1,2}$-connections with curvatures $(\Theta_i)$, and suppose the sequence $(||\Theta_i||_{L^2}:=\int_X \tr(\Theta_i) \wedge \tr(\star \overline{\Theta_i}))$ is bounded. Then there is a sequence of $W^{2,2}$-gauge transformations $(g_i)$ and a subsequence $(\nabla_{i_k})$ such that $(g_{i_k}\cdot \nabla_{i_k})$ weakly converges to some $\nabla_\infty$ (that is, $\int_X \tr(g_{i_k}\cdot \Theta_{i_k}) \wedge \tr(\star A) \to\int_X \tr(\Theta_\infty) \wedge \tr(\star A) $ for all $W^{1,2}$-connections $A$).
\end{theorem}
\begin{proof}
	\autocite[p. 41]{Uhlenbeck}.
\end{proof}
So let $(\nabla_i)$ be a sequence in $O_{\nabla_0}$ with curvatures $(\Theta_i)$, such that $J(\nabla_i)\to \inf J(\O_{\nabla_0})$. In order to use the theorem, we need to check that $||\Theta_i||_{L^2}$ is bounded. To this end, we first observe that $N(\star \Theta_i)$ is bounded, since $J(\nabla_i)$ is and $N$ is a norm. Now note that \[\nu^2(\star \Theta_i) \vol =\tr(\sqrt{(\star \Theta_i)(\star \Theta_i)^*})^2\vol,\] and similarly, \[\tr(\Theta_i) \wedge \star\tr(\overline{\Theta_i}) = \tr(\star\Theta )(\star\Theta)^*\vol. \] Since all norms are equivalent in finite dimensions, it follows that there is some $m,M>0$ such that for any matrix $A$ we have \[m\tr(AA^*)\leq \tr\sqrt{AA^*}^2 \leq M \tr(AA^*),   \] thus since $\{N(\star\theta_i)\}$ is bounded, it follows that $\{||\Theta_i||_{L^2}\}$ is also bounded. Thus (replacing $\nabla_i$ with $g\cdot \nabla_i$, and replacing the sequence with a weakly convergent subsequence) we may assume without loss of generality $\nabla_i$ converges weakly to some $\nabla_\infty$, which is a unitary connection and hence defines a holomorphic bundle, say $\F$ with signature $(n,d)$.
\begin{proposition}
	Let $\E, \F$ be as above.
	\begin{enumerate}
		\item Then $\inf J(O_{\nabla_0}) \geq \inf J(O_{\nabla_\infty})$.
		\item The group $\Hom(\E, \F)$ is nonzero.
	\end{enumerate}
\end{proposition}
\begin{proof}[Proof Sketch]
	To prove (i), we first observe that for any $\varepsilon > 0$ the set $C_\varepsilon = \{\alpha\in \End E \mid N(\alpha + \diag(\mu)) < J(\nabla_\infty) - \varepsilon\}$ is convex and closed, and thus by the Hahn-Banach separation theorem, we can separate $\frac{\star \Theta}{2\pi i}$ from $C_\varepsilon$ by a hyperplane. Now if \[\inf J(O_{\nabla_\infty}) > \inf J(O_{\nabla_0}) = \liminf_{n\to \infty} J(\nabla_i), \] then picking some $\varepsilon_0$ such that $\inf J(O_{\nabla_\infty})  - \varepsilon_0 > \inf J(O_{\nabla_0})$, we find that infinitely many $\frac{\star \Theta_i}{2\pi i}$ lie in $C_{\varepsilon_0}$. But that means $\Theta_i$ cannot converge weakly to $\Theta_\infty$ in $L^2$, and since the curvature is a bounded linear operator, it follows that weak convergence is preserved, and thus we have a contradiction. This proves (i).
	
	To prove (ii), we first observe that an element of $\Hom(\E, \F)$ is just a global section of $\fancyHom(\E, \F) = \E^\vee \otimes \F$. Now the underlying smooth bundle of $\E^\vee \otimes \F$ is just $\fancyEnd(E) = E^\vee \otimes E$, and it is easy to see that given any Dolbeault operators $\delbar_{\E}$ and $\delbar_{\F}$ giving rise to the holomorphic structures on $\E$ and $\F$, the operator \[\delbar_{\E^\vee \otimes \F} := 1\otimes\delbar_{\F} - \delbar_{\E}\otimes 1\] is a Dolbeault operator for $\E^\vee \otimes \F$. Since $\E$ and $\F$ have Chern connections $\nabla_0$ and $\nabla_\infty$ respectively, and since the $\nabla_i$ for $i \geq 0$ all give rise to the same (more precisely isomorphic) holomorphic structures, we can take the $(0,1)$ part of these connections to build the Dolbeault operators $\delbar_{i, \infty}:= (1\otimes \nabla_\infty - \nabla_i \otimes 1)_{0,1}$. Now to say $\Hom(\E, \F) = 0$ is to say that the Dolbeault operators $\delbar_{i, \infty}$ for $i \geq 0$, considered as maps $\End E \rightarrow \End E \otimes H^0(X, \Omega^{0,1})$ have trivial kernel. One can then apply the theory of elliptic operators and the Sobolev embedding theorem to deduce that $\delbar_{0, i}:= 1\otimes\nabla_i -\nabla_0\otimes 1 $ also has no kernel. But that would imply $\End \E = 0$, a clear contradiction.
\end{proof}
With this result in hand, we fix a nonzero homomorphism $\varphi: \E \rightarrow \F$ and apply Proposition \ref{canonical-extension-proposition}, so that we have the following commutative diagram:
	\begin{equation}\label{proper-factorisation-equation-donaldson}
	\begin{tikzcd}
		0 \arrow[r] & \cal{E'}\arrow[r] & \E \arrow[r] \arrow[d, "\varphi"]& \E'' \arrow[d]\arrow[r]& 0\\
		0  & \F''\arrow[l]& \F \arrow[l] & \F'\arrow[l] &\arrow[l] 0
	\end{tikzcd}
\end{equation}
with exact rows, $\E' \cong \ker \varphi$, $\E''\cong \im \varphi$ and $\rk \E'' = \rk \F'$, $\deg \E'' \leq \deg \F'$. The key now is to apply estimates to these rows. 
\begin{proposition}[First Estimate]
	Consider the following short exact sequence of vector bundles: \[ 0 \rightarrow \F' \rightarrow \F \rightarrow \F'' \rightarrow 0,\] and suppose $\mu(\F') \geq \mu(\F)$. Then if $\nabla_\F$ is a unitary connection on $E$ giving rise to the holomorphic structure of $\F$, we have \[J(\nabla_\F) \geq \rk \F'(\mu(\F') - \mu(\F)) + \rk \F''(\mu(\F) - \mu(\F'')). \] Equality holds only if the sequence splits.
\end{proposition}
\begin{proof}
	Firstly, we fix a local unitary frame $s_\alpha$ compatible with a local holomorphic splitting and consider the matrix of one-forms of $\nabla_\F$, which is skew-hermitian. One can show that it has the shape \[\omega_\alpha = \begin{pmatrix*}
		\omega_\alpha' & \beta_\alpha \\
		-\beta^*_\alpha & \omega_\alpha''
	\end{pmatrix*}, \] where the $\beta_\alpha$ glue to the second fundamental form (c.f. Remark \ref{ext-dolbeault}) $\beta $, and the $\omega_\alpha'$ are the 1-forms of a Chern connection $\nabla'$ on $\F'$, and similarly with $\omega_\alpha''$. If we compute the curvature, we see that it is of the form \[\Theta_\F = \begin{pmatrix*}
	\Theta' - \beta \wedge \beta^* & \nabla_{\F'', \F}\beta\\
	-\nabla_{\F'', \F}\beta^*& \Theta'' - \beta^* \wedge \beta
	\end{pmatrix*}, \] where $\Theta'$ and $\Theta''$ are the curvatures of $\nabla'$ and $\nabla''$ respectively and $\nabla_{\F'', \F}: \Omega^1(\fancyHom(\F'', \F))\rightarrow \Omega^2(\fancyHom(\F'', \F))$ is built from the connections $\nabla', \nabla''$ (see \autocite[p. 78]{GriHa} for details). Now by Corollary \ref{trace-norm-properties}, it follows that \[\nu\left(\frac{\star\Theta_\F}{2\pi i} + \diag_{\rk \F}(\mu)\right) \geq \left|\tr\left(\frac{\star(\Theta' - \beta \wedge \beta^*)}{2\pi i} + \diag_{\rk \F'}(\mu)\right)\right| + \left|\tr\left(\frac{\star(\Theta''- \beta^*\wedge \beta)}{2\pi i} + \diag_{\rk \F''}(\mu)\right)\right|, \] where $\mu = \mu(\F)$. Applying H\"older's inequality, we deduce
	\begin{align*}
		J(\nabla_\F) &\geq \int_X\nu\left(\frac{\star\Theta_\F}{2\pi i} + \diag_{\rk \F}(\mu)\right) \vol\\
		&= \left|\int_X\tr\left(\frac{\star(\Theta' - \beta \wedge \beta^*)}{2\pi i} + \diag_{\rk \F'}(\mu)\right)\vol\right| + \left|\int_X	·\tr\left(\frac{\star(\Theta''- \beta^*\wedge \beta)}{2\pi i} + \diag_{\rk \F''}(\mu)\right)\vol\right|
	\end{align*}
	 Let us consider the term $\star(\beta \wedge \beta^*)$. Observe that $\beta$ is a (0,1)-form and so $\beta \wedge \beta^*$ has entries of the form $|f|d\bar{z} \wedge dz$ for any holomorphic coordinate $z$. Since, by our conventions, our orientation is $ i dz \wedge d\bar{z}$, this means that $-i\tr \star(\beta \wedge \beta^*)$ will be nonnegative.
	 
	 Next we observe that by Theorem \ref{chern-weil-degree}, we have \[\int_X\tr\left(\frac{\star \Theta'}{2\pi i}\right)\vol= -\deg \F' \leq -\rk \F' \mu(\F) = -\tr \diag_{\rk \F}(\mu) = -\int_X \tr \diag_{\rk \F'}(\mu) \vol,\] where the last equality follows from the assumption $\int_X \vol = 1$. Hence 
	 \begin{align*}
	 	\left|\int_X\tr\left(\frac{\star(\Theta' - \beta \wedge \beta^*)}{2\pi i} + \diag_{\rk \F'}(\mu)\right)\vol\right| &= -\int_X\tr\left(\frac{\star(\Theta' - \beta \wedge \beta^*)}{2\pi i} + \diag_{\rk \F'}(\mu)\right)\vol\\
	 	&=\rk\F'(\mu(\F')- \mu(\F)) +\frac{1}{2\pi i} \tr \star(\beta \wedge \beta^*)
	 \end{align*} and note that $\frac{1}{2\pi i} \tr \star(\beta \wedge \beta^*) \geq 0$ by the above discussion. Similarly, note that \[\left|\int_X	\tr\left(\frac{\star(\Theta''- \beta^*\wedge \beta)}{2\pi i} + \diag_{\rk \F''}(\mu)\right)\vol\right| = \rk \F''(\mu(\F) - \mu(\F'')) +\frac{1}{2\pi i} \tr \star(\beta\wedge \beta^*) \]

	And putting it all together we get 
	\begin{align*}
		J(\nabla_\F) &\geq \rk\F'(\mu(\F')- \mu(\F))+\rk \F''(\mu(\F) - \mu(\F'')) +\frac{1}{\pi i} \tr \star(\beta \wedge \beta^*) \\&\geq \rk\F'(\mu(\F')- \mu(\F))+\rk \F''(\mu(\F) - \mu(\F'')).
	\end{align*}
	as desired. Finally, if equality occurs, that means $\beta = 0$, but $\beta$ defines an element of $\Ext^1(\F'', \F')$ via the Dolbeault cohomology representation of sheaf cohomology (Remark \ref{ext-dolbeault}), and in particular if it is zero then the sequence splits. 
\end{proof}

And in fact, from this we may already deduce one direction of the Donaldson-Narasimhan-Seshadri theorem:
\begin{corollary}
	Suppose $\E$ is indecomposable, and there is a Chern connection $\nabla$ giving rise to $\E$ such that $J(\nabla) = 0$. Then $\E$ is stable.  
\end{corollary}
\begin{proof}
	Suppose for contradiction $\E$ is not stable. Then there is some subbundle $\E'$ such that $\mu(\E') \geq \mu(\E)$, whence $\mu(\E) \geq \mu(\E/\E')$. Then \[0 = J(\nabla) \geq \rk \E'(\mu(\E') - \mu(\E)) + \rk(\E/\E')(\mu(\E) - \mu(\E/\E')) \geq 0,\] which means the sequence \[0 \rightarrow \E' \rightarrow \E \rightarrow \E/\E' \rightarrow 0 \] splits, by the above proposition, contradicting the indecomposability of $\E$.
\end{proof}
\begin{remark}
	In fact, we can deduce that if $\E$ has a Chern connection which is a zero of $J$, then $\E$ must be polystable, since the proposition tells us that $\E$ can be written as a direct sum of two subbundles of equal slope.
\end{remark}

Our next estimate applies to the top row. However, it is more technical and requires the stronger hypothesis that the Donaldson-Narasimhan-Seshadri theorem has been proven for bundles of smaller rank:
\begin{proposition}[Second Estimate]
	Consider the following short exact sequence of vector bundles: \[ 0 \rightarrow \E' \rightarrow \E \rightarrow \E'' \rightarrow 0,\] suppose this exension is proper, that $\E$ is stable and the Donaldson-Narasimhan-Seshadri theorem has been proven for bundles of rank less than $\rk \E$. Then there exists a unitary connection $\nabla_{\E}$ on $E$ giving rise to $\E$ such that \[J(\nabla_{\E}) < \rk \E'(\mu(\E) - \mu(\E')) + \rk \E''(\mu(\E'') - \mu(\E)). \]
\end{proposition}
\begin{proof}[Proof Sketch]	
	The idea here is to use the Harder-Narasimhan and Jordan-H\"older filtrations and the inductive hypothesis to build this $\nabla_{\E}$. So let $(\E_i')$ be the Harder-Narasimhan filtration of $\E'$, and for each $i$ let $(\E_{ij}')$ denote the Jordan-H\"older filtration of $\E_i'/\E_{i-1}'$. Now since $\rk \E_{i,{j}}'/\E_{i,j-1}< \rk \E$, by assumption we know that there is a projectively flat Chern connection $\nabla'_{ij}$ on $\E_{i,{j}}'/\E_{i,j}$. Now given any \[0 \rightarrow \E'_{i,j}\rightarrow \E'_{i,{j+1}}\rightarrow \E'_{i,{j+1}}/\E'_{i,j}\rightarrow 0 \] with second fundamental form $B_{i,j}$, one can inductively (starting with $j = 0$) build a connection on $\E'_{i,{j+1}}$ from one on $\E'_{i,j}$ and the on $\E'_{i,{j+1}}/\E'_{i,j}$ given to us, and now letting the $i$ vary we can build a connection on each $\E'_i$. Now given any short exact sequence of vector bundles \[0 \rightarrow \F' \rightarrow \F \rightarrow \F''\rightarrow 0\] with second fundamental form $B\in \Ext(\F'', \F')$, one can scale $B$ by any nonzero constant $t\in \C\setminus \{0\}$ and the resulting bundle in the middle is isomorphic to $\F$, by the proof of Theorem \ref{if-extension-then-jump}. In particular, given any short exact sequence from any of our filtrations above (which either looks like \[0 \rightarrow \E'_{i,j}\rightarrow \E'_{i,{j+1}}\rightarrow \E'_{i,{j+1}}/\E'_{i,j}\rightarrow 0 \] or \[0 \rightarrow \E'_{i}\rightarrow \E'_{i+1}\rightarrow \E'_{i+1}/\E'_{i}\rightarrow 0), \] the connection built in the middle is of the form  \[\begin{pmatrix*}
		\nabla_1, B \\
		-B^*, \nabla_2
	\end{pmatrix*}\] where $\nabla_1$ and $\nabla_2$ are connections on the left and right respectively and $B$ is the second fundamental form. Now as mentioned, we may scale $B$ by any nonzero constant $t> 0$ and retain the same extension class, so the matrix\[\begin{pmatrix*}
		\nabla_1, tB \\
	-tB^*, \nabla_2
	\end{pmatrix*}\] gives rise another Chern connection on the middle bundle. Doing this (with a fixed $t > 0$) for every step of both filtrations, we have a collection of Chern connections $\{\nabla'_t\}$ on $\E'$, but their limit $\nabla'_0$ is a Chern connection for $\bigoplus_{i,j} \E'_{ij}$, and moreover by construction we have \[\star \Theta'_0 = -2\pi i \diag(\mu (\E_{ij}')), \] where $\Theta'_t$ is the curvature of $\nabla'_t$. Similarly, we can build a collection of Chern connections $\nabla''_t$ on $\E''$ that converge to a connection $\nabla''_0$ on some $\bigoplus_{i',j'} \E''_{i'j'}$ and $\star\Theta''_0 =- 2\pi i \diag(\mu(\E''_{i',j'}))$.

	Let $[\beta]\in \Ext^1(\F'', \F')$ denote the extension class of $\E$. Now for each $\nabla_t', \nabla''_t$, one can build a connection $\nabla^t_{\E'', \E'}$ on $\fancyHom(\E'', \E')$, and for each $\nabla^t_{\E'', \E'}$, standard arguments from Hodge theory tell us that there is a representative of $[\beta]$, call it $\beta_t$, such that $\nabla^t_{\E'', \E'}(\beta_t) = 0$. Now letting $s>0$ be another variable, we have connections depending on $s$ and $t$ \[\nabla_{s,t} =\begin{pmatrix*}
		\nabla_t' & s\beta_t \\
		-s\beta_t^*& \nabla_t''
	\end{pmatrix*} \] with curvature \[\Theta_{s,t} = \begin{pmatrix*}
	\Theta_t' - s^2\beta_t\wedge \beta^*_t & 0 \\
	0 & \Theta_{t}'' - s^2\beta_t^*\wedge\beta_t
\end{pmatrix*} \] that converge to $\nabla_{0,0}$ with curvature $\Theta_{0,0} =\diag(\Theta_0', \Theta_0'')$. Now we observe \[\tr (\frac{\star \Theta_0'}{2\pi i} + \diag_{\rk \E'}(\mu(\E)))= \sum (\mu(\E) - \mu(\E'_{ij})) = \rk \E'(\mu(\E) - \mu(\E')) > 0\] and similarly \[\tr (\frac{\star \Theta_0''}{2\pi i} + \diag_{\rk \E''}(\mu(\E)))  = \rk \E''(\mu(\E) - \mu(\E'')) < 0, \] and put together this tells us \[J(\nabla_{0,0}) = \rk \E'(\mu(\E) - \mu(\E')) + \rk \E''(\mu(\E'') - \mu(\E)). \] Our task now is to show that for sufficiently small $s,t$ we have $J(\nabla_{s,t}) < J(\nabla_{0,0})$. To this end, we first note that since $A' := \diag(\mu - \mu(\E'_{ij}) )$ is a diagonal matrix with positive entries and hence has negative eigenvalues, it follows that $\nu(A') = \tr A'$, and hence the same is true for matrices sufficiently close to $A'$.  Now it can be shown that the $i\tr \star(\beta_t^*\wedge \beta_t)$ {\color{red} are uniformly bounded}, and hence for sufficiently small $s,t$, it follows 
\begin{align*}
	\nu(\frac{\star(\Theta_{t}' - s^2 \beta_t\wedge \beta_t^*)}{2\pi i} + \diag_{\rk \E'}(\mu)) &=\tr (\frac{\star\Theta_{t}' - s^2 \beta_t\wedge \beta_t^*}{2\pi i} + \diag_{\rk \E'}(\mu)) \\&= \rk \E'(\mu(\E) - \mu(\E')) - s^2\tr\star(\frac{\beta_t \wedge \beta_t^*}{2\pi i}) + \varepsilon_1(t),
\end{align*}
where $\varepsilon_1(t)$ is some error term that vanishes as $t \to 0$, and the uniform bound is used to control the $|\tr\star(\frac{\beta_t \wedge \beta_t^*}{2\pi i})|$, so that the matrix does not deviate from $A'$ too much. Similarly,
\begin{align*}
	\nu(\frac{\star(\Theta_{t}' + s^2 \beta_t\wedge \beta_t^*)}{2\pi i} + \diag_{\rk \E'}(\mu)) &=-\tr (\frac{\star\Theta_{t}'' + s^2 \beta_t\wedge \beta_t^*}{2\pi i} + \diag_{\rk \E'}(\mu)) \\&= \rk \E''(\mu(\E'') - \mu(\E)) - s^2\tr\star(\frac{\beta_t \wedge \beta_t^*}{2\pi i}) + \varepsilon_2(t), 
\end{align*} and hence \[\nu(\frac{\star \Theta_{s,t}}{2\pi i} + \diag_{\rk\E}(\mu)) = J(\nabla_{0,0}) - s^2\tr\star(\frac{\beta_t \wedge \beta_t^*}{\pi i}) + \varepsilon(t).\] Integrating, we find 
\begin{align*}
	J(\nabla_{s,t})^2 &= \int_X \nu^2\left(\frac{\star \Theta}{2\pi i} + \diag(\mu)\right)\vol \\&= \int_X \left(J(\nabla_{0,0}) - s^2\tr\star(\frac{\beta_t \wedge \beta_t^*}{\pi i}) + \varepsilon(t)\right)^2 \vol\\
	&= J(\nabla_{0,0})^2 +  \varepsilon'(s,t) + \int_X  \left(s^4\tr\star(\frac{\beta_t \wedge \beta_t^*}{\pi i})^4 - C_ts^2\tr\star(\frac{\beta_t \wedge \beta_t^*}{\pi i})^2\right) \vol
\end{align*}
where $C_t$ is some term depending on $t$ which is positive and bounded for sufficiently small $t$ and $\varepsilon'$ is some error term depending on $s$ and $t$ which goes to zero. In particular, one can choose an $s,t$ so small that the term in the integral is negative (since $s^4$ is much smaller than $s^2$ for sufficiently small $s$) and $\varepsilon'$ is negligible, whence $J(\nabla_{s,t})< J(\nabla_{0,0})$, as desired.
\end{proof}
\begin{corollary}
	Suppose the Donaldson-Narasimhan-Seshadri theorem has been proven for lower-rank bundles. If $\E$ is stable, we have $\E \cong \F$. In particular, $J(\nabla_\infty) = \inf J(O_{\nabla_0})$.
\end{corollary}
\begin{proof}
	Recalling Proposition \ref{canonical-extension-proposition}, $\varphi$ factors through
	\begin{equation*}
		\begin{tikzcd}
			0 \arrow[r] & \cal{E'}\arrow[r] & \E \arrow[r] \arrow[d, "\varphi"]& \E'' \arrow[d]\arrow[r]& 0\\
			0  & \F''\arrow[l]& \F \arrow[l] & \F'\arrow[l] &\arrow[l] 0.
		\end{tikzcd}
	\end{equation*}
	Now applying the first estimate to the bottom row, we find that \[ J(\nabla_\infty) \geq \rk \F'(\mu(\F') - \mu(\F)) + \rk \F''(\mu(\F) - \mu(\F'')),  \] and similarly, by the second estimate there is some Chern connection $\nabla_{\E}$ on $\E$ such that \[J(\nabla_{\E}) < \rk \E'(\mu(\E) - \mu(\E')) + \rk \E''(\mu(\E'') - \mu(\E)). \] But by assumtion, $J(\nabla_{\infty}) = \inf J(O_{\nabla_{0}})\leq J(\nabla_{\E})$, and so \[\rk \F'(\mu(\F') - \mu(\F)) + \rk \F''(\mu(\F) - \mu(\F'')) <  \rk \E'(\mu(\E) - \mu(\E')) + \rk \E''(\mu(\E'') - \mu(\E)). \] But using the additivity of ranks and degrees, the fact that $\deg \E'' \leq \deg \F'$ and the fact that $\E$ and $\F$ have the same signatures, we deduce
\end{proof}


			
%			
% --------------------------------------------------------------

% --------------------------------------------------------------



\appendix%%% start appendices here 
\chapter{Preliminaries}
Since notations and conventions vary between sources, the purpose of this section is to collect together the basic definitions and results which will be used throughout the thesis. 

\section{Smooth and Holomorphic Vector Bundles}
In this section, we will recall basic definitions and results. Let $X$ be a complex manifold, which may also be regarded as a smooth manifold.
\begin{definition}\index{vector bundle!}
	Let $\bb{K}\in \{\R, \C\}$. If $\bb{K} = \R$, let $\scr{P} = \text{smooth}$, and if $\bb{K} = \C$, let $\scr{P} \in \{\text{smooth, holomorphic}\}$. A $\scr{P}$-\textit{vector bundle}, or just $\scr{P}$-bundle of rank $n$ over $X$ is a $\scr{P}$-manifold $E$ equipped with a surjective $\scr{P}$-map $\pi: E \rightarrow X$ such that at each $p\in X$, the fibre $E_p:=\pi^{-1}(p)$ has the structure of an $n$-dimensional $\bb{K}$-vector space, and there is an open (in the usual topology) cover $\{U_\alpha\}$ and $\scr{P}$-isomorphisms $\{\Phi_\alpha: \pi^{-1}(U_\alpha) \rightarrow U_\alpha \times \bb{K}^n\}$ such that at every $p\in U_\alpha$ the induced map $\Phi_\alpha|_p: E_p \rightarrow \bb{K}^n$ is a linear isomorphism and the following diagram commutes:
	\begin{equation*}
		\begin{tikzcd}
			\pi^{-1}(U_\alpha) \arrow[r, "\Phi_\alpha"] \arrow[d, "\pi"]& U_\alpha\times \bb{K}^n \arrow[ld] \\
			U_\alpha
		\end{tikzcd}
	\end{equation*}
	where the arrow from $U_\alpha\times \bb{K}^n $ to $U_\alpha$ denotes projection onto the first factor. 
	\\\\
	A $\scr{P}$-\textit{(local) section} over an open subset $U$ is a $\scr{P}$-map $s: U \rightarrow E$ such that $\pi \circ s = \id_U$. A $\scr{P}$-\textit{global section} is a section over $X$. A $\scr{P}$-\textit{frame} is a tuple of sections $(s_1,...,s_n)$ over $U$ such that for all $p\in U$ the set $\{s_1(p),...,s_n(p)\}$ is linearly independent.
	\\\\
	A $\scr{P}$-\textit{morphism} of $\scr{P}$-vector bundles $\pi_E: E \rightarrow X$ and $\pi_F: F \rightarrow X$ is a $\scr{P}$-map $f: E \rightarrow F$ such that the following diagram commutes:
	\begin{equation*}
		\begin{tikzcd}
			E \arrow[r, "f"]\arrow[d, "\pi_E"] & F \arrow[ld, "\pi_F"] \\
			X
		\end{tikzcd}
	\end{equation*}
	and for all $p\in X$ we have that $f|_{E_p}$ is a linear map. A $\scr{P}$-\textit{isomorphism} of $\scr{P}$-bundles is a morphism with a two-sided inverse. $E$ and $F$ are \textit{isomorphic} if there is an isomorphism between them. If $E$ is smooth, then an automorphism of $E$ is known as a \textit{gauge transformation}. The group of smooth automorphisms is known as the \textit{gauge group}.
\end{definition}
For now we give only the most basic example: 
\begin{example}
	The most basic example is $E = X \times \bb{K}^n$ with $\pi$ being projection onto the first factor, known as the \textit{trivial bundle}. A bundle is \textit{trivial} if it is isomorphic to the trivial bundle.
\end{example}
\begin{lemma}
	A bundle is trivial if and only if there is a global frame.
\end{lemma}
\begin{proof}
	Clearly $((x, e_i))_{i = 1}^n$ is a frame for $X \times \bb{K}^n$. Conversely, suppose $(s_i)_{i = 1}^n$ is a frame for $E$. Then every $p\in E$ can be written uniquely as $\sum a_is_i(\pi(p))$ where the $a_i$ are smooth. It is not hard to check that $$\sum a_is_i(\pi(p)) \mapsto (\pi(p),\sum a_ie_i)$$ is a $\scr{P}$-isomorphism. 
\end{proof}
In fact, the notions of local frame and local trivialisation are equivalent: given a local trivialisation $\Phi_i: \pi^{-1}(U_\alpha)\rightarrow U_\alpha\times \bb{K}^n$, we can define $s_i(p):= (p, e_i)$, and conversely, given a local frame $(s_i)_\alpha$, we can define $\Phi_i(\sum a_i s_i(p)) := (p, \sum a_i e_i)$. We will be using this equivalence without further comment. 
\begin{definition}\index{transition map}
	Let $E$ be a $\scr{P}$-vector bundle and let $s_\alpha = (s_i)_\alpha$ and $t_\beta = (t_i)_\beta$ on $U_\alpha$ and $U_\beta$. We define the \textit{transition function} $g_{\alpha\beta}: U_\alpha \cap U_\beta\rightarrow \GL_n(\bb{K})$ equal to be the $\scr{P}$-map \[g_{\alpha\beta}(x):= \Phi_{\alpha, x} \circ \Phi_{\beta, x}^{-1}\]
\end{definition}
Observe that if $t_j = \sum g_{ij} s_i$, then \[g_{\alpha\beta}(e_j) = \Phi_{\alpha} \circ \Phi_{\beta}^{-1}(e_j) = \Phi_{\alpha}(t_j) = \Phi_{\alpha}(\sum g_{ij} s_i) = \sum g_{ij} e_i \] and hence $g_{\alpha\beta} = (g_{ij})$.
\\\\
\par If $\{U_\alpha\}$ is an open cover of $X$ with local frames $s_{\alpha}$, then it is not hard to check that the transition functions satisfy the following conditions:
\begin{enumerate}
	\item $g_{\alpha\beta} = g_{\beta\alpha}^{-1}$.
	\item $g_{\alpha\beta} g_{\beta\gamma}= g_{\alpha\gamma}$ 
\end{enumerate}
These are known as the \textit{cocycle conditions}. Conversely:
\begin{lemma}[Clutching Construction]
	Let $\{U_\alpha\}$ be an open cover of $X$ and suppose for any $\alpha, \beta$ we have a $\scr{P}$-map $$g_{\alpha\beta}: U_\alpha \cap U_\beta \rightarrow \GL_n(\bb{K})$$satisfying the cocycle conditions. Then there exists a unique bundle $E \rightarrow X$ trivial on each $U_\alpha$ with transition functions $g_{\alpha\beta}$.
\end{lemma}
\begin{proof}
	We define $$E^\sharp:= \coprod_\alpha U_\alpha \times \bb{K}^n$$with the induced $\scr{P}$-structure. Now we put an equivalence relation $\sim$ on $E^\sharp$ by declaring $(x,u)_\alpha \sim (y,v)_\beta$ if and only if $x=y$ and $u = g_{\alpha\beta}(x)v$. The cocycle conditions guarantee that this is an equivalence relation. We then define $E:= E^\sharp / \sim$. Since $\scr{P}$-ness is local, we obtain a $\scr{P}$-vector bundle. Now for each $\alpha$ define the local frame $(s_i)_\alpha$ on $U_\alpha$ by \[s_i(p):= (p, e_i)_\alpha \mod \sim\] and observe that with respect to the frame $(s_i)_\alpha$, we have $(t_i)_\beta = g_{\alpha\beta}$ as desired. To check uniqueness, suppose $F$ is another vector bundle with local frames $\{(s_i)_\alpha\}$ that satisfy the same transition functions. We then define an isomorphism $E \rightarrow F$ given by $(x, e_i) \mapsto s_i(x)$ and extend by linearity. It is not hard to check that this is well defined and an isomorphism. 
\end{proof}
\begin{example}\label{tangent-bundle}
	We define the $\scr{P}$-\textit{tangent bundle} $\pi: T_X \rightarrow X$ of $X$ as follows: let $\{(U_\alpha, \phi_\alpha: U_\alpha \rightarrow \bb{K}^n)\}$ be a $\scr{P}$-chart for $X$. Then the tangent bundle at $X$ is the unique bundle that is trivial on each $U_\alpha$ and has transition function $g_{\alpha\beta}$ equal to the Jacobian of $\phi_\alpha\circ\phi_{\beta}$. This may be interpreted as follows: on $\bb{K}^n$, the tangent space is spanned by $\partial_i = \partial/\partial x_i$, where $x_i$ are the $\scr{P}$-coordinates of $\bb{K}^n$. Thus on $U_\alpha$, we define the local frame $(s_i)_\alpha$ by $s_i:= \phi_\alpha^{-1}(\partial_i)$, and hence we obtain a local trivialisation $\Phi_\alpha(s_i):= e_i$. Identifying $e_i$ with $\partial_i$, we obtain \[\Phi_\alpha \circ \Phi_\beta = \phi_\alpha \circ \phi_\beta \] as desired.
	
	Similarly, we may define the \textit{cotangent bundle} $T^*_X$ to be the unique bundle with transition function $\varphi_\beta \circ \varphi_\alpha$; we may interpret the fibre at each point to be the set of linear functionals from $T_pX$. Locally, we may find a basis $dx_i$ dual to $\partial_i$, and on overlaps these satisfy the required transition map. 
 
 	In future, we will often denote the smooth tangent bundle by $T_X$ and the holomorphic tangent bundle by $\mathcal{T}_X$.
\end{example}



Our next theorem connects us back to algebraic geometry:
\begin{theorem}\label{locally-free-sheaves-vector-bundles}
	Let $\Ox$ denote the sheaf of complex $\scr{P}$-functions on $X$, and let $E$ be a vector bundle of rank $n$. Then:
	\begin{enumerate}
		\item The presheaf $\cal{E}$ given by $$\mathcal{E}(U):= \{\text{sections of } E \text{ over } U\}$$ is a locally free $\Ox$-module of rank $n$. Call this the \textbf{sheaf of sections} of $E$.
		\item Any locally free $\Ox$-module of rank $n$ is isomorphic to the sheaf of sections of some unique vector bundle of rank $n$.
		\item The association $E\mapsto \cal{E}$ is an equivalence of categories between the category of vector bundles and locally free sheaves.
	\end{enumerate}
\end{theorem}
\begin{proof}
	It is not hard to check that $\cal{E}$ is indeed a sheaf (indeed, a section over $U$ is a function satisfying a local property), and since each fibre $E_p$ is a vector space, this defines the $\Ox$-module structure. To see that it is locally free, suppose $E$ is trivial on $U$. We define an isomorphism $$\varphi: \bigoplus_{i = 1}^n\Ox|_U \rightarrow  \E|_{U}$$ as follows: let $V$ be an open subset of $U$. Then given $(f_i)\in\bigoplus_{i = 1}^n\Ox(V)$, we interpret this as a map $V \rightarrow \bb{K}^n$. This naturally defines a section $s:V \rightarrow E$ given by $$p \mapsto \Phi^{-1}(p, f_1(p),...,f_n(p))$$ where $\Phi: \pi^{-1}(V) \rightarrow V\times \bb{K}^n$ is a local trivialisation. We then define $\varphi_V(f_i):= s$. It is easy to check that this is a homomorphism of $\Ox(V)$-modules and that it commutes with restriction, hence we have a morphism of sheaves. Conversely, given a section $s: V \rightarrow E$, composing with $\Phi: \pi^{-1}(V) \rightarrow V\times \bb{K}^n$ and projecting onto $\bb{K}^n$ we have a map $V \rightarrow \bb{K}^n$, which is exactly an element of $\bigoplus \O_X(V)$. It is clear that this is the inverse to $\varphi$, hence this proves (i).
	\\\\
	Now let $\E$ be a locally free $\Ox$-module. Cover $X$ with open subsets $\{U_\alpha\}$ on which $\E$ is free. For any $\alpha$, fix an isomorphism of $\Ox(U_\alpha)$-modules $\Phi_\alpha: \Gamma(U_\alpha, \E)\rightarrow \Ox(U_\alpha)^n$. Now we define $$g_{\alpha\beta}:= \Phi_\alpha\circ  \Phi^{-1}_\beta|_{U_\alpha \cap U_\beta} \in \GL_n(\Ox(U_\alpha\cap U_\beta))$$In other words, $g_{\alpha\beta}$ is a matrix of $\bb{K}$-valued $\scr{P}$-functions, which may also be interpreted as a $\scr{P}$-map $$g_{\alpha\beta}:U_\alpha\cap U_\beta \rightarrow\GL_n(\bb{K})$$It is clear they satisfy the cocycle condition, so by the Clutching Construction this gives us a unique vector bundle $F$. Now let $\F$ denote the sheaf of sections of $F$, so that $\F$ is trivial on each $U_\alpha$, and fix a frame $(s_i)_\alpha$ for $F$ on each $U_\alpha$. We define a morphism $\varphi:\F \rightarrow \E$ given by $$\varphi_{U_\alpha}(s_i):= \Phi_{\alpha}^{-1}(e_i)$$where $e_i\in \Ox(U_\alpha)^n$ is the obvious constant function, and extend by linearity. It is then not hard to check that these are isomorphisms of $\Ox(U_\alpha)$-modules and that they glue on overlaps, hence we have a morphism of sheaves. Now these are isomorphisms locally, hence $\varphi$ is an isomorphism of sheaves. This proves (ii).
	\\\\
	To prove 3, it suffices to show that the embedding is fully faithful. Let $\varphi: E \rightarrow F$ denote a morphism of vector bundles. Now given a section $s$ of $E$ over $U$, this induces a section $\varphi^\sharp(s)$ of $F$ by defining \[\varphi^\sharp(s)(p):= \varphi(s(p))\] and it is not hard to check that this is an $\Ox(U)$-module homomorphism. Since it clearly commutes with restriction, we have a morphism of sheaves $\varphi^\sharp:\E \rightarrow \F$. It is clear that $\varphi \mapsto \varphi^\sharp$ is functorial, and that it is faithful. To show that it is full, suppose $\varphi^\sharp: \E \rightarrow \F$ is a homomorphism of sheaves. Now we define a morphism $\varphi: E \rightarrow F$ as follows: let $p\in E$ be a point, and suppose $s$ is a section such that $s(\pi(p)) = p$. We then define $\varphi(p):= \varphi^\sharp(s)(\pi(p))$. To see this is well-defined, suppose $s'$ also goes through $p$. Then \[0 = \varphi(0)(p) = \varphi^\sharp(s - s')(p) = \varphi^\sharp(s)(p) - \varphi^\sharp(s')(p) \] and hence $\varphi$ is well-defined. It is not hard to check that $\varphi$ is $\scr{P}$ and linear on each fibre, and is hence a morphism of vector bundles as desired. This proves (iii).
\end{proof}
\begin{example}
	Given two bundles $E,F$, we may interpret these as locally free sheaves $\E,\F$. Then the \textit{tensor product} of $E$ and $F$, denoted $E\otimes F$ is the bundle associated to $\E \otimes \F$. Similarly, we may define the \textit{direct sum} of $E$ and $F$ to be the bundle associated to $\E \oplus \F$, and the \textit{exterior powers} of $E$, denoted $\bigwedge^p E$ to be the bundle associated to the sheaf $\bigwedge^p \E$. We may also define $\fancyHom(E,F)$ to be the bundle associated to the sheaf-hom $\fancyHom(\E,\F)$, and similarly define $\fancyEnd(E)$ to be $\fancyHom(E,E)$. Note that these differ from the \textbf{groups} $\Hom(E,F)$ and $\End(E)$.
\end{example}
Henceforth, we will make very little distinction between locally free sheaves and vector bundles.
\begin{example}
	In this example, we extend the constructions in Example \ref{tangent-bundle}. Let $T_X$ and $T^*_X$ be the \textbf{smooth} tangent and cotangent bundles. We define the \textit{bundle of $p$-forms}, to be the $p$-th exterior power of $T^*_X$. The sections of this bundle will be called \textit{differential $p$-forms}. 
\end{example}
We will conclude this section with a study of the interplay between the smooth and holomorphic tangent bundles. Let $X^\flat$ denote the underlying smooth manifold of $X$. Picking a local holomorphic chart for $X$, we have a local diffeomorphism $\C^n \rightarrow \R^{2n}$ given by \[(z_1 = x_1+iy_1,...,z_n = x_n+iy_n) \mapsto (x_1, y_1,...,x_n, y_n). \] The holomorphic and smooth tangent bundles are then related as follows: observe that the symbols $\partial x_i, \partial y_i$ act on real-valued functions. Tensoring $T_{X^\flat}$ with the trivial bundle $X^\flat \times \C$, these symbols may be interpreted as acting on complex-valued smooth functions, given by \[\frac{\partial}{\partial x_i}(f + ig) =\frac{\partial f}{\partial x_i} + i\frac{\partial g }{\partial x_i}:= \frac{\partial f}{\partial x_i}\otimes 1 + \frac{\partial g }{\partial x_i}\otimes i \] where $f,g$ are real-valued. Then by the chain rule we have \[\frac{\partial}{\partial z_i} = \frac{\partial}{\partial x_i} - i \frac{\partial}{\partial y_i}. \] We thus conclude that the holomorphic tangent space is spanned by $\partial x_i - i \partial y_i$. However, we also see that \[\frac{\partial}{\partial \bar{z_i}} = \frac{\partial}{\partial x_i} + i \frac{\partial}{\partial y_i}. \] We call the vector bundle spanned locally by the $\partial \bar{z_i}$ the \textit{antiholomorphic tangent bundle}. Now observe that $T_{X^\flat}\otimes \C$ is the direct sum of the holomorphic and antiholomorphic tangent spaces. We will often write \[T_{X^\flat}\otimes \C = T^{1,0}_X \oplus T^{0,1}_X\] for this decomposition.
	
\index{forms of type $p,q$}In fact, this extends to differential forms. A \textit{complex $(p,q)$-form}, or a \textit{complex form of type $(p,q)$}, or simply $(p,q)$-form is a section of $\Omega^{p,q}_X:=(\bigwedge^p T_X^{1,0}) \oplus (\bigwedge^q T_X^{0,1}) $. Locally, a $(p,q)$-form looks like \[f_1 dz_{i_1}+...+f_p dz_{i_p} + g_1 d\bar{z_{j_1}}+...+g_q \bar{z_{j_q}}\] where the $f_i, g_i$ are smooth complex-valued functions. A \textit{complex $r$-form} is a complex $(p,q)$-form such that $p+q=r$. We will denote the bundle of complex $r$-forms by $\Omega^r_X$. Observe that we have a decomposition \[\Omega^r_X = \bigoplus_{p + q = r} \Omega^{p,q}_X. \] Finally, we define the operators $\partial^{p,q}: \Omega^{p,q}_X \rightarrow \Omega^{p+1,q}_X$ and $\delbar^{p,q}:  \Omega^{p,q}_X \rightarrow\Omega^{p, q+1}_X$ to be \[\partial^{p,q}:= \pi_{p+1, q}\circ d \] \[\delbar^{p,q}:= \pi_{p, q+1}\circ d, \] where $d^{p+q}: \Omega^{p+q}_X\rightarrow \Omega^{p+q+1}_X$ is the usual exterior derivative, and the projections are the obvious projections. Note that \[d^0 = \partial^0 + \delbar^0.\] We will often omit the superscripts.


\section{Connections}
In this section, we fix a smooth complex bundle $E$. We define the \textit{bundle of complex $E$-valued $p$-forms}, denoted $\Omega^p_E$, to be $E \otimes \Omega^p_X$, where $\Omega^p_X$ is the sheaf of complex $p$-forms. Note that $\Omega^0_E \cong E$.
\begin{definition}\index{connection}
	A \textit{connection} on $E$ is a morphism of abelian sheaves (\textbf{NOT} as sheaves of modules) $\nabla: \Omega^0_E\rightarrow \Omega^1_E$ that satisfies the following \textit{Leibniz rule} for any $f\in C^\infty(U)$ and local seciton $s$: \[\nabla(fs) = df \otimes s + f\nabla(s) \]
	
	Now let $\{U_\alpha\}$ be a trivialising open cover, and let $(s_i)_\alpha$ be a collection of local frames on $U_\alpha$. We define the \textit{local connection 1-form}, denoted $\omega_\alpha$ to be the matrix such that \[\nabla(s_i) = \sum (\omega_\alpha)_{ij}s_j \] Observe that if $s = \sum a_i s_i$ is a local section, then \[\nabla(s) = \sum_j da_j \otimes s_j + \sum_i a_i (\omega_\alpha)_{ij} s_j\] and hence the local 1-form carries the data of the entire connection.
\end{definition}
Observe that the set of connections is an affine space with underlying vector space $H^0(X, \Omega^1_{\fancyEnd(E)})$. In other words, any two connections differ by an $\End(E)$-valued global 1-form, and conversely, if $\nabla$ is a connection and $L\in H^0(X, \Omega^1_{\fancyEnd(E)})$, then $\nabla + L$ is a connection.
\begin{proposition}\label{connection-transformatino-rule}
	Let $\nabla$ be a connection, and $s_\alpha, t_\beta$ two frames and suppose $g: U_\alpha\cap U_\beta \rightarrow \GL_n(\C)$ satisfies $t_i = \sum g_{ij} s_j$. Then if $\omega_\alpha$ and $\omega_\beta$ are the respective local 1-forms, then we have 
	\begin{equation}\label{connection-transformation-rule-equation}
		\omega_\beta = (dg)  g^{-1} + g\omega_\alpha g^{-1} 
	\end{equation}
	Conversely, if $\{U_\alpha\}$ is a trivialising open cover with local frames $\{s_\alpha\}$, and for each $\alpha$ we have a local 1-form $\omega_\alpha$ that satisfy (\ref{connection-transformation-rule-equation}), then there exists a unique connection with local 1-forms $\omega_\alpha$.
\end{proposition}
\begin{proof}
	\autocite[p. 72]{GriHa}
\end{proof}
A connection $\nabla$ induces an operator $\Omega^p_E\rightarrow \Omega^{p+1}_E$, by asserting, for any $\eta\in \Omega^p_E$ and $s\in \Omega^0_E$ that \[\nabla(\eta s) := d\eta \otimes s + \eta \wedge \nabla(s) \] This allows us to make the following definition:
\begin{definition}\label{curvature}
	Let $\nabla$ be a connection. We define the \textit{curvature} of $\nabla$ to be \[\nabla^2: \Omega^0_E\rightarrow \Omega^2_E \]
\end{definition}
\begin{remark}
	This is \textbf{not} the Laplacian.
\end{remark}
Let us compute the curvature locally. Let $s_\alpha = (s_i)_\alpha$ be a local frame with local 1-form $\omega = \omega_\alpha$. We define the \textit{local curvature matrix} $\Theta_\alpha$ such that \[\nabla^2(s_i) = (\Theta_\alpha)_{ij} s_j \] Let us compute the curvature locally: \[\nabla^2(s_i) = \nabla(\sum_j \omega_{ij} s_j) = \sum_j d \omega_{ij} s_j + \omega_{ij}\nabla(s_j) = \sum_j d \omega_{ij} s_j + \sum_j\sum_k\omega_{ij}\wedge \omega_{jk} s_k  = \sum_j (d \omega_{ij}  + \sum_k\omega_{ik}\wedge \omega_{kj}) s_j  \] and hence \[(\Theta_\alpha)_{ij} = d \omega_{ij}  + \sum_k\omega_{ik}\wedge \omega_{kj} \] we commonly just write \[\Theta_\alpha = d\omega + \omega \wedge \omega \]
\begin{proposition}
	The curvature operator is an $\Ox$-module homomorphism, where $\Ox$ is the sheaf of smooth functions on $X$. In particular, the $\Theta_\alpha$ glue together to a global $\fancyEnd(E)$-valued 2-form $\Theta$.
\end{proposition}
\begin{proof}
	One simply checks that $\nabla^2$ is $C^\infty$-linear. 
\end{proof}
%\begin{theorem}\thlabel{matsu}
%	Let $k$ be a field, $A$ a finitely generated $k$-algebra that is also an integral domain and $\p$ a prime ideal of $A$. Then \[\height \p + \dim A/\p = \dim A\].
%\end{theorem}
%Where $\dim A$ is the Krull Dimension of $A$
%\begin{proof}
%	\autocite[p.92]{Matsu}
%\end{proof}
%
%This is an optional chapter for any additional material that does not fit 
%conveniently into the body of the text (e.g., data, copies of computer programmes). 
%Note that appendices won't necessarily be marked.
\printindex
%\addcontentsline{toc}{chapter}[Index]
\renewcommand{\bibname}{References} % changes the header; default: Bibliography
\printbibliography


\end{document}
