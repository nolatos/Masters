%We have constructed the spaces $V^s_{n,d}$ of stable vector bundles on a curve. The next question to ask is \enquote{what does this space look like}? It turns out in the case $k = \C$ (This choice is necessary, as it gives us access to methods from analysis and differential geometry), there is a correspondence known as the \textit{Narasimhan-Seshadri Correspondence}, or the \textit{Narasimhan-Seshadri Theorem}, which asserts a bijection between stable holomorphic vector bundles (which is equivalent to algebraic vector bundles, by GAGA) of degree zero and irreducible unitary representations of the fundamental group, which turns out to be an homeomorphism of moduli spaces, {\color{red}although we do not show this. }So let us fix once and for all a compact Riemann surface $X$ of genus $g$ (which may be identified as the $\C$-valued points of a nonsingular projective curve over $\C$, given the complex topology). 
In Part I, we have contructed the moduli spaces $V^s_{n,d}$ as a projective GIT quotient. In Part II, we will give another na\"ive moduli space construction for the underlying na\"ive problem of $\mathcal{V}^s_{n,0}$ in the special case $k = \C$. Specifically, if $X$ is a compact Riemann surface of genus $g$ (which may be identified with the $\C$-points of a nonsingular projective curve over $\C$) which we will fix in this chapter, one can identify stable vector bundles of degree zero on $X$ with two other spaces, the basic result being the following:
\begin{theorem}[Narasimhan-Seshadri, 1965]
	Let $X$ be a compact Riemann surface of genus $g$ and suppose $g\geq 2$. Then there is a bijection between $V^s_{n,0}$ with irreducible representations $\pi_1(X) \rightarrow U(n)$ up to conjugation.
\end{theorem}
This is first given as Corollary 1 of \autocite{NS}. Since then, Donaldson gave a different proof in \autocite{Donaldson} by studying unitary connections, and using a correspondence theorem known as the \textit{Riemann-Hilbert Correspondence}. His result is stated as follows:
\begin{theorem}[Donaldson, 1983]
	An indecomposable holomorphic bundle $\E$ over $X$ with a hermitian metric $h$ is stable if and only if there is a unitary connection on $\E$ with curvature equal to a constant multiple of the volume form. Such a connection is unique up to isomorphism.
\end{theorem}
In this chapter, we will use these bijections to give the space of stable bundles another geometric structure, specifically a manifold structure, inherited from the character space $\Hom(\pi_1(X), U(n))$. It turns out that this second structure is homeomorphic to $V^s_{n,0}(\C)$, the latter given the usual complex topology, but very unfortunately we will not be proving this.
\\\\
\indent Of course, since we are moving into analytic territory, some comments are in order. \textit{Open} and \textit{closed} will always mean with respect to the usual complex topology. We will make use of the correpondence between vector bundles and locally free sheaves (of an appropriate structure sheaf) without comment, and we will also use without comment the correspondence between holomorphic vector bundles on $X$ and algebraic vector bundles on $X$. 

\section{Holomorphic Structures on a Smooth Bundle}
\indent The goal of this section is to study the space of holomorphic bundles that restrict to a given smooth bundle. Indeed, as we will see, the degree of a holomorphic bundle is actually a smooth invariant, and in fact, along with the rank, completely classifies $E$! Thus $V_{n,d}(\C)$, the isomorphism classes of all holomorphic bundles with signature $(n,d)$, is equal to the set of holomorphic structures on $E$, up to isomorphism. It turns out that this space is, in turn, canonically identified with the space of unitary connections (to be defined) on $E$, and this correspondence, known as the \textit{Chern Correspondence}, gives us a tool to turn the study of holomorphic bundles into the study of connections. 
\subsection{The Chern Correspondence}
Recall some notation. Let $\Omega^{p,q}_X$ denote the sheaf of smooth $(p,q)$-forms on $X$, and let \[\Omega^{p,q}_E:= \Omega^{p,q}_X \otimes E.\] Note that $\Omega^0_E \cong E$.
\begin{definition}\index{Dolbeault operator}
	A \textit{Dolbeault operator} on $E$ is a homomorphism of abelian sheaves (in particular, \textbf{NOT} as $\Ox$-modules) \[\delbar_E: \Omega^0_E \rightarrow \Omega^{0,1}_E\] such that for any smooth $f\in C^\infty(U)$ and local section $s\in \Omega^0_U$, we have \[\delbar_E(fs) =  \delbar(f)\otimes s + f\delbar_E(s).\]
\end{definition}
\begin{example}
	Let $E$ be the trivial bundle $E = X \times \C^n$. Then the usual $\delbar$ operator is a Dolbeault oeprator
\end{example}
\begin{example}
	Let $\nabla$ be any connection on $E$. Composing $\nabla$ with the projection to its $(0,1)$-component we obtain a Dolbeault operator. This is often known as the \textit{$(0,1)$-component of} $\nabla$.
\end{example}
\begin{example}
	Let $\E$ be a holomorphic bundle whose underlying smooth bundle is $E$. We may think of $\E$ as $E$ equipped with a collection of distinguished local frames, which we deem to be holomorphic. Then there is a natural Dolbeault operator, known as the \textit{canonical Dolbeault operator}, characterised by $\delbar_E(s) = 0$ for any homomorphic section $s$. To see that this is well-defined, let $u: U \rightarrow E$ be a smooth section. Covering $U$ with sufficiently small open subsets $\{U_\alpha\}$ we may assume there are local holomorphic frames $\{(s_i)_\alpha\}$. Now we can write $u_\alpha:=u|_{U_\alpha} = \sum a_i s_i$ where $s_\alpha = (s_i)_\alpha$ is a holomorphic frame on $U_\alpha$ and $a = (a_i)$ is smooth. Then we see \[\delbar_E(u_{\alpha}) = \delbar_E(\sum a_i s_i) = \sum \delbar(a_i)\otimes s_i \] Repeating on $U_\beta$ with local holomorphic frame $t_\beta = (t_i)_\beta$ such that $u_\beta = \sum b_i t_i$ we have $\delbar_E(u_\beta) = \sum \delbar(b_i)\otimes t_i$. Now if $g = g_{\alpha\beta}$ is the transition map, letting $g_{ij}$ denote the $i,j$-th entry of $g$, we observe \[\sum_j \delbar(b_j) \otimes t_j = \sum_j\sum_i \delbar(b_j)\otimes (g_{ij}s_i)=  \sum_i \sum_j\delbar(g_{ij}b_j)\otimes s_i = \sum_i \delbar(a_i)\otimes s_i \] as desired (note that since $g$ is holomorphic we have $\delbar(g) = 0$) and hence by the sheaf axioms this defines $\delbar_E(u)$ uniquely. 
\end{example} 
\begin{caution}
	However, it is important to note that this Dolbeault operator on $E$ is in fact dependent on the \textbf{choice} of frames, and not just the isomorphism class of $\E$. Indeed, we will see very soon that there are different Dolbeault operators which can give rise to isomorphic holomorphic bundles, and in fact we will describe exactly when two distinct Dolbeault operators give rise to the same holomorphic bundle.
\end{caution}
In fact, the converse of the above example is true for Riemann surfaces: if $\delbar_E$ is a Dolbeault operator and there exists an open cover $U_\alpha$ with frames $s_\alpha$ such that $\delbar_E(s_\alpha) = 0$, then it is not hard to show that the transition maps are holomorphic and hence the $\{s_\alpha\}$ define a holomorphic structure on $E$. Such a Dolbeault operator is said to be \textit{integrable}. It can be shown (\autocite[p.555]{AtiBot}) that every Dolbeault operator on a Riemann surface is integrable. Hence Dolbeault operators parameterise holomorphic structures on $E$ (however, we will soon see that they actually over-parameterise holomorphic structures).
\\\\
Next we recall the following definition:
\begin{definition}
	A \textit{Hermitian metric} on $E$, denoted $h$ is the assignment of a complex inner product $h_p(\cdot, \cdot): E_p^2 \rightarrow E_p$ to every $p\in X$ such that for any smooth sections $s,t$ we have that $p \mapsto h_p(s,t)$ is smooth. A \textit{hermitian bundle} is a vector bundle $E$ equipped with a hermitian metric. A \textit{morphism} of hermitian bundles is a smooth morphism of bundles $\varphi: E \rightarrow F$ such that $h_F(\varphi(s),\varphi(t)) = h_E(s,t)$ for any sections $s,t$ if $E$. A hermitian automorphism is known as a \textit{unitary gauge transformation}, and the group of all such automorphisms is known as the \textit{unitary gauge group}. 
	
	A frame $(s_i)_\alpha$ is \textit{unitary} if $h(s_i, s_j) = \delta_{ij}$. A \textit{unitary connection} is a connection $\nabla$ such that \[dh(s,t) = h(\nabla s, t )+ h(s, \nabla t)\] for any smooth sections $s,t$. Given a local frame $(s_i)_\alpha$ on $U_\alpha$, we define the \textit{local matrix of} $h$, denoted $h_\alpha$ such that $(h_\alpha)_{ij}:= h(s_i, s_j)$. 
\end{definition}
\begin{lemma}\label{unitary-iff-skew-hermitian}
	Let $\nabla$ be a connection. Then $\nabla$ is a unitary connection if and only if for any unitary frame $(s_i)_\alpha$, the local 1-form $\omega_\alpha$ is skew-Hermitian.
\end{lemma}
\begin{proof}
	Suppose $\nabla$ is unitary. Note that for any $i,j$, we have $$0 = dh(s_i, s_j) = h(\sum_k(\omega_\alpha)_{ki} s_k, s_j) + h(s_i, \sum_k\omega_\alpha)_{kj} s_k) = (\omega_\alpha)_{ij} + (\overline{\omega}_\alpha)_{ji}$$ and thus $(\omega_\alpha)_{ij} =- (\overline{\omega}_\alpha)_{ji}$ as required.
	
	Conversely, suppose $\omega_\alpha$ is skew-Hermitian and let $s = \sum a_i s_i$ and $t = \sum b_i s_i$ be local sections. Then we can check
	\begin{align*}
		h(\nabla(s), t) + h(s, \nabla(t)) &= h(\nabla(\sum a_i s_i), \sum b_i s_i) + h(\sum a_i s_i, \nabla(\sum b_i s_i))\\
		&=h(\sum_i da_i s_i + a_i\sum_j (\omega_\alpha)_{ij} s_j, \sum_i b_i s_i) \\
		&+ h(\sum_i a_i s_i, \sum_i db_i + b_i \sum_j((\omega_\alpha)_{ij}, s_j))\\
		&= (\sum_i b_ida_i + a_idb_i) + (\sum_j\sum_i a_i (\omega_\alpha)_{ij} b_j) + (\sum_j\sum_i a_j (\overline{\omega}_\alpha)_{ij}b_i)\\
		&= (\sum_i b_ida_i + a_idb_i) + (\sum_j\sum_i a_i (\omega_\alpha)_{ij} b_j) - (\sum_j\sum_i a_j (\omega_\alpha)_{ji}b_i)\\
		&= \sum_i b_ida_i + a_idb_i\\
		&= dh(s,t)
	\end{align*}
	as desired.
\end{proof}
It can be shown by a partition of unity argument (\autocite[III Theorem 1.2]{Wells}) that hermitian metrics exist on any bundle. Similarly, a standard Gram-Schmidt argument will show that smooth unitary frames always exist locally, and finally, we will show that unitary connections always exist:
\begin{theorem}[Chern Correspondence]\index{Chern correspondence}
	Let $\E$ be a holomorphic vector bundle, let $\delbar_E$ be a Dolbeault operator on $E$ giving rise to $\E$, and $h$ a hermitian metric. Then there is a unique unitary connection $\nabla$ on $\E$ with $(0,1)$ component $\delbar_E$. Moreover, if $(s_i)_\alpha$ is a local holomorphic frame, then this connection is described by $\omega_\alpha =(\partial h_\alpha) h_\alpha^{-1}$
\end{theorem}
\begin{proof}
	We first prove uniqueness. Suppose $\nabla$ is such a connection, and let $(s_i)_\alpha$ be a holomorphic frame defined on $U_\alpha$. Then the corresponding matrix of 1-forms $\omega_\alpha$ satisfies \[\nabla(s_i) = \sum_j (\omega_\alpha)_{ij}s_j.\] Observe that since all the $s_i$ are holomorphic, all the entries of $\omega_\alpha$ must be of type $(1,0)$. Now we compute: 
	\begin{align*}
		dh(s_i, s_j) &= h(\nabla s_i, s_j) + h(s_i, \nabla s_j) \\
		&= h(\sum_k (\omega_\alpha)_{ik}s_k, s_j) + h(s_i, \sum_k(\omega_\alpha)_{jk}s_k) \\
		&= \sum_k(\omega_\alpha)_{ik} h(s_k, s_j) + \overline{(\omega_\alpha)_{jk}} h(s_i,s_k)
	\end{align*}
	But $dh(s_i, s_j)  = \partial (h_\alpha)_{ij} + \delbar (h_\alpha)_{ij}$, and thus comparing types we must have $(\partial h_\alpha)_{ij} = \sum_k(\omega_\alpha)_{ik} (h_{\alpha})_{kj}$ and letting $i,j$ vary, we observe \[ \omega_\alpha h_\alpha = \partial h_\alpha \] as required. This proves uniqueness.
	
	To prove existence, we define the connection to be $\omega_\alpha:=  (\partial h_\alpha)h_\alpha^{-1}$ for any holomorphic frame $(s_i)_\alpha$, and extend by the Leibniz rule. By the proof of uniqueness, this satisfies the properties in the theorem, thus we just need to check that this is well-defined. So let $(t_i)_\beta$ be another holomorphic frame, and suppose $g$ satisfies $t_i = \sum g_{ij}s_j$. Then it is not hard to see \[h_\beta = gh_\alpha g^*\] where $g^*$ is the conjugate transpose of $g$. By Proposition \ref{connection-transformatino-rule}, it suffices to show that $\omega_\beta = (dg)g^{-1}+ g\omega_\alpha g^{-1}$. We compute:
	\[\partial h_\beta = \partial (gh_\alpha g^*) = (\partial g) h_\alpha g^* + g(\partial h_\alpha)g^* + gh_\alpha (\partial g^*) = (\partial g) h_\alpha g^* + g(\partial h_\alpha)g^* \]  since $g$ is holomorphic. Thus \[\omega_\beta = (\partial h_\beta )h_\beta^{-1} = ((\partial g) h_\alpha g^* + g(\partial h_\alpha)g^* )(g^*)^{-1}h_\alpha g^{-1} = (\partial g)g^{-1}+ g(\partial h_\alpha)h_{\alpha} ^{-1}g^{-1} = (dg)g^{-1}+ g\omega_\alpha g^{-1} \] as desired. 
\end{proof}
\begin{definition}\index{Chern connection}
	The unitary connection in the above theorem is the \textit{Chern connection}.
\end{definition}
\begin{remark}
	Again, it is important to note that different unitary connections could give rise to the same holomorphic bundle. This is why we will often use the phrase \enquote{a Chern connection}.
\end{remark}
We sum up the above results as follows: as established, there is a 1-1 correspondence between holomorphic structures and Dolbeault operators. Now the natural follow-up question to that is given a Dolbeault operator $\delbar_E$, is there a canonical connection we can put on $E$ with (0,1)-component $\delbar_E$? The Chern correspondence answers this in the affirmative, with the choice of a hermitian metric placed on $E$. Thus, in order to study holomorphic structures, we can study unitary connections instead. 

Finally, we describe the gauge group action on the space of connections. To begin, we consider the gauge group action on the space of Dolbeault operators. So let $u$ be a gauge transformation, and $\delbar_E$ a Dolbeault operator. We define \[(u \cdot \delbar_E)(s):= u\delbar_E(u^{-1}(s))\] it is easy to check that this is a group action that results in a Dolbeault operator. 
\begin{proposition}
	Let $\delbar_1, \delbar_2$ be two Dolbeault operators on $E$, and $\E_1, \E_2$ their associated holomorphic bundles. Then $\E_1 \cong \E_2$ if and only if there is a gauge transformation $u$ such that $\delbar_2 = u \cdot \delbar_1$. Moreover, $u: \E_1 \rightarrow \E_2$ is one such isomorphism.
\end{proposition}
\begin{proof}
	Suppose firstly that $\delbar_2 = u\cdot\delbar_1 = u\delbar_1u^{-1}$. Now let $s$ be a holomorphic section of $\E_1$. Observe \[0 = \delbar_1(s) = \delbar_1(u^{-1} u(s)) = u\delbar_1(u^{-1}u(s)) = \delbar_2(u(s))\] and hence $u(s)$ is a holomorphic section of $\E_2$. By the same argument, if $t$ is a holomorphic section of $\E_2$, then $u^{-1}(t)$ is holomorphic in $\E_1$ as desired.
	
	Conversely, suppose $\E_1 \cong \E_2$ and let $u: \E_1 \rightarrow \E_2$ be an isomorphism. Now let $s_\alpha = (s_i)_\alpha$ be a holomorphic frame of $\E_1$; whence by the above calculation we have $u\delbar_1(u^{-1}u(s_i)) = 0$. But also $\delbar_2(u(s_i))= 0$ (since $u(s_i))$ is holomorphic). Since $u\delbar_1u^{-1}$ and $\delbar_2$ agree on a collection of frames on an open cover of $X$, they must be equal.
\end{proof}
We now want to extend this to the space of all unitary connections, so let $\nabla$ be a unitary connection with Dolbeault operator $\delbar_E$. By the Chern correspondence, the Dolbeault operator $u\cdot \delbar_E$ corresponds to a unique unitary connection; thus we simply need to find a unitary connection with $(0,1)$-component $u\cdot \delbar_E$. To this end, observe that as in the space of all connections, the space of unitary connections is an affine space. The underlying vector space is the subspace of $H^0(X, \Omega^1_{\fancyEnd(E)})$ consisting of 1-forms with values in a skew-hermitian endomorphism; that is, an endomorphism $F$ such that \[h(F(s), t) + h(s, F(t)) = 0\] Observe then, that \[-(\delbar_{\fancyEnd E}u)u^{-1} +  ((\delbar_{\fancyEnd E}u)u^{-1} )^*\] is clearly skew-hermitian, (where $(\delbar_{\fancyEnd E}u)(s):= \delbar_E(u(s))- u\delbar_E(s)$ is the \textit{induced Dolbeault operator}) and $u^*$ satisfies $h(us, t) = h(s, u^*t)$) and has $(0,1)$-component equal to \[(-(\delbar_{\fancyEnd E}u)u^{-1} +  (\delbar_{\fancyEnd E}u)u^{-1} )^*)^{0,1} = -(\delbar_{\fancyEnd E}u)u^{-1} \] since $((\delbar_{\fancyEnd E}u)u^{-1} )^*$ is of type (1,0). Hence $u\cdot \nabla$ defined by
\begin{equation}\label{gauge-group-unitary-connection}
	(u\cdot \nabla)(s) := \nabla(s) -(\delbar_{\fancyEnd E}u)u^{-1}(s) +  ((\delbar_{\fancyEnd E}u)u^{-1} )^*(s)
\end{equation}
is a unitary connection, and its associated Dolbeault operator is equal to  \[\delbar_E(s) -(\delbar_{\fancyEnd E}u)u^{-1}(s) = \delbar_E(s) - \delbar_E(uu^{-1}(s)) + u\delbar_E(u^{-1}(s)) = u\delbar_E(u^{-1}(s)) \] as desired. Hence defining the action of the gauge group by the formula in (\ref{gauge-group-unitary-connection}) (and it is easy to check that this is indeed a group action) extends the gauge group action on Dolbeault operators, and in particular two unitary connections induce isomorphic holomorphic structures if and only if they lie in the same gauge orbit. In summary, we have the following bijection:\[\{\text{Holomorphic structures on }E\}/\text{isomorphism} \leftrightarrow \{\text{unitary connections on }E \}/ \text{gauge equivalence}. \]
{\color{red} We conclude this section with a discussion about the Sobolev completions of various spaces we have been working with, and how the above bijection extends. To begin, we first recall that since $X$ is a Riemann surface, it is K\"ahler with its Fubini-Study form. This fixes a volume form $\vol$ and hence a Hodge star product $\star$. The \textit{Hodge inner product} is defined on $\Omega^p_X$ as \[\langle\alpha, \beta\rangle := \int_X \alpha \wedge \star \beta. \] Completing this with }
\subsection{Degree of a Smooth Bundle}
In this section, we will be giving a classification of all smooth vector bundles on $X$. In particular, we will show that the degree of a bundle is actually a \textbf{smooth} invariant, and in fact, along with the rank, the only smooth invariant there is! Thus for each signature $(n,d)\in \N \times \Z$, there is a unique smooth bundle with that signature. The vehicle for showing this is the invariant known as the \textit{first Chern class}. To define it, we fix a smooth bundle $E$. We begin with a result:
\begin{lemma}
	Let $\nabla_1, \nabla_2$ be connections on $E$ with curvature forms $\Theta_1, \Theta_2$ respectively. Then $\tr \Theta_1$ and $\tr \Theta_2$ are cohomologous.
\end{lemma}
\begin{proof}
	Since $X$ is a Riemann surface, clearly $\tr \Theta_i$, being a two-form, is closed, thus the statement makes sense. Now $\nabla_1-\nabla_2$ is a global $\fancyEnd(E)$-valued 1-form; call it $A$. Let $\omega_1, \omega_2$ be respective local 1-forms of $\nabla_1$ and $\nabla_2$ on a local frame. Then locally, we have \[\Theta_2 = d\omega_2 + \omega_2\wedge \omega_2 = d(\omega_1+A) + (\omega_1 + A)\wedge (\omega_1 + A) = \Theta_1 + dA + A\wedge A + \omega_1 \wedge A + A \wedge \omega_1 \] hence \[\tr(\Theta_2) = \tr(\Theta_1) + \tr(dA) + \tr(A\wedge A) + \tr(\omega_1 \wedge A + A \wedge \omega_1). \] It is not hard to show that $\tr(A\wedge A)$ and $\tr(\omega_1 \wedge A + A \wedge \omega_1)$ are both zero, from the antisymmetry of the wedge product. Since $A$ is a global form, the $dA$ glue to a global exact 2-form and hence the result follows.
\end{proof}
\begin{definition}\index{Chern class! first}
	The \textit{first Chern class} of $E$ is defined to be the cohomology class \[ c_1(E):=\left[\tr(\frac{i}{2\pi} \Theta)\right]\in H^2_{\text{DR}}(X) \] where $\Theta$ is the curvature form for any connection. By the above lemma, this does not depend on the connection.
\end{definition}
The key theorem we will be proving in this section is the following:
\begin{theorem}\label{chern-weil-degree}
	For any holomorphic bundle $\E$ with underlying smooth bundle $E$, we have \[\int_X c_1(E) = \deg(\E),\] where $X$ is given the standard orientation $idz \wedge d\bar{z}$ for any holomorphic coordinate $z$.
\end{theorem}
Let us take a moment to appreciate this result. We defined the degree of a line bundle $\L$ on a curve $X$ over an algebraically closed field $k$ to be the degree of its corresponding divisor class, and we defined the degree of a vector bundle $\E$ to be the degree of $\det(\E)$. Note that this is purely algebraic. The theorem states that in the case $k = \C$, where we have access to analytic tools, when we equip $\E$ with a connection (any connection, in fact), take its curvature, take the trace of the curvature, multiply by $i/2\pi$ and integrate it, we get, not only an integer, but the same integer representing the degree of the divisor associated to its determinant bundle!

The way to prove this is to reduce to the case of line bundles, and to do this, we present the following well-known result:
\begin{theorem}[Structure Theorem of Smooth Vector Bundles]
	There is a diffeomorphism \[E \cong \det E \oplus \Ox^{\rk E - 1}. \]
\end{theorem}
Before we give the proof, we first recall it can be shown (\autocite[II, Theorem 15.3]{Bredon}) that every section $s$ has a section $s'$ which intersects $s$ transversally (i.e. if $s$ and $s'$ intersect at $P\in X$, then $\im s_{*,P}\oplus \im s'_{*,P} = T_{s(P)}E$; in other words the images of the differential of $s, s'$ at $P$ generate the tangent space of $s(P)$ in $E$). 
\begin{proof}
	We proceed by induction on the rank, with the rank 1 case being trivial. Now supposing $rk E > 1$, we observe that by the above there is a section $s$ which intersects the zero section transversally. However, the real dimension of $E$ is $\dim_\R E = 2+ 2\rk E > 4$ and thus $s$ and the zero section cannot intersect at all (otherwise the space spanned by the images of their differential is at most 4), or in other words $s$ is nonvanishing. In particular, the bundle spanned by $s$, which is a trivial line bundle, is a subbundle of $E$, and thus we may write $E = \Ox \oplus E'$ for some complement $E'$ (for example, placing a hermitian metric $h$ on $E$ and taking the orthogonal complement of $s$ with respect to $h$), and by the inductive hypothesis the result follows.
\end{proof}
\par Before we proceed, we review the Snake Lemma; in particular how the \enquote{snake} is constructed. We recall the statement:
\begin{proposition}[Snake Lemma]\index{Snake Lemma}
	Suppose we have the following diagram of $A$-modules with exact rows:
	\begin{equation*}
		\begin{tikzcd}
			0 \arrow[r] &A^0  \arrow[r]\arrow[d, "d^A"] &B^0\arrow[d, "d^B"] \arrow[r]& C^0 \arrow[d, "d^C"]\arrow[r] & 0 \\
			0 \arrow[r] &A^1  \arrow[r] &B^1 \arrow[r, "f"]& C^1 \arrow[r] & 0
		\end{tikzcd}
	\end{equation*}
	Then there is a map $\delta: \ker d^C \rightarrow \coker d^A$ such that the following sequence is exact:
	\begin{equation*}
		\begin{tikzcd}[column sep = small]
,`				0 \arrow[r] & \ker d^A \arrow[r] & \ker d^B  \arrow[r] & \ker d^C \arrow[dll, 
			rounded corners,
			to path={ -- ([xshift=2ex]\tikztostart.east)
				|- (Z) [near end]\tikztonodes
				-| ([xshift=-2ex]\tikztotarget.west)
				-- (\tikztotarget)}] \\ 
			& \coker d^A \arrow[r] & \coker d^B \arrow[r] & \coker d^C \arrow[r] & 0
		\end{tikzcd}
	\end{equation*}
\end{proposition}
The $\delta$ above is constructed as follows: Take $c\in \ker d^C$. Since the top row is exact, there exists some $b\in B^0$ that maps to $c$. Now observe that since $d^C(c)
= 0$, it must follow that the image of $d^B(b)\in \ker f$, by the commutativity of the diagram. Since the bottom row is exact, this pulls back to some unique $a\in A^1$, and moreover it can be shown that the image of $a$ in $\coker d^A$ does not depend on our choice of $b$, and in particular it is well-defined. Hence we define $\delta(c):= b$, and one can show that the resulting sequence is exact.
\\\\
\par Now we begin our investigation. We clearly have the short exact sequence of abelian groups:
\begin{equation*}
	\begin{tikzcd}
		0 \arrow[r] & \Z \arrow[r] & \C \arrow[r, "\exp"] & \C^* \arrow[r] &0
	\end{tikzcd}
\end{equation*}
where $\exp$ above is the map $x \mapsto \exp(2\pi i x)$. Now let $\scr{P}$ denote the property of smoothness or holomorphicity (in particular, statements made about $\scr{P}$ will be valid in both the smooth and holomorphic settings). Taking the sheaf of $\scr{P}$-functions with values in the above groups, we have the following short exact sequence of sheaves, known as the \textit{exponential sheaf sequence}: 
\begin{equation}\label{exponential-sequence}
	\begin{tikzcd}
		0 \arrow[r] & \underline{\Z} \arrow[r] & \Ox \arrow[r, "\exp"] & \Ox^* \arrow[r] &0
	\end{tikzcd}
\end{equation}
where $\Ox$ is the sheaf of $\scr{P}$-functions on $X$, and $\underline{\Z}$ is the constant sheaf $\Z$. Recall that $H^i(X, \underline{\Z}) \cong H^i_{\text{sing}}(X, \Z)$ (\autocite[pp. 42-43]{GriHa}). Now taking cohomology of (\ref{exponential-sequence}), we have:
\begin{equation}\label{exponential-sequence-cohomology}
	\begin{tikzcd}[column sep = small]
		0 \arrow[r] & H^{0}(X, \Z) \arrow[r] & H^{0}\left(X, \Ox\right)  \arrow[r] & H^{0}\left(X, \Ox^*\right) \arrow[dll, 
		rounded corners,
		to path={ -- ([xshift=2ex]\tikztostart.east)
			|- (Z) [near end]\tikztonodes
			-| ([xshift=-2ex]\tikztotarget.west)
			-- (\tikztotarget)}] \\ 
		& H^{1}\left(X, \Z\right) \arrow[r] & H^{1}\left(X, \Ox\right) \arrow[r] & H^{1}\left(X, \Ox^*\right) ...
	\end{tikzcd}
\end{equation}
and in particular, we have a map $\delta: H^1(X, \Ox^*)\rightarrow H^2(X, \Z)$. Now recall that $H^1(X, \Ox^*)$ is the Picard group of $X$; in particular it parameterises the isomorphism classes of $\scr{P}$-line bundles on $X$. 
\begin{proposition}\label{exp-sequence-chern-class}
	Under the inclusion $H^2(X, \Z) \subseteq H^2(X, \R) \cong H^2_{\operatorname{DR}}(X)$, we have \[\delta(L) = -c_1(L) \] for any $\scr{P}$-line bundle $L$
\end{proposition}
\begin{proof}
	This follows the proof given in \autocite[pp. 141-142]{GriHa}. Let $\{U_\alpha\}$ be a sufficiently fine open cover (in particular, one where we can take local logarithms on overlaps) and let $\{g_{\alpha\beta}\}$ be a \v{C}ech cocycle $L$. We define local inverses \[h_{\alpha\beta}:= \frac{1}{2\pi i}\log g_{\alpha\beta} \] and by the construction of the snake map of the Snake Lemma, it follows that $\{h_{\alpha\beta} - h_{\alpha\gamma} + h_{\beta\gamma}\}$ is a 2-cocycle of $\underline{\Z}$ representing $\delta(L)$. 
	
	Next we look at $c_1(L)$. Fix a connection and let $\{\omega_\alpha\}$ be the associated 1-forms. By Proposition \ref{connection-transformatino-rule}, we have \[\omega_\beta = (dg_{\alpha\beta})  g_{\alpha\beta}^{-1} + g_{\alpha\beta}\omega_\alpha g_{\alpha\beta}^{-1} = g_{\alpha\beta} dg_{\alpha\beta} + \omega_\alpha \] hence \[\omega_\beta - \omega_\alpha = g_{\alpha\beta}^{-1}dg_{\alpha\beta} = d \log g_{\alpha\beta} \] and in particular $d\omega_\alpha = d\omega_\beta$. Now we compute the curvature. Since we are working with a line bundle, $\omega$ is just a usual 1-form; it follows $\omega \wedge \omega = 0$, hence \[\Theta = d\omega_\alpha = d \omega_\beta. \] Finally, we reconcile de Rham and \v{C}ech cohomology. Note that by the Poincare Lemma we have the following short exact sequences of sheaves (recall exactness of sheaves is measured locally):
	\begin{equation}\label{poincare-zero-forms}
		0 \rightarrow \underline{\R} \rightarrow \Omega^0_X \rightarrow \mathcal{Z}^1 \rightarrow 0
	\end{equation}
	and
	\begin{equation}\label{poincare-1-forms}
		0 \rightarrow \mathcal{Z}^1 \rightarrow \Omega^1_X \rightarrow \mathcal{Z}^2 \rightarrow 0
	\end{equation}
	where $\mathcal{Z}^p$ denotes the sheaf of closed $p$-forms on $X$. Now it can be shown \autocite[p. 42]{GriHa} that $\Omega^p_X$ is acyclic, hence we have 
	\begin{equation}\label{0-cohomology-2-forms}
		H^0(X, \mathcal{Z}^2)/ H^0(X, \Omega^1_X) \cong H^1(X, \mathcal{Z}^1)
	\end{equation}
	by the long exact sequence of (\ref{poincare-1-forms}) and 
	\begin{equation}\label{2-cohomology-1-forms}
		H^1(X, \mathcal{Z}^1) \cong H^2(X, \R) 
	\end{equation}
	by the long exact sequence of (\ref{poincare-zero-forms})
	Now we compute. Using (\ref{0-cohomology-2-forms}), and the construction of the snake map, the image of $\Theta$ in $H^1(X, \cal{Z}^1)$ is represented by the \v{C}ech 1-cocycle $\{i(\omega_\beta - \omega_\alpha)/ 2\pi\}$, and using (\ref{2-cohomology-1-forms}), the image of $\Theta$ in $H^2(X, \R)$ is \[\{\frac{i}{2\pi} (\log g_{\alpha\beta} - \log g_{\alpha\gamma} + \log g_{\beta\gamma})\}= -\{h_{\alpha\beta} - h_{\alpha\gamma} + h_{\beta\gamma}\} = - \delta(L) \] as desired.
\end{proof}
\begin{corollary}
	In the smooth category, or in the holomorphic category with $g = 0$, every line bundle is uniquely determined by its first Chern class.
\end{corollary}
\begin{proof}
	In both cases $\Ox$ is acyclic.
\end{proof}

Next we prove Theorem \ref{chern-weil-degree} for line bundles:
\begin{theorem}\label{chern-weil-degree-line}
	If $\L$ is a holomorphic line bundle with underlying smooth bundle $L$, then we have:
	\begin{equation}\label{chern-degree-equation-line}
		\int_X c_1(L) = \deg \L.
	\end{equation}
\end{theorem}
\begin{proof}
	We first prove the case where $\L$ is the line bundle of a prime divisor. Let $D$ be a prime divisor, supported on $P\in X$, suppose $\L = \L(D)$. Since $D$ is effective, by Proposition \ref{divisor-of-zeroes-effective}, it is the divisor of zeroes of some global section $s\in H^0(X, \L)$ that vanishes exactly once, at $P$. In particular, $s$ is a frame on $X \setminus \{P\}$. Now fix a hermitian metric $h$. By the Chern correspondence, a Chern connection of $(\E, h)$ is given locally by $h^{-1}_\alpha\partial h_\alpha$ with respect to a holomorphic frame, and the curvature by $dh^{-1}_\alpha \partial h_\alpha= d\partial \log h_\alpha$ (here we may take logarithms because $h_\alpha > 0$). Now for an $\varepsilon> 0$, write $U_\varepsilon$ for the open set $\{p\in X\mid h(s(p), s(p)) > \varepsilon \}$ and $\overline{U}_\varepsilon$ for its closure, define $s_\varepsilon$ to be the frame $s|_{U_\varepsilon}$ and finally write $h_\varepsilon:= h(s_\varepsilon, s_\varepsilon) > 0$. We compute: \[\int_X c_1(L) = \lim_{\varepsilon \to 0} \int_{\overline{U}_\varepsilon}c_1(L) = \lim_{\varepsilon \to 0} \int_{\overline{U}_\varepsilon}\frac{i}{2\pi} d\partial \log h_\varepsilon = \lim_{\varepsilon \to 0}\int_{\partial U_\varepsilon} \frac{i}{2\pi}\partial \log h_\varepsilon \] where the final equality follows from Stoke's theorem. Now since $s|_{X \setminus \overline{U}_\varepsilon}$ vanishes exactly at $P$, we may pick a holomorphic coordinate chart centred at $P$ such that $s = z$. Hence 
	\begin{equation}\label{orientation-chern-class}
		\int_{\partial U_\varepsilon} \partial \log h_\varepsilon= - \int_{|z| = \varepsilon} \partial (\log z + \log \bar{z} + \log h_z(1,1))
	\end{equation} integrating anticlockwise as usual. The negative sign is there due to our choice of orientation. Now observe that since $\log h_z(1,1)$ is smooth, it follows $\partial \log (h_z(1,1))/\partial z$ is continuous, and since $\{|z| \leq \delta \}$ is compact, for some sufficiently small $\delta$, it follows $\partial \log (h_z(1,1))/\partial z$ is bounded on $\{|z| \leq \delta \}$. Thus \[\lim_{\varepsilon \to 0}\left| \int_{|z| = \varepsilon} \partial \log h(1,1)\right| = \lim_{\varepsilon \to 0} \left|\int_{|z| = \varepsilon} \frac{\partial \log h_z(1,1)}{\partial z} dz\right| \leq \lim_{\varepsilon\to 0} 2\pi \varepsilon \sup\{ \left|\frac{\partial \log h_z(1,1)}{\partial z}\right| \mid |z|\leq \delta \} = 0. \] Finally, we have, by Cauchy's integration formula, \[\int_{|z| = \varepsilon} \partial (\log z + \log \bar{z}) =  \int_{|z| = \varepsilon} \frac{dz}{z} = 2\pi i  \] as desired. 
	
	Finally, we observe that $H^2(X, \Z)\cong \Z$, and by Proposition \ref{exp-sequence-chern-class}, it follows \[c_1(L_1 \otimes L_2) = c_1(L_1)+ c_2(L_2)\] Hence if $D = \sum n_i P_i$ is any divisor and $\L = \L(D)$, it follows $\L = \otimes \L(P_i)^{\otimes n_i}$ hence \[\int_X c_1(L) = \int_X c_1(\otimes \L(P_i)^{\otimes n_i}) = \int_X \sum n_i c_1(\L(P_i)) = \sum n_i  = \deg \L \] as desired.
\end{proof}

\begin{example}\label{chern-class-line-bundle-on-p1}
	Suppose $X = \P^1$. We will compute the Chern class of $\Ox(n)$. Suppose the homogeneous coordinates on $X$ are $[x_0:x_1]$, and let $U_0 = \{x_0\neq 0\}$ and similarly with $U_1$. We define $z_0:= x_1/x_0$ to be the affine coordinate on $U_0 \cong \A^1$ and similarly for $z_1$. Now by definition of $\Ox(n)$, we may find frames $s_0, s_1$ on the corresponding affine patches such that \[g_{0,1}:= \frac{s_1}{s_0} = z_0^n = z^{-n}_1\] We define the hermitian metric on $\Ox(n)$ to be \[h_0 := h(s_0, s_0) = (1+|z_0|^2)^{-n} \]  on $U_0$, and \[h_1:= h(s_1, s_1) = (1+|z_1|^2)^{-n}. \] Of course, we need to check this is well-defined, that is \[h_0 = h(s_0, s_0) = h(z_1^n s_1, z_1^ns_1) = |z_1|^{2n}h_1. \] To this end observe \[h_0 = (1+|z_0|^2)^{-n} = (1+\frac{1}{|z_1|^2})^{-n} = (\frac{|z_1|^2+1}{|z_1|^2})^{-n} = |z_1|^{2n}(|z_1|^2+1)^{-n} = |z_1|^{2n}h_1\] as claimed. Hence for brevity, we will simply write \[h = (1+|z|^2)^{-n}\] understanding that this works for any affine patch, and finally observe that both affine patches have complement measure zero, and thus we may ultimately just work on one affine patch. 
	
	We continue. By the Chern correspondence, the unitary connection $\omega$ is given by \[\omega = h^{-1}\partial h = -\frac{n\bar{z}}{1+ |z|^2}dz \] and now computing the curvature $\Theta$: \[\Theta = d\omega = \frac{-n}{(1+|z|^2)^2} d\bar{z} \wedge dz =\frac{n}{(1+|z|^2)^2} dz\wedge d\bar{z} \]	and hence \[c_1(\Ox(n)) = \frac{ni}{2\pi(1+|z|^2)^2} dz\wedge d\bar{z} \] Finally, we integrate. Write $z = x+iy$ and we interpret the real coordinates $x,y$ as the standard coordinates of $\R^2$. Then $idz\wedge d\bar{z} = 2dx\wedge dy$, and letting $r,\theta$ denote the polar coordinates, we deduce \[\int_X c_1(\Ox(n)) = \int_U \frac{n}{\pi(1+|z|^2)^2}dx\wedge dy = \int_0^ \infty \int_0^{2\pi} \frac{nr}{\pi(1+r^2)^2}d\theta dr =n \] as desired.
\end{example}

\begin{proof}[Proof of Theorem \ref{chern-weil-degree}]
	By the structure theorem we know $E = \det E \oplus \Ox^{\rk E -1}$. Then given connections on $\det E$ and $\Ox^{rk E - 1}$, we can build a connection on $E$ with a diagonal matrix of associated 1-forms. More precisely, if $\nabla$ is a connection on $\det E$ with local 1-forms $\{\omega_\alpha\}$, then $\{\diag(\omega_\alpha,0,...,0)\}$ is the matrix of 1-forms for a connection on $\det E \oplus \Ox^{\rk E - 1}= E$. Thus \[\int_X c_1(E) = \int_X c_1(\det E) = \deg(E) \] as desired.
\end{proof}

\begin{corollary}
	For each $(n,d)\in \N \times \Z\cong \N \times H^2(X, \Z)$, there exists a unique smooth bundle $E$ over $X$ with signature $(n,d)$.
\end{corollary}

To summarise, we have given an interpretation of $\mathcal{V}_{n,d}$, the set of isomorphism classes of holomorphic vector bundles of signature $(n,d)$ as the set of holomorphic structures on the unique smooth bundle $E$ of signature $(n,d)$. Our earlier work on the Chern correspondence in turn describes this set as equal to the affine space of unitary connections on $E$ modulo gauge equivalence. Now the next question is, where does stability fit in all of this? 
\section{The Riemann-Hilbert Correspondence}
In this section, we will be constructing and demonstrating the equivalence between the moduli space of flat unitary connections and the so-called \textit{character variety} (which, despite the name, is not actually a variety by any definition) of $U(n)$, the space of representations of $\pi_1(X)$ modulo conjugacy. Since some of the more enlightening examples will not be on a Riemann surface, we will instead work with an arbitrary smooth manifold $Y$. To begin, we give the following informal example:
\begin{example}\label{circle-riemann-hilbert}
	Being the free group on one element, there is a canonical isomorphism $\Hom_{\Gps}(\Z, G )\cong G$ for any group $G$. In the case $G = \R^*$, there is a perverse way to see this: firstly, let $U_1 = S^1 \setminus \{1\}$ and $U_2 = S^1\setminus\{-1\}$. Let $u\in \R^*$, and let $E\rightarrow S^1$ be the (real) line bundle defined by gluing $U_1\times \R$ and $U_2 \times \R$ together with locally constant transition function \[g_{1,2}(z):=\begin{cases}
		1&\text{ if } \Im(z) > 0\\
		u&\text{ if } \Im (z) < 0
	\end{cases} \] on the intersection of $U_1$ and $U_2$ (which is just the intersection of $S^1$ and the union of the two half-planes). Observe that if $u < 0$ we get a M\"obius strip, but if $u > 0$ we get a cylinder. Now for any $x\in \R$, define the section $s_x$ to be the section $s_x(z) := (z, x)\in U_1 \times \R$, and define the path $\gamma: [0,1)\rightarrow E$ as \[\gamma(t) := \begin{cases}
	(1, x) \in U_2 \times \R&\text{ if } t = 0 \\
	s_x(\exp(2i\pi t))&\text{ otherwise}
	\end{cases}. \] Observe that the section is smooth, and \[\lim_{t\to 1} \gamma(t) = (1, ux)\in U_2\times \R; \] and in particular by \enquote{transporting} the vector $x$ along the loop, we have caused it to increase by a factor of $u$. In particular, this may be thought of as a map $\pi_1(S^1) \rightarrow \R^*$, taking a loop and seeing how much it deforms a vector transported along the loop. In this particular case, $\pi_1(S^1) =\Z$, and the map $\pi_1(S^1) \cong \Z\rightarrow \R^*$ is given by $1\mapsto u$. 
\end{example}
And indeed, this is a very simple example of the Riemann-Hilbert correspondence, which relates flat connections to representations of the fundamental group. The key concept here is the idea of \textit{parallel transport}, which we will now study, beginning with the following definition:
\begin{definition}
	Let $\nabla$ be a connection on a smooth vector bundle $E$, let $\gamma:[0,1]\rightarrow U$ be a piecewise smooth path and suppose $\gamma^\sharp: [0,1] \rightarrow E$ is a lift of $\gamma$ (that is, $\pi\circ\gamma^\sharp = \gamma$). We say $\gamma^\sharp$ is \textit{parallel} if \[\nabla_{\gamma'(t)}(\gamma^\sharp(t)) = 0 \] for every $t\in [0,1]$ such that $\gamma(t)$ is smooth.  
\end{definition}
\begin{proposition}
	Let $\gamma$ be a piecewise smooth path in $Y$ and let $(p,v)\in E$. Then there exists a unique parallel lift $\gamma^\sharp$ such that $\gamma^\sharp(0) = (p,v)$.
\end{proposition}
\begin{proof}
	Since we may break up our path into finitely many pieces, we may suppose without generality $\gamma$ is contained in some open subset $U_\alpha$ on which $E$ is trivial. Let $(s_1,...,s_n)_\alpha$ be a frame for $E$ on $U_\alpha$ and suppose $(\omega_\alpha)_{ij}$ is the associated matrix of 1-forms for $\nabla$. We are solving the first-order linear ODE \[\sum_j da_j(\gamma(t))(\gamma'(t))\otimes s_j(\gamma(t)) + \sum_i a_i(\gamma(t)) (\omega_\alpha)_{ij}(\gamma'(t))s_j \] in $t$ with an initial condition, and hence there exists a unique solution.
\end{proof}
\begin{example}
	We will construct a connection $\nabla$ such that that $s_x\circ \gamma$ in Example \ref{circle-riemann-hilbert} is a parallel section with respect to the loop $\gamma(t) = \exp(2\pi i t)$. To this end, let $s,t$ be the sections $s(z): = (z,x)\in U_1 \times \R$ and $t(z):=(z,x)\in U_2\times\R$. One then defines both 1-forms of $\nabla$ to be zero, it is clear they satisfy the transformation rule and that $s_x\circ \gamma$ is parallel. 
\end{example}
And now we can define parallel transport:
\begin{definition}
	Let $\gamma$ be a piecewise smooth path in $Y$, let $P\in E$ and let $\gamma^\sharp$ be the unique parallel lift along $\gamma$ beginning at $P$. The \textit{parallel transport} of $P$ is the element $\gamma^\sharp(1)$. 
\end{definition}
Now observe that since the ODE for parallel transport is linear, this means that parallel transport itself is linear. In fact, we have the following:
\begin{proposition}
	Let $E$ be a bundle over $X$, and let $\nabla$ be a unitary connection. Then for any $P\in X$ and any loop beginning and ending at $P$, parallel transport $T: E_P \rightarrow E_P$ is unitary. 
\end{proposition}

\section{An Overview of Donaldson's Proof}
\par Finally, we will conclude the thesis with an exposition of Donaldson's paper \autocite{Donaldson}. This builds on earlier work by Atiyah and Bott in \autocite{AtiBot}, and provides {\color{red} a short proof of the theorem of Narasimhan and Seshadri}. We fix the following data: let $\E$ be an indecomposable holomorphic vector bundle of signature $(n,d)$ with underlying smooth bundle $E$ and fix a hermitian metric $h$. Furthermore, since $X$ is a compact Riemann surface, it is K\"ahler, and we make the further assumption that the volume of $X$ is 1 (that is, we fix a volume form such that $\int_X \vol = 1$). The result is the following:
\begin{theorem}[Donaldson-Narasimhan-Seshadri]
	The bundle $\E$ is stable if and only if there is some unitary connection $\nabla$ on $E$ giving rise to $\E$ with curvature $\Theta\in H^0(\Omega^2_X)\otimes \End E$ satisfying 
	\begin{equation}\label{NS-equality}
		\Theta = -2\pi i \mu \vol \otimes \id_E 
	\end{equation}
	Moreover, $\nabla$ is unique up to the action of the unitary gauge group.
\end{theorem}
\begin{example}
	Of course, over $\P^1$ the only stable bundles are line bundles. So let $\Ox(n)$ be a line bundle. In Example \ref{chern-class-line-bundle-on-p1}, we defined a hermitian metric and computed the Chern class, Chern connection and curvature. Now we will need to compute the volume form. Of course, $X$ is obviously K\"ahler with its Fubini-Study metric (which can be realised as the metric of $\mathcal{T}_X = \Ox(2)$ or its dual $\Ox(-2) = \Omega_X^1$ described in Example \ref{chern-class-line-bundle-on-p1}), and so by taking the real part of this complex inner product, we have a natural Riemannian structure. Locally, if we pick an affine patch with holomorphic coordinate $z = x + iy$, and frame $(\partial x, \partial y)$ of $T_X$ (the real smooth tangent space) the metric is given by \[g = \begin{pmatrix*}
		\frac{1}{\sqrt{\pi}(1 + x^2+y^2)^2} & 0 \\
		0 & \frac{1}{\sqrt{\pi}(1 + x^2+y^2)^2} 
	\end{pmatrix*} \] The $\sqrt{\pi}$ is there so that the resulting volume is 1. An orthonormal frame is given by $(\sqrt{\pi}(1 + x^2+y^2)\partial x , \sqrt{\pi}(1 + x^2+y^2)\partial y)$, and hence the volume form is \[\vol = \frac{dx \wedge dy}{\pi(1 + x^2 + y^2)^2} = \frac{idz \wedge d\bar{z}}{2\pi(1 + |z|^2)^2}. \] Now we computed the curvature of the Chern connection on $\Ox(n)$ to be \[ \Theta =\frac{n}{(1+|z|^2)^2} dz \wedge d\bar{z} = -2\pi i \deg(\Ox(n)) \vol \] as expected. Hence the theorem is true for $\P^1$.
\end{example}
In fact, we will first prove the theorem for line bundles in general:
\begin{theorem}
	The Donaldson-Narasimhan-Seshadri theorem is true for line bundles.
\end{theorem}
\begin{proof}
	Of course, if $\L$ is a line bundle with hermitian metric $h$, underlying smooth bundle $L$ and Chern connection $\nabla$, then it is already stable. Thus we reduce to showing that a connection $\nabla'$ in the orbit of $\nabla$ with curvature in the form (\ref{NS-equality}) exists. 
	
	To this end, we observe that the curvature $\Theta$ of $\nabla$ is just an imaginary global (1,1)-form (since $\fancyEnd \L$ is trivial; the identity endomorphism is a global frame), so $i\Theta$ differs from its harmonic representative $i\Theta'$ by a real exact  1-form, say $i\Theta - d\eta= i\Theta'$. Now observe that since $\Theta'$ is harmonic, $d\star \Theta' = 0$, and so $\star\Theta'$ is a constant; necessarily equal to $-2\pi i\mu$. So we reduce once again to showing that there is a gauge transformation $g$ such that $\Theta'$ is the curvature of $g \cdot \nabla$. 
	
	Observe that $d\eta$ is real and closed and therefore the following Poisson equation has a solution (\autocite[Theorem 4.7]{Aubin}): \[2\delbar\partial f = \Delta f = id\eta. \] Now write $g:= \exp f$, let $\nabla' = g\cdot \nabla$, and write $\Theta'$ for the curvature of $\nabla'$. Firstly, observe that since $\fancyEnd \L$ is trivial, the operators induced by the connection, $\partial_{\fancyEnd \L}$ and $\delbar_{\fancyEnd \L}$, are just the usual $\partial$ and $\delbar$ operators. Hence \[\Theta' = \theta - d(\delbar g)g^{-1} + d\overline{(\delbar g)g^{-1}} = \Theta - \partial \delbar f +  \]
\end{proof}

Observe that the condition (\ref{NS-equality}) is a little awkward to work with, so we introduce the \textit{Donaldson $J$-functional} on the {\color{red} Sobolev }space of unitary connections, which satisfies the property that $J(\nabla) = 0$ if and only if $\nabla$ satisfies (\ref{NS-equality}). It is defined as follows: Firstly recall that the \textit{trace norm} (which despite its name, is not a norm in general) of a square matrix $M\in \C^{r\times r}$ is defined to be \[\nu(M) := \tr((MM^*)^{\frac{1}{2}}), \] where $(MM^*)^{\frac{1}{2}}$ is the unique positive semidefinite matrix $B$ such that $B^2 = MM^*$, which exists since $MM^*$ is hermitian (and hence diagonalisable) and positive semidefinite. In fact, if $M$ is diagonalisable, it is easy to see that \[\nu(M) = \sum |\lambda_i|, \] where the sum is taken across all eigenvalues of $M$, counting multiplicity. The key property is the following:
\begin{lemma}
	For any hermitian matrix $M$, we have \[\nu(M) = \sup_{\{s_i\}}\sum_{i = 1}^n |\langle Ms_i, s_i \rangle|, \] where the supremum is taken across all unitary bases $\{s_i\}$ of $\C^n$.
\end{lemma}
\begin{proof}
	We first observe that $M$ has a unitary basis of eigenvectors, say $\{v_i\}$ and letting $\{s_i\} = \{v_i\}$ we deduce \[\nu(M) = \sum|\lambda_i| = \sum_{i = 1}^n |\langle Mv_i, v_i \rangle| \leq  \sup_{\{s_i\}}\sum_{i = 1}^n |\langle Ms_i, s_i \rangle|. \] For the reverse inequality, let $\{s_i\}$ be a unitary basis, and let $(g_{ij})\in U(n)$ denote the matrix taking $\{v_i\}$ to $\{s_i\}$; that is, $s_i = \sum g_{ij}v_j.$ We compute:
	\begin{align*}
		\sum_{i = 1}^n |\langle Ms_i, s_i\rangle | &= \sum_{i = 1}^n|\langle M\sum_{j = 1}^n g_{ij} v_j,\sum_{k = 1}^n g_{ik}v_k \rangle | \\
		&= \sum _{i = 1}^n |\sum_{j = 1}^ng_{ij}\langle Mv_j, \sum_{k = 1}^n g_{ik}v_k \rangle| \\
		&= \sum _{i = 1}^n |\sum_{j = 1}^ng_{ij}\langle Mv_j, g_{ij}v_j \rangle|\\
		&= \sum _{i = 1}^n |\sum_{j = 1}^n\lambda_j\langle g_{ij}v_j, g_{ij}v_j \rangle|\\
		&\leq \sum_{i = 1}^n\sum_{j = 1}^n |\lambda_j||\langle g_{ij}v_j, g_{ij}v_j \rangle|\\
		&= \sum_{i = 1}^n \sum_{j = 1}^n |\lambda_j||g_{ij}|^2 = \sum_{j = 1}^n |\lambda_j|
	\end{align*} 
	as desired.
\end{proof}
Of course, this in itself is not particularly interesting or useful, but it does give us two very important corollaries:
\begin{corollary}\label{trace-norm-properties}
	Let $H(n)$ denote the vector space of hermitian $n$-by-$n$ matrices.
	\begin{enumerate}
		\item $\nu$ is a norm on $H(n)$.
		\item If $M\in H(n)$ is can be written in the form \[M = \begin{pmatrix*}
			A & B\\
			B^* & C
		\end{pmatrix*}, \] then $\nu(M) \geq |\tr A| + |\tr C|$.
	\end{enumerate}
\end{corollary}
\begin{proof}
	To prove (i), we need only check the triangle inequality. So suppose $M,N\in H(n)$ are given. Then \[\nu(M +N) = \sup_{\{e_i\}} \sum |\langle (M+N)e_i, e_i \rangle| = \leq \sup_{\{e_i\}} \sum |\langle Me_i, e_i \rangle|+ |\langle Ne_i, e_i \rangle|\leq \nu(M) + \nu(N) \] as desired. To prove (ii), let $\{e_i\}$ denote the standard basis of $\C^n$. Then \[\nu(M) \geq \sum_{i = 1}^n |\langle M e_i, e_i \rangle | \geq |\sum_{i = 1}^{\rk A} \langle M e_i, e_i \rangle | + |\sum_{i = \rk A + 1}^n \langle M e_i, e_i \rangle | = |\tr A | + |\tr C| \] as desired.
\end{proof}

With this in mind, we define the \textit{$N$-norm} on the space of self-adjoint smooth endomorphisms of $E$, as \[N(s):= \left(\int_X \nu^2(s) \vol\right)^{\frac{1}{2}}. \] By the above corollary, this is a norm.

Now let $\nabla$ be a unitary $W^{1,2}$-connection with curvature $\Theta\in H^0(\Omega_{\fancyEnd E}^{2})$. Since the matrix of a unitary connection with respect to a unitary frame is skew-hermitian, its curvature $\Theta$ is also skew-hermitian, and since the volume form is real, it follows that $\star\Theta$ is also skew-hermitian. In particular, it follows that $\frac{\star \Theta}{2\pi i}$ is actually \textbf{hermitian}. Thus we define the \textit{Donaldson $J$-functional} as \[J(\nabla):= N(\frac{\star \Theta}{2\pi i} + \diag(\mu))= \left(\int_X \nu^2\left(\frac{\star \Theta}{2\pi i} + \diag(\mu)\right)\vol\right)^{\frac{1}{2}}, \] where $\nu^2(s) := (\nu(s))^2$. Observe that $J=0$ if and only if $\nabla$ is a unitary connection of the type we want (known as, \textit{projectively flat}, or \textit{Yang-Mills connections}). Thus we have turned our problem into one of finding zeroes of $J$.

The rough idea of the proof is as follows: we fix a reference Chern connection $\nabla_0$ of $\E$, and use gauge transformations to find our desired $\nabla$. Denote the $W^{2,2}$-gauge orbit of $\nabla_0$ by $O_{\nabla_0}$. We will show that if $\E$ is stable, then the infimum of $J(O_{\nabla_0})$ is attained; that is there is some $\nabla\in O_{\nabla_0}$ such that $J(\nabla) = \inf J(O_{\nabla_0})$. One then deduces that the infimum must be zero, by looking at near $\nabla$. In order to deduce that the infimum is attained, we take a minimising sequence (that is, a sequence $\nabla_i$ such that $J(\nabla_i)\to \inf J(O_\nabla)$) in $O_{\nabla_0}$ and extract, using Uhlenbeck's weak compactness theorem (to be stated), a weakly convergent subsequence that converges to $\nabla$. Now $\nabla$ defines a holomorphic bundle, say $\F$, and the key property is that $\Hom(\E, \F)\neq 0$. So we take a nonzero $\varphi: \E \rightarrow \F$, and apply Proposition \ref{canonical-extension-proposition} to get a factorisation of $\varphi$ through two exact rows, and apply estimates to these rows to deduce that $\E$ is stable if and only if $\E \cong\F$. The converse (that if there is some connection annihilating $J$ then $\E$ is stable) also follows from these estimates.
\\\\
\par Now we begin with a {\color{red} statement of Uhlenbeck's compactness theorem}:
\begin{theorem}[Uhlenbeck's weak compactness]
	Let $(\nabla_i)$ be a sequence of $W^{1,2}$-connections with curvatures $(\Theta_i)$, and suppose the sequence $(||\Theta_i||_{L^2}:=\int_X \tr(\Theta_i) \wedge \tr(\star \overline{\Theta_i}))$ is bounded. Then there is a sequence of $W^{2,2}$-gauge transformations $(g_i)$ and a subsequence $(\nabla_{i_k})$ such that $(g_{i_k}\cdot \nabla_{i_k})$ weakly converges to some $\nabla_\infty$ (that is, $\int_X \tr(g_{i_k}\cdot \Theta_{i_k}) \wedge \tr(\star A) \to\int_X \tr(\Theta_\infty) \wedge \tr(\star A) $ for all $W^{1,2}$-connections $A$).
\end{theorem}
\begin{proof}
	\autocite[p. 41]{Uhlenbeck}.
\end{proof}
So let $(\nabla_i)$ be a sequence in $O_{\nabla_0}$ with curvatures $(\Theta_i)$, such that $J(\nabla_i)\to \inf J(\O_{\nabla_0})$. In order to use the theorem, we need to check that $||\Theta_i||_{L^2}$ is bounded. To this end, we first observe that $N(\star \Theta_i)$ is bounded, since $J(\nabla_i)$ is and $N$ is a norm. Now note that \[\nu^2(\star \Theta_i) \vol =\tr(\sqrt{(\star \Theta_i)(\star \Theta_i)^*})^2\vol,\] and similarly, \[\tr(\Theta_i) \wedge \star\tr(\overline{\Theta_i}) = \tr(\star\Theta )(\star\Theta)^*\vol. \] Since all norms are equivalent in finite dimensions, it follows that there is some $m,M>0$ such that for any matrix $A$ we have \[m\tr(AA^*)\leq \tr\sqrt{AA^*}^2 \leq M \tr(AA^*),   \] thus since $\{N(\star\theta_i)\}$ is bounded, it follows that $\{||\Theta_i||_{L^2}\}$ is also bounded. Thus (replacing $\nabla_i$ with $g\cdot \nabla_i$, and replacing the sequence with a weakly convergent subsequence) we may assume without loss of generality $\nabla_i$ converges weakly to some $\nabla_\infty$, which is a unitary connection and hence defines a holomorphic bundle, say $\F$ with signature $(n,d)$.
\begin{proposition}
	Let $\E, \F$ be as above.
	\begin{enumerate}
		\item Then $\inf J(O_{\nabla_0}) \geq \inf J(O_{\nabla_\infty})$.
		\item The group $\Hom(\E, \F)$ is nonzero.
	\end{enumerate}
\end{proposition}
\begin{proof}[Proof Sketch]
	To prove (i), we first observe that for any $\varepsilon > 0$ the set $C_\varepsilon = \{\alpha\in \End E \mid N(\alpha + \diag(\mu)) < J(\nabla_\infty) - \varepsilon\}$ is convex and closed, and thus by the Hahn-Banach separation theorem, we can separate $\frac{\star \Theta}{2\pi i}$ from $C_\varepsilon$ by a hyperplane. Now if \[\inf J(O_{\nabla_\infty}) > \inf J(O_{\nabla_0}) = \liminf_{n\to \infty} J(\nabla_i), \] then picking some $\varepsilon_0$ such that $\inf J(O_{\nabla_\infty})  - \varepsilon_0 > \inf J(O_{\nabla_0})$, we find that infinitely many $\frac{\star \Theta_i}{2\pi i}$ lie in $C_{\varepsilon_0}$. But that means $\Theta_i$ cannot converge weakly to $\Theta_\infty$ in $L^2$, and since the curvature is a bounded linear operator, it follows that weak convergence is preserved, and thus we have a contradiction. This proves (i).
	
	To prove (ii), we first observe that an element of $\Hom(\E, \F)$ is just a global section of $\fancyHom(\E, \F) = \E^\vee \otimes \F$. Now the underlying smooth bundle of $\E^\vee \otimes \F$ is just $\fancyEnd(E) = E^\vee \otimes E$, and it is easy to see that given any Dolbeault operators $\delbar_{\E}$ and $\delbar_{\F}$ giving rise to the holomorphic structures on $\E$ and $\F$, the operator \[\delbar_{\E^\vee \otimes \F} := 1\otimes\delbar_{\F} - \delbar_{\E}\otimes 1\] is a Dolbeault operator for $\E^\vee \otimes \F$. Since $\E$ and $\F$ have Chern connections $\nabla_0$ and $\nabla_\infty$ respectively, and since the $\nabla_i$ for $i \geq 0$ all give rise to the same (more precisely isomorphic) holomorphic structures, we can take the $(0,1)$ part of these connections to build the Dolbeault operators $\delbar_{i, \infty}:= (1\otimes \nabla_\infty - \nabla_i \otimes 1)_{0,1}$. Now to say $\Hom(\E, \F) = 0$ is to say that the Dolbeault operators $\delbar_{i, \infty}$ for $i \geq 0$, considered as maps $\End E \rightarrow \End E \otimes H^0(X, \Omega^{0,1})$ have trivial kernel. One can then apply the theory of elliptic operators and the Sobolev embedding theorem to deduce that $\delbar_{0, i}:= 1\otimes\nabla_i -\nabla_0\otimes 1 $ also has no kernel. But that would imply $\End \E = 0$, a clear contradiction.
\end{proof}
With this result in hand, we fix a nonzero homomorphism $\varphi: \E \rightarrow \F$ and apply Proposition \ref{canonical-extension-proposition}, so that we have the following commutative diagram:
	\begin{equation}\label{proper-factorisation-equation-donaldson}
	\begin{tikzcd}
		0 \arrow[r] & \cal{E'}\arrow[r] & \E \arrow[r] \arrow[d, "\varphi"]& \E'' \arrow[d]\arrow[r]& 0\\
		0  & \F''\arrow[l]& \F \arrow[l] & \F'\arrow[l] &\arrow[l] 0
	\end{tikzcd}
\end{equation}
with exact rows, $\E' \cong \ker \varphi$, $\E''\cong \im \varphi$ and $\rk \E'' = \rk \F'$, $\deg \E'' \leq \deg \F'$. The key now is to apply estimates to these rows. 
\begin{proposition}[First Estimate]
	Consider the following short exact sequence of vector bundles: \[ 0 \rightarrow \F' \rightarrow \F \rightarrow \F'' \rightarrow 0,\] and suppose $\mu(\F') \geq \mu(\F)$. Then if $\nabla_\F$ is a unitary connection on $E$ giving rise to the holomorphic structure of $\F$, we have \[J(\nabla_\F) \geq \rk \F'(\mu(\F') - \mu(\F)) + \rk \F''(\mu(\F) - \mu(\F'')). \] Equality holds only if the sequence splits.
\end{proposition}
\begin{proof}
	Firstly, we fix a local unitary frame $s_\alpha$ compatible with a local holomorphic splitting and consider the matrix of one-forms of $\nabla_\F$, which is skew-hermitian. One can show that it has the shape \[\omega_\alpha = \begin{pmatrix*}
		\omega_\alpha' & \beta_\alpha \\
		-\beta^*_\alpha & \omega_\alpha''
	\end{pmatrix*}, \] where the $\beta_\alpha$ glue to the second fundamental form (c.f. Remark \ref{ext-dolbeault}) $\beta $, and the $\omega_\alpha'$ are the 1-forms of a Chern connection $\nabla'$ on $\F'$, and similarly with $\omega_\alpha''$. If we compute the curvature, we see that it is of the form \[\Theta_\F = \begin{pmatrix*}
	\Theta' - \beta \wedge \beta^* & \nabla_{\F'', \F}\beta\\
	-\nabla_{\F'', \F}\beta^*& \Theta'' - \beta^* \wedge \beta
	\end{pmatrix*}, \] where $\Theta'$ and $\Theta''$ are the curvatures of $\nabla'$ and $\nabla''$ respectively and $\nabla_{\F'', \F}: \Omega^1(\fancyHom(\F'', \F))\rightarrow \Omega^2(\fancyHom(\F'', \F))$ is built from the connections $\nabla', \nabla''$ (see \autocite[p. 78]{GriHa} for details). Now by Corollary \ref{trace-norm-properties}, it follows that \[\nu\left(\frac{\star\Theta_\F}{2\pi i} + \diag_{\rk \F}(\mu)\right) \geq \left|\tr\left(\frac{\star(\Theta' - \beta \wedge \beta^*)}{2\pi i} + \diag_{\rk \F'}(\mu)\right)\right| + \left|\tr\left(\frac{\star(\Theta''- \beta^*\wedge \beta)}{2\pi i} + \diag_{\rk \F''}(\mu)\right)\right|, \] where $\mu = \mu(\F)$. Applying H\"older's inequality, we deduce
	\begin{align*}
		J(\nabla_\F) &\geq \int_X\nu\left(\frac{\star\Theta_\F}{2\pi i} + \diag_{\rk \F}(\mu)\right) \vol\\
		&= \left|\int_X\tr\left(\frac{\star(\Theta' - \beta \wedge \beta^*)}{2\pi i} + \diag_{\rk \F'}(\mu)\right)\vol\right| + \left|\int_X	·\tr\left(\frac{\star(\Theta''- \beta^*\wedge \beta)}{2\pi i} + \diag_{\rk \F''}(\mu)\right)\vol\right|
	\end{align*}
	 Let us consider the term $\star(\beta \wedge \beta^*)$. Observe that $\beta$ is a (0,1)-form and so $\beta \wedge \beta^*$ has entries of the form $|f|d\bar{z} \wedge dz$ for any holomorphic coordinate $z$. Since, by our conventions, our orientation is $ i dz \wedge d\bar{z}$, this means that $-i\tr \star(\beta \wedge \beta^*)$ will be nonnegative.
	 
	 Next we observe that by Theorem \ref{chern-weil-degree}, we have \[\int_X\tr\left(\frac{\star \Theta'}{2\pi i}\right)\vol= -\deg \F' \leq -\rk \F' \mu(\F) = -\tr \diag_{\rk \F}(\mu) = -\int_X \tr \diag_{\rk \F'}(\mu) \vol,\] where the last equality follows from the assumption $\int_X \vol = 1$. Hence 
	 \begin{align*}
	 	\left|\int_X\tr\left(\frac{\star(\Theta' - \beta \wedge \beta^*)}{2\pi i} + \diag_{\rk \F'}(\mu)\right)\vol\right| &= -\int_X\tr\left(\frac{\star(\Theta' - \beta \wedge \beta^*)}{2\pi i} + \diag_{\rk \F'}(\mu)\right)\vol\\
	 	&=\rk\F'(\mu(\F')- \mu(\F)) +\frac{1}{2\pi i} \tr \star(\beta \wedge \beta^*)
	 \end{align*} and note that $\frac{1}{2\pi i} \tr \star(\beta \wedge \beta^*) \geq 0$ by the above discussion. Similarly, note that \[\left|\int_X	\tr\left(\frac{\star(\Theta''- \beta^*\wedge \beta)}{2\pi i} + \diag_{\rk \F''}(\mu)\right)\vol\right| = \rk \F''(\mu(\F) - \mu(\F'')) +\frac{1}{2\pi i} \tr \star(\beta\wedge \beta^*) \]

	And putting it all together we get 
	\begin{align*}
		J(\nabla_\F) &\geq \rk\F'(\mu(\F')- \mu(\F))+\rk \F''(\mu(\F) - \mu(\F'')) +\frac{1}{\pi i} \tr \star(\beta \wedge \beta^*) \\&\geq \rk\F'(\mu(\F')- \mu(\F))+\rk \F''(\mu(\F) - \mu(\F'')).
	\end{align*}
	as desired. Finally, if equality occurs, that means $\beta = 0$, but $\beta$ defines an element of $\Ext^1(\F'', \F')$ via the Dolbeault cohomology representation of sheaf cohomology (Remark \ref{ext-dolbeault}), and in particular if it is zero then the sequence splits. 
\end{proof}

And in fact, from this we may already deduce one direction of the Donaldson-Narasimhan-Seshadri theorem:
\begin{corollary}
	Suppose $\E$ is indecomposable, and there is a Chern connection $\nabla$ giving rise to $\E$ such that $J(\nabla) = 0$. Then $\E$ is stable.  
\end{corollary}
\begin{proof}
	Suppose for contradiction $\E$ is not stable. Then there is some subbundle $\E'$ such that $\mu(\E') \geq \mu(\E)$, whence $\mu(\E) \geq \mu(\E/\E')$. Then \[0 = J(\nabla) \geq \rk \E'(\mu(\E') - \mu(\E)) + \rk(\E/\E')(\mu(\E) - \mu(\E/\E')) \geq 0,\] which means the sequence \[0 \rightarrow \E' \rightarrow \E \rightarrow \E/\E' \rightarrow 0 \] splits, by the above proposition, contradicting the indecomposability of $\E$.
\end{proof}
\begin{remark}
	In fact, we can deduce that if $\E$ has a Chern connection which is a zero of $J$, then $\E$ must be polystable, since the proposition tells us that $\E$ can be written as a direct sum of two subbundles of equal slope.
\end{remark}

Our next estimate applies to the top row. However, it is more technical and requires the stronger hypothesis that the Donaldson-Narasimhan-Seshadri theorem has been proven for bundles of smaller rank:
\begin{proposition}[Second Estimate]
	Consider the following short exact sequence of vector bundles: \[ 0 \rightarrow \E' \rightarrow \E \rightarrow \E'' \rightarrow 0,\] suppose this exension is proper, that $\E$ is stable and the Donaldson-Narasimhan-Seshadri theorem has been proven for bundles of rank less than $\rk \E$. Then there exists a unitary connection $\nabla_{\E}$ on $E$ giving rise to $\E$ such that \[J(\nabla_{\E}) < \rk \E'(\mu(\E) - \mu(\E')) + \rk \E''(\mu(\E'') - \mu(\E)). \]
\end{proposition}
\begin{proof}[Proof Sketch]	
	The idea here is to use the Harder-Narasimhan and Jordan-H\"older filtrations and the inductive hypothesis to build this $\nabla_{\E}$. So let $(\E_i')$ be the Harder-Narasimhan filtration of $\E'$, and for each $i$ let $(\E_{ij}')$ denote the Jordan-H\"older filtration of $\E_i'/\E_{i-1}'$. Now since $\rk \E_{i,{j}}'/\E_{i,j-1}< \rk \E$, by assumption we know that there is a projectively flat Chern connection $\nabla'_{ij}$ on $\E_{i,{j}}'/\E_{i,j}$. Now given any \[0 \rightarrow \E'_{i,j}\rightarrow \E'_{i,{j+1}}\rightarrow \E'_{i,{j+1}}/\E'_{i,j}\rightarrow 0 \] with second fundamental form $B_{i,j}$, one can inductively (starting with $j = 0$) build a connection on $\E'_{i,{j+1}}$ from one on $\E'_{i,j}$ and the on $\E'_{i,{j+1}}/\E'_{i,j}$ given to us, and now letting the $i$ vary we can build a connection on each $\E'_i$. Now given any short exact sequence of vector bundles \[0 \rightarrow \F' \rightarrow \F \rightarrow \F''\rightarrow 0\] with second fundamental form $B\in \Ext(\F'', \F')$, one can scale $B$ by any nonzero constant $t\in \C\setminus \{0\}$ and the resulting bundle in the middle is isomorphic to $\F$, by the proof of Theorem \ref{if-extension-then-jump}. In particular, given any short exact sequence from any of our filtrations above (which either looks like \[0 \rightarrow \E'_{i,j}\rightarrow \E'_{i,{j+1}}\rightarrow \E'_{i,{j+1}}/\E'_{i,j}\rightarrow 0 \] or \[0 \rightarrow \E'_{i}\rightarrow \E'_{i+1}\rightarrow \E'_{i+1}/\E'_{i}\rightarrow 0), \] the connection built in the middle is of the form  \[\begin{pmatrix*}
		\nabla_1, B \\
		-B^*, \nabla_2
	\end{pmatrix*}\] where $\nabla_1$ and $\nabla_2$ are connections on the left and right respectively and $B$ is the second fundamental form. Now as mentioned, we may scale $B$ by any nonzero constant $t> 0$ and retain the same extension class, so the matrix\[\begin{pmatrix*}
		\nabla_1, tB \\
	-tB^*, \nabla_2
	\end{pmatrix*}\] gives rise another Chern connection on the middle bundle. Doing this (with a fixed $t > 0$) for every step of both filtrations, we have a collection of Chern connections $\{\nabla'_t\}$ on $\E'$, but their limit $\nabla'_0$ is a Chern connection for $\bigoplus_{i,j} \E'_{ij}$, and moreover by construction we have \[\star \Theta'_0 = -2\pi i \diag(\mu (\E_{ij}')), \] where $\Theta'_t$ is the curvature of $\nabla'_t$. Similarly, we can build a collection of Chern connections $\nabla''_t$ on $\E''$ that converge to a connection $\nabla''_0$ on some $\bigoplus_{i',j'} \E''_{i'j'}$ and $\star\Theta''_0 =- 2\pi i \diag(\mu(\E''_{i',j'}))$.

	Let $[\beta]\in \Ext^1(\F'', \F')$ denote the extension class of $\E$. Now for each $\nabla_t', \nabla''_t$, one can build a connection $\nabla^t_{\E'', \E'}$ on $\fancyHom(\E'', \E')$, and for each $\nabla^t_{\E'', \E'}$, standard arguments from Hodge theory tell us that there is a representative of $[\beta]$, call it $\beta_t$, such that $\nabla^t_{\E'', \E'}(\beta_t) = 0$. Now letting $s>0$ be another variable, we have connections depending on $s$ and $t$ \[\nabla_{s,t} =\begin{pmatrix*}
		\nabla_t' & s\beta_t \\
		-s\beta_t^*& \nabla_t''
	\end{pmatrix*} \] with curvature \[\Theta_{s,t} = \begin{pmatrix*}
	\Theta_t' - s^2\beta_t\wedge \beta^*_t & 0 \\
	0 & \Theta_{t}'' - s^2\beta_t^*\wedge\beta_t
\end{pmatrix*} \] that converge to $\nabla_{0,0}$ with curvature $\Theta_{0,0} =\diag(\Theta_0', \Theta_0'')$. Now we observe \[\tr (\frac{\star \Theta_0'}{2\pi i} + \diag_{\rk \E'}(\mu(\E)))= \sum (\mu(\E) - \mu(\E'_{ij})) = \rk \E'(\mu(\E) - \mu(\E')) > 0\] and similarly \[\tr (\frac{\star \Theta_0''}{2\pi i} + \diag_{\rk \E''}(\mu(\E)))  = \rk \E''(\mu(\E) - \mu(\E'')) < 0, \] and put together this tells us \[J(\nabla_{0,0}) = \rk \E'(\mu(\E) - \mu(\E')) + \rk \E''(\mu(\E'') - \mu(\E)). \] Our task now is to show that for sufficiently small $s,t$ we have $J(\nabla_{s,t}) < J(\nabla_{0,0})$. To this end, we first note that since $A' := \diag(\mu - \mu(\E'_{ij}) )$ is a diagonal matrix with positive entries and hence has negative eigenvalues, it follows that $\nu(A') = \tr A'$, and hence the same is true for matrices sufficiently close to $A'$.  Now it can be shown that the $i\tr \star(\beta_t^*\wedge \beta_t)$ {\color{red} are uniformly bounded}, and hence for sufficiently small $s,t$, it follows 
\begin{align*}
	\nu(\frac{\star(\Theta_{t}' - s^2 \beta_t\wedge \beta_t^*)}{2\pi i} + \diag_{\rk \E'}(\mu)) &=\tr (\frac{\star\Theta_{t}' - s^2 \beta_t\wedge \beta_t^*}{2\pi i} + \diag_{\rk \E'}(\mu)) \\&= \rk \E'(\mu(\E) - \mu(\E')) - s^2\tr\star(\frac{\beta_t \wedge \beta_t^*}{2\pi i}) + \varepsilon_1(t),
\end{align*}
where $\varepsilon_1(t)$ is some error term that vanishes as $t \to 0$, and the uniform bound is used to control the $|\tr\star(\frac{\beta_t \wedge \beta_t^*}{2\pi i})|$, so that the matrix does not deviate from $A'$ too much. Similarly,
\begin{align*}
	\nu(\frac{\star(\Theta_{t}' + s^2 \beta_t\wedge \beta_t^*)}{2\pi i} + \diag_{\rk \E'}(\mu)) &=-\tr (\frac{\star\Theta_{t}'' + s^2 \beta_t\wedge \beta_t^*}{2\pi i} + \diag_{\rk \E'}(\mu)) \\&= \rk \E''(\mu(\E'') - \mu(\E)) - s^2\tr\star(\frac{\beta_t \wedge \beta_t^*}{2\pi i}) + \varepsilon_2(t), 
\end{align*} and hence \[\nu(\frac{\star \Theta_{s,t}}{2\pi i} + \diag_{\rk\E}(\mu)) = J(\nabla_{0,0}) - s^2\tr\star(\frac{\beta_t \wedge \beta_t^*}{\pi i}) + \varepsilon(t).\] Integrating, we find 
\begin{align*}
	J(\nabla_{s,t})^2 &= \int_X \nu^2\left(\frac{\star \Theta}{2\pi i} + \diag(\mu)\right)\vol \\&= \int_X \left(J(\nabla_{0,0}) - s^2\tr\star(\frac{\beta_t \wedge \beta_t^*}{\pi i}) + \varepsilon(t)\right)^2 \vol\\
	&= J(\nabla_{0,0})^2 +  \varepsilon'(s,t) + \int_X  \left(s^4\tr\star(\frac{\beta_t \wedge \beta_t^*}{\pi i})^4 - C_ts^2\tr\star(\frac{\beta_t \wedge \beta_t^*}{\pi i})^2\right) \vol
\end{align*}
where $C_t$ is some term depending on $t$ which is positive and bounded for sufficiently small $t$ and $\varepsilon'$ is some error term depending on $s$ and $t$ which goes to zero. In particular, one can choose an $s,t$ so small that the term in the integral is negative (since $s^4$ is much smaller than $s^2$ for sufficiently small $s$) and $\varepsilon'$ is negligible, whence $J(\nabla_{s,t})< J(\nabla_{0,0})$, as desired.
\end{proof}
\begin{corollary}
	Suppose the Donaldson-Narasimhan-Seshadri theorem has been proven for lower-rank bundles. If $\E$ is stable, we have $\E \cong \F$. In particular, $J(\nabla_\infty) = \inf J(O_{\nabla_0})$.
\end{corollary}
\begin{proof}
	Recalling Proposition \ref{canonical-extension-proposition}, $\varphi$ factors through
	\begin{equation*}
		\begin{tikzcd}
			0 \arrow[r] & \cal{E'}\arrow[r] & \E \arrow[r] \arrow[d, "\varphi"]& \E'' \arrow[d]\arrow[r]& 0\\
			0  & \F''\arrow[l]& \F \arrow[l] & \F'\arrow[l] &\arrow[l] 0.
		\end{tikzcd}
	\end{equation*}
	Now applying the first estimate to the bottom row, we find that \[ J(\nabla_\infty) \geq \rk \F'(\mu(\F') - \mu(\F)) + \rk \F''(\mu(\F) - \mu(\F'')),  \] and similarly, by the second estimate there is some Chern connection $\nabla_{\E}$ on $\E$ such that \[J(\nabla_{\E}) < \rk \E'(\mu(\E) - \mu(\E')) + \rk \E''(\mu(\E'') - \mu(\E)). \] But by assumtion, $J(\nabla_{\infty}) = \inf J(O_{\nabla_{0}})\leq J(\nabla_{\E})$, and so \[\rk \F'(\mu(\F') - \mu(\F)) + \rk \F''(\mu(\F) - \mu(\F'')) <  \rk \E'(\mu(\E) - \mu(\E')) + \rk \E''(\mu(\E'') - \mu(\E)). \] But using the additivity of ranks and degrees, the fact that $\deg \E'' \leq \deg \F'$ and the fact that $\E$ and $\F$ have the same signatures, we deduce
\end{proof}


