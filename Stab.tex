In this chapter, we will put together everything we have learnt so far to construct the moduli space of vector bundles. Let $X$ be a fixed nonsingular curve of genus $g$ over an algebraically closed field $k$ of characteristic zero (that is, an integral scheme of dimension 1 whose local rings at $k$-points are regular, quasi-projective over $k$). 


\section{Divisors and Line Bundles}
In our study of vector bundles, the natural place to start is with line bundles. In this section, we will review the relation between divisors and line bundles, which will ultimately help us define the degree of a vector bundle. Our exposition closely follows the one found in \autocite[II, Section 6]{Hart}
\begin{definition}\index{divisor ! Weil divisor}
	We define the \textit{group of Weil divisors} on $X$, denoted $\Div(X)$ to be the free abelian group on the $k$-points of $X$. The elements of $\Div X$ are known as \textit{Weil divisors}. The \textit{support} of a Weil divisor $D = \sum n_i p_i$, denoted $\Supp D$, is the set $\{p_i\mid n_i \neq 0\}$, and the \textit{degree} of $D$ is defined to be $\deg D:=\sum n_i$. We say $D$ is \textit{effective} if $n_p > 0$ for all $p\in \Supp D$. A \textit{prime divisor} is a Weil divisor of degree 1.
\end{definition}
Since $X$ is nonsingular, for any $k$-point $p$, the local ring $\O_{X, p}$ is regular of dimension 1, and in particular it is a DVR, with fraction field $K(X)$, the function field of $X$. Denote the valuation $v_p$. For any $f\in K(X)^*$, we define the \textit{divisor of $f$}, denoted $\div(f)$ to be \[\div f := \sum_{p\in X(k)} v_p(f) p \]we can show (\autocite[II Lemma 6.1]{Hart}) that all but finitely many of the $v_p(f)$ vanish, hence we get a Weil divisor. A divisor of the form $\div f$ is known as a \textit{principal divisor}; clearly the principal divisors form a subgroup. Two Weil divisors are \textit{linearly equivalent} if their difference is a principal divisor. The group of Weil divisors modulo linear equivalence is the \textit{divisor class group}, denoted $\Cl X$. A \textit{divisor class}, is an element of $\Cl X$. By a \textit{divisor}, we will abuse language and refer to either a Weil divisor or its divisor class; it will either be clear from context or unimportant which is meant. \index{divisor! divisor class}
\begin{example}
	If $X$ is affine, equal to $\Spec A$ where $A$ is necessarily a Dedekind domain (\autocite[I, Proposition 11.5]{Neuk}), then $\Cl X = 0$ if and only if $A$ is a PID. Indeed, if $A$ is a PID, then if $D = \sum n_p p$ is a divisor, then for each $p\in \Supp D$, the maximal ideal $\mf{m}_p$ of $p$ is principal, generated by, say $\varpi_p$, which must be a uniformiser of $\O_{X,p}$. Now let \[f = \prod_{p\in \Supp D} \varpi^{n_p}\in \Frac A = K(X) \] then clearly $D = \div f$. 
	
	Conversely, if $A$ is not a PID, then there is some maximal ideal $\mf{m}_p$ corresponding to some $p\in X(k)$ generated by two elements (\autocite[I, 3, Ex 6.]{Neuk}), say $\mf{m}_p = \langle f, g \rangle$. Now localising at $p$, it is clear that either $f$ or $g$ must be the uniformiser of $\O_{X,p}$; suppose it is $f$ without loss of generality. Now if $p$ was principal, then clearly $p$ must be the divisor of some associate of $f$ in $\O_{X,p}$, say $F$. But $F$ is not prime, and by the unique ideal factorisation property of Dedekind domains (\autocite[I, Theroem 3.3]{Neuk}), we may write \[\langle F \rangle = \mf{m}_p \prod \mf{m}_{p_i}^{n_i}\] where at least one $n_i$ is nonzero, but all but finitely many of them are. But that means \[\div F = p + \sum n_i p_i \] which is a contradiction.
	
	More concretely, if $X = \A^1 = \Spec k[x]$, then any $D = \sum n_i a_i$ can be written $D = \div \prod (x - a_i)$, but if $X = \Spec k[x,y]/ \langle y^2 = x^3-x\rangle$ and $\operatorname{char} k \neq 2$, then we claim $p = (0,0)$ is not principal. Indeed, $\mf{m}_p = \langle x, y \rangle$ and clearly $y^2 = x(x-1)(x+1)$, so $v_p(x) = 2$, so $y$ is a uniformiser of $\O_{X,p}$. But \[\div y = (0,0) + (1,0) + (-1,0) \neq p,\] and similarly for any other uniformiser.
\end{example}
We now state a key result, which will allow us to define the degree of a line bundle:
\begin{proposition}\label{degree-independent-divisor-representative}
	The degree map $\deg: \Div X \rightarrow \Z$ descends to a map $\Cl \rightarrow \Z$.
\end{proposition}
\begin{proof}
	\autocite[p. 138]{Hart}.
\end{proof}
\begin{example}
	Let $X = \P^1 = \Proj k[x_0,x_1]$. We claim that the degree map is an isomorphism; in particular its kernel is trivial. So suppose $D = \sum n_p p$ is a degree zero divisor. Now each $p$ can be written $[p_0: p_1]$, corresponding to $\langle x_0p_1 - x_1p_0 \rangle$. Write \[f = \prod_{p\in \Supp D} (x_0p_1-x_1p_0)^{n_p}\in k(x_0, x_1) \] Now observe that \[K(X) = k(\frac{x_0}{x_1}) \] in particular, $K(X)$ is the subfield of $k(x_0, x_1)$ consisting of degree 0 elements. Since $\sum n_p = 0$, it follows $\deg f = 0$ too, hence $D = \div f$ as desired.
\end{example}

Next we describe the relation between divisors and line bundles: Let $[D]$ be a divisor class. We define the \textit{line bundle associated to} $[D]$, denoted $\L(D)$ as follows: let $D = \sum n_p p$ be a divisor in the class of $[D]$. For any open set $U$, we define \[\L(D)(U):= \{f\in K(X) \mid v_p(f) + n_p \geq 0 \text{ for all }p\in U(k)\}\] Firstly, we need to check that this is indeed a line bundle. To see this, observe the following: if $U\cap \Supp D = \emptyset$ then $\L(D)(U) = \Ox(U)$. Now for any $p\in \Supp D$, we choose some uniformiser $\varpi_p$ of $\O_{X, p}$, and we may assume $\varpi_p\in \Ox(U_p)$ for some open set $U_p$ containing $p$. We may pick $U_p$ sufficiently small that $(\div \varpi_p^{n_p})|_{U} = n_p p$ (in other words, $U_p \cap \Supp (\div \varpi_p) = \{p\}$). Now observe $\L(D)(U_p) = \varpi_p^{-1} \Ox(U)$, and in particular they are isomorphic. Now covering $X$ with the open sets $U_p$ and open sets which do not intersect $\Supp D$, we deduce that $\L(D)$ is a line bundle. 

Finally, we need to check that $\L(D)$ does not depend on our choice of $D$. So let $D' = \sum n_p' p$ be a divisor linearly equivalent to $D$, so that $D - D' = \div F$. Observe that then $v_p(F) = n_p - n_p'$. Now we define the isomorphism $\varphi: \L(D)\rightarrow \L(D')$ to be $f\mapsto  Ff$. To see that this is an isomorphism, observe that given an open set $U$, we have \[ \L(D)(U)= \{f\in K(X) \mid v_p(f) + n_p \geq 0 \text{ for all }p\in U(k)\} \] and \[\L(D')(U):= \{f\in  K(X) \mid v_p(f) + n_p' \geq 0 \text{ for all }p\in  U(k)\} \] now if $f\in \L(D)(U)$, then \[v_p(fF) + n_p' = v_p(f) + v_p(F) + n_p' \geq -n_p + v_p(F) + n_p' = 0 \] hence we have a well-defined morphism of sheaves, and moreover, clearly division by $F$ is its inverse. 
\par Before we state our next result, we recall that the \textit{Picard group}, denoted $\Pic X$ is the group of line bundles on $X$, with the group operation given by tensor product, and inversion given by dualising. \index{Picard group}%Using \v{C}ech cohomology, we can show $\Pic X \cong H^1(\Ox^*)$, where $\Ox^*$ is the sheaf of nowhere-vanishing functions on $X$. Indeed, the transition functions of a line bundle $\L$ on a sufficiently fine cover define a \v{C}ech 1-cocycle of $\Ox^*$, and is not hard to show two line bundles are isomorphic if and only if their transition functions are cohomologous, and that the $\v{C}ech$ cohomology group remains constant under refeinement. 
\begin{theorem}
	The map $D \mapsto \L(D)$ is an isomorphism $\Cl X \rightarrow \Pic X$.
\end{theorem}
\begin{proof}
	This follows the proof given in \autocite[pp. 144, 145]{Hart}. First of all, let $\K$ denote the constant sheaf $K(X)$, which is an $\Ox$-module. We claim $\L \otimes \K\cong \K$. To see this, we consider the base of the topology $\{U_\alpha = \Spec A_\alpha\}$ on $X$ consisting of the affine open sets where $\L$ is trivial. Then $K(X) = \Frac A_\alpha$ for each $\alpha$, so locally we have the natural isomorphism $A_\alpha \otimes K(X) \cong K(X)$, given by $a\otimes b \mapsto b$, and it is very easy to see that these agree on overlaps, and since the $U_\alpha$ form a base, by \autocite[Proposition I-12]{EisHar}, this extends to an isomorphisms of sheaves. It thus follows that $\L$ can be embedded inside $\K$. 
	
	First we prove surjectivity. By our discussion above, we may consider $\L$ as a subsheaf of $\K$; in particular, every local section is an element of $K(X)$. Since $\L$ is invertible, it follows that $\L(U)$ is a free $\Ox(U)$-submodule of $K(X) = \Frac \Ox(U)$, for sufficiently small $U$, and in particular it is generated by some $\varpi_U^{-1}\in K(X)$. Now we define the divisor $D$ locally as $\div \varpi_U$, and let $U$ vary across a cover. To see this is well-defined, note that if $\varpi_U^{-1}$ generates $\L(U)$ and $\varpi_V^{-1}$ generates $\L(V)$, then both $\varpi_U^{-1}$ and $\varpi_V^{-1}$ generate $\L(U \cap V)$, and in particular they are associates in $\Ox(U\cap V)$, hence they generate the same Weil divisor on $U \cap V$. It is then clear that $\L(D) = \L$.
	
	Next we prove the homomorphism property. But this is obvious since if $\L(D_1)$ and $\L(D_2)$ are generated locally by $\{\varpi_1^{-1}\}$ and $\{\varpi_2^{-1}\}$ respectively, then $\L(D_1 + D_2)$ are generated locally by $\{\varpi_1^{-1}\varpi_2^{-1}\}$. But $\L(D_1)\otimes \L(D_2)$ is also generated locally by $\{\varpi_1^{-1}\varpi_2^{-1}\}$, and hence they are isomorphic.
	
	Finally, we prove injectivity. Since we have shown this is a group homomorphism, it suffices to show that the kernel is trivial. So suppose $\L(D) \cong \Ox$, and fix an isomorphism $\varphi: \Ox \rightarrow \L(D)$. We claim $D = \div \varphi(1)^{-1}$. Indeed, choosing a sufficiently fine open affine cover $\{U_\alpha\}$ such that $\#(U_\alpha \cap \Supp D) \leq 1$, we suppose $D|_{U_\alpha}$ is the divisor of $\varpi_\alpha$. Then $\varpi_\alpha^{-1}$ generates $\L(D)(U_\alpha)$. But $\varphi(1)$ also generates $\L(D)(U_\alpha)$, hence $\varphi(1)$ and $\varpi_\alpha^{-1}$, and hence both generate the same Weil divisor. In particular, $\varphi(1)^{-1}$ generates $D$, as desired.
\end{proof}

Now let $\L$ be a line bundle. We will take a look at $H^0(X, \L)$, the space of global sections. For each nonzero $s\in H^0(X, \L)$, we define the \textit{divisor of zeroes} of $s$, denoted $\div s$ as follows: on any open subset $U$ on which $\L$ is trivial, we let $\Phi_U: \L|_U \rightarrow \O_U$ be an isomorphism, and define \[\div s|_U:= \div \Phi_U(s). \] Of course, this is well-defined: let $p\in U(k)$, and let $\Phi_V: \L|_V \rightarrow \O_V$ be another trivialisation, with $p\in V(k)$. Then at $p$, $\Phi_V(s)$ and $\Phi_U(s)$ differ by an invertible element of $\O_{X, p}$, and thus they have the same valuation, and since $p$ is arbitrary, this means we get a well-defined Weil divisor. Furthermore, observe that $\div s$ is effective, since it is locally the divisor of some section of $\Ox$, which must have nonnegative valuation at any $k$-point. We are now in a position to state our result:
\begin{proposition}\label{divisor-of-zeroes-effective}
	Let $\L = \L(D)$ be the line bundle associated to a divisor $D$. Then:
	\begin{enumerate}
		\item The divisor of zeroes of any nonzero $s\in H^0(X, \L)$ is linearly equivalent to $D$.
		\item Any effective divisor $D_0$ linearly equivalent to $D$ is the divisor of zeroes of some nonzero $s\in H^0(X, \L)$.
	\end{enumerate}
\end{proposition}
\begin{proof}
	\autocite[p. 157]{Hart}.
\end{proof}
We now come to the definition of the degree:
\begin{definition}\index{vector bundle ! degree}
	Let $\E$ be a line bundle. We define the \textit{degree} of $\E$, denoted $\deg \E$ as follows: if $\E$ is a line bundle, we define $\deg \E$ to be the degree of the divisor corresponding to $\E$ via the isomorphism $\Pic X \cong \Cl X$. This is well-defined, by Proposition \ref{degree-independent-divisor-representative}. In general, we define the \textit{determinant line bundle} of a rank $n$ vector bundle $\E$ to be the line bundle \[\det \E := \wedge ^n \E\] and we define $\deg \E := \deg(\det \E)$. Note that $\det \L = \L$ if $\L$ is a line bundle, hence our definition is consistent. We define the \textit{signature} of $\E$ to be the pair $(\rk \E, \deg \E)$.
\end{definition}
\begin{example}
	Let $X = \P^1 = \Proj k[x_0, x_1]$. We will show $\deg \Ox(n) = n$. Let $D =n [0:1]$. We compute $\L(D)$ as follows: on $U_1 = \{x_1\neq 0\} = \Spec k[x_0/x_1]$, we have \[D|_{U_1} = \div (\frac{x_0}{x_1})^n \] and on $U_0 = \{x_0 \neq 0\} = \Spec k[x_1/x_0]$, we have $D|_{U_0} = 0$, hence \[\L(D)(U_0) = \Ox(U_0) = k[x_1/x_0] \] and \[\L(D)(U_1) = (\frac{x_1}{x_0})^nk[x_0/x_1] \] Meanwhile, \[\Ox(n)(U_0) = x^n_0 k[x_1/x_0]\] and \[\Ox(n)(U_1) = x_1^n k[x_0/x_1] \] hence we define the isomorphism $\L(D) \rightarrow \Ox(n)$ by $f\mapsto x_0^n f$. It is easy to check that this is well-defined and agrees on overlaps (which is just localising), hence we get an isomorphism of line bundles.
	
	We also observe that $D$ is the divisor of zeroes of $x_0^n\in H^0(X, \Ox(n))$, as expected.
\end{example} 
\begin{example}
	Let $X$ be the elliptic curve $X=\Proj k[x,y,z]/ \langle y^2z = x^3 - xz^2 \rangle$, and let $\Ox(1)$ be the pullback of $\O_{\P^2}(1)$ induced by the embedding (or equivalently the twisting sheaf of Serre). We will show $\deg \Ox(1) = 3$ and is isomorphic to $\L = \L(3[0:1:0])$. Note that $x\neq 0$ implies $z\neq 0$, hence we can cover $X$ by two open sets $U_z = \{z\neq 0\}$ and $U_y = \{y \neq 0\}$. Thus we have \[\Ox(1)(U_z) = zk[y/x, z/x]/ (y^2z/x^3 = 1 - z^2/x^2) = z\Ox(U_z) \] and \[\Ox(1)(U_y) = yk[x/y, z/y]/ (z/y = (x/y)^3 - xz^2/y^3) = y\Ox(U_y).\] Now observe that \[\L(U_z) = \Ox(U_z)\] but \[\L(U_y) = (y/z)\Ox(U_y) \] since $D = 3[0:1:0]$ is the divisor of $z/y$ on $U_y$. It then follows we have a global isomorphism $\L \rightarrow \Ox(1)$ defined by $f \mapsto zf$.
\end{example}

We conclude this section with some technical results about the degree:


\begin{proposition}
	Let \[0 \rightarrow \E \rightarrow \F \rightarrow \G \rightarrow 0 \] be a short exact sequence of vector bundles. Then \[\deg \F = \deg \E + \deg \G\]
\end{proposition}
\begin{proof}
	If $\{g_{\alpha\beta}\in \GL_n(\Ox(U_\alpha\cap U_\beta)) \}$ is the set of transition morphisms on a sufficiently fine cover representing $\F$, then it is not difficult to check $\{\det g_{\alpha\beta}: U_\alpha\cap U_\beta \rightarrow \bb{G}_m \}$ is the set of transition functions of $\det \F$. Now it can be shown (\autocite[p. 68]{GriHa}) that each $g_{\alpha\beta}$ has the shape \[\begin{pmatrix*}
		h_{\alpha\beta} & k_{\alpha\beta} \\
		0 & j_{\alpha\beta}
	\end{pmatrix*} \] where $\{h_{\alpha\beta} \}$ and $\{j_{\alpha\beta}\}$ are the transition morphisms representing $\E$ and $\F$ respectively. Hence\[ \det g_{\alpha\beta} = \det h_{\alpha\beta} \det j_{\alpha\beta} \] But $\{\det h_{\alpha\beta} \det j_{\alpha\beta}\}$ is the set of transition functions for $\det \E \otimes \det \G$, which means $\det \F = \det \E \otimes \det \G$, and thus the result follows.
\end{proof}
\begin{corollary}
	Let $\E$ and $\F$ be vector bundles. then $\deg (\E\oplus \F) = \deg \E + \deg \F$.
\end{corollary}
\begin{proof}
	Apply the above proposition to \[0 \rightarrow \E \rightarrow \E \oplus \F \rightarrow \F \rightarrow 0 \] and the result follows immediately.
\end{proof}
\begin{corollary}\label{twisting-by-line-bundle-degree}
	Let $\E$ be a vector bundle and $\L$ a line bundle. Then \[\deg(\E \otimes \L) = \deg \E + \deg \L \rk \E \] 
\end{corollary}
\begin{proof}
	As in the proof of the proposition, we look at the transition functions, and the result immediately follows.
\end{proof}


\begin{lemma}
	Let $\E$ and $\F$ be vector bundles of rank $n$, and suppose there is a homomorphism $ \E \rightarrow \F$ with nonzero determinant (i.e. the determinant is not identically zero). Then $\deg \E \leq \deg \F$.
\end{lemma}
\begin{proof}
	Taking the determinant, we get a nonzero homomorphism $\det \E \rightarrow \det \F$, hence we may assume without loss of generality $n = 1$. Now a nonzero homomorphism $\E \rightarrow \F$ is nothing more than a global section of $\fancyHom(\E, \F) = \E ^\vee \otimes \F$, and since we are assuming $\E$ and $\F$ are line bundles, it follows $\deg(\E ^\vee \otimes \F) = \deg \F - \deg \E$; and thus we reduce to proving that if $\L$ is a line bundle with a nonzero global section, then $\deg \L \geq 0$. 
	
	So let $\L$ be a line bundle, associated to a divisor $D$ and suppose $s\in H^0(X, \L)$ is a nonzero global section. Then by Proposition \ref{divisor-of-zeroes-effective}, the divisor of zeroes of $s$, say $D_0$, is effective, and linearly equivalent to $D$. But since $D_0$ is effective, it has nonnegative degree, and thus $\L = \L(D) = \L(D_0)$ has nonnegative degree, as desired.
\end{proof}
\begin{proposition}\label{canonical-extension-proposition}
	Let $\E \rightarrow \F$ be a nonzero homomorphism of vector bundles. Then there is a factorisation:
	\begin{equation}
		\begin{tikzcd}
			0 \arrow[r] & \cal{E'}\arrow[r] & \E \arrow[r] \arrow[d, "\varphi"]& \E'' \arrow[d]\arrow[r]& 0\\
			0  & \F''\arrow[l]& \F \arrow[l] & \F'\arrow[l] &\arrow[l] 0
		\end{tikzcd}
	\end{equation}
	where each sheaf above is locally free, the rows are exact, and $\E' \cong \ker \varphi$, $\E''\cong \im \varphi$ and $\rk \E'' = \rk \F'$, $\deg \E'' \leq \deg \F'$.
\end{proposition}
\begin{proof}
	Define $\E'$ and $\E''$ as in the theorem statement, so that the top row is exact. Since $\F$ is locally free and since $X$ is covered by the spectra of Dedekind domains (which are hereditary), it follows that $\E'' = \im \varphi\subseteq \F$ is locally free. Now the sheaf $\coker \varphi$ is coherent, and hence if $U = \Spec A$ is an affine open subset of $X$, where $A$ is a Dedekind domain, then $\coker \varphi$ is isomorphic to $\widetilde{M}$ for some finitely-generated $A$-module $M$. Let $M'$ be the torsion submodule of $M$ and we define $\F''$ to locally be the sheaf $(M/M')^\sim$. It is not hard to see that this is well-defined. Observe that since $A$ is a Dedekind domain and $M/M'$ is a torsion-free module, it is also a projective module, and hence $\F''$ is locally free, by the Serre-Swan Theorem. We then define $\F'$ to be the kernel of $\F \rightarrow \F''$. Now since the map $\E'' \rightarrow \F''$ defined by composing the obvious maps is zero, by the universal property of kernels there is unique homomorphism $\E'' \rightarrow \F'$ making everything commute. This unique homomorphism has nonzero determinant, because $\varphi$ is nonzero, and $\E'' \rightarrow \F$ is the inclusion of the image. Finally, in light of the above lemma, it suffices to show that $\rk \E'' = \rk \F'$. But this follows directly from the observation that the local ring at any $k$-point is a DVR, and thus a PID, and so any finitely generated module splits into its torsion and free parts.
\end{proof}


\section{Stable and Semistable Bundles}
We now have the tools we need to define stability. To see why we require this definition in the first place though, we consider the moduli problem of vector bundles over $X$. To formalise:
\begin{definition}
	The \textit{moduli problem of vector bundles of signature $(n,d)$ on $X$} is the functor $\mathcal{V}_{n,d}: \Sch/k \rightarrow \Sets$ defined as follows: for any scheme $S/k$, we define a family of vector bundles over $S$ to be a coherent sheaf $\E$ on $X\times S$, flat over $S$ such that for any $s\in S(k)$, the fibre $\E_s $ defined to be the pullback of $\E$ along the map $s\times \id: \Spec k \times X \rightarrow S \times X$ is a locally free sheaf of signature $(n,d)$ on $X$. Two families $\E, \F$ are \textit{equivalent}, written $\E \sim \F$, if there is a line bundle $\L$ on $S$ such that \[\E \cong \F \otimes \pi_S^*\L. \] Note that dualising commutes with pullbacks, so this is an equivalence relation. We may also pull back families in the obvious way, and clearly equivalent families are pulled back to equivalent families. We then define $\mathcal{V}_{n,d}$ to be \[\mathcal{V}_{n,d}(S):= \{\text{families over }S\}/ \sim. \]
\end{definition}
However, even in the simple case of signature (2,0)-bundles over $\P^1 = \Proj k[x,y]$, there is a jump phenomenon:
\begin{example}
	We consider the following family $\E_t$ over the affine line $\A^1 = \Spec k[t]$: let $U_0$ (resp. $U_1$) be the open subset where $x$ (resp. $y$) does not vanish. Then we glue $\O_{U_0\times \A^1}^2$ and $\O_{U_1\times \A^1}^2$ on the overlap via the transition function \[g_{0,1}:= \begin{pmatrix*}
		\frac{x}{y} & t \\
		0 & \frac{y}{x}
	\end{pmatrix*} \in \GL_2(k[t, \frac{y}{x}, \frac{x}{y}])= \GL_2(\Gamma(U_0 \cap U_1, \O_{\A^1\times \P^1})). \] More precisely, on $U_0$ we have the standard basis $r_1, r_2\in \Gamma(U_0, \O_{U_0\times \A^1}^2)$, and on $U_1$ the standard basis $s_1, s_2\in \Gamma(U_0, \O_{U_1\times \A^1}^2)$. We then define isomorphism given by $s_1 \mapsto ( x/y)r_1$ and $s_2 \mapsto rs_1 + (y/x) s_2$, and we define $\E_t$ to be the gluing of the two sheaves on the overlap. Since $\E_t$ is locally free, it is flat over $\A^1 \times \P^1$ and by the transitivity of flatness is also flat over $\A^1$. 
	
	Now we claim that the fibre $\E_0$ is isomorphic to $\O_{\P^1}(-1) \oplus \O_{\P^1}(1)$, but every other fibre is isomorphic to $\O_{\P^1} \oplus \O_{\P^1}$ (and of course, these are not isomorphic, the latter has a nowhere-vanishing global section, but the former does not). To see this, first observe that clearly when $t = 0$, the transition map is just $\diag(y/x, x/y)$, which is the transition map of $\O_{\P^1}(-1) \oplus \O_{\P^1}(1)$. However, for a nonzero $\lambda$, we will define an isomorphism $\varphi: \E_\lambda \rightarrow \O_{\P^1} \oplus \O_{\P^1}$ as follows: let $e_1, e_2$ be the standard basis of $\O_{\P^1}\oplus \O_{\P^1}$. We define $\varphi$ to be the map 
	
	\begin{align*}
		r_1 &\mapsto \frac{y}{x}e_1 - e_2, \\ r_2 &\mapsto \lambda e_1
	\end{align*} 
	on $U_0$ and
	\begin{align*}
		s_1& \mapsto e_1 - \frac{x}{y}e_2,\\   s_2 &\mapsto \lambda e_2
	\end{align*}
	on $U_1$. To see that these glue, observe that $r_1 = (y/x)s_1$ and $r_2 = \lambda s_1 + (x/y)s_2$, and so on the overlap we have \[\varphi(\frac{y}{x}s_1) = \frac{y}{x}(e_1- \frac{x}{y}e_2) =\frac{y}{x}e_1 - e_2  \] \[\varphi(\lambda s_1 + \frac{x}{y} s_2) = \lambda e_1 - \frac{\lambda x}{y} e_2 + \frac{x}{y} \lambda e_2 = \lambda e_1 \] as desired. Finally, observe that $\varphi$ maps local free generators to local free generators, and thus this is an isomorphism.
\end{example}
\begin{remark}
	Of course, the fibres of the family $\E_t$ given above are canonically identified with the elements of the group \[\Ext^1(\O_{\P^1}(1), \O_{\P^1}(-1)) \cong \Ext^1(\O_{\P^1},\O_{\P^1}(-2) ) \cong H^1(\P^1, \O_{\P^1}(-2)) \cong k.\] {\color{red} We may generalise this idea as follows}: let $\E, \F$ be two vector bundles over $X$, so that $\Ext^1(\F, \E) \cong H^1(X, \fancyHom(\F, \E))$. Let $\{U_\alpha\}$ be a sufficiently fine open cover of $X$, let $e_{\alpha\beta}$ (resp. $f_{\alpha\beta}$) be a set of transition functions representing $\E$ (resp. $\F$), and let $\{g_{\alpha\beta}\}$ be a \v{C}ech 1-cocycle representing some element of $H^1(X, \fancyHom(\G, \F))$. 
\end{remark}
\begin{definition}\index{vector bundle ! slope}
	Let $\E$ be a vector bundle. The \textit{slope} of $\E$, denoted $\mu(\E)$, is defined to be \[\mu(\E):= \frac{\deg \E}{\rk \E} \]
\end{definition}
\begin{lemma}\label{subbundle-quotient-slope}
	Let \[0 \rightarrow \E \rightarrow \F \rightarrow \G \rightarrow 0 \] be a short exact sequence of vector bundles. Then $\mu(\E) \leq \mu(\F)$ if and only if $\mu(\G) \geq \mu(\F)$, with equality holding in one if and only if in the other.
\end{lemma}
\begin{proof}
	Trivial.
\end{proof}
And finally, the definition:
\begin{definition}\index{vector bundle ! stability}
	Let $\E$ be a vector bundle. Then $\E$ is \textit{stable} (resp. \textit{semistable}) if for every proper subbundle $\F$, we have 
	\begin{align*}
		\mu(\F) &\,< \mu(\E)\\
		&(\leq)
	\end{align*}
	$\E$ is \textit{polystable} if it is a direct sum of stable bundles of the same slope.
\end{definition}
We observe a few things. Firstly, note that this definition resembles, in some way, the Hilbert Mumford criterion; and indeed {\color{red} we will show later that this is not a coincidence}. Next observe that we can alternatively define stability in terms of quotient bundles: by Lemma \ref{subbundle-quotient-slope}, $\E$ is (semi) stable if and only if for every quotient bundle $\G$ we have:
\begin{align*}
	\mu(\E) &\,< \mu(\G)\\
	&(\leq)
\end{align*}
\par Let us mention some more basic properties of stability:
\begin{lemma}\label{stability-properties}
	Let $\E$ be a vector bundle. 
	\begin{enumerate}
		\item $\E$ is (semi) stable if and only if $\E \otimes \L$ is (semi) stable for every line bundle $\L$.
		\item If $\rk \E = 1$ (i.e. $\E$ is a line bundle) then $\E$ is always stable.
		\item If $\E'$ is another vector bundle then $\E \oplus \E'$ is not stable. It is semistable only if $\mu(\E) = \mu(\E')$ and both are semistable. 
		\item If $\varphi: \E\rightarrow \F$ is a nonzero morphism of vector bundles, and both bundles are semistable, then $\mu(\E)\leq \mu(\F)$.
		\item If $\E$ is stable, then it is simple (that is, $\End(\E) = k$).
	\end{enumerate}
\end{lemma}
\begin{proof}
	Note that every subbundle of $\E\otimes \L$ has the form $\F \otimes \L$ (indeed, if $\G$ is any subbundle of $\E \otimes \L$ we take $\F = \G \otimes \L^\vee$). Now observe \[\mu(\F \otimes \L) = \frac{\deg \F + \rk \F \deg \L}{\rk \F} = \mu(\F) + \deg \L\] and similarly for $\mu(\E \otimes \L)$ and hence \[\mu(\E \otimes \L) - \mu(\F \otimes \L) = \mu(\E) - \mu(\F) \] which proves (i). (ii) is trivial. To prove (iii), suppose without loss of generality $\mu(\E) \leq \mu(\E')$. Now \[\mu(\E \oplus \E') = \frac{\deg(\E) + \deg(\E')}{\rk(\E) + \rk(\E')} \leq \mu(\E') \] hence $\E \oplus \E'$ is not stable. If it is semistable, then equality must hold above, and clearly both must both be semistable, which proves (iii).
	
	To prove (iv), observe that by Proposition \ref{canonical-extension-proposition}, the map $\varphi$ factors as follows:
	\begin{equation}\label{proper-factorisation-equation}
		\begin{tikzcd}
			0 \arrow[r] & \cal{E'}\arrow[r] & \E \arrow[r] \arrow[d, "\varphi"]& \E'' \arrow[d]\arrow[r]& 0\\
			0  & \F''\arrow[l]& \F \arrow[l] & \F'\arrow[l] &\arrow[l] 0
		\end{tikzcd}
	\end{equation}
	where $\E' \cong \ker \varphi$, $\E''\cong \im \varphi$ and $\rk \E'' = \rk \F'$, $\deg \E'' \leq \deg \F'$. Since $\E$ and $\F$ are both stable, it follows 
	\begin{equation}\label{chain-of-slop-inequalities-equation}
		\mu(\E) \leq \mu(\E'') \leq \mu(\F') \leq \mu(\F)
	\end{equation}
	as desired. 
	
	Finally, to prove (v), suppose $\varphi: \E \rightarrow \E$ is an endomorphism. If $\varphi$ is zero, we are done. If not, we must have equality in (\ref{chain-of-slop-inequalities-equation}), and by the stability of $\E$ we have $\E = \E'' = \im \varphi$, so in particular $\varphi$ is an isomorphism. In particular, this implies that any nonzero endomorphism of $\E$ is an isomorphism. Now fix some $p\in X(k)$; we have an induced map of fibres $\E_p/\mf{m}_p\E_p\rightarrow \E_p/\mf{m}_p\E_p$, and since $k$ is algebraically closed, this map has an eigenvector, say $\lambda$. But $\varphi - \lambda: \E \rightarrow \E$ is no longer an isomorphism, hence must be zero, and thus $\varphi = \lambda$ as desired. 
\end{proof}

\section{The Construction}
Our first job is to construct a bounded family of semistable bundles; that is, a family that parameterises every semistable bundle of given signature, so that we may take a GIT quotient of the parameter space. To this end, we will first require the Riemann-Roch theorem:
\begin{theorem}[Riemann-Roch for Vector Bundles]\index{Riemann-Roch Theorem}
	Let $\F$ be a vector bundle on $X$ of signature $(n,d)$. Then \[\chi(\F) := h^0(\F) - h^1(\F) = d + n(1-g). \]
\end{theorem}
\begin{proof}
	We induct on $n$. For $n=1$, the result is classical, and is proven in, for example, pages 295-296 of \autocite{Hart}. Now supposing true up to some $n-1$, suppose $\F$ has rank $n$. Let $\L$ be a line subbundle of maximal degree. Then we claim $\F/\L$ must be locally free. Indeed, we may apply Proposition \ref{canonical-extension-proposition} taking $\varphi$ in the proposition statement to be the inclusion $\L\subseteq \F$, so that there is a nonzero map $\L \rightarrow \F'$ (where $\F'$ is as in the proposition statement). But both these bundles are line bundles, and hence are stable, and so $\deg \L \leq \deg \F'$, and since $\L$ is of maximal degree, equality must hold, and hence $\F/\L = \F''$ is locally free as claimed. Now applying the inductive hypothesis we have \[\chi(\F) = \chi(\L) + \chi(\F/\L) = \deg \L + 1 - g + (d - \deg \L) +(n-1)(1-g) = d + n(1-g) \] as desired.
\end{proof}
\begin{corollary}[Classical Riemann-Roch Theorem]
	For any line bundle $\L$ of degree $d$, we have \[h^0(\L) + h^0(\L^\vee\otimes \omega_X) = d + 1-g, \]  where $\omega_X$ is the canonical bundle (which is just the cotangent bundle in this case). In particular, the degree of the canonical bundle is $2g-2$.
\end{corollary}
\begin{proof}
	Combine the above theorem and the Serre duality theorem (\autocite[III, Corollary 7.7]{Hart}). 
\end{proof}
The importance of the Riemann-Roch theorem is that it allows us to relate the quantities we are interested in. More precisely, by the Serre vanishing theorem, for any coherent sheaf $\F$, and any ample line bundle $\L$, we have $H^i(X, \F\otimes \L^{\otimes m}) = 0$ for any sufficiently large $m$ and any $ i > 0$. Moreover, by the definition of ampleness, we know that $\F\otimes \L^{\otimes m}$ is generated by global sections, for any sufficiently large $m$. In particular, if $\F$ is locally free then the Riemann-Roch theorem tells us exactly how many global sections generate $\F\otimes \L^{\otimes m}$. The main issue is that the \enquote{sufficiently large} criterion depends on $\F$ and $\L$. This is remedied by the following result:
\begin{lemma}\label{bounding-family}
	Let $\E$ be a semistable vector bundle of signature $(n,d)$.
	\begin{enumerate}
		\item If $d >n(2g-2)$ then $H^1(X, \E) = 0$.
		\item If $d > n(2g-1)$ then $\E$ is generated by global sections.
	\end{enumerate} 
\end{lemma}
\begin{proof}
	This follows the proof given in \autocite[p. 68]{ModuliNotes}. Suppose for contradiction $H^1(X, \E)\neq 0$. By the Serre duality theorem, we have \[H^1(X, \E)\cong H^0(X, \E^\vee \otimes \omega_X) = \Hom(\E, \omega_X),\] where $\omega_X$ is the canonical bundle (which is just the cotangent bundle here). This means that there is a nonzero homomorphism $\varphi: \E \rightarrow \omega_X$. Now by Lemma \ref{stability-properties} (iv), we have \[2g-2 = \frac{n(2g-2)}{n} <\frac{d}{n} = \mu(\E) \leq \mu(\omega_X) = 2g-2 \] which is a contradiction. This proves (i). 
	
	To prove (ii), we note that since $k$ is algebraically closed, we need only check the stalk is generated by global sections at $k$-points. So let $p$ be a $k$-point, with local ring $\O_{X,p}$ and maximal ideal $\mf{m}_p$. The composition $\Ox(U) \rightarrow \O_{X,p} \rightarrow \O_{X,p}/\mf{m}_p$ induces the following short exact sequence of sheaves:
	\[ 0 \rightarrow \mathcal{I}_p \rightarrow \Ox \rightarrow k_p \rightarrow 0, \] where $k_p$ is the skyscraper sheaf $k$ sitting over $p$, and $\mathcal{I}_p$ is the kernel. Since $\E$ is locally free, tensoring is exact, and moreover since it is of rank $n$, and $k_p$ is a skyscraper sheaf, it follows $k_p\otimes \E\cong k_p \otimes \Ox^n \cong k_p^n$. Thus tensoring the above with $\E$ we have 
	\begin{equation}\label{skyscraper-ses}
		0 \rightarrow \mathcal{I}_p\otimes \E \rightarrow \E \rightarrow k_p^n \rightarrow 0.
	\end{equation}  
	Now we claim that $\mathcal{I}_p\cong \L(-p)$. To see this, let $U = \Spec A$ be an open affine subset. If $p\notin U$, then $\mathcal{I}_p|_U = \Ox|_U$. Otherwise, $p$ is cut out by some $f\in A$, and hence $\mathcal{I}_p|_U = f\Ox|_U$. But this is exactly the definition of $\L(-p)$, as claimed. Now observe that $\mathcal{I}_p\otimes \E = \E \otimes \L(-p)$ is semistable and has degree $n(2g-2)$ and thus by part (i) we have $H^1(X, \E\otimes \L(-p)) = 0$. Now taking cohomology of (\ref{skyscraper-ses}), it follows we have a surjection \[H^0(X, \E ) \rightarrow H^0(X, k_p^n) = k^n. \] Finally, we apply Nakayama's lemma (\autocite[Corollary 4.8]{EisAlg}) on the local ring $\O_{X,p}$ to deduce that the map $H^0(X, \E)\rightarrow \E_p$ is surjective too.
\end{proof}
The consequence of the above lemma combined with the Riemann-Roch theorem is that every semistable vector bundle of signature $(n,d)$ for $d$ sufficiently large is a quotient of $\Ox^{d+n(1-g)}$. Now it just so happens that there is a scheme that parameterises all quotients of a given coherent sheaf, with certain constraints. To define it however, we require the following result:
\begin{proposition}
	Let $Y$ be a projective variety, let $\F$ be a coherent sheaf on $Y$ and let $\O(1)$ be a very ample line bundle. For any $m\in \Z$, write $\F(m):=\F\otimes \O(m):= \F \otimes \O(1)^{\otimes m}$. Then the map \[P:m\mapsto \chi(\F(m)) \] coincides with a polynomial with rational coefficients.
\end{proposition}
\begin{proof}
	Embed $Y$ into $\P^n = \Proj k[x_0,...,x_n]$ via $\O(1)$. We induct on $r=\dim \Supp \F$. If $r = 0$, then $\F$ is supported on a discrete subset, and hence tensoring with $\O(1)$ does nothing, and so $P$ is constant, equal to $\chi(\F)$. Now supposing true for some $r > 0$, we suppose $\dim \Supp \F = r+1$, and write $M$ for the graded module $M:= \bigoplus_{m\in \Z} H^0(X, \F(m))$, so that $\widetilde{M} \cong \F$ (\autocite[II. Proposition 5.15]{Hart}). We note that the map $M(-1)\rightarrow M$ given by multiplication by $x_i$ induces the following short exact sequence of sheaves \[ 0 \rightarrow \E \rightarrow \F(-1)\rightarrow \F \rightarrow \G \rightarrow 0 \] where $\E$ and $\G$ are the kernel and cokernel respectively. Now if $\dim \Supp \E = r+1$, then {\color{red} TO FINISH }.
\end{proof}
\begin{definition}\index{Hilbert polynomial}
	The polynomial $P$ above is known as the \textit{Hilbert polynomial} of $\F$ with respect to $\O(1)$. 
\end{definition}
\begin{example}
	Take $Y = \P^r$, take $\F = \Oy$ and take $\O(1)$ to be the usual twisting sheaf of Serre. Observe that if $m \geq 0$ then $h^0(\O(m)) = \binom{m+r}{r},$ and {\color{red}$h^i(\O(m)) = 0$} for all $i > 0$, and hence\[ P = \binom{z+r}{r}:= \frac{1}{r!}\prod_{i = 1}^{r} (z+i)\in \Q[z].\]
\end{example}
\begin{example}
	Let $\Ox(1)$ be a very ample line bundle on $X$ of degree $m$, and let $\E$ be a of signature $(n,d)$. By the Riemann-Roch theorem we know \[\chi(\E(m)) = d+nm + n(1-g), \] and so $\E$ has Hilbert polynomial $P = d + nz + n(1-g)\in \Q[z]$ with respect to $\Ox(1)$. Conversely, if $\F$ is a vector bundle with Hilbert polynomial $P = d + nz + n(1-g)\in \Q[z]$, then $\chi(\F) = d+n(1-g)$ and $\chi(\F(1)) = d+nm+n(1-g)$, and so we know $\rk(\F) = n$ by Corollary \ref{twisting-by-line-bundle-degree} and $\deg(\F) = d$ by Riemann-Roch. In particular, the data of the Hilbert polynomial on $\F$ (with respect to $\Ox(1)$) is equivalent to the data of the signature of $\F$.
\end{example}
The last example illustrates that the degree of $\E$ is encoded by the Hilbert polynomial. We now have what we need to define a bounded family. First, we define the following moduli problem:
\begin{definition}
	Let $\F$ be a coherent sheaf on a projective scheme $Y$ with an embedding $Y\subseteq \P^n$, and let $P\in \Q[z]$ be a numerical polynomial. For any scheme $S$ of finite type over $k$, a \textit{family of quotients of $\F$ with Hilbert polynomial $P$ parameterised by $S$} is a coherent sheaf $\G$ on $S \times Y$, flat and with proper support over $S$, equipped with a surjective map $\F_S \rightarrow \G$, where $\F_S$ is the pullback of $\F$ to $S\times X$ via the projection, such that all closed fibres $\G_p$ are coherent quotient sheaves of $\F$ with Hilbert polynomial $P$. Two families over $S$ are \textit{equivalent} if they have the same kernel. It is clear how families pull back, and so we have a functor \[\mathcal{Q}uot^{P}_Y(\F): \sf{C} \rightarrow \Sets, \] where $\sf{C}$ is the category of schemes of finite type over $k$, sending a scheme $S$ to the equivalence classes of families over $S$.
\end{definition}
\begin{theorem}[Grothendieck]
	The $\mathcal{Q}uot^{P}_Y(\F)$ functor is representable. 
\end{theorem}
\begin{definition}
	Let $P\in \Q[z]$ be a numerical polynomial. The \textit{quot scheme of $\F$ with respect to $P$}, denoted $\Quot_Y^P(\F)$, is the fine moduli space of $\mathcal{Q}uot^{P}_Y(\F)$. A \textit{Hilbert scheme} is a Quot scheme of the form $\Quot_Y^P(\Oy)$, which we will simply denote $\Hilb_Y^P$.
\end{definition}
The idea of the construction, which can be found in \autocite{HilbQuot}, is to define an injective natural transformation from $\mathcal{Q}uot^{P}_Y(\F)$ into a certain Grassmannian functor, and show that this defines a scheme structure. However, this construction is far beyond the scope of this thesis, so we will be content with our examples from Chapter 1, where we showed that $\Quot_{\Spec k}^1(k^n) = \P^n$ and $\Hilb_{\P^2}^{2z+1} = \P^5$.

In particular, from Lemma \ref{bounding-family}, it follows that all semistable vector bundles of signature $n,d$ with $d > n(2g-1) =: N$ are found in the universal family over the scheme $Q = \Quot_X^{d+nz +n(1-g)}(\Ox^N).$ However, $Q$ also parameterises other sheaves, which we would like to ignore. The following result allows us to do so:
\begin{proposition}
	Let $S$ be a scheme and $\F$ a family of quotients of $\Ox^N$ with Hilbert polynomial $d+nz +n(1-g)$ over $S$. Then the locus $s$ of locally free semistable quotients $s: \Ox^N \rightarrow \F_s$ such that $H^0(s)$ is an isomorphism $k$-vector spaces is open.
\end{proposition}
\begin{proof}
	\autocite[p. 35 and p. 45]{HuyLehn}.
\end{proof}
In particular, taking $S = Q$ and $\F$ as the universal family, say $\mf{E}$ over $Q$, we have an open subscheme $Q^{ss}$, and a restricted family $\mf{E}^{ss}$ whose fibres over $k$-points are all semistable vector bundles of signature $(n,d)$, and conversely, every such vector bundle is contained in this family, by Lemma \ref{bounding-family}. Such a family is known as a \textit{bounded family}, and our plan of attack from here is clear: show that this family is locally universal, find a group action parameterising isomorphic fibres, linearise this action and take a GIT quotient. We will sketch these steps below:

