In this chapter, we will put together everything we have learnt so far to construct the moduli space of vector bundles. Let $X$ be a fixed nonsingular curve of genus $g$ over an algebraically closed field $k$ of characteristic zero (that is, an integral scheme of dimension 1 whose local rings at $k$-points are regular, quasi-projective over $k$). 


\section{Divisors and Line Bundles}
In our study of vector bundles, the natural place to start is with line bundles. In this section, we will review the relation between divisors and line bundles, which will ultimately help us define the degree of a vector bundle. Our exposition closely follows the one found in \autocite[II, Section 6]{Hart}
\begin{definition}\index{divisor ! Weil divisor}
	We define the \textit{group of Weil divisors} on $X$, denoted $\Div(X)$ to be the free abelian group on the $k$-points of $X$. The elements of $\Div X$ are known as \textit{Weil divisors}. The \textit{support} of a Weil divisor $D = \sum n_i p_i$, denoted $\Supp D$, is the set $\{p_i\mid n_i \neq 0\}$, and the \textit{degree} of $D$ is defined to be $\deg D:=\sum n_i$. We say $D$ is \textit{effective} if $n_i > 0$ for all $p_i\in \Supp D$. A \textit{prime divisor} is an effective Weil divisor of degree 1.
\end{definition}
Since $X$ is nonsingular, for any $k$-point $p$, the local ring $\O_{X, p}$ is regular of dimension 1, and in particular it is a DVR, with fraction field $K(X)$, the function field of $X$. Denote the valuation $v_p$. For any $f\in K(X)^*$, we define the \textit{divisor of $f$}, denoted $\div(f)$ to be \[\div f := \sum_{p\in X(k)} v_p(f) p. \]We can show (\autocite[II Lemma 6.1]{Hart}) that all but finitely many of the $v_p(f)$ vanish, hence we get a Weil divisor. A divisor of the form $\div f$ is known as a \textit{principal divisor}; clearly the principal divisors form a subgroup. Two Weil divisors are \textit{linearly equivalent} if their difference is a principal divisor. The group of Weil divisors modulo linear equivalence is the \textit{divisor class group}, denoted $\Cl X$. A \textit{divisor class}, is an element of $\Cl X$. By a \textit{divisor}, we will abuse language and refer to either a Weil divisor or its divisor class; it will either be clear from context or unimportant which is meant. \index{divisor! divisor class}
\begin{example}
	If $X$ is affine, equal to $\Spec A$ where $A$ is necessarily a Dedekind domain (\autocite[I, Proposition 11.5]{Neuk}), then $\Cl X = 0$ if and only if $A$ is a PID. Indeed, if $A$ is a PID, then if $D = \sum n_p p$ is a divisor, then for each $p\in \Supp D$, the maximal ideal $\mf{m}_p$ of $p$ is principal, generated by, say $\varpi_p$, which must be a uniformiser of $\O_{X,p}$. Now let \[f = \prod_{p\in \Supp D} \varpi^{n_p}\in \Frac A = K(X) \] then clearly $D = \div f$. 
	
	Conversely, if $A$ is not a PID, then there is some maximal ideal $\mf{m}_p$ corresponding to some $p\in X(k)$ generated by two elements (\autocite[I, 3, Ex 6.]{Neuk}), say $\mf{m}_p = \langle f, g \rangle$. Now localising at $p$, it is clear that either $f$ or $g$ must be the uniformiser of $\O_{X,p}$; suppose it is $f$ without loss of generality. Now if $p$ was principal, then clearly $p$ must be the divisor of some associate of $f$ in $\O_{X,p}$, say $F$. But $F$ is not prime, and by the unique ideal factorisation property of Dedekind domains (\autocite[I, Theroem 3.3]{Neuk}), we may write \[\langle F \rangle = \mf{m}_p \prod \mf{m}_{p_i}^{n_i}\] where at least one $n_i$ is nonzero, but all but finitely many of them are. But that means \[\div F = p + \sum n_i p_i \] which is a contradiction.
	
	More concretely, if $X = \A^1 = \Spec k[x]$, then any $D = \sum n_i a_i$ can be written $D = \div \prod (x - a_i)$, but if $X = \Spec k[x,y]/ \langle y^2 = x^3-x\rangle$ and $\operatorname{char} k \neq 2$, then we claim $p = (0,0)$ is not principal. Indeed, $\mf{m}_p = \langle x, y \rangle$ and clearly $y^2 = x(x-1)(x+1)$, so $v_p(x) = 2$, so $y$ is a uniformiser of $\O_{X,p}$. But \[\div y = (0,0) + (1,0) + (-1,0) \neq p,\] and similarly for any other uniformiser.
\end{example}
We now state a key result, which will allow us to define the degree of a line bundle:
\begin{proposition}\label{degree-independent-divisor-representative}
	The degree map $\deg: \Div X \rightarrow \Z$ descends to a map $\Cl \rightarrow \Z$.
\end{proposition}
\begin{proof}
	\autocite[p. 138]{Hart}.
\end{proof}
\begin{example}
	Let $X = \P^1 = \Proj k[x_0,x_1]$. We claim that the degree map is an isomorphism; in particular its kernel is trivial. So suppose $D = \sum n_p p$ is a degree zero divisor. Now each $p$ can be written $[p_0: p_1]$, corresponding to $\langle x_0p_1 - x_1p_0 \rangle$. Write \[f = \prod_{p\in \Supp D} (x_0p_1-x_1p_0)^{n_p}\in k(x_0, x_1) \] Now observe that \[K(X) = k(\frac{x_0}{x_1}) \] in particular, $K(X)$ is the subfield of $k(x_0, x_1)$ consisting of degree 0 elements. Since $\sum n_p = 0$, it follows $\deg f = 0$ too, hence $D = \div f$ as desired.
\end{example}

Next we describe the relation between divisors and line bundles: Let $[D]$ be a divisor class. We define the \textit{line bundle associated to} $[D]$, denoted $\L(D)$ as follows: let $D = \sum n_p p$ be a divisor in the class of $[D]$. For any open set $U$, we define \[\L(D)(U):= \{f\in K(X) \mid v_p(f) + n_p \geq 0 \text{ for all }p\in U(k)\}\] Firstly, we need to check that this is indeed a line bundle. To see this, observe the following: if $U\cap \Supp D = \emptyset$ then $\L(D)(U) = \Ox(U)$. Now for any $p\in \Supp D$, we choose some uniformiser $\varpi_p$ of $\O_{X, p}$, and we may assume $\varpi_p\in \Ox(U_p)$ for some open set $U_p$ containing $p$. We may pick $U_p$ sufficiently small that $(\div \varpi_p^{n_p})|_{U} = n_p p$ (in other words, $U_p \cap \Supp (\div \varpi_p) = \{p\}$). Now observe $\L(D)(U_p) = \varpi_p^{-1} \Ox(U)$, and in particular they are isomorphic. Now covering $X$ with the open sets $U_p$ and open sets which do not intersect $\Supp D$, we deduce that $\L(D)$ is a line bundle. 

Finally, we need to check that $\L(D)$ does not depend on our choice of $D$. So let $D' = \sum n_p' p$ be a divisor linearly equivalent to $D$, so that $D - D' = \div F$. Observe that then $v_p(F) = n_p - n_p'$. Now we define the isomorphism $\varphi: \L(D)\rightarrow \L(D')$ to be $f\mapsto  Ff$. To see that this is an isomorphism, observe that given an open set $U$, we have \[ \L(D)(U)= \{f\in K(X) \mid v_p(f) + n_p \geq 0 \text{ for all }p\in U(k)\} \] and \[\L(D')(U):= \{f\in  K(X) \mid v_p(f) + n_p' \geq 0 \text{ for all }p\in  U(k)\} \] now if $f\in \L(D)(U)$, then \[v_p(fF) + n_p' = v_p(f) + v_p(F) + n_p' \geq -n_p + v_p(F) + n_p' = 0 \] hence we have a well-defined morphism of sheaves, and moreover, clearly division by $F$ is its inverse. 
\par Before we state our next result, we recall that the \textit{Picard group}, denoted $\Pic X$ is the group of line bundles on $X$, with the group operation given by tensor product, and inversion given by dualising. \index{Picard group}%Using \v{C}ech cohomology, we can show $\Pic X \cong H^1(\Ox^*)$, where $\Ox^*$ is the sheaf of nowhere-vanishing functions on $X$. Indeed, the transition functions of a line bundle $\L$ on a sufficiently fine cover define a \v{C}ech 1-cocycle of $\Ox^*$, and is not hard to show two line bundles are isomorphic if and only if their transition functions are cohomologous, and that the $\v{C}ech$ cohomology group remains constant under refeinement. 
\begin{theorem}
	The map $D \mapsto \L(D)$ is an isomorphism $\Cl X \rightarrow \Pic X$.
\end{theorem}
\begin{proof}
	This follows the proof given in \autocite[pp. 144, 145]{Hart}. First of all, let $\K$ denote the constant sheaf $K(X)$, which is an $\Ox$-module. We claim $\L \otimes \K\cong \K$. To see this, we consider the base of the topology $\{U_\alpha = \Spec A_\alpha\}$ on $X$ consisting of the affine open sets where $\L$ is trivial. Then $K(X) = \Frac A_\alpha$ for each $\alpha$, so locally we have the natural isomorphism $A_\alpha \otimes K(X) \cong K(X)$, given by $a\otimes b \mapsto b$, and it is very easy to see that these agree on overlaps, and since the $U_\alpha$ form a base, by \autocite[Proposition I-12]{EisHar}, this extends to an isomorphisms of sheaves. It thus follows that $\L$ can be embedded inside $\K$. 
	
	First we prove surjectivity. By our discussion above, we may consider $\L$ as a subsheaf of $\K$; in particular, every local section is an element of $K(X)$. Since $\L$ is invertible, it follows that $\L(U)$ is a free $\Ox(U)$-submodule of $K(X) = \Frac \Ox(U)$, for sufficiently small $U$, and in particular it is generated by some $\varpi_U^{-1}\in K(X)$. Now we define the divisor $D$ locally as $\div \varpi_U$, and let $U$ vary across a cover. To see this is well-defined, note that if $\varpi_U^{-1}$ generates $\L(U)$ and $\varpi_V^{-1}$ generates $\L(V)$, then both $\varpi_U^{-1}$ and $\varpi_V^{-1}$ generate $\L(U \cap V)$, and in particular they are associates in $\Ox(U\cap V)$, hence they generate the same Weil divisor on $U \cap V$. It is then clear that $\L(D) = \L$.
	
	Next we prove the homomorphism property. But this is obvious since if $\L(D_1)$ and $\L(D_2)$ are generated locally by $\{\varpi_1^{-1}\}$ and $\{\varpi_2^{-1}\}$ respectively, then $\L(D_1 + D_2)$ are generated locally by $\{\varpi_1^{-1}\varpi_2^{-1}\}$. But $\L(D_1)\otimes \L(D_2)$ is also generated locally by $\{\varpi_1^{-1}\varpi_2^{-1}\}$, and hence they are isomorphic.
	
	Finally, we prove injectivity. Since we have shown this is a group homomorphism, it suffices to show that the kernel is trivial. So suppose $\L(D) \cong \Ox$, and fix an isomorphism $\varphi: \Ox \rightarrow \L(D)$. We claim $D = \div \varphi(1)^{-1}$. Indeed, choosing a sufficiently fine open affine cover $\{U_\alpha\}$ such that $\#(U_\alpha \cap \Supp D) \leq 1$, we suppose $D|_{U_\alpha}$ is the divisor of $\varpi_\alpha$. Then $\varpi_\alpha^{-1}$ generates $\L(D)(U_\alpha)$. But $\varphi(1)$ also generates $\L(D)(U_\alpha)$, hence $\varphi(1)$ and $\varpi_\alpha^{-1}$, and hence both generate the same Weil divisor. In particular, $\varphi(1)^{-1}$ generates $D$, as desired.
\end{proof}

Now let $\L$ be a line bundle. We will take a look at $H^0(X, \L)$, the space of global sections. For each nonzero $s\in H^0(X, \L)$, we define the \textit{divisor of zeroes} of $s$, denoted $\div s$ as follows: on any open subset $U$ on which $\L$ is trivial, we let $\Phi_U: \L|_U \rightarrow \O_U$ be an isomorphism, and define \[\div s|_U:= \div \Phi_U(s). \] Of course, this is well-defined: let $p\in U(k)$, and let $\Phi_V: \L|_V \rightarrow \O_V$ be another trivialisation, with $p\in V(k)$. Then at $p$, $\Phi_V(s)$ and $\Phi_U(s)$ differ by an invertible element of $\O_{X, p}$, and thus they have the same valuation, and since $p$ is arbitrary, this means we get a well-defined Weil divisor. Furthermore, observe that $\div s$ is effective, since it is locally the divisor of some section of $\Ox$, which must have nonnegative valuation at any $k$-point. We are now in a position to state our result:
\begin{proposition}\label{divisor-of-zeroes-effective}
	Let $\L = \L(D)$ be the line bundle associated to a divisor $D$. Then:
	\begin{enumerate}
		\item The divisor of zeroes of any nonzero $s\in H^0(X, \L)$ is linearly equivalent to $D$.
		\item Any effective divisor $D_0$ linearly equivalent to $D$ is the divisor of zeroes of some nonzero $s\in H^0(X, \L)$.
	\end{enumerate}
\end{proposition}
\begin{proof}
	\autocite[p. 157]{Hart}.
\end{proof}
We now come to the definition of the degree:
\begin{definition}\index{vector bundle ! degree}
	Let $\E$ be a line bundle. We define the \textit{degree} of $\E$, denoted $\deg \E$ as follows: if $\E$ is a line bundle, we define $\deg \E$ to be the degree of the divisor corresponding to $\E$ via the isomorphism $\Pic X \cong \Cl X$. This is well-defined, by Proposition \ref{degree-independent-divisor-representative}. In general, we define the \textit{determinant line bundle} of a rank $n$ vector bundle $\E$ to be the line bundle \[\det \E := \wedge ^n \E\] and we define $\deg \E := \deg(\det \E)$. Note that $\det \L = \L$ if $\L$ is a line bundle, hence our definition is consistent. We define the \textit{signature} of $\E$ to be the pair $(\rk \E, \deg \E)$.
\end{definition}
\begin{example}
	Let $X = \P^1 = \Proj k[x_0, x_1]$. We will show $\deg \Ox(n) = n$. Let $D =n [0:1]$. We compute $\L(D)$ as follows: on $U_1 = \{x_1\neq 0\} = \Spec k[x_0/x_1]$, we have \[D|_{U_1} = \div (\frac{x_0}{x_1})^n \] and on $U_0 = \{x_0 \neq 0\} = \Spec k[x_1/x_0]$, we have $D|_{U_0} = 0$, hence \[\L(D)(U_0) = \Ox(U_0) = k[x_1/x_0] \] and \[\L(D)(U_1) = (\frac{x_1}{x_0})^nk[x_0/x_1] \] Meanwhile, \[\Ox(n)(U_0) = x^n_0 k[x_1/x_0]\] and \[\Ox(n)(U_1) = x_1^n k[x_0/x_1] \] hence we define the isomorphism $\L(D) \rightarrow \Ox(n)$ by $f\mapsto x_0^n f$. It is easy to check that this is well-defined and agrees on overlaps (which is just localising), hence we get an isomorphism of line bundles.
	
	We also observe that $D$ is the divisor of zeroes of $x_0^n\in H^0(X, \Ox(n))$, as expected.
\end{example} 
\begin{example}
	Let $X$ be the elliptic curve $X=\Proj k[x,y,z]/ \langle y^2z = x^3 - xz^2 \rangle$, and let $\Ox(1)$ be the pullback of $\O_{\P^2}(1)$ induced by the embedding (or equivalently the twisting sheaf of Serre). We will show $\deg \Ox(1) = 3$ and is isomorphic to $\L = \L(3[0:1:0])$. Note that $x\neq 0$ implies $z\neq 0$, hence we can cover $X$ by two open sets $U_z = \{z\neq 0\}$ and $U_y = \{y \neq 0\}$. Thus we have \[\Ox(1)(U_z) = zk[y/x, z/x]/ (y^2z/x^3 = 1 - z^2/x^2) = z\Ox(U_z) \] and \[\Ox(1)(U_y) = yk[x/y, z/y]/ (z/y = (x/y)^3 - xz^2/y^3) = y\Ox(U_y).\] Now observe that \[\L(U_z) = \Ox(U_z)\] but \[\L(U_y) = (y/z)\Ox(U_y) \] since $D = 3[0:1:0]$ is the divisor of $z/y$ on $U_y$. It then follows we have a global isomorphism $\L \rightarrow \Ox(1)$ defined by $f \mapsto zf$.
\end{example}

We conclude this section with some technical results about vector bundles. This library of results will be used throughout the rest of this chapter.
\begin{proposition}\label{subbundle exists}
	Let $\E$ be a vector bundle. Then $\E$ has a line subbundle (a \textbf{subbundle} is a subsheaf $\F$ of $\E$ such that $\E/\F$ is also locally free; note that this is a property of the inclusion $\F \subseteq \E$, not of $\F$ itself).
\end{proposition}
\begin{proof}
	Tensoring with a sufficiently high power of an ample line bundle (and then tensoring with its dual at the end), we may assume without loss of generality $H^0(X, \E)\neq 0$. Let $s\in H^0(X, \E)$ be a nonzero global section and let $\L$ be the subsheaf generated by $s$, so that $\L$ is locally free of rank 1. Then $s$ is a global section of $\L$ and thus if $D$ is the divisor of zeroes of $s$, we know $\L = \L(D)$. Now tensoring with $\L(-D)$, we see  \[ \Ox \cong \L \otimes \L(-D)\subseteq \E \otimes\L(-D), \] and in particular there is a nonzero global section of $\E \otimes \L(-D)$, hence $\E \otimes\L(-D)/\Ox$ is locally free (since we are locally annihilating a free generator), and we have the following short exact sequence of vector bundles \[ 0 \rightarrow \Ox \rightarrow \E \otimes \L(-D)\rightarrow \E \otimes\L(-D)/\Ox\rightarrow 0. \] Tensoring with $\L(D)$ then proves the result.
\end{proof}
\begin{remark}\label{subbundle-exists-remark}
	In fact, we have proven that if $s$ is a global section of $\E$, then there is a line subbundle of $\E$ isomorphic to the subsheaf generated by $s$.
\end{remark}

\begin{proposition}
	Let \[0 \rightarrow \E \rightarrow \F \rightarrow \G \rightarrow 0 \] be a short exact sequence of vector bundles. Then \[\deg \F = \deg \E + \deg \G\]
\end{proposition}
\begin{proof}
	If $\{g_{\alpha\beta}\in \GL_n(\Ox(U_\alpha\cap U_\beta)) \}$ is the set of transition morphisms on a sufficiently fine cover representing $\F$, then it is not difficult to check $\{\det g_{\alpha\beta}: U_\alpha\cap U_\beta \rightarrow \bb{G}_m \}$ is the set of transition functions of $\det \F$. Now it can be shown (\autocite[p. 68]{GriHa}) that each $g_{\alpha\beta}$ has the shape \[\begin{pmatrix*}
		h_{\alpha\beta} & k_{\alpha\beta} \\
		0 & j_{\alpha\beta}
	\end{pmatrix*} \] where $\{h_{\alpha\beta} \}$ and $\{j_{\alpha\beta}\}$ are the transition morphisms representing $\E$ and $\F$ respectively. Hence\[ \det g_{\alpha\beta} = \det h_{\alpha\beta} \det j_{\alpha\beta} \] But $\{\det h_{\alpha\beta} \det j_{\alpha\beta}\}$ is the set of transition functions for $\det \E \otimes \det \G$, which means $\det \F = \det \E \otimes \det \G$, and thus the result follows.
\end{proof}
\begin{corollary}
	Let $\E$ and $\F$ be vector bundles. then $\deg (\E\oplus \F) = \deg \E + \deg \F$.
\end{corollary}
\begin{proof}
	Apply the above proposition to \[0 \rightarrow \E \rightarrow \E \oplus \F \rightarrow \F \rightarrow 0 \] and the result follows immediately.
\end{proof}
\begin{corollary}\label{twisting-by-line-bundle-degree}
	Let $\E$ be a vector bundle and $\L$ a line bundle. Then \[\deg(\E \otimes \L) = \deg \E + \deg \L \rk \E \] 
\end{corollary}
\begin{proof}
	As in the proof of the proposition, we look at the transition functions, and the result immediately follows.
\end{proof}

\begin{lemma}\label{negative-degree-no-h0}
	Let $\L$ be a line bundle of rank $n$ and suppose $H^0(X, \L) \neq 0$. Then $\deg \L \geq 0$, with equality holding if and only if $\L = \Ox$.
\end{lemma}
\begin{proof}
	Suppose $\L$ is the line bundle associated to a divisor $D$ and suppose $s\in H^0(X, \L)$ is a nonzero global section. Then by Proposition \ref{divisor-of-zeroes-effective}, the divisor of zeroes of $s$, say $D_0$, is effective, and linearly equivalent to $D$. But since $D_0$ is effective, it has nonnegative degree, and thus $\L = \L(D) = \L(D_0)$ has nonnegative degree. Moreover, if the degree of $D_0$ is zero, then it is trivial, and thus so is $\L$.
\end{proof}
\begin{lemma}
	Let $\E$ and $\F$ be vector bundles of rank $n$, and suppose there is a homomorphism $ \E \rightarrow \F$ with nonzero determinant (i.e. the determinant is not identically zero). Then $\deg \E \leq \deg \F$.
\end{lemma}
\begin{proof}
	Taking the determinant, we get a nonzero homomorphism $\det \E \rightarrow \det \F$. Now a nonzero homomorphism $\det \E \rightarrow \det \F$ is nothing more than a global section of $\fancyHom(\det\E, \det\F) = (\det\E) ^\vee \otimes \det\F$, and since $\deg((\det\E) ^\vee \otimes \det \F) = \deg \F - \deg \E$, the result follows from Lemma \ref{negative-degree-no-h0}.
\end{proof}
\begin{proposition}\label{canonical-extension-proposition}
	Let $\E \rightarrow \F$ be a nonzero homomorphism of vector bundles. Then there is a factorisation:
	\begin{equation}
		\begin{tikzcd}
			0 \arrow[r] & \cal{E'}\arrow[r] & \E \arrow[r] \arrow[d, "\varphi"]& \E'' \arrow[d]\arrow[r]& 0\\
			0  & \F''\arrow[l]& \F \arrow[l] & \F'\arrow[l] &\arrow[l] 0
		\end{tikzcd}
	\end{equation}
	where each sheaf above is locally free, the rows are exact, and $\E' \cong \ker \varphi$, $\E''\cong \im \varphi$ and $\rk \E'' = \rk \F'$, $\deg \E'' \leq \deg \F'$.
\end{proposition}
\begin{proof}
	Define $\E'$ and $\E''$ as in the theorem statement, so that the top row is exact. Since $\F$ is locally free and since $X$ is covered by the spectra of Dedekind domains (which are hereditary), it follows that $\E'' = \im \varphi\subseteq \F$ is locally free. Now the sheaf $\coker \varphi$ is coherent, and hence if $U = \Spec A$ is an affine open subset of $X$, where $A$ is a Dedekind domain, then $\coker \varphi$ is isomorphic to $\widetilde{M}$ for some finitely-generated $A$-module $M$. Let $M'$ be the torsion submodule of $M$ and we define $\F''$ to locally be the sheaf $(M/M')^\sim$. It is not hard to see that this is well-defined. Observe that since $A$ is a Dedekind domain and $M/M'$ is a torsion-free module, it is also a projective module, and hence $\F''$ is locally free, by the Serre-Swan Theorem. We then define $\F'$ to be the kernel of $\F \rightarrow \F''$. Now since the map $\E'' \rightarrow \F''$ defined by composing the obvious maps is zero, by the universal property of kernels there is unique homomorphism $\E'' \rightarrow \F'$ making everything commute. This unique homomorphism has nonzero determinant, because $\varphi$ is nonzero, and $\E'' \rightarrow \F$ is the inclusion of the image. Finally, in light of the above lemma, it suffices to show that $\rk \E'' = \rk \F'$. But this follows directly from the observation that the local ring at any $k$-point is a DVR, and thus a PID, and so any finitely generated module splits into its torsion and free parts.
\end{proof}
\begin{proposition}
	Let $\Ox^n\rightarrow \E$ be a surjective homomorphism of vector bundles, let $\F$ be a subbundle pf $\E$ and
\end{proposition}
\begin{proof}
	The existence and uniqueness of the homomorphism follows from the universal property of kernels. To prove the homomorphism is surjective, it suffices to check this on the level of stalks. So let $p\in X$ be a point. Since $\E$ is a subbundle, there exists a basis $\{v_1,..,v_n\}$ of the stalk $\G_p\cong \O_{X,p}^n$, where $n = \rk \G$, such that $\{v_1,...,v_m\}$ is a generating set for $\E_p$, and $\{v_{m+1},...,v_{n}\}$ is a generating set for $(\G/\E)_p$. 
\end{proof}

\section{Stable and Semistable Bundles}
To motivate the definitions that follow, we begin by stating our moduli problem. We now make the assumption that $X$ is complete, and we fix a very ample line bundle $\Ox(1)$ on $X$.
\begin{definition}
	The \textit{moduli problem of vector bundles of signature $(n,d)$ on $X$} is the functor $\mathcal{V}_{n,d}: \Sch/k \rightarrow \Sets$ defined as follows: for any scheme $S/k$, we define a family of vector bundles over $S$ to be a coherent sheaf $\E$ on $X\times S$, flat over $S$ such that for any $s\in S(k)$, the fibre $\E_s $ defined to be the pullback of $\E$ along the map $s\times \id: \Spec k \times X \rightarrow S \times X$ is a locally free sheaf of signature $(n,d)$ on $X$. Two families $\E, \F$ are \textit{equivalent}, written $\E \sim \F$, if there is a line bundle $\L$ on $S$ such that \[\E \cong \F \otimes \pi_S^*\L. \] Note that dualising commutes with pullbacks, so this is an equivalence relation. We may also pull back families in the obvious way, and clearly equivalent families are pulled back to equivalent families. We then define $\mathcal{V}_{n,d}$ to be \[\mathcal{V}_{n,d}(S):= \{\text{families over }S\}/ \sim. \]
\end{definition}
However, even in the simple case of signature (2,0)-bundles over $\P^1 = \Proj k[x,y]$, there is a jump phenomenon:
\begin{example}
	We consider the following family $\E_t$ over the affine line $\A^1 = \Spec k[t]$: let $U_0$ (resp. $U_1$) be the open subset where $x$ (resp. $y$) does not vanish. Then we glue $\O_{U_0\times \A^1}^2$ and $\O_{U_1\times \A^1}^2$ on the overlap via the transition function \[g_{0,1}:= \begin{pmatrix*}
		\frac{x}{y} & t \\
		0 & \frac{y}{x}
	\end{pmatrix*} \in \GL_2(k[t, \frac{y}{x}, \frac{x}{y}])= \GL_2(\Gamma(U_0 \cap U_1, \O_{\A^1\times \P^1})). \] More precisely, on $U_0$ we have the standard basis $r_1, r_2\in \Gamma(U_0, \O_{U_0\times \A^1}^2)$, and on $U_1$ the standard basis $s_1, s_2\in \Gamma(U_0, \O_{U_1\times \A^1}^2)$. We then define an isomorphism sending $s_1 \mapsto ( x/y)r_1$ and $s_2 \mapsto rs_1 + (y/x) s_2$, and define $\E_t$ to be the gluing of the two sheaves on the overlap via this isomorphism. Since $\E_t$ is locally free, it is flat over $\A^1 \times \P^1$ and by the transitivity of flatness is also flat over $\A^1$. 
	
	Now we claim that the fibre $\E_0$ is isomorphic to $\O_{\P^1}(-1) \oplus \O_{\P^1}(1)$, but every other fibre is isomorphic to $\O_{\P^1} \oplus \O_{\P^1}$ (and of course, these are not isomorphic, the latter has a nowhere-vanishing global section, but the former does not). To see this, first observe that clearly when $t = 0$, the transition map is just $\diag(y/x, x/y)$, which is the transition map of $\O_{\P^1}(-1) \oplus \O_{\P^1}(1)$. However, for a nonzero $\lambda$, we will define an isomorphism $\varphi: \E_\lambda \rightarrow \O_{\P^1} \oplus \O_{\P^1}$ as follows: let $e_1, e_2$ be the standard basis of $\O_{\P^1}\oplus \O_{\P^1}$. We define $\varphi$ to be the map 
	
	\begin{align*}
		r_1 \mapsto \frac{y}{x}e_1 - e_2,\; r_2 \mapsto \lambda e_1
	\end{align*} 
	on $U_0$ and
	\begin{align*}
		s_1 \mapsto e_1 - \frac{x}{y}e_2,\;   s_2 \mapsto \lambda e_2
	\end{align*}
	on $U_1$. To see that these glue, observe that $r_1 = (y/x)s_1$ and $r_2 = \lambda s_1 + (x/y)s_2$, and so on the overlap we have \[\varphi(\frac{y}{x}s_1) = \frac{y}{x}(e_1- \frac{x}{y}e_2) =\frac{y}{x}e_1 - e_2  \] \[\varphi(\lambda s_1 + \frac{x}{y} s_2) = \lambda e_1 - \frac{\lambda x}{y} e_2 + \frac{x}{y} \lambda e_2 = \lambda e_1 \] as desired. Finally, observe that $\varphi$ maps local free generators to local free generators, and thus this is an isomorphism.
\end{example}
\begin{remark}
	Of course, the fibres of the family $\E_t$ given above are canonically identified with the elements of the group \[\Ext^1(\O_{\P^1}(1), \O_{\P^1}(-1)) \cong \Ext^1(\O_{\P^1},\O_{\P^1}(-2) ) \cong H^1(\P^1, \O_{\P^1}(-2)) \cong k.\] {\color{red} We may generalise this idea as follows}: let $\E, \F$ be two vector bundles over $X$, so that $\Ext^1(\F, \E) \cong H^1(X, \fancyHom(\F, \E))$. Let $\{U_\alpha\}$ be a sufficiently fine open cover of $X$, let $e_{\alpha\beta}$ (resp. $f_{\alpha\beta}$) be a set of transition functions representing $\E$ (resp. $\F$), and let $\{g_{\alpha\beta}\}$ be a \v{C}ech 1-cocycle representing some element of $H^1(X, \fancyHom(\G, \F))$. 
\end{remark}
\begin{definition}\index{vector bundle ! slope}
	Let $\E$ be a vector bundle. The \textit{slope} of $\E$, denoted $\mu(\E)$, is defined to be \[\mu(\E):= \frac{\deg \E}{\rk \E} \]
\end{definition}
\begin{lemma}\label{subbundle-quotient-slope}
	Let \[0 \rightarrow \E \rightarrow \F \rightarrow \G \rightarrow 0 \] be a short exact sequence of vector bundles. Then $\mu(\E) \leq \mu(\F)$ if and only if $\mu(\G) \geq \mu(\F)$, with equality holding in one if and only if in the other.
\end{lemma}
\begin{proof}
	Trivial.
\end{proof}
And finally, the definition:
\begin{definition}\index{vector bundle ! stability}
	Let $\E$ be a vector bundle. Then $\E$ is \textit{stable} (resp. \textit{semistable}) if for every proper subbundle $\F$, we have 
	\begin{align*}
		\mu(\F) &\,< \mu(\E)\\
		&(\leq)
	\end{align*}
	$\E$ is \textit{polystable} if it is a direct sum of stable bundles of the same slope.
\end{definition}
Observe that we can alternatively define stability in terms of quotient bundles: by Lemma \ref{subbundle-quotient-slope}, $\E$ is (semi) stable if and only if for every quotient bundle $\G$ we have:
\begin{align*}
	\mu(\E) &\,< \mu(\G)\\
	&(\leq)
\end{align*}

\par Let us mention some more basic properties of stability:
\begin{lemma}\label{stability-properties}
	Let $\E$ be a vector bundle. 
	\begin{enumerate}
		\item $\E$ is (semi)-stable if and only if $\E \otimes \L$ is (semi)-stable for every line bundle $\L$.
		\item If $\rk \E = 1$ (i.e. $\E$ is a line bundle) then $\E$ is always stable.
		\item If $\E'$ is another vector bundle then $\E \oplus \E'$ is not stable. It is semistable only if $\mu(\E) = \mu(\E')$ and both are semistable. 
		\item If $\varphi: \E\rightarrow \F$ is a nonzero morphism of vector bundles, and both bundles are semistable, then $\mu(\E)\leq \mu(\F)$.
		\item If $\E$ is stable, then it is simple (that is, $\End(\E) = k$).
		\item $\E$ is semistable if and only if for any \textbf{subsheaf} $\F$ we have $\mu(\F) \leq \mu(\E)$. Stability holds if and only if strict inequality holds. 
	\end{enumerate}
\end{lemma}
\begin{proof}
	Note that every subbundle of $\E\otimes \L$ has the form $\F \otimes \L$ (indeed, if $\G$ is any subbundle of $\E \otimes \L$ we take $\F = \G \otimes \L^\vee$). Now observe \[\mu(\F \otimes \L) = \frac{\deg \F + \rk \F \deg \L}{\rk \F} = \mu(\F) + \deg \L\] and similarly for $\mu(\E \otimes \L)$ and hence \[\mu(\E \otimes \L) - \mu(\F \otimes \L) = \mu(\E) - \mu(\F) \] which proves (i). (ii) is trivial. To prove (iii), suppose without loss of generality $\mu(\E) \leq \mu(\E')$. Now \[\mu(\E \oplus \E') = \frac{\deg(\E) + \deg(\E')}{\rk(\E) + \rk(\E')} \leq \mu(\E') \] hence $\E \oplus \E'$ is not stable. If it is semistable, then equality must hold above, and clearly both must both be semistable, which proves (iii).
	
	To prove (iv), observe that by Proposition \ref{canonical-extension-proposition}, the map $\varphi$ factors as follows:
	\begin{equation}\label{proper-factorisation-equation}
		\begin{tikzcd}
			0 \arrow[r] & \cal{E'}\arrow[r] & \E \arrow[r] \arrow[d, "\varphi"]& \E'' \arrow[d]\arrow[r]& 0\\
			0  & \F''\arrow[l]& \F \arrow[l] & \F'\arrow[l] &\arrow[l] 0
		\end{tikzcd}
	\end{equation}
	where $\E' \cong \ker \varphi$, $\E''\cong \im \varphi$ and $\rk \E'' = \rk \F'$, $\deg \E'' \leq \deg \F'$. Since $\E$ and $\F$ are both stable, it follows 
	\begin{equation}\label{chain-of-slop-inequalities-equation}
		\mu(\E) \leq \mu(\E'') \leq \mu(\F') \leq \mu(\F)
	\end{equation}
	as desired. 
	
	To prove (v), suppose $\varphi: \E \rightarrow \E$ is an endomorphism. If $\varphi$ is zero, we are done. If not, we must have equality in (\ref{chain-of-slop-inequalities-equation}), and by the stability of $\E$ we have $\E = \E'' = \im \varphi$, so in particular $\varphi$ is an isomorphism. We have proven that any nonzero endomorphism of $\E$ is an isomorphism. Now fix some $p\in X(k)$; we have an induced map of fibres $\E_p/\mf{m}_p\E_p\rightarrow \E_p/\mf{m}_p\E_p$, and since $k$ is algebraically closed, this map has an eigenvalue, say $\lambda$. But $\varphi - \lambda: \E \rightarrow \E$ is no longer an isomorphism, hence must be zero, and thus $\varphi = \lambda$ as desired. 
	
	Finally, to prove (vi), suppose $\E$ is semistable and let $\F$ be a subsheaf of $\E$. Then applying Proposition \ref{canonical-extension-proposition} to the inclusion $\F \subseteq \E$, we deduce there is a subbundle $\F'$ of $\E$ with $\deg (\F') \geq \deg(\F)$ and $\rk(\F') = \rk(\F)$, and hence $\mu(\F)\leq \mu(\F') \leq \mu(\E)$. If $\E$ is stable, then the inequality is strict. The converse is trivial.
\end{proof}
\begin{example}
	It is well-known that every vector bundle on $\P^1$ is a direct sum of line bundles, and moreover every line bundle is of the form $\O_{\P^1}(n)$. Thus we may write any vector bundle $\E$ on $\P^1$ as \[\E = \bigoplus_{n\in \Z} \O_{\P^1}(n)^{r_n}, \] where $\O_{\P^1}(n)^{r_n}$ means $r_n$ direct copies of $\O_{\P^1}(n)$, and all but finitely many of the $r_n$ are zero. In this situation, stability has a very easy description: the only stable bundles are line bundles by (ii) in the above lemma, semistable and polystable bundles coincide and they all look like $\O_{\P^1}(n)^r$, and as soon as two distinct summands $\O_{\P^1}(n)$ and $\O_{\P^1}(m)$ turn up the bundle is unstable. 
\end{example}
We consider the above example in greater detail. Given a bundle $\E = \bigoplus_{n\in \Z} \O_{\P^1}(n)^{r_n}$, where $n_1<...<n_{\ell}$ are the indices where $r_{n_i}\neq 0$, it is very tempting to build a filtration out of it, and indeed there is a very natural and reasonable way to do so, namely: \[0 \subseteq \E_1=\O_{\P^1}(n_\ell)^{r_{n_\ell}} \subseteq \E_2=\bigoplus_{i= \ell-1}^\ell \O_{\P^1}(n_i)^{r_{n_i}}\subseteq...\subseteq\E_{\ell} = \bigoplus_{i= 1}^\ell \O_{\P^1}(n_i)^{r_{n_i}} = \E. \] Note that this has the nice property that every quotient (i.e. each $\E_i/\E_{i-1} $) of the filtration is semistable, and it is ordered such that $\mu(\E_{i+1}/\E_{i})> \mu(\E_{i}/\E_{i-1})$. In fact, it is also easy to see that this is the \textit{unique} filtration such that these hold. To generalise:
\begin{proposition}\label{existence-of-harder-narasimhan}
	Let $\E$ be a vector bundle on $X$. Then there exists a unique filtration \[ 0 = \E_0\subseteq \E_1\subseteq....\subseteq \E_r = \E \] with the property that each $\E_i/\E_{i-1}$ is semistable and $\mu(\E_{i+1}/\E_i) > \mu(\E_{i}/\E_{i-1})$ for each relevant $i$.
\end{proposition}
Before we prove this, we extract the following lemma:
\begin{lemma}[\autocite{LePotier}, Lemma 5.4.1]
	Let $\E$ be a vector bundle on $X$. Then the set of degrees of subsheaves of $\E$ is bounded above.
\end{lemma}
\begin{proof}[Proof of Proposition \ref{existence-of-harder-narasimhan}]
	If $\E$ is semistable we are done. If $\E$ is a line bundle then it is stable and we are done. We now induct on the rank. Suppose $\E$ is unstable. By the above lemma, the slopes of subbundles of $\E$ is bounded above, so we choose a maximal slope $\mu$ and among these a maximal rank subbundle $\E_1$ with slope $\mu_{\max}$. It is clear that $\E_1$ is semistable, and moreover we claim $\E/\E_1$ has no subbundle of slope $\mu_{\max}$. Indeed, if it has such a subbundle, say $\F$, then by the correspondence theorem applied locally (it is not hard to check that we can do this and patch everything together), then $\F$ corresponds to a subbundle $\E_1\subsetneq \F_1 \subseteq \E$, and it is clear that $\F_1$ has strictly larger rank (since for every $p\in X$ the stalks of both $\E_{1,p}$ and $\F_{1,p}$ can be generated by elements of a basis of $\E_p \cong \O_{X,p}^{\rk \E}$). 
	
	 Now by the inductive hypothesis, $\E/\E_1$ has a filtration \[0 \subseteq (\E/\E_1)_2 \subseteq ...\subseteq (\E/\E_1)_n = \E/\E_1, \] and it is not hard to see that this lifts. Indeed, $(\E/\E_{1})_i$ is a subbundle of $\E/\E_1$ and applying the correspondence theorem locally and patching together, we find an $\E_i$ such that $\E_1\subseteq \E_i \subseteq \E$ and $\E_i/\E_1 = (\E/\E_1)_i$. Since $(\E_i/\E_{i+1}) = (\E/\E_1)_i/(\E/\E_1)_{i+1}$ it follows that this filtration satisfies the desired properties. 
	 
	 To prove uniqueness, suppose $(\E_i)_{i = 1}^n$ and $(\E_i')_{i = 1}^m$ are two filtrations satisfying the property. Let $\mu = \mu(\E_1)$ and let $\mu' = \mu(\E_1')$. Supposing without loss of generality $\mu \geq  \mu'$, we consider the map $\E_1 \rightarrow \E \rightarrow \E/\E_{m-1}'$. By assumption, $\mu \geq \mu' > \mu(\E/\E_{m-1}')$ and since both bundles are semistable, by (iv) of Lemma \ref{stability-properties} it follows the map is zero; in particular, $\E_1$ is contained in $\E_{m-1}'$. But applying the same argument inductively with $\E_{m-i}'$ in place of $\E$ and $\E_{m-i-1}'$ in place of $\E_{m-1}'$, we deduce $\E_1$ is contained in $\E_1'$. Now if $\mu > \mu'$, then by the same argument $\E_1 = 0$ which obviously cannot happen, and thus $\mu = \mu'$ and $\E_1$ is contained in $\E_1'$. But reversing the roles of $\E_1$ and $\E_1'$ we deduce $\E_1'$ is contained in $\E_1$ and thus they are equal. The result then follows by inductively applying the argument to $(\E_i/\E_1)$ and $(\E_i'/\E_1')$.  
\end{proof}
\begin{definition}
	The filtration above is known as the \textit{Harder-Narasimhan filtration} of $\E$. 
\end{definition}
%We next study the interplay between bundles of various stability. To begin, we have the following result:
%\begin{proposition}
%	Semistability and stability are open conditions. 
%\end{proposition} 
%\begin{proof}
%	Let $\F$ be a family of bundles over a scheme $S$ of signature $(n,d)$. 
%\end{proof}

\section{Constructing the Moduli Space}

{\color{red} We will now construct the moduli space of stable bundles of signature $(n,d)$. Firstly, line bundles are always stable, and in fact we have a natural isomorphism of functors $\mathcal{V}_{1,d}\cong \mathcal{V}_{1,0}$ by tensoring with a degree 1 line bundle.} {\color{blue}Thus $\mathcal{V}_{1,d}$ is represented the Jacobian of $X$}. {\color{red}Thus we may now assume $n \geq 2$. We have a trichotomy: $g = 0, g=1, g\geq 2$ (these correspond to the cases of the anticanonical bundle being ample, neither the canonical nor the anticanonical bundle are ample, or the canonical bundle is ample, as is usual in algebraic geometry). If $g=0$, then the result is trivial; indeed, since every vector bundle over $\P^1$ is a direct sum of line bundles, the only stable bundles on $\P^1$ are line bundles, thus $V^s_{n,d} = \emptyset$. 

If $g = 1$...}

We will now suppose $g \geq 2$. Our first job is to construct a bounded family of semistable bundles; that is, a family that parameterises every semistable bundle of given signature, so that we may take a GIT quotient of the parameter space. To this end, we will first require the Riemann-Roch theorem:
\begin{theorem}[Riemann-Roch for Vector Bundles]\index{Riemann-Roch Theorem}
	Let $\F$ be a vector bundle on $X$ of signature $(n,d)$. Then \[\chi(\F) = d + n(1-g), \] where $\chi(\F) := \sum_{i = 0}^\infty (-1)^i h^i(\F) = h^0(\F) - h^1(\F) $ is the Euler characteristic of $\F$. 
\end{theorem}
\begin{proof}
	We induct on $n$. For $n=1$, the result is classical, and is proven in, for example, pages 295-296 of \autocite{Hart}. Now supposing true up to some $n-1$, suppose $\F$ has rank $n$. Let $\L$ be a line subbundle of maximal degree. Then we claim $\F/\L$ must be locally free. Indeed, we may apply Proposition \ref{canonical-extension-proposition} taking $\varphi$ in the proposition statement to be the inclusion $\L\subseteq \F$, so that there is a nonzero map $\L \rightarrow \F'$ (where $\F'$ is as in the proposition statement). But both these bundles are line bundles, and hence are stable, and so $\deg \L \leq \deg \F'$, and since $\L$ is of maximal degree, equality must hold, and hence $\F/\L = \F''$ is locally free as claimed. Now applying the inductive hypothesis we have \[\chi(\F) = \chi(\L) + \chi(\F/\L) = \deg \L + 1 - g + (d - \deg \L) +(n-1)(1-g) = d + n(1-g) \] as desired.
\end{proof}
\begin{corollary}[Classical Riemann-Roch Theorem]
	For any line bundle $\L$ of degree $d$, we have \[h^0(\L) - h^0(\L^\vee\otimes \omega_X) = d + 1-g, \]  where $\omega_X$ is the canonical bundle (which is just the cotangent bundle in this case). In particular, the degree of the canonical bundle is $2g-2$.
\end{corollary}
\begin{proof}
	Combine the above theorem and the Serre duality theorem (\autocite[III, Corollary 7.7]{Hart}). 
\end{proof}
The importance of the Riemann-Roch theorem is that it allows us to relate the quantities we are interested in. More precisely, by the Serre vanishing theorem, for any coherent sheaf $\F$, and any ample line bundle $\L$, we have $H^i(X, \F\otimes \L^{\otimes m}) = 0$ for any sufficiently large $m$ and any $ i > 0$. Moreover, by the definition of ampleness, we know that $\F\otimes \L^{\otimes m}$ is generated by global sections, for any sufficiently large $m$. In particular, if $\F$ is locally free then the Riemann-Roch theorem tells us exactly how many global sections generate $\F\otimes \L^{\otimes m}$. The main issue is that the \enquote{sufficiently large} criterion depends on $\F$ and $\L$. This is remedied by the following result:
\begin{lemma}\label{bounding-family}
	Let $\E$ be a semistable vector bundle of signature $(n,d)$.
	\begin{enumerate}
		\item If $d >n(2g-2)$ then $H^1(X, \E) = 0$.
		\item If $d > n(2g-1)$ then $\E$ is generated by global sections.
	\end{enumerate} 
\end{lemma}
\begin{proof}
	This follows the proof given in \autocite[p. 68]{ModuliNotes}. Suppose for contradiction $H^1(X, \E)\neq 0$. By the Serre duality theorem, we have \[H^1(X, \E)\cong H^0(X, \E^\vee \otimes \omega_X) = \Hom(\E, \omega_X),\] where $\omega_X$ is the canonical bundle (which is just the cotangent bundle here). This means that there is a nonzero homomorphism $\varphi: \E \rightarrow \omega_X$. Now by Lemma \ref{stability-properties} (iv), we have \[2g-2 = \frac{n(2g-2)}{n} <\frac{d}{n} = \mu(\E) \leq \mu(\omega_X) = 2g-2 \] which is a contradiction. This proves (i). 
	
	To prove (ii), we note that since the local ring of the generic point is a field, we need only check the stalk is generated by global sections at closed points, or equivalently $k$-points, since $k$ is algebraically closed. So let $p$ be a $k$-point, with local ring $\O_{X,p}$ and maximal ideal $\mf{m}_p$. The composition $\Ox(U) \rightarrow \O_{X,p} \rightarrow \O_{X,p}/\mf{m}_p$ induces the following short exact sequence of sheaves:
	\[ 0 \rightarrow \mathcal{I}_p \rightarrow \Ox \rightarrow k_p \rightarrow 0, \] where $k_p$ is the skyscraper sheaf $k$ sitting over $p$, and $\mathcal{I}_p$ is the kernel. Since $\E$ is locally free, tensoring is exact, and moreover since it is of rank $n$, and $k_p$ is a skyscraper sheaf, it follows $k_p\otimes \E\cong k_p \otimes \Ox^n \cong k_p^n$. Thus tensoring the above with $\E$ we have 
	\begin{equation}\label{skyscraper-ses}
		0 \rightarrow \mathcal{I}_p\otimes \E \rightarrow \E \rightarrow k_p^n \rightarrow 0.
	\end{equation}  
	Now we claim that $\mathcal{I}_p\cong \L(-p)$. To see this, let $U = \Spec A$ be an open affine subset. If $p\notin U$, then $\mathcal{I}_p|_U = \Ox|_U$. Otherwise, $p$ is cut out by some $f\in A$, and hence $\mathcal{I}_p|_U = f\Ox|_U$. But this is exactly the definition of $\L(-p)$, as claimed. Now observe that $\mathcal{I}_p\otimes \E = \E \otimes \L(-p)$ is semistable and has degree $n(2g-2)$ and thus by part (i) we have $H^1(X, \E\otimes \L(-p)) = 0$. Now taking cohomology of (\ref{skyscraper-ses}), it follows we have a surjection \[H^0(X, \E ) \rightarrow H^0(X, k_p^n) = k^n. \] Finally, we apply Nakayama's lemma (\autocite[Corollary 4.8]{EisAlg}) on the local ring $\O_{X,p}$ to deduce that the map $H^0(X, \E)\rightarrow \E_p$ is surjective too.
\end{proof}
The consequence of the above lemma combined with the Riemann-Roch theorem is that every semistable vector bundle of signature $(n,d)$ for $d$ sufficiently large is a quotient of $\Ox^{d+n(1-g)}$. Now it just so happens that there is a scheme, known as the \textit{Quot scheme} that that is a fine moduli space parameterising quotients of a given coherent sheaf, with certain constraints.
\subsection{The Quot Functor and its Scheme}
To define the moduli problem the Quot scheme represents, we first require the following result.
\begin{proposition}
	Let $Y$ be a projective variety, let $\F$ be a coherent sheaf on $Y$ and let $\O(1)$ be a very ample line bundle. For any $m\in \Z$, write $\F(m):=\F\otimes \O(m):= \F \otimes \O(1)^{\otimes m}$. Then the map \[P:m\mapsto \chi(\F(m)) \] coincides with a polynomial with rational coefficients.
\end{proposition}
\begin{proof}[Proof Sketch]
	Embed $Y$ into $\P^n = \Proj k[x_0,...,x_n]$ via $\O(1)$. We induct on $r=\dim \Supp \F$. If $r = 0$, then $\F$ is supported on a discrete subset, and hence tensoring with $\O(1)$ does nothing, and so $P$ is constant, equal to $\chi(\F)$. Now supposing true for some $r \geq 0$, we suppose $\dim \Supp \F = r+1$, and write $M$ for the graded module $M:= \bigoplus_{m\in \Z} H^0(X, \F(m))$, so that $\widetilde{M} \cong \F$ (\autocite[II. Proposition 5.15]{Hart}). We note that the map $M(-1)\rightarrow M$ given by multiplication by $x_i$ induces the following short exact sequence of sheaves \[ 0 \rightarrow \E \rightarrow \F(-1)\rightarrow \F \rightarrow \G \rightarrow 0 \] where $\E$ and $\G$ are the kernel and cokernel respectively. Now if $\dim \Supp \E = r+1$, then {\color{red} TO FINISH }.
\end{proof}
\begin{definition}\index{Hilbert polynomial}
	The polynomial $P$ above is known as the \textit{Hilbert polynomial} of $\F$ with respect to $\O(1)$. 
\end{definition}
\begin{example}
	Take $Y = \P^r$, take $\F = \Oy$ and take $\O(1)$ to be the usual twisting sheaf of Serre. Observe that if $m \geq 0$ then $h^0(\O(m)) = \binom{m+r}{r},$ and {\color{red}$h^i(\O(m)) = 0$} for all $i > 0$, and hence\[ P = \binom{z+r}{r}:= \frac{1}{r!}\prod_{i = 1}^{r} (z+i)\in \Q[z].\]
\end{example}
\begin{example}
	Let $\E$ be a vector bundle of signature $(n,d)$ on $X$. By the Riemann-Roch theorem we know \[\chi(\E(m)) = d+nm\deg \Ox(1) + n(1-g), \] and so $\E$ has Hilbert polynomial $P = d + n\deg \Ox(1)z + n(1-g)\in \Q[z]$ with respect to $\Ox(1)$. Conversely, if $\F$ is a vector bundle with Hilbert polynomial $P = d + n\deg \Ox(1)z + n(1-g)\in \Q[z]$, then $\chi(\F) = d+n(1-g)$ and $\chi(\F(1)) = d+n\deg \Ox(1)+n(1-g)$, and so we know $\rk(\F) = n$ by Corollary \ref{twisting-by-line-bundle-degree} and $\deg(\F) = d$ by Riemann-Roch. In particular, the data of the Hilbert polynomial on $\E$ (with respect to $\Ox(1)$) is equivalent to the data of the signature of $\E$.
\end{example}
A key property of the Hilbert polynomial is the following:
\begin{theorem}
	Let $X \rightarrow S$ be a projective flat morphism of noetherian schemes, and let $\F$ be a coherent sheaf on $X$. Then the map \[s \mapsto \chi(\F_s) = \sum_{i = 0}^\infty (-1)^i \dim_{k(s)}H^i(S, \F_s), \] where $\F_s$ is the fibre of $\F$ over $s\in S$ and $k(s)$ is the residue field of $s$ is locally constant. In particular, the Hilbert polynomials of $\F_s$ all agree on a connected component.
\end{theorem}
\begin{proof}
	content...
\end{proof}
Next, we define the following moduli problem:
\begin{definition}
	Let $\F$ be a coherent sheaf on a projective scheme $Y$ with an embedding $Y\subseteq \P^n$, and let $P\in \Q[z]$ be a numerical polynomial. For any scheme $S$ of finite type over $k$, a \textit{family of quotients of $\F$ with Hilbert polynomial $P$ parameterised by $S$} is a coherent sheaf $\G$ on $S \times Y$, flat and with proper support over $S$, equipped with a surjective map $\F_S \rightarrow \G$, where $\F_S$ is the pullback of $\F$ to $S\times X$ via the projection, such that all closed fibres $\G_p$ are coherent quotient sheaves of $\F$ with Hilbert polynomial $P$. Two families over $S$ are \textit{equivalent} if they have the same kernel. It is clear how families pull back, and so we have a functor \[\mathcal{Q}uot^{P}_Y(\F): \sf{C} \rightarrow \Sets, \] where $\sf{C}$ is the category of schemes of finite type over $k$, sending a scheme $S$ to the equivalence classes of families over $S$.
\end{definition}
\begin{theorem}[Grothendieck]
	The $\mathcal{Q}uot^{P}_Y(\F)$ functor is representable. 
\end{theorem}
\begin{definition}
	Let $P\in \Q[z]$ be a numerical polynomial. The \textit{quot scheme of $\F$ with respect to $P$}, denoted $\Quot_Y^P(\F)$, is the fine moduli space of $\mathcal{Q}uot^{P}_Y(\F)$. A \textit{Hilbert scheme} is a Quot scheme of the form $\Quot_Y^P(\Oy)$, which we will simply denote $\Hilb_Y^P$.
\end{definition}
The idea of the construction, which can be found in \autocite{HilbQuot}, is to define an injective natural transformation from $\mathcal{Q}uot^{P}_Y(\F)$ into a certain Grassmannian functor, and show that this defines a scheme structure. However, this construction is far beyond the scope of this thesis, so we will be content with our examples from Chapter 1, where we showed that $\Quot_{\Spec k}^1(k^{n+1}) = \P^n$ and $\Hilb_{\P^2}^{2z+1} = \P^5$.

In particular, from Lemma \ref{bounding-family}, it follows that all semistable vector bundles of signature $n,d$ with $d > n(2g-1)$ are found in the universal family over the scheme \[Q = \Quot_X^{P}(\Ox^N),\] where $N := d+n(1-g)$ and $P$ is the polynomial $P(z) =d+nz\deg \Ox(1) + n(1-g)$. However, $Q$ also parameterises other quotients, which we would like to ignore. It turns out there is an $\SL_N$ action on $Q$ and a linearisation such that if $d$ is large enough, the (semi)stable points are exactly locally free (semi)stable quotients where the induced map of global sections is an isomorphism. We will give the construction below, following \autocite[68-84]{ModuliNotes}, where more details may be found. Various proofs below will also be found in the quoted citation, so we will not cite them individually.
\subsection{The $\SL_N$ action on $Q$ and its stability}
%The following result allows us to do so:
%\begin{proposition}
%	Let $S$ be a scheme and $\F$ a family of quotients of $\Ox^N$ with Hilbert polynomial $d+nz +n(1-g)$ over $S$. Then the locus $s$ of locally free semistable quotients $s: \Ox^N \rightarrow \F_s$ such that $H^0(s)$ is an isomorphism $k$-vector spaces is open.
%\end{proposition}
%\begin{proof}
%	\autocite[p. 35 and p. 45]{HuyLehn}.
%\end{proof}
%In particular, taking $S = Q$ and $\F$ as the universal family, say $\mf{E}$ over $Q$, we have an open subscheme $Q^{ss}$, and a restricted family $\mf{E}^{ss}$ whose fibres over $k$-points are all semistable vector bundles of signature $(n,d)$, and conversely, every such vector bundle is contained in this family, by Lemma \ref{bounding-family}. Such a family is known as a \textit{bounded family}, and our plan of attack from here is clear: show that this family is locally universal, find a group action parameterising isomorphic fibres, linearise this action and take a GIT quotient. We will sketch these steps below:
\par{}To define this $\SL_N$-action, observe that since $Q$ is a fine moduli space, a morphism $\SL_N \times Q \rightarrow Q$ is exactly a family of quotients of $\O_X^N$ parameterised by $\SL_N\times Q$. To this end, observe that the group $\SL_N(\Gamma(\SL_N,\O_{\SL_N}))$ of  $\Gamma(\SL_N,\O_{\SL_N}) = k[x_{ij}, \, 1 \leq i,j\leq N]/\langle \det(x_{ij}) - 1 \rangle$-valued points of $\SL_N$ is just the abstract group of automorphisms $\SL_N \rightarrow \SL_N$, and is dual to the group of automorphisms of $k[x_{ij}, \, 1 \leq i,j\leq N]/\langle \det(x_{ij}) - 1 \rangle$. Now $\SL_N(\Gamma(\SL_N,\O_{\SL_N}))$ may also be seen as the group of $\Gamma(\SL_N,\O_{\SL_N})$-linear automorphisms of the module $\Gamma(\SL_N,\O_{\SL_N})^N$, and thus by extension the free sheaf of rank $N$ on $\SL_N$. Thus the inversion morphism $\SL_N \rightarrow \SL_N$ corresponds to an automorphism $\iota$ of $\Gamma(\SL_N,\O_{\SL_N})^N$, specifically given by the inverse of the matrix $(x_{ij})$, and thus for any $k$-point $(g_{ij})\in \SL_N(k)$, the fibre of this morphism is exactly $(g_{ij})^{-1}\in \SL_N(k)$. 

Now let $U:\O_{Q\times X}^N \rightarrow\scr{E}$ denote the universal family on $Q = \Quot_X^{d+nz\deg \Ox(1) +n(1-g)}(\Ox^N)$, and let $\pi$ with subscripts denote the projection from $\SL_N \times Q \times X$ onto the subscripts. Now we define the action $\sigma: \SL_N \times Q \rightarrow Q$ as the morphism associated to the family \[ \pi_{Q\times X}^*(U)\circ \pi_{\SL_N}^*(\iota): \O_{\SL_N \times Q \times X}^N\rightarrow \pi_{Q\times X}^*(\scr{E}). \]
One can check that this is indeed a group action. 

The next question to ask is what this does to $k$-points. Let $(g,p)$ be a $k$-point in $\SL_N \times Q$ (by abuse of notation, we identify $p$ with its fibre in its universal family). Then we one can check that $\sigma(g,p)$ is the quotient \[\sigma(g,p) = p\circ g^{-1}, \] where $g$ acts on $\O_{\SL_N\times Q \times X}^N$ in the obvious way; indeed, pulling $\iota$ back via $g: \Spec k \rightarrow \SL_N$ is just the map $g^{-1}: k^N \rightarrow k^N$, and thus pulling $\pi^*_{\SL_N}(\iota)$ back via $g\times \id \times \id: \Spec k \times Q \times X \rightarrow \SL_N \times Q \times X$ is just $g^{-1}: \O_{Q \times X}^N \rightarrow \O_{Q\times X}^N$ (where we abused a lot of notation). Finally, pulling the universal quotient back via $p$, we obtain the required  \[\sigma(g,p) = p\circ g^{-1}. \]
\par{} Finally, since we are doing a stability analysis, we will need to study the 1-PS's of this action. So let $\lambda: \bb{G}_m \rightarrow \SL_N$ be a 1-PS, and as before, we have a weight space decomposition \[k^N = V = \bigoplus_{r\in \Z} V_r. \] Write $V_{\leq r} = \bigoplus_{s\leq r} V_s$, so that we have a filtration $V_{\leq r} \subseteq V_{\leq r+1}$. Now if $q: \Ox^N \rightarrow \E$ is a $k$-point in $Q$, write \[\E_{\leq r}:= \im q|_{V_{\leq r}\otimes_k \Ox}\] and \[\E_r:= \E_{\leq r}/\E_{\leq r-1}. \] We are now in a position to state:

\begin{proposition}\label{HM weight of large linearisation}
	There exists a natural number $M_0 > 0$ such that for any $M\geq M_0$, there is a very ample linearisation $\L_M$ of this $\SL_N$ action, depending on $M$, such that for a $k$-point $q: \Ox^N \rightarrow \E$ and 1-PS $\lambda: \bb{G}_m \rightarrow \SL_N$ inducing a weight space decomposition and filtration as above, we have \[\mu^{\L_M}(q, \lambda) = \sum_{r\in \Z} P_{\E_{\leq r}}(M) - \frac{\dim V_{\leq r}}{N}P(M), \] where $P_{\E_{\leq r}}$ is the Hilbert polynomial of $\E_{\leq r}$ and $P$ is the Hilbert polynomial $d + nz\deg \Ox(1) + n(1-g)$ of $\E$.
\end{proposition}
\begin{proof}
	\autocite[p. 76]{ModuliNotes}.
\end{proof}
Thus the way forward is clear: we fix one such $M$ and use this expression for the weight and the Hilbert-Mumford criterion to calculate the stability of this linearised action, and eventually relate it to the usual vector bundle stability. To begin, let us investigate the sum $\sum_{r\in \Z} P_{\E_{\leq r}}(M) - \frac{\dim V_{\leq r}}{N}P(M)$. Suppose the weights of $\lambda$ are labelled $r_1<...<r_m$. Then for any $r < r_1$, it follows $V_{\leq r}  = 0$ whence $\E_{\leq r} = 0$ too, hence \[ P_{\E_{\leq r}}(M) - \frac{\dim V_{\leq r}}{N}P(M) = P_0(M) - \frac{0}{N}P(M) = 0.\] On the other end, if $r > r_m$, then it follows $\E_{\leq r} = \E$ and $\dim V_{\leq r} = N$, and so \[ P_{\E_{\leq r}}(M) - \frac{\dim V_{\leq r}}{N}P(M) = P(M) - \frac{N}{N}P(M) = 0\] too. In particular, we do get a finite sum. Now for any $r$ such that $r_i \leq r < r_{i+1}$, it follows $V_{\leq r_i} = V_{\leq r}$, hence $\E_{\leq r} = \E_{\leq r_i}$, and thus \[\sum_{r_i \leq r \leq r_{i+1}} P_{\E_{\leq r}}(M) - \frac{\dim V_{\leq r}}{N}P(M) = (r_{i+1} - r_i)\left(P_{\E_{\leq r_i}}(M) - \frac{\dim V_{\leq r_i}}{N}P(M)\right) \] whence \[\mu^{\L_M} = \sum_{r\in \Z} P_{\E_{\leq r}}(M) - \frac{\dim V_{\leq r}}{N}P(M) = \sum_{1 \leq i \leq m-1} (r_{i+1} - r_i)\left(P_{\E_{\leq r_i}}(M) - \frac{\dim V_{\leq r_i}}{N}P(M)\right).\] From this and the above proposition, we deduce:
\begin{proposition}\label{GIT criterion stability}
	Let $q: \Ox^N \rightarrow \E$ be a $k$-point in $Q$. Then $q$ is semistable with respect to $\L_M$ if and only if for any subspace $V' \subseteq V$ we have 
	\begin{equation}\label{git-semi-stability-quot-scheme}
		P_{\E'}(M) - \frac{\dim V'}{N}P(M) \geq 0.
	\end{equation} 
	where $\E' = \im q|_{V' \otimes \Ox}$. Stability holds if and only if the inequality is strict.
\end{proposition}
\begin{proof}
	Firstly, suppose the inequality holds. Now let $\lambda$ be a 1-PS with weights $r_1<...<r_m$. Now observe that for any $1 \leq i\leq m-1$, we have \[P_{\E_{\leq r_i}}(M) - \frac{\dim V_{\leq r_i}}{N}P(M) \geq 0, \] hence \[ \mu^{\L_M}(q, \lambda) = \sum_{1 \leq i \leq m-1} (r_{i+1} - r_i)\left(P_{\E_{\leq r_i}}(M) - \frac{\dim V_{\leq r_i}}{N}P(M)\right) \geq 0,\] and semistability follows from the Hilbert-Mumford criterion. If strict inequality holds in (\ref{git-semi-stability-quot-scheme}), then strict inequality holds above, and thus stability holds, as desired. 
	
	Conversely, suppose there is some $V'\subseteq V$ such that the strict reverse inequality holds in (\ref{git-semi-stability-quot-scheme}). Fix a complement $W$ such that $V = V' \oplus W$, and define the 1-PS $\lambda$ to act with equal weight $r_1$ on $V'$ and $r_2> r_1$ on $W$ (since we may scale bases as we want, this is always possible). Then \[ \mu^{\L_M}(q, \lambda) = (r_2 - r_1)\left(P_{\E_{\leq r_1}}(M) - \frac{\dim V_{\leq r_1}}{N}P(M)\right)  = (r_2 - r_1)\left(P_{\E'}(M) - \frac{\dim V'}{N}P(M)\right) \leq 0. \] and hence $q$ is unstable. If no strict reverse inequality holds, but equality holds, then $q$ is semistable, but not stable. 
\end{proof}

Note that this is starting to look like our notion of stability already, since we are relating quantities of $\E$ to quantities of subsheaves of $\E$. For example, if $q: \Ox^N \rightarrow \E$ is a point, $\E$ is locally free and $\E'$ is one such subsheaf and $H^1(X, \E') = 0$, then the Riemann-Roch theorem tells us that the degree of $\E'$ cannot be too small. And in fact, we can already concretely deduce:
\begin{corollary}
	Let $q: \Ox^N \rightarrow \E$ be a semistable $k$-point in $Q$ with respect to a sufficiently large $M$. Then:
	\begin{enumerate}
		\item $\E$ is a locally free sheaf.
		\item $H^0(q)$ is an isomorphism.
		%\item $\E$ is a semistable vector bundle.
	\end{enumerate}
\end{corollary} 
\begin{proof}
	Choose $M\geq M_0$, and such that $P(M) > N^2$. 
	
	Firstly suppose $\E$ is not locally free. Then $\E$ is not torsion free, so let $\F$ be a nonzero torsion subsheaf of $\E$. Then $\F$ is supported on a discrete set, since we are on a curve, and so $H^0(\F)\neq 0$. Now let $V' := H^0(q)^{-1}(H^0(\F))$ and let $\E' = \im q|_{V' \otimes \Ox}$. Since $q$ is surjective, it follows that $\E'$ is a subsheaf of $\F$ and hence is also torsion. But then $\E'$ has constant Hilbert polynomial (since it is supported on a discrete set, twisting does nothing), and moreover $h^1(\E) = 0$ (\autocite[III, Theorem 2.7]{Hart}), and so \[0 < P_{\E'}(M) = h^0(\E') \leq N \leq \frac{P(M)}{N} < \frac{\dim V'}{N}P(M). \] Hence $\E$ is unstable, which proves (i). 
	
	To prove (ii), let $V' = \ker H^0(q)$. Then $\E' = \im q|_{V'\otimes \Ox} =0$, and so \[0 = P_{\E'}(M) \geq \frac{\dim V'}{N}P(M), \] whence $\dim V' = 0$, as desired. To prove surjectivity, observe that by the Riemann-Roch theorem, it suffices to show that $H^1(X, \E) = 0$, whence $h^0( E) = N = \dim \im H^0(q)$. So suppose for contradiction $H^1(X, \E)\neq 0$; then by the Serre duality theorem there is a nonzero map $\varphi: \E \rightarrow \omega_X$. Let $\F$ denote the image of this map. Then $\F$ is a line bundle, and in particular we know that its Hilbert polynomial $P'$ is \[P'(z) = \deg \F + z\deg \Ox(1) +  1-g\leq g-1+z\deg \Ox(1), \] since $\deg \F \leq 2g-2$ by Lemma \ref{stability-properties} (iv). It thus follows that $\ker \varphi$ has Hilbert polynomial $P - P'$. We claim $h^0(\ker \varphi) \neq 0$. Indeed, since $\F$ injects into $\omega_X$, it follows that $h^0(\F)\leq g$, and hence \[ h^0(\ker \varphi) \geq h^0(\E) - h^0(\F) \geq d+n(1-g) - g > n(2g-1) + n(1-g) -g = ng -g \geq  0, \] as claimed, since we are assuming $g \geq 2$ and $n \geq 2$. Now let $V' := H^0(q)^{-1}(H^0(X, \ker \varphi))$, and let $\E':= \im q_{V' \otimes \Ox}\neq 0$, so that $\E' \subseteq \ker \varphi$, and thus $\E/\E'$ surjects onto $\F$. 
%	
%	
%	
%	Finally, to prove (iii), suppose $\E$ is unstable, and let $\F$ be a {\color{red} destabilising subbundle} with $\mu(\F)> \mu(\E)$. 
\end{proof}

Our next job is to relate stability of $Q$ as above back to vector bundle stability. To do this, observe that Proposition \ref{GIT criterion stability} defines stability in terms of the Hilbert polynomial of certain subsheaves and the dimension of the space of global sections, and thus we need some results to tie these quantities back into our usual degree and rank invariants. To begin, we have the following useful bound:
\begin{lemma}
	Let $\E$ be a semistable vector bundle of signature $(n,d)$ and slope $\mu$. Then
	\begin{equation}\label{le-potier-bounds}
		\frac{h^0(\E)}{n} \leq [\mu+1]_+ := \sup \{0, \mu +1\}.
	\end{equation} 
\end{lemma}
\begin{proof}
	We induct on the degree. Firstly, if $d < 0$, we claim $H^0(X, \E) = 0$. Indeed, if not, let $s\in H^0(X, \E)$ be nonzero. Then by Remark \ref{subbundle-exists-remark}, there exists a line subbundle $\L$ isomorphic to the subsheaf generated by $s$. In particular, $\L$ has a nonzero global section. But by semistability, $\deg \L < 0$, which contradicts Lemma \ref{negative-degree-no-h0}. Thus $H^0(X, \E)= 0$ as claimed. Now, suppose $d \geq 0$ and supposing this has been proven for all smaller values, we fix some $p\in X(k)$ and as in the proof of Lemma \ref{bounding-family}, we have the following short exact sequence \[0 \rightarrow \L(-p)\rightarrow \Ox \rightarrow k_p \rightarrow 0. \] Tensoring with $\E$  and taking cohomology, we obtain \[h^0(\E) \leq h^0(\E\otimes \L(-p)) +n, \] and finally, observing that \[\deg( \E \otimes \L(-p)) = d - n < d\] by Corollary \ref{twisting-by-line-bundle-degree} and applying the induction hypothesis, we obtain \[\frac{h^0(\E)}{n} \leq \frac{h^0(\E\otimes \L(-p))}{n} +1 \leq [\frac{d-n}{n}+1]_++1 = [\mu+1]_+ \] as desired.
\end{proof}

We now give another criterion for vector bundle (semi)-stability.
\begin{proposition}
	Let $n\geq 2$ be fixed, let $d > gn^2+n(2g-2)$ and let $\E$ be a vector bundle of signature $(n,d)$ and slope $\mu$. Then $\E$ is semistable if and only if for every subsheaf $\F$ of rank $n'$ we have \[h^0(\F)\leq \frac{n'}{n}h^0(\E). \] If strict inequality holds, then $\E$ is stable. 
\end{proposition}
\begin{proof}
	Firstly suppose $\E$ is unstable, and let $\F$ be a semistable subbundle of signature $(n', d')$ such that $\mu(\F)> \mu(\E)$ (for example, we can take $\F$ to be the first term in the Harder-Narasimhan filtration of $\E$). Now observe that \[d' > \mu n' \geq n' (gn + 2g-2)> n'(2g-2), \] so in particular $H^1(X, \F) = 0$ by Lemma \ref{bounding-family}. Thus by the Riemann-Roch theorem we have \[h^0(\F) = d'+n'(1-g) = n'(\mu(\F)+ 1-g) > n'(\mu(\E)+1-g) = \frac{n'}{n}(d+n(1-g)) = \frac{n'}{n}h^0(\E). \] If $\E$ is strictly semistable, then we repeat the argument with $\F$ of equal slope and deduce equality. 
	
	Conversely, suppose $\E$ is semistable. Let $\F$ be a subsheaf of signature $(n', d')$ and let $(\F_i)_{i = 0}^m$ be the Harder-Narasimhan filtration of $\F$. Observe that \[h^0(\F) =\sum_{i = 1}^m h^0(\F_i) - h^0(\F_{i-1})\leq \sum_{i = 1}^m h^0(\F_i/\F_{i-1}) \leq \sum_{i = 1}^m \rk(\F_i/\F_{i-1})[\mu(\F_i/\F_{i-1})+1 ]_+,   \] and furthermore note that $\mu(\F_{1}) \geq \mu(\F_i/\F_{i-1})$ by the construction of the Harder-Narasimhan filtration. In particular, since $\E$ is semistable we have $\mu(\F_i/\F_{i-1})\leq \mu$ and applying this to the above we have \[h^0(\F)\leq  \sum_{i = 1}^m \rk(\F_i/\F_{i-1})[\mu(\F_i/\F_{i-1})+1 ]_+ \leq (n'-1)(\mu+1) + [\mu(\F/\F_{m-1})+1]_+. \] Now if $\mu(\F/\F_{m-1}) \leq \mu - gn$, then \[h^0(\F)\leq (n'-1)(\mu+1) + (\mu-gn+1) \leq n'(\mu+1)-gn'= \frac{n'}{n}(d+n(1-g)) = \frac{n'}{n} h^0(\E), \] since $H^1(X, \E) = 0$ by Lemma \ref{bounding-family}. Otherwise, if $\mu(\F/\F_{m-1})>\mu-gn >2g-2$, we claim that $H^1(X, \F) = 0$, whence \[h^0(\F) = d'+n'(1-g)\leq \frac{n'd}{n} + n'(1-g) = \frac{n'}{n}(d+n(1-g)) =  \frac{n'}{n}h^0(\E).\] In fact, we will show inductively that each $H^1(X, \F_i) = 0$. Firstly note that $\mu(\F_1) \geq \mu(\F/\F_{m-1}) > 2g-2$ (with equality holding on the left if and only if $\F$ is semistable) whence $H^1(X, \F_1) = 0$ by Lemma \ref{bounding-family}. Now supposing we have shown $H^1(X, \F_{i-1}) = 0$, we observe that $H^1(X, \F_{i}/\F_{i-1}) = 0$ too, by the same reasoning as before, and thus by the long exact sequence of cohomology applied to \[0 \rightarrow \F_{i-1} \rightarrow \F_i \rightarrow \F_{i}/\F_{i-1} \rightarrow 0 \] we deduce that $H^1(X, \F_i) = 0$ too. This completes the proof.
\end{proof}

And finally, we have:
\begin{theorem}
	Let $n\geq 2$ be fixed, let $d > gn^2+n^2(2g-2)$, write $N:= d+n(1-g)$, let $P\in \Q[z]$ be the polynomial \[P:=d+nz\deg \Ox(1) + n(1-g), \] and finally let $Q = \Quot_X^{P}(\Ox^N)$. Then there exists an $M_0$ such that for any $M \geq M_0$, a $k$-point $q: \Ox^N \rightarrow \E$ of $Q$ is semistable with respect to the linearised $\SL_N$-action on $\L_M$ if and only if $\E$ is a semistable locally free sheaf and $H^0(q)$ is an isomorphism. 
\end{theorem}

